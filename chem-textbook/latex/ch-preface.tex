% ch-preface.tex — Preface
% Note: \chapter{} command is in the main chem-textbook.tex file.
% This file contains the chapter body only.

This is not a typical chemistry textbook. Most introductory chemistry courses ask you to memorize a large number of facts --- names of elements, rules for balancing equations, definitions you reproduce on exams and then forget. This book takes a different approach. Instead of asking \textit{what do you need to remember?}, it asks \textit{how do you need to think?}

The core idea is simple. Chemistry, like any science, rests on a surprisingly small set of \textbf{reasoning moves} --- portable cognitive tools that let you figure things out rather than look them up. When a doctor reads a drug interaction warning, when an environmental scientist evaluates a pollution claim, when you decide whether a cleaning product is safe to mix with another one, the underlying reasoning is the same: a handful of moves about composition, structure, energy, and change, applied in combination. This book teaches you those moves explicitly, so you can use them long after the course is over.

We have organized the material around \textbf{five question-driven domains}, each corresponding to a chapter, plus an optional sixth capstone:

\begin{enumerate}[nosep]
  \item \textbf{What is stuff made of?} (Composition)
  \item \textbf{How is it put together?} (Structure)
  \item \textbf{What drives change?} (Energy)
  \item \textbf{How much? How big?} (Scale)
  \item \textbf{What happens?} (Change)
  \item \textbf{Chemistry meets life} (an optional capstone that applies the same primitives to DNA, proteins, and biochemistry)
\end{enumerate}

Each chapter introduces between 10 and 15 reasoning moves, grounded in everyday examples --- tap water, cooking, breathing, batteries, medication. You will learn, for instance, that ``given a molecular formula, count atoms of each element to determine the formula's composition'' is a reasoning move, and that it composes naturally with moves from other domains to explain why ice floats, why vinegar reacts with baking soda, or why carbon dioxide traps heat.

The mathematical ceiling is intentionally low. You will need proportional reasoning (if one mole weighs this much, how much do two moles weigh?), basic algebra (solve for an unknown in a simple equation), and a brief encounter with logarithms when we discuss pH. That is all. No calculus, no advanced math, no assumed fluency with scientific notation beyond what you encountered in high school.

No prior chemistry is required beyond what you saw in a general high school science course. If you remember that water is \ce{H2O} and that the periodic table exists, you have enough background to begin.

A note on how this book was developed. The reasoning moves, their dependencies, and their composition patterns were built using a systematic taxonomy approach inspired by formal methods in logic and computer science. Every primitive was catalogued, every dependency was traced, and every cross-domain pattern was verified for completeness before a single chapter was drafted. The result is a textbook where nothing is introduced without a clear reason, nothing is redundant, and every concept connects to something you already know.

Welcome to chemistry as a way of thinking.
