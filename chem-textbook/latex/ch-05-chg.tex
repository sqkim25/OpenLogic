% ch-05-chg.tex — Chapter 5: What Happens? (Change/Transformation Domain)
% Converted from CH-05-CHG.md

% Note: \chapter{} command is in the main chem-textbook.tex file.
% This file contains the chapter body only.

\noindent\textit{Domain: CHG (Change/Transformation) --- Where atoms rearrange, bonds break and form, and chemistry actually does something}

\bigskip

You strike a match. The head scrapes across the rough strip, and for a fraction of a second nothing seems to happen. Then --- flame. A bright, hot tongue of fire leaps from the match tip. The red phosphorus and potassium chlorate that were sitting peacefully in the match head for months have, in an instant, become something entirely different: hot gases, a wisp of sulfur-scented smoke, and a charred wooden stick. The atoms that were there before are still there --- not one has been created or destroyed. But everything about them has changed. They have been rearranged. Old bonds have broken. New bonds have formed. Energy that was locked in chemical bonds has been released as heat and light.

This is a chemical reaction. It is what chemistry \textit{does}.

Every previous chapter in this book has been building toward this moment. You learned what stuff is made of (Chapter~\ref{ch:com}, Composition). You learned how atoms are connected and arranged (Chapter~\ref{ch:str}, Structure). You learned what drives change --- where energy comes from, where it goes, and whether a process is favorable (Chapter~\ref{ch:nrg}, Energy). You learned how to measure amounts and concentrations (Chapter~\ref{ch:scl}, Scale). Now we put it all together. This chapter asks the question that most people think of when they think of chemistry: \textbf{What happens when substances interact?}

The answer has many layers. Some reactions go to completion; others reach a balance point and stop partway. Some are fast; others are imperceptibly slow. Some transfer protons; others transfer electrons. Some rearrange atoms within molecules; others rearrange protons and neutrons within atomic nuclei. This chapter gives you the reasoning tools to navigate all of these transformations --- not by memorizing hundreds of individual reactions, but by recognizing patterns and deploying a small set of powerful reasoning moves.

We start with the most fundamental skill: reading and balancing a chemical equation.


%% ============================================================
\section{CHG.1: How Do We Describe What Happens in a Reaction?}
\label{sec:chg1}
%% ============================================================

\subsection*{The Big Idea: Reactions Are Atom-Rearrangement Stories}

A chemical reaction is a process in which substances (the \textbf{reactants}) are transformed into different substances (the \textbf{products}). At the atomic level, what happens is straightforward: bonds break, atoms rearrange, and new bonds form. No atoms are created. No atoms are destroyed. Every atom that enters a reaction exits the reaction --- just in a different arrangement.

This is the conservation principle you learned in Chapter~\ref{ch:com} (PRIM-COM006), now put to work. Conservation of atoms is not just an abstract idea --- it is the rule that governs how we write and balance chemical equations. Before we can ask how far a reaction goes, how fast it proceeds, or what drives it, we need a language for describing what happens. That language is the balanced chemical equation.

%% ---- Reasoning Move: PRIM-CHG001 ----
\begin{reasoningmove}{PRIM-CHG001}{Equation Reading and Balancing}
  \textbf{Reasoning move}: Given a chemical equation with reactants and products, balance it so that atoms of each element are conserved on both sides, and extract the quantitative mole ratios.
\end{reasoningmove}

\depends{PRIM-COM005}{chemical formula reading --- you must be able to decode formulas into atom identities and counts}{1}
\depends{PRIM-COM006}{conservation of atoms --- atom conservation is the principle that equation balancing enforces}{1}

\textbf{Why do we balance equations?}

Because atoms do not appear from nothing and do not vanish into nothing. If you start with 3 carbon atoms, 8 hydrogen atoms, and some oxygen atoms on the left side of an equation, you must end with exactly 3 carbon atoms, 8 hydrogen atoms, and the same total of oxygen atoms on the right side. Balancing is not a mathematical trick --- it is the direct consequence of conservation.

Think of it this way: a chemical equation is a recipe. The left side lists ingredients (reactants). The right side lists what you make (products). The coefficients --- the numbers in front of each formula --- tell you the proportions. And just as you cannot bake a cake and end up with more flour than you started with, you cannot run a reaction and end up with more atoms than you started with.

\textbf{How to read a chemical equation:}

Consider the combustion of propane, the fuel in a backyard grill:

\reaction{C3H8 + 5 O2 -> 3 CO2 + 4 H2O}

Reading this equation from left to right:

\begin{center}
\begin{tabular}{ll}
  \toprule
  \textbf{Component} & \textbf{What It Means} \\
  \midrule
  \ce{C3H8} & One molecule (or one mole) of propane \\
  + & ``reacts with'' \\
  \ce{5 O2} & Five molecules (or moles) of oxygen gas \\
  $\to$ & ``to produce'' \\
  \ce{3 CO2} & Three molecules (or moles) of carbon dioxide \\
  + \ce{4 H2O} & and four molecules (or moles) of water \\
  \bottomrule
\end{tabular}
\end{center}

The coefficients (1, 5, 3, 4) are the \textbf{mole ratios}. They tell you that for every 1 mole of propane consumed, 5 moles of oxygen are consumed, 3 moles of \ce{CO2} are produced, and 4 moles of \ce{H2O} are produced. These ratios are the quantitative heart of the equation.

\textbf{How to balance an equation:}

Balancing is a systematic process. You adjust the coefficients (never the subscripts --- changing a subscript changes the substance itself) until every element has the same atom count on both sides.

\textbf{Worked example: Balancing propane combustion.}

Start with the unbalanced equation:

\reaction{C3H8 + O2 -> CO2 + H2O}

Step 1: Count atoms on each side.

\begin{center}
\begin{tabular}{lll}
  \toprule
  \textbf{Element} & \textbf{Left side} & \textbf{Right side} \\
  \midrule
  C & 3 & 1 \\
  H & 8 & 2 \\
  O & 2 & 3 \\
  \bottomrule
\end{tabular}
\end{center}

Not balanced. Carbon, hydrogen, and oxygen all disagree.

Step 2: Balance carbon first. There are 3 C on the left, so put a 3 in front of \ce{CO2}:

\reaction{C3H8 + O2 -> 3 CO2 + H2O}

Step 3: Balance hydrogen. There are 8 H on the left, so put a 4 in front of \ce{H2O} (since each \ce{H2O} has 2 H, and $4 \times 2 = 8$):

\reaction{C3H8 + O2 -> 3 CO2 + 4 H2O}

Step 4: Balance oxygen last (it appears in multiple products). Count O on the right: $3 \times 2 = 6$ (from \ce{CO2}) + $4 \times 1 = 4$ (from \ce{H2O}) = 10 oxygen atoms. So we need 10 O on the left. Since \ce{O2} has 2 per molecule, the coefficient is 5:

\reaction{C3H8 + 5 O2 -> 3 CO2 + 4 H2O}

Step 5: Verify.

\begin{center}
\begin{tabular}{llll}
  \toprule
  \textbf{Element} & \textbf{Left side} & \textbf{Right side} & \textbf{Balanced?} \\
  \midrule
  C & 3 & 3 & Yes \\
  H & 8 & 8 & Yes \\
  O & 10 & 10 & Yes \\
  \bottomrule
\end{tabular}
\end{center}

Every atom is accounted for. Conservation holds.

\begin{hook}{Why incomplete combustion kills}
  The balanced equation above describes \textit{complete} combustion --- when there is plenty of oxygen. But what happens in a poorly ventilated space, like a garage with a running car or a charcoal grill used indoors? There is not enough \ce{O2} to fully combust the fuel. Instead of \ce{CO2}, the carbon atoms form carbon monoxide (\ce{CO}):

  \reaction{2 C3H8 + 7 O2 -> 6 CO + 8 H2O}

  This is incomplete combustion. \ce{CO} is colorless, odorless, and lethal --- it binds to hemoglobin in your blood 200 times more tightly than oxygen does, preventing oxygen delivery to your tissues. The difference between a safe barbecue and a deadly one is, at the molecular level, a question of oxygen supply. Balanced equations make this visible: compare the oxygen coefficients. Complete combustion needs \ce{5 O2} per propane. When oxygen is limited, the products change, and \ce{CO2} is replaced by the deadly \ce{CO}.
\end{hook}

\textbf{What coefficients tell you (and what they do not):}

Coefficients give you mole ratios --- the proportions in which reactants are consumed and products are formed. They do not tell you how fast the reaction proceeds (that is CHG.3), how far it goes (CHG.2), or whether it is favorable (that was Chapter~\ref{ch:nrg}). They tell you only the \textit{stoichiometric recipe}: if the reaction does occur, these are the proportions.

%% ---- Reasoning Move: PRIM-CHG002 ----
\begin{reasoningmove}{PRIM-CHG002}{Reaction Type Recognition}
  \textbf{Reasoning move}: Given reactants, classify the transformation type --- synthesis, decomposition, single replacement, double replacement, or combustion --- and use the classification to predict the general form of the products.
\end{reasoningmove}

\depends{PRIM-CHG001}{you must be able to read and balance equations before classifying them}{5}
\depends{PRIM-STR001}{bond type reasoning --- distinguishing ionic from covalent bonds helps identify which type of rearrangement is occurring}{2}

Think of reaction types as a \textbf{field guide}. Just as a birdwatcher learns to recognize families of birds --- raptors, songbirds, waterfowl --- before identifying individual species, a chemist learns to recognize families of reactions before analyzing individual transformations. Knowing the family tells you the general pattern. The pattern tells you what to expect for products.

Here are the five major reaction types:

\textbf{Type 1: Synthesis (Combination)}

\begin{center}
Pattern: A + B $\to$ AB
\end{center}

Two or more simple substances combine to form a more complex product. Think of it as assembly.

\textit{How to recognize it}: Two reactants, one product. Simple substances combine into a compound.

\textit{Example}: Iron reacts with oxygen to form iron(III) oxide (rust):
\reaction{4 Fe + 3 O2 -> 2 Fe2O3}

\textit{Prediction}: If you see two elements or simple compounds on the left and nothing else, predict they combine into a single compound.

\textbf{Type 2: Decomposition}

\begin{center}
Pattern: AB $\to$ A + B
\end{center}

A single compound breaks apart into simpler substances. The reverse of synthesis.

\textit{How to recognize it}: One reactant, two or more products.

\textit{Example}: Hydrogen peroxide decomposes:
\reaction{2 H2O2 -> 2 H2O + O2}

\textit{Prediction}: If you see a single compound on the left, predict it breaks into simpler pieces.

\textbf{Type 3: Single Replacement (Single Displacement)}

\begin{center}
Pattern: A + BC $\to$ AC + B
\end{center}

An element displaces another element from a compound, taking its place.

\textit{How to recognize it}: A free element plus a compound on the left. One of the elements in the compound has been ``kicked out'' and replaced.

\textit{Example}: Zinc displaces copper from copper sulfate solution:
\reaction{Zn + CuSO4 -> ZnSO4 + Cu}

\textit{Prediction}: The more reactive element (recall PRIM-COM003, periodic position reasoning, Chapter~\ref{ch:com}) displaces the less reactive one.

\textbf{Type 4: Double Replacement (Double Displacement / Metathesis)}

\begin{center}
Pattern: AB + CD $\to$ AD + CB
\end{center}

Two compounds exchange partners. The positive part of one compound pairs with the negative part of the other, and vice versa. This is an ionic exchange --- recall PRIM-STR001 (bond type reasoning, Chapter~\ref{ch:str}): the substances involved are typically ionic compounds whose ions swap partners in solution.

\textit{How to recognize it}: Two ionic compounds on the left, two different ionic compounds on the right. The cations and anions have switched.

\textit{Example}: Baking soda reacts with vinegar (acetic acid):
\reaction{NaHCO3 + CH3COOH -> CH3COONa + H2O + CO2}

This is the classic baking soda and vinegar ``volcano.'' The fizzing is \ce{CO2} gas escaping. At the molecular level, the sodium ion from baking soda pairs with the acetate ion from vinegar, while the \ce{H+} and \ce{HCO3-} combine and decompose into water and carbon dioxide.

\textit{Prediction}: When two ionic compounds are mixed, swap their cation-anion partners. If one of the potential products is a gas, a precipitate (insoluble solid), or water, the reaction is likely to proceed.

\textbf{Type 5: Combustion}

\begin{center}
Pattern: Fuel + \ce{O2} $\to$ \ce{CO2} + \ce{H2O}
\end{center}

A carbon-containing (organic) compound reacts with oxygen to produce carbon dioxide and water. This is the reaction class that powers grills, engines, and your body's metabolism.

\textit{How to recognize it}: A hydrocarbon or other organic compound reacts with \ce{O2}. Products are \ce{CO2} and \ce{H2O} (and sometimes other oxides if the fuel contains nitrogen or sulfur).

\textit{Example}: Methane (natural gas) combustion:
\reaction{CH4 + 2 O2 -> CO2 + 2 H2O}

\textit{Prediction}: If the reactant contains C and H (and possibly O), and it reacts with \ce{O2}, the products are \ce{CO2} and \ce{H2O}.

\textbf{Field guide summary:}

\begin{center}
\begin{tabular}{lllp{4cm}}
  \toprule
  \textbf{Type} & \textbf{Pattern} & \textbf{How to Recognize} & \textbf{Product Prediction} \\
  \midrule
  Synthesis & A + B $\to$ AB & Two simple $\to$ one complex & Elements/simple compounds combine \\
  Decomposition & AB $\to$ A + B & One complex $\to$ simpler pieces & Compound breaks apart \\
  Single replacement & A + BC $\to$ AC + B & Element + compound & More reactive element takes the spot \\
  Double replacement & AB + CD $\to$ AD + CB & Two compounds swap ions & Cation-anion partners exchange \\
  Combustion & Fuel + \ce{O2} $\to$ \ce{CO2} + \ce{H2O} & Organic + \ce{O2} & Carbon $\to$ \ce{CO2}, hydrogen $\to$ \ce{H2O} \\
  \bottomrule
\end{tabular}
\end{center}

\textbf{Why classify?} Classification is not an end in itself. The point is prediction. When you encounter unfamiliar reactants, recognizing the reaction type immediately tells you what general form the products will take. You do not need to memorize thousands of individual reactions. You need to recognize patterns.

%% ---- Reasoning Chain ----
\begin{reasoningchain}{From Reactants to Products}
  \chainitem{PRIM-COM005}{Chemical formula reading}{Read the molecular formulas of the reactants (Chapter~\ref{ch:com}).}
  \chainitem{PRIM-COM004}{Substance classification}{Identify whether the reactants are elements, compounds, or a mix (Chapter~\ref{ch:com}).}
  \chainitem{---}{Pattern matching}{Match the combination to a reaction type: if two compounds are present, check for double replacement; if an element and a compound, check for single replacement; if a compound reacts with \ce{O2}, check for combustion.}
  \chainitem{PRIM-CHG002}{Reaction type recognition}{Use the type pattern to predict the general form of the products.}
  \chainitem{PRIM-CHG001 + PRIM-COM006}{Balancing and conservation}{Balance the equation to confirm atom conservation.}
\end{reasoningchain}

%% ---- Practice Questions: CHG.1 ----
\begin{practicequestions}
  \practiceq{Balance the following equation for the combustion of ethanol (\ce{C2H5OH}):

  \reaction{C2H5OH + O2 -> CO2 + H2O}

  How many moles of \ce{O2} are needed per mole of ethanol? How many moles of \ce{CO2} are produced?}

  \practiceq{Classify each reaction by type:
  \begin{itemize}[nosep]
    \item \ce{2 Na + Cl2 -> 2 NaCl}
    \item \ce{CaCO3 -> CaO + CO2}
    \item \ce{AgNO3 + NaCl -> AgCl + NaNO3}
  \end{itemize}
  For each, explain how you recognized the type from the reactant-product pattern.}

  \practiceq{A student writes the equation: \ce{H2 + O2 -> H2O}. They claim it is balanced because ``hydrogen and oxygen are on both sides.'' What mistake are they making? Balance the equation correctly and verify by counting atoms.}
\end{practicequestions}

\medskip

\noindent\textit{You can now read, balance, and classify chemical equations. But a balanced equation tells you nothing about how far the reaction actually goes. Does it run to completion, using up all the reactants? Or does it stop partway, leaving some reactants behind? That is the question of equilibrium, and it is where we turn next.}


%% ============================================================
\section{CHG.2: How Far Does a Reaction Go?}
\label{sec:chg2}
%% ============================================================

\subsection*{The Big Idea: Not All Reactions Run to Completion}

When you strike a match, the phosphorus and oxygen react completely --- you do not find leftover phosphorus sitting on the match tip when the flame goes out. But many reactions do not work this way. Many reactions proceed partway and then appear to stop, even though reactants are still present.

Consider opening a can of soda. The carbonated water inside was sealed under pressure with dissolved \ce{CO2}. The moment you crack the tab, the pressure drops, and \ce{CO2} begins escaping as bubbles. But the soda does not go flat instantly. It takes time. And if you reseal the can (with a clip, say), the fizzing slows and eventually seems to stop --- even though there is still dissolved \ce{CO2} in the liquid.

What is happening? The \ce{CO2} molecules have not stopped moving. They are still escaping from the liquid into the gas above. But at the same time, \ce{CO2} molecules in the gas above are dissolving back into the liquid. When the rate of escape equals the rate of re-dissolving, the system reaches a balance point. The macroscopic concentration of dissolved \ce{CO2} stabilizes, even though individual molecules are constantly shuttling back and forth. This balance point is called \textbf{chemical equilibrium}.

%% ---- Reasoning Move: PRIM-CHG003 ----
\begin{reasoningmove}{PRIM-CHG003}{Equilibrium Reasoning}
  \textbf{Reasoning move}: Given a reversible process, explain why macroscopic properties stabilize even though molecular-level activity continues.
\end{reasoningmove}

\depends{PRIM-CHG001}{you must understand chemical equations before reasoning about forward and reverse reactions}{5}
\depends{PRIM-SCL003}{concentration reasoning --- equilibrium is defined in terms of concentrations remaining constant}{4}
\depends{PRIM-NRG005}{spontaneity reasoning --- the direction of a reaction depends on whether it is energetically/entropically favorable}{3}

\textbf{The critical threshold: ``static'' versus ``dynamic''}

Here is the single most important conceptual leap in this section: \textbf{equilibrium is NOT a state where reactions have stopped.} It is a state where the forward and reverse reactions are occurring at equal rates, so the macroscopic concentrations do not change.

An analogy: imagine a footbridge over a stream, with people walking in both directions. At rush hour, 50 people per minute cross from east to west, and 50 people per minute cross from west to east. The bridge looks static from the air --- the number of people on each side stays constant. But the individuals are moving constantly. Equilibrium is like this: a dynamic steady state, not a frozen one.

\textbf{What equilibrium looks like:}

For any reversible reaction:

\reaction{A + B <=> C + D}

The double arrow (\ce{<=>}) means the reaction can proceed in both directions. Initially, if you start with only A and B, the forward reaction dominates: A and B are consumed, and C and D accumulate. But as C and D build up, the reverse reaction begins to compete. Eventually, the forward rate equals the reverse rate. At that point:

\begin{itemize}[nosep]
  \item The concentrations of A, B, C, and D stop changing.
  \item The reaction has not stopped --- molecules are still reacting in both directions.
  \item The system has reached equilibrium.
\end{itemize}

\textbf{What determines the balance point?}

Every reaction has a characteristic equilibrium position. Some reactions heavily favor products --- nearly all reactants are converted. We say these reactions have a \textbf{large equilibrium constant ($K$)}. Other reactions heavily favor reactants --- very little product forms. These have a \textbf{small $K$}.

Think of $K$ as a dial:

\figurebox{The equilibrium dial: $K$ very small (mostly reactants) on the left, $K \approx 1$ (both present) in the center, $K$ very large (mostly products) on the right.}{fig:k-dial}

You do not need to calculate $K$ in this course. You need to reason qualitatively: when someone tells you $K$ is large, you understand that at equilibrium, products dominate. When $K$ is small, reactants dominate.

\begin{hook}{Why soda goes flat faster when warm}
  Recall the soda can. The equilibrium between dissolved \ce{CO2} and gaseous \ce{CO2} depends on temperature. At higher temperatures, the equilibrium shifts toward gas --- \ce{CO2} molecules have more kinetic energy to escape the liquid. This is why warm soda goes flat faster than cold soda. The equilibrium position has changed: at higher temperature, less \ce{CO2} remains dissolved. A cold soda, kept sealed, maintains more dissolved \ce{CO2} because the equilibrium favors the dissolved form at lower temperatures.

  This connects directly to PRIM-NRG005 (spontaneity reasoning, Chapter~\ref{ch:nrg}): the dissolution of \ce{CO2} in water releases a small amount of heat (it is slightly exothermic). Warming the system disfavors the exothermic direction, shifting equilibrium toward the gas phase. The energy reasoning and the equilibrium reasoning tell the same story.
\end{hook}

%% ---- Reasoning Move: DEF-CHG003 ----
\begin{reasoningmove}{DEF-CHG003}{Le Chatelier's Principle}
  \textbf{Reasoning move}: Given a system at equilibrium plus a stress (change in concentration, temperature, or pressure), predict the direction the equilibrium shifts to partially counteract the stress.
\end{reasoningmove}

\depends{PRIM-CHG003}{you must understand equilibrium before predicting how it responds to stress}{5}

Le Chatelier's principle states: \textbf{when a system at equilibrium is subjected to a stress, the system shifts in the direction that partially counteracts the stress.}

The key word is \textit{partially}. The system does not fully undo the stress. It shifts to reduce the effect of the stress, but the stress is never completely reversed.

\textbf{The three types of stress:}

\textbf{Stress 1: Concentration change.}

If you add more of a reactant, the system shifts toward products (consuming the added reactant). If you remove a product, the system shifts toward products (replenishing what was removed).

\textit{Example}: In the reaction \ce{N2 + 3 H2 <=> 2 NH3}, adding more \ce{N2} shifts the equilibrium to the right --- more \ce{NH3} is produced. Removing \ce{NH3} as it forms (by cooling it into a liquid) also shifts equilibrium to the right.

\textbf{Stress 2: Temperature change.}

Think of heat as a ``reactant'' in endothermic reactions and a ``product'' in exothermic reactions.

\begin{itemize}[nosep]
  \item If the forward reaction is exothermic (releases heat), raising the temperature shifts equilibrium toward reactants (the system counteracts the added heat by favoring the endothermic, reverse direction).
  \item If the forward reaction is endothermic (absorbs heat), raising the temperature shifts equilibrium toward products.
\end{itemize}

\textit{Example}: The Haber process for making ammonia (\ce{N2 + 3 H2 <=> 2 NH3}) is exothermic in the forward direction. Raising the temperature shifts equilibrium to the left, producing less \ce{NH3}. This is why the Haber process uses moderate temperatures --- hot enough for a reasonable rate, but not so hot that the equilibrium shifts too far toward reactants.

\textbf{Stress 3: Pressure change (for reactions involving gases).}

Increasing pressure shifts equilibrium toward the side with fewer gas molecules. Decreasing pressure shifts toward the side with more gas molecules.

\textit{Example}: In \ce{N2(g) + 3 H2(g) <=> 2 NH3(g)}, there are 4 moles of gas on the left ($1 + 3$) and 2 moles of gas on the right. Increasing pressure shifts equilibrium toward the right (fewer gas molecules = less pressure). This is why the Haber process operates at high pressure --- it pushes the equilibrium toward the desired product.

\textbf{Common misconception: Le Chatelier fully reverses the stress.}

If you add more \ce{N2} to the Haber reaction, the system shifts right, consuming some of that added \ce{N2}. But the new equilibrium concentration of \ce{N2} is still higher than the original concentration. The shift partially counteracts the addition --- it does not restore the original state.

\begin{hook}{Hyperventilation and blood chemistry}
  Your blood maintains a critical equilibrium:

  \reaction{CO2 + H2O <=> H2CO3 <=> H+ + HCO3-}

  When you hyperventilate (breathe too rapidly), you exhale excess \ce{CO2}. Removing \ce{CO2} from the left side shifts the equilibrium to the left. This reduces \ce{H+} concentration, raising blood pH. The result --- respiratory alkalosis --- causes tingling, dizziness, and lightheadedness. Breathing into a paper bag reintroduces \ce{CO2} to the system, shifting equilibrium back to the right and restoring normal blood pH.

  Le Chatelier's principle, applied to a reaction occurring inside your bloodstream, explains why a simple paper bag can resolve a medical emergency.
\end{hook}

%% ---- Practice Questions: CHG.2 ----
\begin{practicequestions}
  \practiceq{Consider the reaction: \ce{2 SO2(g) + O2(g) <=> 2 SO3(g)}, which is exothermic in the forward direction. Predict the effect of each stress on the equilibrium:
  \begin{itemize}[nosep]
    \item Adding more \ce{O2}
    \item Raising the temperature
    \item Increasing the total pressure
  \end{itemize}}

  \practiceq{A student says, ``At equilibrium, the concentrations of reactants and products are equal.'' Is this correct? Explain using the concept of $K$ (large vs.\ small).}

  \practiceq{Explain, using Le Chatelier's principle, why a sealed bottle of soda remains fizzy in the refrigerator but goes flat quickly once opened and left on a warm countertop. (Hint: consider both temperature and pressure changes.)}
\end{practicequestions}

\medskip

\noindent\textit{Equilibrium tells you how far a reaction goes. But it says nothing about how quickly it gets there. A reaction might strongly favor products (large $K$), yet proceed so slowly that nothing seems to happen for centuries. That disconnect between ``favorable'' and ``fast'' is the subject of the next section.}


%% ============================================================
\section{CHG.3: What Controls How Fast a Reaction Proceeds?}
\label{sec:chg3}
%% ============================================================

\subsection*{The Big Idea: Favorable Does Not Mean Fast}

Diamond is thermodynamically unstable relative to graphite. At room temperature and pressure, the conversion of diamond to graphite is spontaneous --- the free energy scorecard (Chapter~\ref{ch:nrg}) says it should happen. Yet every diamond on Earth has been sitting unchanged for millions, sometimes billions, of years. The ``reaction'' proceeds so slowly that it is undetectable on any human timescale.

This is a profound lesson: \textbf{whether a reaction is favorable (thermodynamics) and how fast it proceeds (kinetics) are separate questions.} Equilibrium (CHG.2) and spontaneity (Chapter~\ref{ch:nrg}) tell you where a reaction wants to go. Rate tells you how quickly it gets there. A reaction can be favorable but slow (diamond to graphite), fast but unfavorable (essentially impossible), or both favorable and fast (the match you struck at the start of this chapter).

%% ---- Reasoning Move: PRIM-CHG004 ----
\begin{reasoningmove}{PRIM-CHG004}{Rate Reasoning}
  \textbf{Reasoning move}: Given conditions (temperature, concentration, surface area, catalyst), predict whether the reaction rate will increase or decrease, using collision theory as the explanatory framework.
\end{reasoningmove}

\depends{PRIM-CHG001}{you must understand the reaction being discussed}{5}
\depends{PRIM-NRG006}{activation energy reasoning --- rate depends on how many molecules have enough energy to overcome the activation barrier}{3}
\depends{PRIM-SCL003}{concentration reasoning --- more molecules per unit volume means more collisions}{4}

\textbf{Collision theory: the simple framework.}

For a reaction to occur, reactant molecules must:

\begin{enumerate}[nosep]
  \item \textbf{Collide} with each other.
  \item Collide with sufficient \textbf{energy} (at least as much as the activation energy, $E_a$ --- recall PRIM-NRG006, Chapter~\ref{ch:nrg}).
  \item Collide with the correct \textbf{orientation} (the reactive parts of the molecules must face each other).
\end{enumerate}

If any of these three conditions is not met, no reaction occurs during that collision. Most collisions, in fact, are unproductive --- the molecules bounce off each other unchanged. The reaction rate depends on how many collisions per second are effective (meeting all three conditions).

\textbf{The four factors that control reaction rate:}

\textbf{Factor 1: Temperature.}

Raising the temperature increases the average kinetic energy of molecules. Faster-moving molecules collide more often, and a larger fraction of those collisions have enough energy to overcome the activation barrier. Both effects increase the rate.

\begin{hook}{Refrigerating food}
  Refrigerating food slows spoilage. The chemical reactions that cause food to decay --- oxidation of fats, enzyme-catalyzed breakdown of proteins --- all slow down at lower temperatures because fewer molecular collisions have sufficient energy.
\end{hook}

\textbf{Factor 2: Concentration.}

Higher concentration means more molecules per unit volume. More molecules in the same space means more frequent collisions. More collisions means a higher reaction rate.

\begin{hook}{Blowing on a campfire}
  Blowing on a campfire (increasing oxygen concentration near the fuel) makes it burn faster. The wood is already reacting with oxygen; blowing provides more \ce{O2} molecules per unit volume, increasing the collision frequency.
\end{hook}

\textbf{Factor 3: Surface area.}

For reactions involving solids, only the molecules at the surface are exposed to other reactants. Grinding a solid into a fine powder dramatically increases the surface area, exposing more molecules to collisions.

\begin{hook}{Sugar dissolution}
  A sugar cube dissolves slowly in water. Granulated sugar dissolves faster. Powdered sugar dissolves almost instantly. Same substance, same mass --- but vastly different surface areas.
\end{hook}

\textbf{Factor 4: Catalyst.} (See DEF-CHG001 below.)

\textbf{Summary of rate factors:}

\begin{center}
\begin{tabular}{lllp{5.5cm}}
  \toprule
  \textbf{Factor} & \textbf{Change} & \textbf{Effect on Rate} & \textbf{Why (via collision theory)} \\
  \midrule
  Temperature & Increase & Faster & More energetic collisions; more exceed $E_a$ \\
  Concentration & Increase & Faster & More molecules per volume; more frequent collisions \\
  Surface area & Increase & Faster & More exposed molecules available to collide \\
  Catalyst & Add & Faster & Lowers $E_a$; more collisions are effective \\
  \bottomrule
\end{tabular}
\end{center}

\textbf{Critical misconception: ``favorable'' does not equal ``fast.''}

This is worth repeating because it is the single most common confusion in introductory chemistry. A reaction can be:

\begin{itemize}[nosep]
  \item Favorable AND fast (combustion of gasoline --- spontaneous and rapid)
  \item Favorable AND slow (diamond to graphite --- spontaneous but takes geological time)
  \item Unfavorable AND irrelevant (you do not need to worry about rate if the reaction will not happen regardless)
\end{itemize}

Thermodynamics tells you \textit{whether}. Kinetics tells you \textit{when}. They are independent assessments.

\begin{hook}{Pressure cookers speed cooking}
  A pressure cooker speeds cooking. Inside the sealed pot, water boils at a higher temperature (recall the pressure cooker example from Chapter~\ref{ch:scl}). The higher temperature increases the rate of all the chemical reactions in the food --- proteins denature faster, starches hydrolyze faster, flavors develop faster. You are not changing what reactions occur; you are changing how quickly they proceed.
\end{hook}

%% ---- Reasoning Move: DEF-CHG001 ----
\begin{reasoningmove}{DEF-CHG001}{Catalyst}
  \textbf{Reasoning move}: A catalyst speeds a reaction by providing a lower-activation-energy pathway, without being consumed in the process.
\end{reasoningmove}

\depends{PRIM-CHG004}{you must understand rate reasoning before understanding how a catalyst modifies it}{5}
\depends{PRIM-NRG006}{activation energy reasoning --- a catalyst works by lowering $E_a$}{3}

A catalyst is a substance that increases the rate of a reaction without being consumed by it. It participates in the reaction mechanism --- it is there during the reaction, facilitating the bond-breaking and bond-forming steps --- but it emerges at the end chemically unchanged.

\textbf{Three key properties of catalysts:}

\textbf{Property 1: A catalyst lowers the activation energy ($E_a$), not the overall energy change.}

Think of a mountain pass. The reactants are on one side, the products are on the other, and the activation energy is the height of the mountain they must cross. A catalyst does not move the starting or ending elevation --- the energy difference between reactants and products (\DH{} or \DG{}) is unchanged. Instead, it carves a lower pass through the mountain. More molecules can get over the lower pass, so the reaction proceeds faster.

\figurebox{Energy diagram showing uncatalyzed reaction with high $E_a$ peak and catalyzed reaction with lower $E_a$ peak. Both pathways connect the same reactant and product energy levels.}{fig:catalyst-energy}

The gap between reactants and products (the energy of the overall reaction) is the same in both pathways. Only the height of the barrier changes.

\textbf{Property 2: A catalyst is NOT consumed.}

Unlike a reactant, a catalyst is not used up. It may temporarily form an intermediate complex with the reactants, but it is regenerated at the end of each catalytic cycle. This is why a small amount of catalyst can accelerate the conversion of an enormous amount of reactant.

\textbf{Property 3: A catalyst does NOT change the equilibrium position ($K$).}

This is subtle but important. A catalyst speeds up both the forward and reverse reactions equally. It helps the system reach equilibrium faster, but it does not change where the equilibrium lies. If a reaction produces 70\% products at equilibrium without a catalyst, it still produces 70\% products with a catalyst --- it just gets there sooner.

\textbf{Misconception: Catalysts ``make reactions happen.''}

Catalysts do not make impossible reactions possible. They accelerate reactions that are already thermodynamically favorable. If the free energy scorecard (Chapter~\ref{ch:nrg}) says a reaction is non-spontaneous, no catalyst can force it to proceed. Catalysts affect rate, not favorability.

\begin{hook}{Catalytic converters}
  Your car's catalytic converter contains platinum (Pt) and palladium (Pd) metals that catalyze the conversion of toxic exhaust gases (\ce{CO}, nitrogen oxides, unburned hydrocarbons) into less harmful products (\ce{CO2}, \ce{N2}, \ce{H2O}). The reactions are thermodynamically favorable, but too slow at exhaust temperatures without the catalyst.
\end{hook}

\begin{hook}{Lactase enzyme}
  Lactose (milk sugar) can be broken down into glucose and galactose. The reaction is favorable. But without the enzyme lactase, it proceeds too slowly to be useful during digestion. People who produce insufficient lactase cannot efficiently digest lactose --- this is lactose intolerance. The enzyme is a biological catalyst: it lowers the activation energy for lactose hydrolysis, allowing the reaction to proceed fast enough for the body to absorb the products. Enzymes are biological catalysts --- proteins whose three-dimensional shapes (recall Chapter~\ref{ch:str}, structure) create precisely shaped active sites that bind specific reactant molecules and facilitate their transformation.
\end{hook}

%% ---- Practice Questions: CHG.3 ----
\begin{practicequestions}
  \practiceq{Explain why a log in a fireplace burns slowly, but sawdust from the same log can explode in a confined space. Which factor from the rate table is most relevant?}

  \practiceq{A student claims that adding a catalyst to a reaction will produce more product at equilibrium. Is this correct? Explain.}

  \practiceq{Use collision theory to explain why food spoils faster on a hot summer day than in a refrigerator. Which of the three collision requirements (collision frequency, energy, orientation) is most affected by the temperature change?}
\end{practicequestions}

\medskip

\noindent\textit{You now understand both where reactions go (equilibrium) and how fast they get there (kinetics). The next three sections explore specific types of chemical change. We start with one of the most practically important: the transfer of protons between molecules.}


%% ============================================================
\section{CHG.4: How Do Acids and Bases Interact?}
\label{sec:chg4}
%% ============================================================

\subsection*{The Big Idea: Proton Transfer Chemistry}

Squeeze a lemon into a glass of water. The water becomes sour. Drop an antacid tablet into that lemon water. The sourness diminishes. What just happened?

At the molecular level, the lemon juice donated protons (\ce{H+} ions) to the water, making it acidic. The antacid accepted those protons, neutralizing the acid. This proton transfer --- one substance donating \ce{H+}, another accepting it --- is the essence of acid-base chemistry. It governs the pH of your blood, the effectiveness of your shampoo, the safety of your swimming pool, and the discomfort of an upset stomach.

%% ---- Reasoning Move: PRIM-CHG005 ----
\begin{reasoningmove}{PRIM-CHG005}{Acid-Base Reasoning}
  \textbf{Reasoning move}: Given two substances, identify the proton donor (Br{\o}nsted acid) and proton acceptor (Br{\o}nsted base), and predict the products of the proton transfer.
\end{reasoningmove}

\depends{PRIM-CHG001}{you must be able to write the reaction equation}{5}
\depends{PRIM-STR002}{bond polarity reasoning --- bond polarity explains why certain molecules release \ce{H+} easily}{2}
\depends{PRIM-SCL003}{concentration reasoning --- the degree of proton transfer determines acid/base strength}{4}

\textbf{The Br{\o}nsted-Lowry definition:}

\begin{itemize}[nosep]
  \item A \textbf{Br{\o}nsted acid} is a proton (\ce{H+}) donor.
  \item A \textbf{Br{\o}nsted base} is a proton (\ce{H+}) acceptor.
\end{itemize}

Every acid-base reaction is a proton transfer from the acid to the base.

\textbf{Why do some molecules donate protons?}

This is where PRIM-STR002 (bond polarity, Chapter~\ref{ch:str}) comes in. Consider hydrochloric acid (\ce{HCl}). The H--Cl bond is polar: chlorine is much more electronegative than hydrogen, so the electron density is pulled toward Cl. This makes it easy for the H to leave as \ce{H+} (essentially a bare proton), with the electron pair staying with Cl to form \ce{Cl-}.

\reaction{HCl -> H+ + Cl-}

The same logic applies to other acids. In acetic acid (\ce{CH3COOH}, the acid in vinegar), the O--H bond in the carboxylic acid group (--COOH) is polar. The oxygen pulls electron density away from the hydrogen, making it available for donation to a base.

\textbf{Strong versus weak acids and bases:}

\begin{itemize}[nosep]
  \item A \textbf{strong acid} donates protons completely. In water, essentially every molecule of \ce{HCl} releases its \ce{H+}. The reaction goes to completion.
  \item A \textbf{weak acid} donates protons partially. In water, only a small fraction of acetic acid molecules release their \ce{H+} at any given time. Most remain as intact \ce{CH3COOH} molecules. This is an equilibrium (PRIM-CHG003): \ce{CH3COOH <=> CH3COO- + H+}.
\end{itemize}

The same distinction applies to bases. A strong base (like \ce{NaOH}) completely dissociates to produce \ce{OH-}. A weak base (like ammonia, \ce{NH3}) only partially accepts protons from water: \ce{NH3 + H2O <=> NH4+ + OH-}.

\textbf{Conjugate pairs:}

When an acid donates a proton, it becomes a \textbf{conjugate base}. When a base accepts a proton, it becomes a \textbf{conjugate acid}. Every acid-base reaction involves two conjugate pairs:

\reaction{HCl + H2O -> Cl- + H3O+}

\begin{center}
\begin{tabular}{lll}
  \toprule
  \textbf{Species} & \textbf{Role} & \textbf{Conjugate} \\
  \midrule
  \ce{HCl} & Acid (donates \ce{H+}) & \ce{Cl-} (conjugate base) \\
  \ce{H2O} & Base (accepts \ce{H+}) & \ce{H3O+} (conjugate acid) \\
  \bottomrule
\end{tabular}
\end{center}

Notice that water acts as the base here, accepting a proton to become \ce{H3O+} (the hydronium ion). Water is versatile --- it can act as either an acid or a base depending on the partner.

\textbf{The neutralization pattern:}

When an acid and a base react, the products are typically a salt and water:

\begin{center}
Acid + Base $\to$ Salt + Water
\end{center}

\begin{hook}{How an antacid works}
  Your stomach produces hydrochloric acid (\ce{HCl}) to help digest food. Sometimes, excess acid causes discomfort. An antacid tablet contains a base --- often calcium carbonate (\ce{CaCO3}) or magnesium hydroxide (\ce{Mg(OH)2}) --- that neutralizes the excess acid:

  \reaction{CaCO3 + 2 HCl -> CaCl2 + H2O + CO2}

  The carbonate ion (\ce{CO3^{2-}}) accepts protons from \ce{HCl}. The products are calcium chloride (a salt), water, and carbon dioxide gas --- which is why you might burp after taking an antacid. The proton transfer from \ce{HCl} to \ce{CaCO3} reduces the \ce{H+} concentration in your stomach, relieving the burning sensation.
\end{hook}

%% ---- Reasoning Move: DEF-CHG002 ----
\begin{reasoningmove}{DEF-CHG002}{pH Scale}
  \textbf{Reasoning move}: Given a solution, classify it as acidic, neutral, or basic using the pH scale (0--14), and relate pH to \ce{H+} concentration.
\end{reasoningmove}

\depends{PRIM-CHG005}{you must understand acid-base proton transfer before classifying solutions by their acidity}{5}
\depends{PRIM-SCL003}{concentration reasoning --- pH is fundamentally a measure of \ce{H+} concentration}{4}

The pH scale is a way to express how acidic or basic a solution is, using a simple number from 0 to 14.

\textbf{The core ideas:}

\begin{itemize}[nosep]
  \item \textbf{pH below 7} = acidic (more \ce{H+} than \ce{OH-})
  \item \textbf{pH equal to 7} = neutral (\ce{H+} and \ce{OH-} are balanced; pure water)
  \item \textbf{pH above 7} = basic (more \ce{OH-} than \ce{H+})
\end{itemize}

\textbf{Lower pH means more \ce{H+}.} Each step of one pH unit represents a \textbf{tenfold change} in \ce{H+} concentration. A solution at pH 3 has ten times more \ce{H+} than a solution at pH 4, and one hundred times more \ce{H+} than a solution at pH 5.

The mathematical definition is $\text{pH} = -\log[\ce{H+}]$, but you are not expected to calculate pH in this course. What matters is the conceptual relationship: lower number = more acidic = more \ce{H+} ions in solution.

\textbf{A pH map of everyday substances:}

\figurebox{The pH scale from 0 to 14, with everyday substances mapped: battery acid (pH~0), stomach acid (pH~1.5), lemon juice (pH~2), vinegar (pH~2.8), tomato juice (pH~4.2), black coffee (pH~5), milk (pH~6.5), pure water (pH~7, neutral), seawater (pH~8.1), baking soda solution (pH~9), milk of magnesia (pH~10.5), ammonia cleaner (pH~11), soapy water (pH~12), bleach (pH~13), drain cleaner/NaOH (pH~14).}{fig:ph-scale}

\begin{hook}{Swimming pool pH}
  Pool water must be maintained between pH 7.2 and 7.8. Below 7.2, the water is too acidic --- it irritates eyes and skin, and corrodes metal fixtures. Above 7.8, the water is too basic --- chlorine (the disinfectant) becomes less effective because it shifts to a less active chemical form. Pool chemistry is a balancing act: monitoring pH (a concentration measurement, connecting to PRIM-SCL003, Chapter~\ref{ch:scl}) and adjusting it with acid (muriatic acid, to lower pH) or base (sodium carbonate, to raise pH).
\end{hook}

\textbf{Why the tenfold scale matters:}

The tenfold-per-unit nature of the pH scale means that seemingly small changes in pH correspond to large changes in \ce{H+} concentration. A drop from pH 7.4 (normal blood) to pH 7.0 represents a 2.5-fold increase in \ce{H+} concentration. A drop to pH 6.8 can be fatal. Your body has elaborate buffer systems --- chemical equilibria (recall PRIM-CHG003) that resist pH changes --- precisely because even small pH shifts have dramatic physiological consequences.

%% ---- Practice Questions: CHG.4 ----
\begin{practicequestions}
  \practiceq{Identify the Br{\o}nsted acid and Br{\o}nsted base in the following reaction. Name the conjugate acid and conjugate base.

  \reaction{NH3 + H2O <=> NH4+ + OH-}}

  \practiceq{A swimming pool has a pH of 8.5. Is this within the ideal range? Would you add an acid or a base to correct it? In which direction would the correction change the \ce{H+} concentration?}

  \practiceq{Stomach acid has a pH of about 1.5. Milk has a pH of about 6.5. How many times greater is the \ce{H+} concentration in stomach acid compared to milk? (Hint: each pH unit is a tenfold change, and the difference is 5 units.)}
\end{practicequestions}

\medskip

\noindent\textit{Acid-base reactions transfer protons. But protons are not the only particles that can be transferred in a chemical reaction. In the next section, we turn to reactions that transfer electrons --- a different kind of exchange that powers batteries, causes corrosion, and sustains life itself.}


%% ============================================================
\section{CHG.5: How Do Electrons Drive Chemical Change?}
\label{sec:chg5}
%% ============================================================

\subsection*{The Big Idea: Electron Transfer Chemistry}

Place a shiny iron nail in a glass of water and wait a few days. A reddish-brown coating appears: rust. The iron has reacted with oxygen and water. But what exactly happened at the atomic level?

Iron atoms lost electrons. Oxygen atoms gained them. The electrons transferred from iron to oxygen, forming iron oxide (rust). This type of reaction --- in which electrons move from one species to another --- is called an \textbf{oxidation-reduction reaction}, or simply a \textbf{redox reaction}. Redox reactions are everywhere: in the battery that powers your phone, in the corrosion that eats through a bridge, in the mitochondria of your cells where glucose is ``burned'' to release energy.

%% ---- Reasoning Move: PRIM-CHG006 ----
\begin{reasoningmove}{PRIM-CHG006}{Oxidation-Reduction Reasoning}
  \textbf{Reasoning move}: Given a process, identify the electron transfer --- which species is oxidized (loses electrons) and which is reduced (gains electrons).
\end{reasoningmove}

\depends{PRIM-CHG001}{you must be able to write and read the reaction equation}{5}
\depends{PRIM-COM007}{valence electron reasoning --- understanding electron configurations helps you see why certain atoms gain or lose electrons readily}{1}

\textbf{The OIL RIG mnemonic:}

\begin{itemize}[nosep]
  \item \textbf{O}xidation \textbf{I}s \textbf{L}oss (of electrons)
  \item \textbf{R}eduction \textbf{I}s \textbf{G}ain (of electrons)
\end{itemize}

These always occur together. If one species loses electrons, another must gain them. You cannot have oxidation without reduction --- they are two halves of the same process.

\textbf{Tracking electrons with oxidation numbers:}

To identify which atoms have gained or lost electrons, chemists assign \textbf{oxidation numbers} --- a bookkeeping tool that tracks where electrons ``belong.''

Key rules for assigning oxidation numbers:

\begin{center}
\begin{tabular}{lp{7cm}}
  \toprule
  \textbf{Rule} & \textbf{Oxidation Number} \\
  \midrule
  Any element in its elemental form (\ce{Fe}, \ce{O2}, \ce{H2}, \ce{Na}) & 0 \\
  Monatomic ion & Equal to the ion's charge (\ce{Na+} = $+1$, \ce{Cl-} = $-1$) \\
  Hydrogen in most compounds & $+1$ \\
  Oxygen in most compounds & $-2$ \\
  Sum of oxidation numbers in a neutral compound & Must equal 0 \\
  Sum of oxidation numbers in a polyatomic ion & Must equal the ion's charge \\
  \bottomrule
\end{tabular}
\end{center}

\textbf{Worked example: Rusting of iron.}

\reaction{4 Fe + 3 O2 -> 2 Fe2O3}

Assign oxidation numbers:
\begin{itemize}[nosep]
  \item Fe on the left: elemental iron $\to$ oxidation number = 0
  \item \ce{O2} on the left: elemental oxygen $\to$ oxidation number = 0
  \item Fe in \ce{Fe2O3}: oxygen is $-2$, and the compound is neutral. So $2(\text{Fe}) + 3(-2) = 0$, which gives Fe = $+3$
  \item O in \ce{Fe2O3}: $-2$ (standard)
\end{itemize}

What changed?
\begin{itemize}[nosep]
  \item Iron went from 0 to $+3$. It lost 3 electrons per atom. \textbf{Iron is oxidized.}
  \item Oxygen went from 0 to $-2$. It gained 2 electrons per atom. \textbf{Oxygen is reduced.}
\end{itemize}

The electrons lost by iron are the same electrons gained by oxygen. Conservation of charge, just like conservation of atoms.

\begin{hook}{How a battery works}
  A common alkaline battery uses zinc (Zn) and manganese dioxide (\ce{MnO2}):

  \begin{itemize}[nosep]
    \item At the negative terminal: Zn is oxidized (loses electrons): \ce{Zn -> Zn^{2+} + 2 e-}
    \item At the positive terminal: \ce{MnO2} is reduced (gains electrons)
    \item The electrons flow through the external circuit from the zinc to the \ce{MnO2} --- and that electron flow is the electrical current that powers your device.
  \end{itemize}

  Recall PRIM-COM007 (valence electron reasoning, Chapter~\ref{ch:com}): zinc has 2 valence electrons in its outer shell. These are the electrons that zinc loses during oxidation. The ease with which zinc gives up these outer electrons is what makes it a good anode material.
\end{hook}

%% ---- Reasoning Chain ----
\begin{reasoningchain}{Why Does Iron Rust but Gold Does Not?}
  \chainitem{PRIM-CHG006}{Oxidation numbers}{Assign oxidation numbers to iron in its elemental form: Fe = 0.}
  \chainitem{PRIM-CHG006}{Electron transfer}{Consider the reaction with oxygen and water. Iron is oxidized: \ce{Fe -> Fe^{3+}} (loses 3 electrons). Oxygen is reduced: \ce{O2 -> O^{2-}} (gains electrons).}
  \chainitem{PRIM-COM003}{Periodic position reasoning}{Iron is a transition metal in period 4. Gold is in period 6, much further right among the transition metals. Gold holds its electrons much more tightly due to its higher nuclear charge and electron configuration.}
  \chainitem{---}{Energy barrier comparison}{Iron readily loses its outer electrons to oxygen --- the energy barrier is low. Gold does not. Gold is so resistant to oxidation that it remains shiny for millennia. This is why gold has been valued since antiquity: it does not corrode.}
  \chainitem{---}{Stainless steel solution}{Stainless steel solves iron's rusting problem by alloying it with chromium (Cr). Chromium is oxidized preferentially, forming a thin, invisible \ce{Cr2O3} layer that protects the underlying iron from further oxidation. The redox chemistry is redirected: chromium sacrifices its electrons so iron does not have to.}
\end{reasoningchain}

\textbf{Redox in biology: You are burning glucose right now.}

Cellular respiration is a redox reaction:

\reaction{C6H12O6 + 6 O2 -> 6 CO2 + 6 H2O + energy}

The carbon atoms in glucose are oxidized (they go from being bonded to hydrogen and oxygen in glucose to being bonded to oxygen in \ce{CO2} --- their oxidation numbers increase). Oxygen is reduced. The electrons released during glucose oxidation are what your mitochondria use to produce ATP --- the energy currency of your cells.

This is the same type of reaction as burning wood or gasoline. The difference is that your body controls the electron transfer through a series of carefully managed enzyme-catalyzed steps (recall DEF-CHG001, catalysts), rather than releasing all the energy at once as flame and heat.

%% ---- Practice Questions: CHG.5 ----
\begin{practicequestions}
  \practiceq{In the reaction \ce{2 Mg + O2 -> 2 MgO}, identify which species is oxidized and which is reduced. Assign oxidation numbers to each element before and after the reaction.}

  \practiceq{A student says, ``In a battery, electrons are created at the negative terminal and destroyed at the positive terminal.'' Using conservation reasoning (PRIM-COM006, Chapter~\ref{ch:com}) and redox reasoning, explain what is wrong with this statement.}

  \practiceq{Explain why silver jewelry tarnishes (reacts with sulfur compounds in the air) but gold jewelry does not. Use periodic position reasoning (PRIM-COM003, Chapter~\ref{ch:com}) and oxidation-reduction reasoning.}
\end{practicequestions}

\medskip

\noindent\textit{Acid-base reactions transfer protons. Redox reactions transfer electrons. Both are types of chemical change --- rearrangements of atoms and electrons that leave the nuclei of atoms untouched. But there is a type of transformation that goes deeper: change within the nucleus itself. The next section, an enrichment topic, explores nuclear change and how it differs from everything we have studied so far.}


%% ============================================================
\section{CHG.E: How Is Nuclear Change Different from Chemical Change?}
\label{sec:chge}
%% ============================================================

\begin{enrichment}{Nuclear Change, Half-Life, and Precipitation}

This material extends the core concepts from CHG.1 and CHG.2 and is not required for subsequent chapters. It provides a conceptual introduction to nuclear chemistry, radioactive decay, half-life, and precipitation reactions. Your instructor will tell you whether this section is assigned.

%% ---- Reasoning Move: PRIM-CHG007 ----
\begin{reasoningmove}{PRIM-CHG007}{Nuclear Change Reasoning \emark}
  \textbf{Reasoning move}: Given an unstable nucleus, predict the type of radiation emitted, write the nuclear equation, and distinguish nuclear change from chemical change.
\end{reasoningmove}

\textit{Tier: Enrichment}

\depends{PRIM-CHG001}{you must be able to write balanced equations --- nuclear equations follow similar conservation rules}{5}
\depends{DEF-COM001}{isotope --- nuclear change involves specific isotopes, so you must understand what an isotope is}{1}

\textbf{Chemical change versus nuclear change:}

In every reaction we have studied so far --- combustion, acid-base, redox --- the \textbf{nuclei of atoms remain unchanged}. The atomic number ($Z$) is preserved. Carbon atoms stay carbon. Iron atoms stay iron. Only the electrons and bonds rearrange.

Nuclear change is fundamentally different. In nuclear reactions, \textbf{the nucleus itself transforms}. Protons and neutrons are rearranged. The atomic number can change --- meaning one element can become another. This is why nuclear chemistry feels so different from the rest of chemistry: it operates at a deeper level of matter.

\begin{center}
\begin{tabular}{lll}
  \toprule
  \textbf{Feature} & \textbf{Chemical Change} & \textbf{Nuclear Change} \\
  \midrule
  What changes & Electron arrangements, bonds & Protons and neutrons in the nucleus \\
  Atomic number ($Z$) & Preserved & Can change \\
  Energy scale & \si{kJ/mol} & Millions of \si{kJ/mol} \\
  Affected by temperature/pressure & Yes & No \\
  Element identity & Preserved & Can change (transmutation) \\
  \bottomrule
\end{tabular}
\end{center}

\textbf{Three types of radioactive decay:}

\textbf{Alpha decay ($\alpha$):} The nucleus emits an alpha particle --- a cluster of 2 protons and 2 neutrons (essentially a helium-4 nucleus, ${}^{4}_{2}\text{He}$). The parent atom's atomic number decreases by 2 and its mass number decreases by 4.

\textit{Example}: Americium-241 (used in smoke detectors):

\[ {}^{241}_{95}\text{Am} \to {}^{237}_{93}\text{Np} + {}^{4}_{2}\text{He} \]

Americium ($Z = 95$) becomes neptunium ($Z = 93$). The element has changed. This is something no chemical reaction can accomplish.

\textbf{Beta decay ($\beta$):} A neutron in the nucleus converts to a proton, emitting an electron (called a beta particle, ${}^{0}_{-1}e$). The atomic number increases by 1 (one more proton), but the mass number stays the same.

\textit{Example}: Carbon-14 decay (used in radiocarbon dating):

\[ {}^{14}_{6}\text{C} \to {}^{14}_{7}\text{N} + {}^{0}_{-1}e \]

Carbon ($Z = 6$) becomes nitrogen ($Z = 7$). Again, transmutation --- impossible in ordinary chemistry.

\textbf{Gamma decay ($\gamma$):} The nucleus releases a high-energy photon (gamma ray) without changing its composition. The atomic number and mass number stay the same. Gamma emission often accompanies alpha or beta decay, as the daughter nucleus releases excess energy.

\begin{hook}{Smoke detectors}
  The americium-241 in a smoke detector undergoes alpha decay, emitting alpha particles that ionize air molecules between two metal plates. This ionized air conducts a small electrical current. When smoke particles enter the chamber, they absorb the alpha particles, reducing the current. The detector senses the drop in current and sounds the alarm. The nuclear decay of Am-241 is what makes the detector work --- and it has a half-life of 432 years, so the source lasts far longer than the battery.
\end{hook}

\textbf{Balancing nuclear equations:}

Nuclear equations must conserve both \textbf{mass number} (total protons + neutrons) and \textbf{atomic number} (total protons) on each side. This is analogous to balancing chemical equations for atom conservation (PRIM-COM006, Chapter~\ref{ch:com}), but the conserved quantities are different.

For the americium example: mass numbers: $241 = 237 + 4$. Atomic numbers: $95 = 93 + 2$. Both balance.

%% ---- Reasoning Move: DEF-CHG004 ----
\begin{reasoningmove}{DEF-CHG004}{Half-Life \emark}
  \textbf{Reasoning move}: Given an isotope and its half-life, predict the amount remaining after a whole number of half-lives.
\end{reasoningmove}

\textit{Tier: Enrichment}

\depends{PRIM-CHG007}{you must understand radioactive decay before reasoning about its rate}{5}

\textbf{What is a half-life?}

The \textbf{half-life} ($t_{1/2}$) of a radioactive isotope is the time it takes for half of the radioactive atoms in a sample to decay. After one half-life, half the original atoms remain. After two half-lives, one quarter remain. After three, one eighth. And so on.

\textbf{The pattern:}

\begin{center}
\begin{tabular}{lll}
  \toprule
  \textbf{Number of Half-Lives} & \textbf{Fraction Remaining} & \textbf{If Starting with \SI{100}{\gram}} \\
  \midrule
  0 & 1 & \SI{100}{\gram} \\
  1 & 1/2 & \SI{50}{\gram} \\
  2 & 1/4 & \SI{25}{\gram} \\
  3 & 1/8 & \SI{12.5}{\gram} \\
  4 & 1/16 & \SI{6.25}{\gram} \\
  5 & 1/32 & \SI{3.125}{\gram} \\
  \bottomrule
\end{tabular}
\end{center}

The calculation is simply: amount remaining = original amount $\times (1/2)^n$, where $n$ is the number of half-lives elapsed. This works for whole-number half-lives only --- we will not use the exponential equation.

\textbf{Key insight: Half-life is constant.}

Unlike chemical reaction rates (PRIM-CHG004), which depend on temperature, concentration, pressure, and catalysts, the half-life of a radioactive isotope is \textbf{completely independent of external conditions}. You cannot speed up or slow down radioactive decay by heating, cooling, compressing, or adding chemicals. Nuclear decay is governed by forces inside the nucleus --- forces that are entirely unaffected by the chemical environment.

This is why radioactive dating works: the ``clock'' ticks at a constant rate regardless of what happens to the material.

\textbf{Real-world examples of half-lives:}

\begin{center}
\begin{tabular}{llp{6cm}}
  \toprule
  \textbf{Isotope} & \textbf{Half-Life} & \textbf{Application} \\
  \midrule
  Carbon-14 (${}^{14}$C) & 5,730 years & Archaeological dating (up to $\sim$50,000 years) \\
  Iodine-131 (${}^{131}$I) & 8 days & Medical imaging of thyroid; short half-life limits radiation exposure \\
  Uranium-238 (${}^{238}$U) & 4.5 billion years & Geological dating of rocks and the age of the Earth \\
  Technetium-99m (${}^{99m}$Tc) & 6 hours & Most common medical imaging isotope; decays quickly after scan \\
  \bottomrule
\end{tabular}
\end{center}

\begin{hook}{Carbon-14 dating}
  All living organisms contain carbon. A small fraction of that carbon is the radioactive isotope ${}^{14}$C, which is constantly replenished from the atmosphere while the organism is alive. When the organism dies, it stops absorbing ${}^{14}$C, and the existing ${}^{14}$C begins to decay with a half-life of 5,730 years. By measuring the ratio of ${}^{14}$C to stable ${}^{12}$C in a sample, scientists can determine how long ago the organism died. A sample with half the expected ${}^{14}$C ratio is about 5,730 years old. A sample with one quarter is about 11,460 years old. The half-life acts as a built-in clock.
\end{hook}

%% ---- Reasoning Move: DEF-CHG005 ----
\begin{reasoningmove}{DEF-CHG005}{Precipitation \emark}
  \textbf{Reasoning move}: Given two ionic solutions, predict whether mixing them will produce an insoluble solid (precipitate) using qualitative solubility rules.
\end{reasoningmove}

\textit{Tier: Enrichment}

\depends{PRIM-CHG002}{precipitation is a specialized form of double replacement --- the ions swap partners, and one combination is insoluble}{5}
\depends{PRIM-STR001}{bond type reasoning --- the reactants are ionic compounds that dissociate into ions in solution}{2}
\depends{PRIM-SCL003}{concentration reasoning --- precipitation depends on ion concentrations in solution}{4}

\textbf{What is precipitation?}

When two solutions of ionic compounds are mixed, the cations and anions from each compound are all floating freely in the same solution. If a particular cation-anion combination forms an insoluble compound, it will crash out of solution as a solid --- a \textbf{precipitate}.

This is a specialized double replacement reaction (recall PRIM-CHG002): the ions swap partners, and one of the new combinations is insoluble in water.

\textbf{Example:}

Mix silver nitrate (\ce{AgNO3}) solution with sodium chloride (\ce{NaCl}) solution:

\reaction{AgNO3(aq) + NaCl(aq) -> AgCl(s) + NaNO3(aq)}

Silver chloride (\ce{AgCl}) is insoluble --- it forms a white solid that settles out of solution. Sodium nitrate (\ce{NaNO3}) remains dissolved. The (s) indicates a solid precipitate; (aq) indicates dissolved in water.

\textbf{How do you know which combinations are insoluble?}

Chemists use \textbf{solubility rules} --- a set of guidelines (not to be memorized, but used as a reference table) that tell you which ionic compounds dissolve in water and which do not.

\textbf{Simplified solubility rules (reference table):}

\begin{center}
\begin{tabular}{ll}
  \toprule
  \textbf{Generally Soluble (dissolve)} & \textbf{Exceptions (do NOT dissolve)} \\
  \midrule
  All compounds of \ce{Na+}, \ce{K+}, \ce{NH4+} & None \\
  All nitrates (\ce{NO3-}) & None \\
  All chlorides (\ce{Cl-}) & \ce{AgCl}, \ce{PbCl2}, \ce{Hg2Cl2} \\
  All sulfates (\ce{SO4^{2-}}) & \ce{BaSO4}, \ce{PbSO4}, \ce{CaSO4} \\
  \midrule
  \textbf{Generally Insoluble} & \textbf{Exceptions (DO dissolve)} \\
  \midrule
  Most carbonates (\ce{CO3^{2-}}) & \ce{Na2CO3}, \ce{K2CO3}, \ce{(NH4)2CO3} \\
  Most hydroxides (\ce{OH-}) & \ce{NaOH}, \ce{KOH}, \ce{Ca(OH)2} (slightly) \\
  Most phosphates (\ce{PO4^{3-}}) & \ce{Na3PO4}, \ce{K3PO4}, \ce{(NH4)3PO4} \\
  Most sulfides (\ce{S^{2-}}) & \ce{Na2S}, \ce{K2S}, \ce{(NH4)2S} \\
  \bottomrule
\end{tabular}
\end{center}

You do not need to memorize this table. In practice, you would look it up --- just as you look up elements on the periodic table. The reasoning skill is: given two ionic solutions, identify the possible ion combinations, check the table, and predict whether a precipitate forms.

\begin{hook}{Hard water and kettle scale}
  ``Hard'' water contains dissolved calcium (\ce{Ca^{2+}}) and magnesium (\ce{Mg^{2+}}) ions. When you boil hard water in a kettle, dissolved calcium bicarbonate decomposes:

  \reaction{Ca(HCO3)2(aq) -> CaCO3(s) + H2O(l) + CO2(g)}

  Calcium carbonate is insoluble --- it precipitates as the white, chalky ``scale'' that builds up inside kettles, pipes, and showerheads. Water softeners work by removing \ce{Ca^{2+}} and \ce{Mg^{2+}} ions from the water before they can form these precipitates, typically by exchanging them for \ce{Na+} ions (which form only soluble compounds --- see the table above).
\end{hook}

%% ---- Practice Questions: CHG.E ----
\begin{practicequestions}
  \practiceq{Radon-222 undergoes alpha decay. Write the nuclear equation. What element is produced? (Hint: radon has $Z = 86$.)}

  \practiceq{A sample of iodine-131 (half-life = 8 days) is used for a medical scan. If the initial dose is \SI{100}{\micro\gram}, how much remains after 24 days? After 32 days?}

  \practiceq{Predict whether a precipitate forms when each pair of solutions is mixed. If so, identify the precipitate.
  \begin{itemize}[nosep]
    \item Potassium chloride (\ce{KCl}) + sodium nitrate (\ce{NaNO3})
    \item Lead(II) nitrate (\ce{Pb(NO3)2}) + potassium iodide (\ce{KI}) [Hint: \ce{PbI2} is insoluble]
  \end{itemize}}

  \practiceq{Explain why radioactive half-life is unaffected by temperature, while chemical reaction rates (PRIM-CHG004) are strongly temperature-dependent. What is fundamentally different about nuclear processes versus chemical processes?}
\end{practicequestions}

\end{enrichment}

\medskip

\noindent\textit{The enrichment section has taken us from the molecular scale into the nucleus and back out to everyday phenomena like smoke detectors, carbon dating, and hard water. We now return to the chapter's overarching theme --- chemical change --- and deploy the reasoning tools from this chapter and all previous chapters together in the capstone sections that follow.}


%% ============================================================
\section{CHG.CP-002: Why Does a Battery Eventually Die?}
\label{sec:chgcp002}
%% ============================================================

\begin{cpcapstone}{CP-002}{Energy-Driven Transformation}

This capstone section combines reasoning tools from the Energy domain (Chapter~\ref{ch:nrg}) and the Change domain (this chapter) to explain why batteries produce energy and why they eventually stop. It follows the ADP four-step pedagogy.

\medskip

\textbf{Primitives required:}

\begin{center}
\begin{tabular}{llp{7cm}}
  \toprule
  \textbf{Primitive} & \textbf{Source} & \textbf{Reasoning Move} \\
  \midrule
  PRIM-CHG006 & Chapter~\ref{ch:chg} (CHG.5) & Oxidation-reduction reasoning \\
  PRIM-NRG005 & Chapter~\ref{ch:nrg} (NRG.3) & Spontaneity reasoning \\
  PRIM-NRG003 & Chapter~\ref{ch:nrg} (NRG.2) & Exo/endothermic classification \\
  PRIM-CHG003 & Chapter~\ref{ch:chg} (CHG.2) & Equilibrium reasoning \\
  \bottomrule
\end{tabular}
\end{center}

\cpstep{The Hook}{You put fresh batteries in a flashlight. It shines brightly. Over the following weeks, the light gradually dims. Eventually, the flashlight goes dark. The batteries are ``dead.''

But what does ``dead'' actually mean? The metal inside the battery has not vanished. The chemicals are still there. You can weigh a dead battery and it weighs essentially the same as a fresh one. So what has changed? Why does a battery eventually stop producing electricity?}

\cpstep{Identify the electron transfer}{Using \textbf{PRIM-CHG006 (oxidation-reduction reasoning)}, we identify the redox reaction inside the battery. In a typical alkaline battery:

\begin{itemize}[nosep]
  \item At the negative terminal, zinc (Zn) is \textbf{oxidized}: \ce{Zn -> Zn^{2+} + 2 e-}. Zinc atoms lose electrons.
  \item At the positive terminal, manganese dioxide (\ce{MnO2}) is \textbf{reduced}: \ce{MnO2} gains the electrons that zinc lost.
\end{itemize}

The electrons travel from the zinc through the external circuit (the wire, the flashlight bulb) to the \ce{MnO2}. That flow of electrons is the electrical current that lights the bulb.}

\cpstep{Explain why the electron transfer happens}{Using \textbf{PRIM-NRG005 (spontaneity reasoning)}, we understand that this redox reaction is spontaneous --- the free energy scorecard is favorable. The electrons ``want'' to flow from zinc (which holds them loosely) to \ce{MnO2} (which holds them more tightly). No external push is needed. The battery produces energy on its own because the reaction is thermodynamically favorable.

This addresses a common misconception: a battery does not create energy from nothing. It converts chemical energy (stored in the zinc and \ce{MnO2}) into electrical energy. The energy was always there, locked in the chemical bonds and electron arrangements.}

\cpstep{Classify the energy flow}{Using \textbf{PRIM-NRG003 (exo/endothermic classification)}, we classify the battery's discharge as exothermic (energy-releasing). Chemical energy stored in the reactants (Zn and \ce{MnO2}) is converted to electrical energy (the useful output) and thermal energy (batteries get warm during use). The total energy released equals the energy difference between the reactant bonds and the product bonds.}

\cpstep{Explain why it stops}{Using \textbf{PRIM-CHG003 (equilibrium reasoning)}, we answer the original question: why does the battery die?

As the battery operates, zinc is consumed (oxidized to \ce{Zn^{2+}}) and \ce{MnO2} is consumed (reduced). The concentrations of reactants decrease while the concentrations of products increase. The system is approaching equilibrium. At equilibrium, the forward and reverse reactions proceed at equal rates, and there is no net electron flow. No net electron flow means no current. No current means no light.

A ``dead'' battery is a battery at equilibrium. The reaction has not stopped at the molecular level --- molecules are still reacting in both directions. But the net flow is zero. The driving force (the spontaneity, the ``downhill'' energy difference) has been used up.

\textbf{Summary chain}: Redox (PRIM-CHG006) identifies the electron transfer $\to$ spontaneity (PRIM-NRG005) explains why it occurs $\to$ exo/endo (PRIM-NRG003) classifies the energy flow $\to$ equilibrium (PRIM-CHG003) explains why it stops. Four primitives, one complete explanation: a battery dies because its redox reaction reaches equilibrium.}

\cpstep{The Bridge}{\textbf{New question}: Why does ice melt at room temperature?

The same four-primitive chain applies, with a different type of transformation:

\textbf{PRIM-CHG006} (redox) does not directly apply here --- melting is not a redox reaction. But the chain structure transfers if we replace ``electron transfer'' with ``energy transfer'' (PRIM-NRG001, Chapter~\ref{ch:nrg}): thermal energy flows from the warm room into the cold ice, providing the energy to break hydrogen bonds in the crystal lattice.

\textbf{PRIM-NRG005 (spontaneity reasoning)}: At temperatures above \SI{0}{\degreeCelsius}, melting is spontaneous. The entropy gain (solid $\to$ liquid means more molecular arrangements) outweighs the energy cost of breaking hydrogen bonds when $T$ is high enough.

\textbf{PRIM-NRG003 (exo/endothermic classification)}: Melting is endothermic --- the ice absorbs energy from the surroundings. That is why a glass of ice water keeps your drink cold: the ice is pulling thermal energy from the liquid to fuel the phase change.

\textbf{PRIM-CHG003 (equilibrium reasoning)}: At exactly \SI{0}{\degreeCelsius}, the system is at equilibrium: ice melts at the same rate that water freezes. Above \SI{0}{\degreeCelsius}, the system is not at equilibrium --- melting dominates until all the ice is gone. Below \SI{0}{\degreeCelsius}, freezing dominates.

Same reasoning chain. Different phenomenon. The tools transfer: spontaneity explains direction, energy classification explains the flow, and equilibrium explains the balance point.}

\end{cpcapstone}


%% ============================================================
\section{CHG.CP-003: Why Does Aspirin Hurt Your Stomach but an Antacid Helps?}
\label{sec:chgcp003}
%% ============================================================

\begin{cpcapstone}{CP-003}{Acid-Base in the Body}

This capstone section combines reasoning tools from the Structure domain (Chapter~\ref{ch:str}), the Change domain (this chapter), and the Scale domain (Chapter~\ref{ch:scl}) to explain how aspirin and antacids interact with stomach chemistry. It follows the ADP four-step pedagogy.

\medskip

\textbf{Primitives required:}

\begin{center}
\begin{tabular}{llp{7cm}}
  \toprule
  \textbf{Primitive} & \textbf{Source} & \textbf{Reasoning Move} \\
  \midrule
  PRIM-STR002 & Chapter~\ref{ch:str} (STR.1) & Bond polarity reasoning \\
  PRIM-STR005 & Chapter~\ref{ch:str} (STR.4) & Structure-to-property inference \\
  PRIM-CHG005 & Chapter~\ref{ch:chg} (CHG.4) & Acid-base reasoning \\
  DEF-CHG002 & Chapter~\ref{ch:chg} (CHG.4) & pH scale \\
  PRIM-SCL003 & Chapter~\ref{ch:scl} (SCL.3) & Concentration reasoning \\
  \bottomrule
\end{tabular}
\end{center}

\cpstep{The Hook}{You have a headache. You take two aspirin tablets. An hour later, the headache is gone --- but your stomach hurts. You reach for an antacid. Within minutes, the stomach discomfort eases.

Why does aspirin --- a pain reliever --- cause stomach pain? And why does an antacid --- a simple chalky tablet --- fix it? The answer involves the molecular structure of aspirin, the chemistry of proton transfer, and the concentration of \ce{H+} ions in your stomach.}

\cpstep{Identify the reactive feature of aspirin}{Using \textbf{PRIM-STR002 (bond polarity reasoning)}, we examine aspirin's molecular structure. Aspirin (acetylsalicylic acid) contains a \textbf{carboxylic acid group} (--COOH). In this group, the O--H bond is polar: oxygen is far more electronegative than hydrogen, pulling the electron density toward itself. This leaves the hydrogen with a partial positive charge and makes it available for donation as \ce{H+}.}

\cpstep{Predict the property consequence}{Using \textbf{PRIM-STR005 (structure-to-property inference)}, we chain from molecular structure to chemical behavior: the polar O--H bond in the carboxylic acid group means aspirin can donate a proton. Aspirin is an acid. In the aqueous environment of your stomach, it releases \ce{H+}.

But wait --- your stomach already contains acid. Stomach acid (\ce{HCl}) maintains a pH of about 1.5 to 2.0 (using \textbf{DEF-CHG002}, the pH scale). The stomach lining is protected by a mucus barrier that shields it from this extremely acidic environment. Aspirin interferes with this protective barrier. It inhibits the production of prostaglandins --- molecules that maintain the mucus layer. With a weakened mucus barrier, the stomach's own acid attacks the underlying tissue.}

\cpstep{Trace the proton transfer}{Using \textbf{PRIM-CHG005 (acid-base reasoning)}, we see the proton transfer in action. Aspirin donates \ce{H+} into the already-acidic stomach environment. More \ce{H+} in a solution with a compromised mucus barrier means more acid in contact with vulnerable tissue. The proton transfer from aspirin adds to the stomach's acid load at the worst possible time.}

\cpstep{Quantify the damage via pH}{Using \textbf{DEF-CHG002 (pH scale)}, we quantify what is happening. A drop from pH 2.0 to pH 1.5 might seem small --- just half a pH unit. But because each pH unit represents a tenfold change in \ce{H+} concentration, a drop of 0.5 units means the \ce{H+} concentration has approximately tripled. In an area where the mucus barrier is compromised, this increase in acid concentration causes irritation, pain, and potentially ulceration.}

\cpstep{Explain dose-dependence}{Using \textbf{PRIM-SCL003 (concentration reasoning)}, we understand why the problem is dose-dependent. One aspirin might not cause noticeable discomfort. Four aspirin on an empty stomach might cause significant pain. The effect depends on concentration: more aspirin means more \ce{H+} donated, more prostaglandin inhibition, and more acid exposure to the stomach lining.}

\cpstep{Explain the antacid reversal}{Now, the antacid. A typical antacid tablet contains calcium carbonate (\ce{CaCO3}), a Br{\o}nsted base. It accepts the excess \ce{H+} ions:

\reaction{CaCO3 + 2 H+ -> Ca^{2+} + H2O + CO2}

Using \textbf{PRIM-CHG005} again, we trace the proton transfer in the reverse direction: the carbonate ion accepts protons from the acid, reducing \ce{H+} concentration. Using \textbf{DEF-CHG002}, the pH rises back toward 2.0 or above. Using \textbf{PRIM-SCL003}, the lower \ce{H+} concentration means less acid attacking the compromised stomach lining. The pain subsides.

\textbf{Summary chain}: Bond polarity (PRIM-STR002) $\to$ structure-to-property (PRIM-STR005) $\to$ acid-base transfer (PRIM-CHG005) $\to$ pH quantification (DEF-CHG002) $\to$ concentration dependence (PRIM-SCL003) $\to$ antacid reversal. Five primitives across three domains, one complete explanation.}

\cpstep{The Bridge}{\textbf{New question}: Why does lemon juice taste sour but soap feels slippery?

\textbf{PRIM-STR002 (bond polarity)}: Citric acid in lemon juice has three carboxylic acid groups (--COOH), each with a polar O--H bond ready to donate \ce{H+}. Soap (sodium hydroxide-based) contains \ce{Na+} and \ce{OH-} ions.

\textbf{PRIM-STR005 (structure-to-property inference)}: The polar O--H bonds in citric acid make it a proton donor (acid). The \ce{OH-} in soap solution is a proton acceptor (base).

\textbf{PRIM-CHG005 (acid-base reasoning)}: Citric acid donates \ce{H+} to your taste receptors. Sour taste is the biological response to \ce{H+} ions --- your tongue has receptors that detect proton concentration. Soap's \ce{OH-} ions react with the natural oils on your skin via a reaction called saponification (a type of base-catalyzed hydrolysis). The slippery feeling is the soap dissolving the oils on your skin surface.

\textbf{DEF-CHG002 (pH)}: Lemon juice has a pH of about 2 (very acidic, high \ce{H+}). Soap solution has a pH of about 12 (very basic, high \ce{OH-}). They sit at opposite ends of the pH scale, and their sensory effects on your body reflect this: acid tastes sour, base feels slippery.

\textbf{PRIM-SCL003 (concentration)}: Dilute lemon juice (lemonade) is pleasantly tart. Concentrated citric acid burns the mouth. Dilute soap is slippery; concentrated sodium hydroxide is caustic. Both effects are concentration-dependent.

Same five primitives. Same reasoning chain. Different phenomenon.}

\end{cpcapstone}


%% ============================================================
\section{CHG.CP-005: Is Fluoride in Drinking Water Safe?}
\label{sec:chgcp005}
%% ============================================================

\begin{cpcapstone}{CP-005}{Dose Makes the Poison}

This capstone section combines reasoning tools from the Structure, Change, and Scale domains to analyze the safety of fluoride at different concentrations. It is a scientific-literacy capstone. It follows the ADP four-step pedagogy.

\medskip

\textbf{Primitives required:}

\begin{center}
\begin{tabular}{llp{7cm}}
  \toprule
  \textbf{Primitive} & \textbf{Source} & \textbf{Reasoning Move} \\
  \midrule
  PRIM-STR005 & Chapter~\ref{ch:str} (STR.4) & Structure-to-property inference \\
  PRIM-CHG005 & Chapter~\ref{ch:chg} (CHG.4) & Acid-base reasoning \\
  PRIM-SCL003 & Chapter~\ref{ch:scl} (SCL.3) & Concentration reasoning \\
  DEF-SCL002 & Chapter~\ref{ch:scl} (SCL.3) & Parts per million (ppm) \\
  \bottomrule
\end{tabular}
\end{center}

\cpstep{The Hook}{A headline reads: ``City Puts Fluoride in Your Drinking Water!'' To some, this sounds alarming. Fluoride is a chemical. It is being added to something you drink. Is it safe?

This is exactly the kind of question where chemistry literacy matters. The answer is not ``yes, fluoride is safe'' or ``no, fluoride is dangerous.'' The answer is: \textbf{it depends on the concentration.} At the right concentration, fluoride strengthens tooth enamel and prevents cavities. At much higher concentrations, it can cause problems. The chemistry is the same in both cases --- what changes is the amount. As the Renaissance physician Paracelsus observed: the dose makes the poison.}

\cpstep{Understand how fluoride interacts with tooth enamel}{Using \textbf{PRIM-STR005 (structure-to-property inference)}, we examine the chemistry. Tooth enamel is made primarily of hydroxyapatite, \ce{Ca5(PO4)3OH}. The hydroxyl group (\ce{OH-}) at the surface of enamel crystals is vulnerable to acid attack from bacteria in the mouth. When bacteria metabolize sugars, they produce acids that dissolve enamel:

\reaction{Ca5(PO4)3OH + acid -> Ca^{2+} + PO4^{3-} + H2O}

This is tooth decay.}

\cpstep{Trace the acid-base chemistry of fluoride protection}{Using \textbf{PRIM-CHG005 (acid-base reasoning)}, we see how fluoride helps. Fluoride ions (\ce{F-}) in saliva replace the hydroxyl groups in enamel to form fluorapatite, \ce{Ca5(PO4)3F}. The fluoride ion is less reactive with acids than the hydroxyl ion. Fluorapatite is more resistant to acid dissolution than hydroxyapatite. In acid-base terms, the \ce{F-} does not accept protons as readily as \ce{OH-} does, so the enamel surface resists acid attack more effectively.}

\cpstep{Quantify the concentration}{Using \textbf{DEF-SCL002 (parts per million)}, we examine the numbers. The U.S.\ EPA recommends community water fluoridation at approximately \textbf{\SI{0.7}{\ppm}} (\SI{0.7}{mg} of fluoride per liter of water). At this concentration, fluoride provides measurable cavity protection. Decades of public health data show that communities with fluoridated water have 20--40\% fewer cavities than communities without it.

Using \textbf{PRIM-SCL003 (concentration reasoning)}, we ask: at what concentration does fluoride become harmful? The EPA's maximum contaminant level for fluoride is \textbf{\SI{4.0}{\ppm}}. Above this level, long-term exposure can cause \textbf{dental fluorosis} (white spots or pitting on teeth) and, at much higher levels, \textbf{skeletal fluorosis} (bone and joint damage).

\figurebox{The fluoride concentration spectrum: \SI{0}{\ppm} (no effect), \SI{0.7}{\ppm} (beneficial, cavity protection), \SI{4.0}{\ppm} (potentially harmful, dental fluorosis), \SI{10}{\ppm}+ (harmful, skeletal fluorosis).}{fig:fluoride-spectrum}

The difference between ``beneficial'' and ``harmful'' is not a change in the molecule. \ce{F-} is \ce{F-} at any concentration. What changes is the amount. At \SI{0.7}{\ppm}, there is enough fluoride to reinforce enamel but not enough to interfere with normal tooth and bone development. At \SI{10}{\ppm}+, there is so much fluoride that it disrupts the crystallization of enamel and bone.

\textbf{The chemist's first question:}

When you read a headline claiming a chemical is ``toxic,'' the first question a chemist asks is: \textbf{at what concentration?} Water is lethal if you drink enough of it (water intoxication, hyponatremia). Oxygen is toxic at high pressures (oxygen toxicity). Vitamin A is essential at normal dietary levels but poisonous in large doses. The identity of the substance is only half the story. The concentration is the other half.

\textbf{Summary chain}: Structure-to-property (PRIM-STR005) $\to$ acid-base chemistry (PRIM-CHG005) $\to$ ppm quantification (DEF-SCL002) $\to$ concentration determines outcome (PRIM-SCL003). Four primitives, one scientific-literacy lesson: the same molecule can be beneficial or harmful depending on concentration.}

\cpstep{The Bridge}{\textbf{New question}: Is chlorine in pool water dangerous?

The same four-primitive chain applies, now with a redox component:

\textbf{PRIM-STR005 (structure-to-property)}: Chlorine dissolved in water forms hypochlorous acid (\ce{HOCl}), a strong oxidizing agent. Its structure enables it to disrupt the cell membranes and enzymes of bacteria and pathogens.

\textbf{PRIM-CHG006 (oxidation-reduction reasoning)} replaces acid-base reasoning here: chlorine kills bacteria through oxidation. \ce{HOCl} oxidizes key molecules in bacterial cells, destroying their function. This is redox chemistry deployed as disinfection.

\textbf{DEF-SCL002 (parts per million)}: The recommended chlorine level for swimming pools is \textbf{1--3 ppm}. At this concentration, chlorine effectively kills bacteria and algae while being safe for swimmers.

\textbf{PRIM-SCL003 (concentration reasoning)}: Above \textbf{\SI{10}{\ppm}}, chlorine causes skin irritation, eye damage, and respiratory distress. At much higher concentrations (as in industrial accidents), chlorine gas is acutely toxic and potentially lethal.

\figurebox{The chlorine concentration spectrum: \SI{0}{\ppm} (no effect), 1--3 ppm (safe for swimming, kills pathogens), \SI{10}{\ppm} (irritation, health risk), 100+ ppm (acute toxicity).}{fig:chlorine-spectrum}

Same pattern as fluoride. Same reasoning chain. The dose makes the poison.}

\end{cpcapstone}


%% ============================================================
\section{CHG.CP-006: Why Do We Cook Food?}
\label{sec:chgcp006}
%% ============================================================

\begin{cpcapstone}{CP-006}{Food Chemistry}

This capstone section is the most domain-rich composition pattern in the course, combining reasoning tools from four domains --- Composition, Energy, Change, and Scale --- to explain the chemistry of cooking and metabolism. It follows the ADP four-step pedagogy.

\medskip

\textbf{Primitives required:}

\begin{center}
\begin{tabular}{llp{7cm}}
  \toprule
  \textbf{Primitive} & \textbf{Source} & \textbf{Reasoning Move} \\
  \midrule
  PRIM-CHG004 & Chapter~\ref{ch:chg} (CHG.3) & Rate reasoning via collision theory \\
  PRIM-NRG002 & Chapter~\ref{ch:nrg} (NRG.2) & Bond energy reasoning \\
  PRIM-NRG003 & Chapter~\ref{ch:nrg} (NRG.2) & Exo/endothermic classification \\
  PRIM-COM006 & Chapter~\ref{ch:com} (COM.3) & Conservation of atoms \\
  PRIM-SCL002 & Chapter~\ref{ch:scl} (SCL.1) & Mole concept / amount \\
  \bottomrule
\end{tabular}
\end{center}

\cpstep{The Hook}{You can eat a raw potato. It will not poison you. But it tastes starchy, chalky, and unappealing. Put that same potato in an oven at \SI{200}{\degreeCelsius} for an hour, and it becomes soft, golden-brown, and delicious. The chemical composition of the potato has not changed dramatically --- it is still mostly starch, water, and fiber. But its molecular structure has been transformed by heat.

Why do we cook food? The simple answer is ``to make it taste better and safer to eat.'' But the chemical answer is deeper: cooking uses heat to accelerate chemical reactions that break down large molecules, develop new flavors, kill pathogens, and make nutrients more accessible to digestion. Cooking is applied chemistry.}

\cpstep{Heat speeds chemical reactions}{Using \textbf{PRIM-CHG004 (rate reasoning)}, we understand the fundamental role of heat in cooking. When you heat food, you increase the kinetic energy of all the molecules in it. This increases the rate of every chemical reaction occurring in the food:

\begin{itemize}[nosep]
  \item \textbf{Maillard reaction}: At temperatures above about \SI{140}{\degreeCelsius}, amino acids and sugars react to produce hundreds of new flavor and aroma compounds --- the ``browning'' of bread crust, roasted coffee, seared steak, and baked potatoes. This reaction is slow at room temperature (raw flour does not brown on its own). Heat accelerates it dramatically.
  \item \textbf{Protein denaturation}: Heat causes proteins to unfold (recall the protein folding discussion from Chapter~\ref{ch:nrg}). In cooking, this is desirable: denatured egg proteins form the firm structure of a boiled egg. Denatured meat proteins become tender and chewable.
  \item \textbf{Starch gelatinization}: Heat causes starch granules to absorb water and swell, transforming raw starch (gritty, difficult to digest) into a soft gel (the fluffy interior of a baked potato).
\end{itemize}

Without heat, these reactions would take days, weeks, or longer. Cooking compresses them into minutes. This is collision theory in the kitchen: higher temperature means more energetic molecular collisions means faster chemical transformations.}

\cpstep{Food stores energy in bonds}{Using \textbf{PRIM-NRG002 (bond energy reasoning)}, we understand where the energy in food comes from. Carbohydrates, fats, and proteins contain chemical bonds --- C--C, C--H, C--O, and others --- that store energy. When these bonds are broken and new bonds are formed during metabolism (the body's controlled combustion of food), energy is released.

Consider the metabolic oxidation of glucose:

\reaction{C6H12O6 + 6 O2 -> 6 CO2 + 6 H2O + energy}

The bonds in glucose and \ce{O2} store more energy than the bonds in \ce{CO2} and \ce{H2O}. The difference is released --- and that released energy is what powers every cell in your body.}

\cpstep{Metabolism is exothermic}{Using \textbf{PRIM-NRG003 (exo/endothermic classification)}, we classify metabolism as exothermic. Energy flows out of the chemical bonds and into the body's systems. This is why you feel warm --- your body temperature of \SI{37}{\degreeCelsius} is maintained by the continuous exothermic oxidation of food molecules. Exercise increases the rate of metabolism (PRIM-CHG004 again), which is why you feel hotter during a run.}

\cpstep{Every atom is conserved}{Using \textbf{PRIM-COM006 (conservation of atoms)}, we trace where the atoms go. The balanced equation for glucose metabolism tells the story:

\reaction{C6H12O6 + 6 O2 -> 6 CO2 + 6 H2O}

\begin{itemize}[nosep]
  \item The 6 carbon atoms in glucose exit your body as 6 molecules of \ce{CO2} (exhaled through your lungs).
  \item The 12 hydrogen atoms in glucose exit as 6 molecules of \ce{H2O} (exhaled as water vapor or excreted).
  \item The oxygen atoms from both glucose and inhaled \ce{O2} are distributed among the \ce{CO2} and \ce{H2O} products.
\end{itemize}

Every atom that goes in comes out. You do not gain carbon atoms from food and keep them forever. You exhale them. When you ``lose weight,'' most of the mass leaves your body as \ce{CO2} --- you literally breathe out the carbon atoms that were stored in your fat.}

\cpstep{Calories quantify bond energy per serving}{Using \textbf{PRIM-SCL002 (mole concept / amount)}, we connect the molecular-level bond energies to the macroscopic measurement on a food label. A food Calorie (capital C, = 1 kilocalorie = \SI{4184}{\joule}) measures the total energy released when the food is completely metabolized. The Calorie count on a food label is, at its core, a measurement of bond energy per serving.

A slice of bread contains about 80 Calories. This means that if your body completely metabolized the carbohydrates, proteins, and fats in that bread --- breaking the C--H, C--C, and C--O bonds and forming \ce{CO2} and \ce{H2O} bonds --- approximately 80 kilocalories of energy would be released. This energy powers your muscles, maintains your body temperature, drives your brain's electrical activity, and sustains every cellular process.

\figurebox{The metabolism flowchart: Food (bond energy stored) $\to$ Metabolism (exothermic oxidation) $\to$ outputs: \ce{CO2} (exhaled, C atoms leave), \ce{H2O} (exhaled/excreted, H atoms leave), Energy (heat, muscle contraction, cellular work). Atoms in = Atoms out (conservation). Energy in bonds = Energy out as heat + work (conservation).}{fig:metabolism-flow}

\textbf{Summary chain}: Rate (PRIM-CHG004) explains why heat speeds cooking $\to$ bond energy (PRIM-NRG002) explains where food energy is stored $\to$ exo/endo (PRIM-NRG003) classifies metabolism as energy-releasing $\to$ conservation (PRIM-COM006) traces every atom from food to \ce{CO2} and \ce{H2O} $\to$ amount (PRIM-SCL002) connects bond energy to Calories on a label. Five primitives across four domains, one integrated explanation: cooking accelerates food chemistry, and metabolism releases the energy stored in food bonds while conserving every atom.}

\cpstep{The Bridge}{\textbf{New question}: Why do food labels list Calories?

The same five-primitive chain answers this directly:

\textbf{PRIM-CHG004 (rate)}: Your body metabolizes food at a rate that depends on activity level (exercise speeds metabolism because muscles demand more energy --- more molecular collisions in mitochondria).

\textbf{PRIM-NRG002 (bond energy)}: Different foods store different amounts of bond energy. Fat stores about 9 Calories per gram (its C--H bonds are highly energetic). Carbohydrates and proteins store about 4 Calories per gram. The Calorie count reflects the total bond energy available.

\textbf{PRIM-NRG003 (exo/endothermic)}: Metabolism is exothermic. The energy released maintains body temperature and powers activity. If you consume more Calories (bond energy) than you metabolize, the excess is stored as fat --- the body's energy reserve. If you metabolize more than you consume, stored fat is broken down and its energy is released.

\textbf{PRIM-COM006 (conservation)}: The carbon atoms in stored fat exit the body as \ce{CO2} when fat is metabolized. Weight loss is, chemically, the exhalation of carbon atoms that were previously locked in fat molecules. The atoms are conserved --- they are not destroyed, just relocated from fat tissue to exhaled air.

\textbf{PRIM-SCL002 (amount)}: Food labels express the bond energy per serving in Calories, allowing you to estimate your daily energy intake. A 2,000-Calorie diet means your body needs approximately 2,000 kilocalories of bond energy per day to maintain its functions. The label translates molecular-level bond energies into a practical macroscopic measurement.

Same five primitives. Same reasoning chain. The food label is a bond-energy report card, and your lungs are the exhaust pipe.}

\end{cpcapstone}


%% ============================================================
%% Chapter Summary
%% ============================================================

\begin{chaptersummary}
\begin{tabular}{llp{4cm}p{3.5cm}l}
  \toprule
  \textbf{ID} & \textbf{Name} & \textbf{Reasoning Move} & \textbf{Real-World Hook} & \textbf{Tier} \\
  \midrule
  PRIM-CHG001 & Equation balancing & Balance equations; extract mole ratios & Propane combustion; CO from incomplete combustion & Core \\
  PRIM-CHG002 & Reaction type recognition & Classify type and predict products & Baking soda + vinegar & Core \\
  PRIM-CHG003 & Equilibrium reasoning & Explain why properties stabilize in reversible processes & Opening a soda can & Core \\
  DEF-CHG003 & Le Chatelier's principle & Predict equilibrium shift given stress & Haber process; hyperventilation & Core \\
  PRIM-CHG004 & Rate reasoning & Predict rate changes from conditions & Refrigerating food; pressure cooker & Core \\
  DEF-CHG001 & Catalyst & Lower-$E_a$ pathway; not consumed & Catalytic converter; lactase & Core \\
  PRIM-CHG005 & Acid-base reasoning & Identify proton donor/acceptor & Antacid neutralizing stomach acid & Core \\
  DEF-CHG002 & pH scale & Classify acidity using 0--14 scale & Swimming pool pH & Core \\
  PRIM-CHG006 & Redox reasoning & Identify electron transfer & Battery; iron rusting; respiration & Core \\
  PRIM-CHG007 & Nuclear change & Write nuclear equation; distinguish from chemical & Smoke detectors (Am-241) & E \\
  DEF-CHG004 & Half-life & Predict remaining amount after $n$ half-lives & C-14 dating; I-131 imaging & E \\
  DEF-CHG005 & Precipitation & Predict precipitate using solubility rules & Hard water kettle scale & E \\
  \bottomrule
\end{tabular}

\bigskip

\textbf{Composition Patterns deployed:}

\begin{tabular}{llp{8cm}}
  \toprule
  \textbf{CP} & \textbf{Pattern Name} & \textbf{What It Demonstrates} \\
  \midrule
  CP-002 & Energy-Driven Transformation & Redox + spontaneity + equilibrium explain battery operation and death \\
  CP-003 & Acid-Base in the Body & Bond polarity + acid-base + pH + concentration explain aspirin and antacid chemistry \\
  CP-005 & Dose Makes the Poison & Structure + acid-base + ppm + concentration reveal fluoride safety as dose-dependent \\
  CP-006 & Food Chemistry & Rate + bond energy + exo/endo + conservation + amount explain cooking and Calories \\
  \bottomrule
\end{tabular}
\end{chaptersummary}

\bigskip

This chapter has been the largest and most integrative of the course. It deployed every domain you have studied: Composition (atom conservation, formula reading), Structure (bond type, bond polarity), Energy (spontaneity, activation energy, bond energy, exo/endothermic), Scale (concentration, ppm, moles), and now Change (equation balancing, equilibrium, rate, acid-base, redox). The primitives you learned in Chapters 1 through 4 were not just background knowledge --- they were active reasoning tools deployed again and again in this chapter to explain real phenomena.

Chemistry is, at its heart, the study of change. Atoms rearrange. Bonds break and form. Protons transfer. Electrons flow. Nuclei decay. Energy is released or absorbed. In every case, the same small set of reasoning moves --- conservation, classification, equilibrium, rate, acid-base, redox --- gives you the power to explain what happened, predict what will happen, and understand why.
