% ch-01-com.tex — Chapter 1: What Is Stuff Made Of? (Composition Domain)
% Full conversion from CH-01-COM.md

% Note: \chapter{} command is in the main chem-textbook.tex file.
% This file contains the chapter body only.

\noindent\textit{Domain: COM (Composition) --- The root of all chemical reasoning}

\bigskip

You are holding a bottle of water. The label says ``purified drinking water'' and lists a few minerals: calcium, magnesium, potassium. Now pick up a can of soda. The ingredients list includes ``carbonated water, high fructose corn syrup, phosphoric acid, natural flavors, caffeine.'' One product has five ingredients. The other has three minerals and water.

Here is a question that sounds simple but turns out to be the foundation of all chemistry: \textbf{What is this stuff actually made of?}

Not ``what does the label call it.'' Not ``what does it taste like.'' What is it made of --- at the deepest level that still matters for chemistry? What are the building blocks, how many of each, and how do we keep track of them?

This chapter answers that question. By the end, you will be able to take any substance --- water, table salt, aspirin, the calcium carbonate in an antacid tablet --- and break it down to its atomic inventory. You will be able to read chemical formulas, classify substances, and verify that atoms are not being created or destroyed in a chemical process. These are the foundational moves that every other chapter in this book depends on.

We start with atoms.


%% ============================================================
\section{COM.1: What Are Atoms, and How Do We Identify Them?}
\label{sec:com1}
%% ============================================================

\subsection*{The Big Idea: Matter Is Made of Countable Pieces}

Pour water into a glass. It looks continuous --- a smooth, unbroken liquid you can divide as many times as you like. Scoop out half. Half again. Half again. It seems like you could keep going forever.

You cannot.

If you could zoom in far enough --- far beyond what any kitchen knife or microscope could show you --- you would eventually reach a point where the water is no longer smoothly divisible. You would find individual particles: distinct, countable units. These are \textbf{molecules}, and each water molecule is itself built from even smaller pieces called \textbf{atoms}.

This is the single most important conceptual shift in this chapter: \textbf{matter is not infinitely divisible goo. It is made of discrete, countable atoms.} Everything you eat, drink, breathe, and touch is made of atoms. The chair you are sitting in. The screen you are reading. Your own body. All of it: atoms.

We are not going to spend time on how scientists figured this out. That is a fascinating story involving vacuum tubes and gold foil and tiny oil drops, but it is a history story, not a reasoning tool. Instead, we accept the atomic model as a starting point and learn to use it.

%% ---- Reasoning Move: PRIM-COM001 ----
\begin{reasoningmove}{PRIM-COM001}{Atomic Composition Analysis}
  \textbf{Reasoning move}: Given a substance or sample, decompose it into its constituent atoms and identify their subatomic particles (protons, neutrons, electrons) to determine what the substance is made of at the most fundamental chemical level.
\end{reasoningmove}

This is the entry point to all chemical reasoning. Before you can ask how atoms connect (that is Chapter~\ref{ch:str}), how energy flows (Chapter~\ref{ch:nrg}), or how substances transform (Chapter~\ref{ch:chg}), you need to know \textbf{what atoms are present}. Atomic composition analysis is the act of taking any sample and reasoning downward to its atomic inventory.

\textbf{What is an atom made of?} Every atom contains three types of subatomic particles:

\begin{center}
\begin{tabular}{llll}
  \toprule
  \textbf{Particle} & \textbf{Charge} & \textbf{Location} & \textbf{Mass (relative)} \\
  \midrule
  Proton   & Positive (+1) & Nucleus (center) & 1 \\
  Neutron  & No charge (0) & Nucleus (center) & 1 \\
  Electron & Negative ($-1$) & Surrounding the nucleus & $\approx 0$ (about 1/1836 of a proton) \\
  \bottomrule
\end{tabular}
\end{center}

The nucleus --- containing protons and neutrons packed tightly together --- sits at the center of the atom. Electrons occupy the space around the nucleus. The atom is mostly empty space. If the nucleus were the size of a marble at the center of a football field, the nearest electrons would be orbiting near the outer walls of the stadium.

Here is the key fact: \textbf{the number of protons determines what element an atom is.} An atom with 6 protons is carbon. Always. An atom with 79 protons is gold. Always. The number of protons is called the \textbf{atomic number}, and we label it~$Z$. This is the single most important number in chemistry.

\begin{hook}{Water quality and atomic composition}
  When a water quality report says ``lead detected at \SI{5}{\ppb},'' the first reasoning step is atomic composition analysis: lead atoms ($Z = 82$) are present in the water. We know they are lead because of their proton count --- not because of their color, weight, or any other property.
\end{hook}

\begin{hook}{Sodium on food labels}
  A food label lists ``sodium'' as an ingredient in crackers. Performing atomic composition analysis: sodium is an element with $Z = 11$, meaning every sodium atom has 11 protons in its nucleus. That one number --- 11 protons --- is what makes it sodium and not potassium ($Z = 19$) or calcium ($Z = 20$).
\end{hook}

%% ---- Reasoning Move: PRIM-COM002 ----
\begin{reasoningmove}{PRIM-COM002}{Elemental Identity}
  \textbf{Reasoning move}: Given an atom's proton count (atomic number~$Z$), identify the element it belongs to and locate it on the periodic table to access that element's characteristic properties.
\end{reasoningmove}

\depends{PRIM-COM001}{you need to know what $Z$ means before you can look it up}{1}

If PRIM-COM001 tells you to count the protons, PRIM-COM002 tells you what to do with that count: \textbf{look it up on the periodic table}.

The periodic table is a map. It is indexed by atomic number~$Z$. Every element has a unique spot on this map, and that spot tells you several things at a glance:

\begin{itemize}[nosep]
  \item The element's \textbf{symbol} (a one- or two-letter abbreviation: C for carbon, Na for sodium, Fe for iron)
  \item The element's \textbf{atomic mass} (the weighted average mass of its naturally occurring atoms)
  \item The element's \textbf{period} (row number --- tells you how many electron shells the atom has)
  \item The element's \textbf{group} (column number --- tells you how many valence electrons, which we will get to shortly)
\end{itemize}

Think of the periodic table as a cheat sheet. You do not need to memorize the properties of 118 elements. You need to know how to read the table. Given~$Z$, you can find the element. Given the element's position (row and column), you can predict its behavior. That prediction power is what the next section is about.

\figurebox{The periodic table of the elements, organized by atomic number. Each box shows the element symbol, atomic number, and atomic mass. Rows (periods) indicate electron shell count; columns (groups) indicate valence electron count.}{fig:periodic-table}

\textbf{How to use the periodic table as a lookup tool:}
\begin{enumerate}[nosep]
  \item Find the atomic number ($Z$). Every box on the table displays~$Z$ prominently.
  \item Read the element symbol and name.
  \item Note the period (row) and group (column).
  \item Read the atomic mass (usually displayed below the symbol).
\end{enumerate}

\begin{hook}{Mercury in fish}
  A news headline reads ``Mercury found in imported fish.'' You look up mercury: $Z = 80$, symbol Hg, period~6, group~12. It is a heavy metal --- you can see it sits in the lower portion of the table among the transition metals. That one lookup immediately distinguishes mercury from lighter, biologically essential elements like calcium ($Z = 20$) or iron ($Z = 26$).
\end{hook}

%% ---- Practice Questions ----
\begin{practicequestions}
  \practiceq{A water quality report lists arsenic (As) as a contaminant. Look up arsenic on the periodic table. What is its atomic number? How many protons does every arsenic atom have?}

  \practiceq{Two atoms are found in a sample. Atom A has 8 protons. Atom B has 8 protons and a different number of neutrons than Atom A. Are they the same element or different elements? How do you know?}

  \practiceq{If someone told you they had discovered a new element with 6 protons, what would you tell them?}

  \practiceq{A nutrition label says a food contains ``iron.'' What is the atomic number of iron? Where on the periodic table would you find it?}
\end{practicequestions}

\medskip

\noindent\textit{You now know that matter is made of countable atoms, that each atom's identity is determined by its proton count, and that the periodic table is your lookup tool for connecting proton count to element identity. But the periodic table does far more than just name elements --- it predicts their behavior. That is where we go next.}


%% ============================================================
\section{COM.2: What Can the Periodic Table Predict?}
\label{sec:com2}
%% ============================================================

The periodic table is not just a list. It is a \textbf{prediction engine}. Elements that sit near each other on the table behave similarly. Elements at opposite ends behave very differently. The table's structure encodes patterns --- trends that let you predict how an element will behave before you ever put it in a test tube.

In this section, we learn three things: how to read periodic trends (size, electronegativity, reactivity), how to count valence electrons from an element's group number, and why atoms bond in the first place.

%% ---- Reasoning Move: PRIM-COM003 ----
\begin{reasoningmove}{PRIM-COM003}{Periodic Position Reasoning}
  \textbf{Reasoning move}: Given an element's row (period) and column (group) on the periodic table, predict its relative atomic size, electronegativity, and typical reactivity compared to neighboring elements.
\end{reasoningmove}

\depends{PRIM-COM002}{you must be able to locate an element on the table before you can use its position to make predictions}{1}

Here is the core insight: \textbf{position predicts properties}. Two simple trends capture most of what you need:

\textbf{Trend 1 --- Across a row (left to right):}
\begin{itemize}[nosep]
  \item Atoms get \textbf{smaller}
  \item Atoms pull on electrons \textbf{more strongly}
  \item Metals give way to nonmetals
\end{itemize}

Why? As you move left to right across a row, every element has one more proton in its nucleus than the last. More protons means a stronger pull from the nucleus on the surrounding electrons. That stronger pull draws the electrons closer, making the atom smaller. It also means the atom holds onto its own electrons more tightly and attracts other atoms' electrons more aggressively.

\textbf{Trend 2 --- Down a column (top to bottom):}
\begin{itemize}[nosep]
  \item Atoms get \textbf{larger}
  \item Atoms pull on electrons \textbf{less strongly}
  \item Metals become more reactive; nonmetals become less reactive
\end{itemize}

Why? Moving down a column, each element has an additional shell of electrons between the nucleus and the outermost electrons. Those inner shells act as a buffer --- they shield the outermost electrons from the full pull of the nucleus. More shielding means a weaker grip, a larger atom, and (for metals) a greater willingness to let go of outer electrons.

\figurebox{Simple arrow summary of periodic trends. Across a row (left to right): smaller atoms, stronger pull. Down a column (top to bottom): larger atoms, weaker pull. Metals on the left give way to nonmetals on the right.}{fig:periodic-trends}

\begin{hook}{Alkali metals and water}
  Lithium (Li), sodium (Na), and potassium (K) are all in group~1 --- the leftmost column. Periodic position reasoning predicts they all react vigorously with water, with potassium reacting most violently (it is furthest down the column, so its outer electron is held most loosely). This is why potassium supplements come as carefully formulated pills, not as chunks of pure metal.
\end{hook}

\begin{hook}{Fluorine vs.\ iodine}
  Fluorine (F) and iodine (I) are both in group~17. Fluorine is at the top of the column and is incredibly reactive --- it attacks almost everything. Iodine, much further down, is far less reactive. You can safely put iodine solution on a cut to disinfect it. You would never do that with fluorine.
\end{hook}

%% ---- Reasoning Move: DEF-COM005 ----
\begin{reasoningmove}{DEF-COM005}{Electronegativity}
  \textbf{Reasoning move}: Given two elements' periodic table positions, compare their electronegativities to predict which atom will attract shared electrons more strongly in a bond between them.
\end{reasoningmove}

\depends{PRIM-COM003}{electronegativity follows periodic trends}{1}

Electronegativity is a number that tells you how strongly an atom attracts electrons when it is sharing them with another atom. It follows the same periodic trends we just learned:

\begin{itemize}[nosep]
  \item \textbf{Increases left to right} (fluorine is the champion --- highest electronegativity of any element)
  \item \textbf{Decreases top to bottom} (larger atoms grip electrons less tightly)
\end{itemize}

Here is a short table of electronegativity values for elements you will encounter frequently:

\begin{center}
\begin{tabular}{lll}
  \toprule
  \textbf{Element} & \textbf{Symbol} & \textbf{Electronegativity} \\
  \midrule
  Fluorine   & F  & 4.0 \\
  Oxygen     & O  & 3.4 \\
  Nitrogen   & N  & 3.0 \\
  Chlorine   & Cl & 3.2 \\
  Carbon     & C  & 2.6 \\
  Hydrogen   & H  & 2.2 \\
  Sodium     & Na & 0.9 \\
  Potassium  & K  & 0.8 \\
  \bottomrule
\end{tabular}
\end{center}

You do not need to memorize these numbers. You need to understand the pattern: \textbf{nonmetals on the upper right have high electronegativity; metals on the lower left have low electronegativity.} The difference in electronegativity between two atoms is what determines how they share electrons when they bond --- but that is a structure question (Chapter~\ref{ch:str}). For now, just know that electronegativity is a periodic trend you can predict from position.

\textbf{Why does this matter?} Oxygen (electronegativity 3.4) is far more electronegative than hydrogen (2.2). That difference is why water molecules are lopsided in how they share electrons --- oxygen hogs them. That lopsidedness is why water dissolves salt, which is why the oceans are salty, which is why marine life evolved the way it did. It all starts with a periodic trend.

\medskip

\noindent\textit{Note}: Electronegativity is the bridge between this chapter (Composition) and the next chapter (Structure). Here in COM, we establish electronegativity as a predictable periodic property. In Chapter~\ref{ch:str}, we will use it to determine whether bonds are polar or nonpolar.

%% ---- Reasoning Move: PRIM-COM007 ----
\begin{reasoningmove}{PRIM-COM007}{Valence Electron Reasoning}
  \textbf{Reasoning move}: Given an element's group number on the periodic table, determine its valence electron count and predict its typical bonding behavior (how many bonds it tends to form or how many electrons it tends to gain or lose).
\end{reasoningmove}

\depends{PRIM-COM002}{elemental identity}{1}
\depends{PRIM-COM003}{periodic position reasoning --- group number determines valence count}{1}

Before we can talk about valence electrons, we need to answer a question: \textbf{Why do atoms bond at all?}

Atoms bond because most of them are not stable on their own. They have incomplete sets of outer electrons, and they ``want'' to complete those sets. (They do not literally want anything --- but the result of electrical forces is that atoms with incomplete outer sets will combine with other atoms to achieve a more stable arrangement.)

The electrons in the outermost shell are called \textbf{valence electrons}. These are the ones that participate in bonding. The inner electrons are tucked away near the nucleus and generally do not participate. For our purposes, only valence electrons matter.

Here is the beautiful shortcut: \textbf{for main-group elements, the number of valence electrons equals the group number} (with a small adjustment for groups 13--18).

\begin{center}
\begin{tabular}{lll}
  \toprule
  \textbf{Group} & \textbf{Valence Electrons} & \textbf{Example Elements} \\
  \midrule
  1  & 1 & H, Li, Na, K \\
  2  & 2 & Be, Mg, Ca \\
  13 & 3 & B, Al \\
  14 & 4 & C, Si \\
  15 & 5 & N, P \\
  16 & 6 & O, S \\
  17 & 7 & F, Cl, Br, I \\
  18 & 8 & He (2), Ne, Ar \\
  \bottomrule
\end{tabular}
\end{center}

\textbf{The Octet Rule (a useful heuristic)}: Atoms tend to gain, lose, or share electrons until they have \textbf{8 valence electrons} (or 2, in the case of hydrogen and helium). This is not a law of physics --- it is a pattern that works well for the main-group elements you will encounter most often. Think of it as a rule of thumb: atoms are ``happiest'' with a full set of~8.

How does this play out?

\begin{itemize}[nosep]
  \item \textbf{Metals (groups 1--2)} have few valence electrons. It is easier for them to \textbf{lose} those electrons than to find 6 or 7 more. Sodium (group~1, 1 valence electron) loses that electron to become \ce{Na+}.
  \item \textbf{Nonmetals (groups 15--17)} are close to~8. It is easier for them to \textbf{gain} a few electrons. Chlorine (group~17, 7 valence electrons) gains one electron to become \ce{Cl-}.
  \item \textbf{Carbon (group~14, 4 valence electrons)} is right in the middle. It neither gains nor loses --- it \textbf{shares}. This is why carbon forms 4 bonds, which is why carbon can build the complex molecular scaffolding of life: sugars, proteins, DNA, plastics.
\end{itemize}

\begin{hook}{Noble gases glow}
  Noble gases (group~18) already have 8 valence electrons. They have no need to bond with anything. This is why neon signs glow --- neon atoms stay as isolated atoms even when electricity passes through them, and the energy excites them to emit light. They do not combine with other atoms because their valence shell is already full.
\end{hook}

\begin{hook}{Reactive metals}
  Sodium metal (group~1, one lonely valence electron) reacts violently with water. Why? That single valence electron is loosely held and eager to be transferred. Potassium (also group~1 but further down the table) reacts even more violently --- its valence electron is held even more loosely, consistent with periodic position reasoning (PRIM-COM003).
\end{hook}

%% ---- Reasoning Chain ----
\begin{reasoningchain}{Why does sodium react with chlorine?}
  \chainitem{PRIM-COM007}{Valence electron reasoning}{Sodium is in group~1 and has 1 valence electron.}
  \chainitem{PRIM-COM007}{Valence electron reasoning}{Chlorine is in group~17 and has 7 valence electrons.}
  \chainitem{}{Octet rule}{Sodium ``wants'' to lose 1 electron; chlorine ``wants'' to gain 1.}
  \chainitem{}{Electron transfer}{Sodium transfers its electron to chlorine: \ce{Na+} and \ce{Cl-} form.}
  \chainitem{}{Ionic attraction}{The oppositely charged ions attract, forming \ce{NaCl} --- table salt.}
\end{reasoningchain}

%% ---- Practice Questions ----
\begin{practicequestions}
  \practiceq{Without looking at an electronegativity table, predict which element is more electronegative: oxygen or sulfur. Explain your reasoning using periodic trends.}

  \practiceq{Calcium is in group~2 of the periodic table. How many valence electrons does it have? When it forms an ion, what charge would you predict?}

  \practiceq{Why do the noble gases (group~18) almost never form compounds with other elements?}

  \practiceq{Phosphorus is in group~15. How many valence electrons does it have? How many more electrons would it need to satisfy the octet rule?}

  \practiceq{Rank these elements from smallest to largest atomic size: Na, Mg, K. Explain your reasoning.}
\end{practicequestions}

\medskip

\noindent\textit{You can now extract predictions from the periodic table: size trends, electronegativity, valence electron counts, and expected bonding behavior. But we have been talking about individual elements. What happens when atoms combine? How do we read and classify the substances they form? That is next.}


%% ============================================================
\section{COM.3: How Do We Read and Classify Chemical Substances?}
\label{sec:com3}
%% ============================================================

You walk into a pharmacy and pick up a bottle of antacid. The label says ``calcium carbonate, \ce{CaCO3}.'' A sports drink lists ``water, sugar, citric acid, sodium chloride, potassium phosphate.'' A bag of potting soil says ``contains perlite, vermiculite, peat moss, and slow-release fertilizer.''

These are three very different kinds of ``stuff.'' To make sense of any of them, you need two skills: the ability to classify what kind of substance you are dealing with, and the ability to read a chemical formula. This section teaches both.

%% ---- Reasoning Move: PRIM-COM004 ----
\begin{reasoningmove}{PRIM-COM004}{Substance Classification}
  \textbf{Reasoning move}: Given a sample description or chemical formula, classify it as an element (one type of atom), compound (two or more types of atoms chemically bonded), or mixture (two or more substances physically combined) to determine what kind of substance you are dealing with.
\end{reasoningmove}

\depends{PRIM-COM001}{you must identify what atoms are present before you can classify}{1}

All matter falls into one of three categories. Here is the decision tree:

\figurebox{Decision tree for substance classification. First question: is the sample a single substance or multiple substances mixed together? If single: is it one type of atom only (element) or two or more types bonded in fixed ratios (compound)? If multiple: it is a mixture.}{fig:substance-classification}

\textbf{Why does this classification matter?} Because it tells you what you can and cannot do with the substance:

\begin{itemize}[nosep]
  \item \textbf{Elements} cannot be broken down into simpler substances by chemical means. Gold is gold. You cannot chemically decompose it into something simpler.
  \item \textbf{Compounds} can be broken down into elements by chemical means, but not by physical means. You can split water (\ce{H2O}) into hydrogen gas and oxygen gas using electricity, but you cannot filter or distill it into hydrogen and oxygen.
  \item \textbf{Mixtures} can be separated by physical means --- filtering, distilling, evaporating, picking things apart. You can boil salt water and collect the steam to get pure water, leaving the salt behind.
\end{itemize}

\textbf{Real-world examples:}

\begin{center}
\begin{tabular}{lll}
  \toprule
  \textbf{Substance} & \textbf{Classification} & \textbf{Reasoning} \\
  \midrule
  The aluminum in a soda can & Element & Only one type of atom (Al) \\
  Table salt (\ce{NaCl}) & Compound & Two types of atoms (Na and Cl) in a fixed 1:1 ratio \\
  Tap water & Mixture & \ce{H2O} plus dissolved minerals, gases, trace chemicals \\
  Distilled water & Compound & Only \ce{H2O} molecules --- one substance, two types of atoms \\
  Air & Mixture & \ce{N2}, \ce{O2}, Ar, \ce{CO2}, and others mixed together \\
  A gold ring & Element (or mixture) & Pure 24K gold is an element; 14K gold is a mixture \\
  \bottomrule
\end{tabular}
\end{center}

%% ---- Reasoning Move: PRIM-COM005 ----
\begin{reasoningmove}{PRIM-COM005}{Chemical Formula Reading}
  \textbf{Reasoning move}: Given a chemical formula (molecular, ionic, or empirical), extract which atoms are present, how many of each, and their ratios to determine the quantitative composition of the substance.
\end{reasoningmove}

\depends{PRIM-COM001}{atomic composition analysis}{1}
\depends{PRIM-COM002}{elemental identity --- you must recognize element symbols}{1}

A chemical formula is a compact code that tells you exactly what atoms are in a substance and how many of each. Reading a formula is a literacy skill --- once you learn the notation, you can decode any formula you encounter.

\textbf{Rule 1: Subscripts tell you how many atoms.}

\begin{itemize}[nosep]
  \item \ce{H2O} means 2 hydrogen atoms and 1 oxygen atom per molecule
  \item \ce{CO2} means 1 carbon atom and 2 oxygen atoms
  \item \ce{C6H12O6} (glucose) means 6 carbon, 12 hydrogen, and 6 oxygen atoms per molecule
\end{itemize}

When no subscript is written, the count is~1. The ``1'' is implied.

\textbf{Rule 2: Parentheses multiply.}

Some formulas use parentheses to group atoms, with a subscript outside the parenthesis that multiplies everything inside:

\begin{itemize}[nosep]
  \item \ce{Ca(OH)2} means: 1 calcium, and then the OH group appears \textbf{twice}. Total: 1~Ca, 2~O, 2~H.
  \item \ce{Mg(NO3)2} means: 1 magnesium, and then the \ce{NO3} group appears twice. Total: 1~Mg, 2~N, 6~O (because $2 \times 3 = 6$).
  \item \ce{Al2(SO4)3} means: 2 aluminum, and then the \ce{SO4} group appears three times. Total: 2~Al, 3~S, 12~O.
\end{itemize}

\textbf{Rule 3: Coefficients multiply the entire formula.}

In chemical equations (which you will see in Chapter~\ref{ch:chg}), a number in front of a formula multiplies every atom in it:

\begin{itemize}[nosep]
  \item \ce{2 H2O} means 2 molecules of water: total 4 H atoms and 2 O atoms
  \item \ce{3 CO2} means 3 molecules of carbon dioxide: total 3 C atoms and 6 O atoms
\end{itemize}

\textbf{Practice: Decode these formulas.}

\begin{center}
\begin{tabular}{lll}
  \toprule
  \textbf{Formula} & \textbf{Elements Present} & \textbf{Atom Count} \\
  \midrule
  \ce{NaCl}           & Na, Cl     & 1 Na, 1 Cl \\
  \ce{H2SO4}          & H, S, O    & 2 H, 1 S, 4 O \\
  \ce{NaHCO3}         & Na, H, C, O & 1 Na, 1 H, 1 C, 3 O \\
  \ce{Ca3(PO4)2}      & Ca, P, O   & 3 Ca, 2 P, 8 O \\
  \ce{C2H5OH}         & C, H, O    & 2 C, 6 H, 1 O \\
  \ce{Fe2O3}          & Fe, O      & 2 Fe, 3 O \\
  \bottomrule
\end{tabular}
\end{center}

\begin{hook}{Baking soda on grocery shelves}
  The back of a baking soda box says ``sodium bicarbonate (\ce{NaHCO3}).'' Reading the formula: 1 sodium atom, 1 hydrogen atom, 1 carbon atom, and 3 oxygen atoms per formula unit. That is atomic composition analysis (PRIM-COM001) plus formula reading (PRIM-COM005) working together --- you identified the elements and counted the atoms, all from a grocery store label.
\end{hook}

%% ---- Reasoning Move: PRIM-COM006 ----
\begin{reasoningmove}{PRIM-COM006}{Conservation of Atoms}
  \textbf{Reasoning move}: Given a before-and-after scenario (a chemical reaction or a physical change), verify that atom counts are preserved --- atoms rearrange but are neither created nor destroyed.
\end{reasoningmove}

\depends{PRIM-COM001}{you must be able to count atoms to verify conservation}{1}

This is one of the most powerful ideas in chemistry: \textbf{in any chemical reaction, atoms rearrange, but they do not appear or disappear.} The total number of each type of atom before the reaction equals the total number after. No exceptions in ordinary chemistry.

This principle has a formal name --- the law of conservation of mass --- but the atomic-level version is more useful for reasoning: \textbf{conservation of atoms}.

%% ---- Reasoning Chain ----
\begin{reasoningchain}{Why does burning wood produce \ce{CO2}?}
  \chainitem{PRIM-COM001}{Atomic composition}{Wood contains carbon atoms and hydrogen atoms.}
  \chainitem{PRIM-COM002}{Element identification}{Carbon is element~6; hydrogen is element~1.}
  \chainitem{PRIM-COM006}{Conservation of atoms}{When burned, carbon atoms combine with oxygen atoms from the air --- atoms rearrange, not destroyed.}
  \chainitem{}{Product formation}{The product \ce{CO2} contains the same carbon atoms --- they moved from wood to gas.}
  \chainitem{}{Water vapor}{The hydrogen atoms combine with oxygen to form \ce{H2O} (water vapor).}
  \chainitem{}{Ash}{The ash is what remains of atoms that did not form gaseous products.}
  \chainitem{PRIM-COM006}{Conservation of atoms}{Total atoms before = total atoms after. Nothing was destroyed; everything was rearranged.}
\end{reasoningchain}

\textbf{Why does this matter?} Conservation of atoms is your verification tool. Whenever you encounter a chemical process --- whether it is burning fuel, cooking food, or a claim on the news --- you can check: are atoms being conserved? If someone claims a process ``destroys'' a chemical, conservation says no: the atoms went somewhere. They were rearranged into different substances, but they still exist.

\begin{hook}{Greenhouse gases from cars}
  When gasoline burns in a car engine, the carbon and hydrogen atoms in the fuel combine with oxygen from the air to produce \ce{CO2} and \ce{H2O}. The gasoline ``disappears'' from the tank, but every single atom is accounted for in the exhaust. This is why cars produce greenhouse gases --- the carbon atoms from fossil fuels do not vanish when burned; they become \ce{CO2} in the atmosphere.
\end{hook}

\textbf{A brief note on the exception}: In nuclear reactions (which we will touch on later and cover more in Chapter~\ref{ch:chg}), atoms can be converted into different elements, and tiny amounts of mass can be converted into energy. This is what happens in nuclear power plants and the sun. But in ordinary chemistry --- cooking, burning, rusting, dissolving, everything you encounter in daily life --- conservation of atoms holds without exception.

%% ---- Practice Questions ----
\begin{practicequestions}
  \practiceq{Classify each of the following as an element, compound, or mixture: (a)~pure copper wire, (b)~a cup of coffee, (c)~carbon dioxide gas (\ce{CO2}), (d)~the nitrogen in Earth's atmosphere (\ce{N2}), (e)~brass (a blend of copper and zinc).}

  \practiceq{How many atoms of each element are in the formula \ce{Al2(CO3)3}?}

  \practiceq{A log weighing \SI{5}{kg} is burned completely, leaving \SI{0.3}{kg} of ash. Has matter been destroyed? Use conservation of atoms (PRIM-COM006) to explain where the ``missing'' mass went.}

  \practiceq{Decode the formula for aspirin: \ce{C9H8O4}. How many total atoms are in one molecule?}

  \practiceq{Is orange juice an element, compound, or mixture? Explain your reasoning.}
\end{practicequestions}

\medskip

\noindent\textit{You can now classify substances and read their formulas. But we have been treating every atom of an element as identical. Are they really? What happens when an atom gains or loses electrons? The next section tackles these variations: isotopes and ions.}


%% ============================================================
\section{COM.4: What Are Isotopes and Ions?}
\label{sec:com4}
%% ============================================================

So far, we have said that every atom's identity is determined by its proton count ($Z$). An atom with 6 protons is always carbon. An atom with 8 protons is always oxygen. That is still true. But atoms of the same element are not all perfectly identical --- they can differ in two ways.

First, they can have different numbers of \textbf{neutrons}, which changes their mass but not their chemistry. These are called \textbf{isotopes}.

Second, they can gain or lose \textbf{electrons}, which gives them an electrical charge but does not change what element they are. These are called \textbf{ions}.

%% ---- Reasoning Move: DEF-COM001 ----
\begin{reasoningmove}{DEF-COM001}{Isotope}
  \textbf{Reasoning move}: Given two atoms of the same element (same $Z$) with different mass numbers, recognize them as isotopes --- same chemical behavior, different mass, potentially different nuclear stability.
\end{reasoningmove}

\depends{PRIM-COM001}{atomic composition analysis --- neutron count}{1}
\depends{PRIM-COM002}{elemental identity --- same $Z$ = same element}{1}

An isotope is a version of an element that has the same number of protons but a different number of neutrons. Since the number of protons determines the element's identity, isotopes of the same element are chemically identical. They undergo the same reactions, form the same compounds, and have the same electronegativity. The only differences are mass and, sometimes, nuclear stability.

The \textbf{mass number} of an atom is the total count of protons plus neutrons:

\begin{quote}
  Mass number = protons + neutrons = $Z$ + neutrons
\end{quote}

Carbon has three naturally occurring isotopes:

\begin{center}
\begin{tabular}{lllll}
  \toprule
  \textbf{Isotope} & \textbf{Protons} & \textbf{Neutrons} & \textbf{Mass Number} & \textbf{Stable?} \\
  \midrule
  Carbon-12 & 6 & 6  & 12 & Yes \\
  Carbon-13 & 6 & 7  & 13 & Yes \\
  Carbon-14 & 6 & 8  & 14 & No (radioactive) \\
  \bottomrule
\end{tabular}
\end{center}

All three are carbon. All three form the same chemical bonds, participate in the same reactions, and build the same molecules. The only difference: carbon-14 is radioactive, meaning its nucleus is unstable and will eventually decay.

\textbf{Real-world applications of isotopes:}

\begin{itemize}[nosep]
  \item \textbf{Carbon dating}: Living organisms constantly take in carbon from the environment, including a tiny fraction of carbon-14. When an organism dies, it stops taking in new carbon, and the carbon-14 slowly decays. By measuring how much carbon-14 remains in an archaeological sample, scientists can estimate when the organism died. This works because carbon-14 and carbon-12 are chemically identical --- the organism treats them the same way.

  \item \textbf{Medical imaging}: Certain radioactive isotopes are used as tracers in medical scans. For example, iodine-131 is used to examine thyroid function. The thyroid absorbs iodine-131 just as it absorbs regular iodine (iodine-127) because they are chemically the same element. Doctors can then detect the radioactive decay to create images of the thyroid.

  \item \textbf{Atomic mass on the periodic table}: The atomic mass listed for each element is not a whole number because it is a weighted average of all naturally occurring isotopes. Carbon's atomic mass is 12.011, not exactly~12, because of the small contributions of carbon-13 and carbon-14.
\end{itemize}

%% ---- Reasoning Move: DEF-COM002 ----
\begin{reasoningmove}{DEF-COM002}{Ion}
  \textbf{Reasoning move}: Given an atom or group of atoms with a net electric charge (from gaining or losing electrons), recognize it as an ion and predict its charge from periodic position.
\end{reasoningmove}

\depends{PRIM-COM001}{atomic composition analysis --- electron count vs.\ proton count}{1}
\depends{PRIM-COM003}{periodic position reasoning --- group predicts charge}{1}

A neutral atom has equal numbers of protons and electrons. When an atom \textbf{gains or loses electrons}, it becomes an \textbf{ion} --- a charged particle.

\begin{itemize}[nosep]
  \item If an atom \textbf{loses electrons}, it has more protons than electrons, giving it a \textbf{positive charge}. This is called a \textbf{cation} (pronounced ``CAT-eye-on'').
  \item If an atom \textbf{gains electrons}, it has more electrons than protons, giving it a \textbf{negative charge}. This is called an \textbf{anion} (pronounced ``AN-eye-on'').
\end{itemize}

The periodic table predicts which ions an element will form:

\begin{center}
\begin{tabular}{llll}
  \toprule
  \textbf{Group} & \textbf{Typical Ion} & \textbf{Reasoning} & \textbf{Examples} \\
  \midrule
  1  & +1 cation & Loses 1 valence electron           & \ce{Na+}, \ce{K+}, \ce{Li+} \\
  2  & +2 cation & Loses 2 valence electrons           & \ce{Ca^{2+}}, \ce{Mg^{2+}} \\
  16 & $-2$ anion & Gains 2 electrons to reach 8       & \ce{O^{2-}}, \ce{S^{2-}} \\
  17 & $-1$ anion & Gains 1 electron to reach 8        & \ce{Cl-}, \ce{Br-}, \ce{I-} \\
  \bottomrule
\end{tabular}
\end{center}

Notice the pattern: metals (left side) lose electrons to form cations. Nonmetals (right side) gain electrons to form anions. This connects directly to valence electron reasoning (PRIM-COM007) --- metals have few valence electrons and lose them; nonmetals are close to a full set and gain the remaining ones.

\textbf{Key point}: An ion is still the same element. \ce{Na+} is still sodium ($Z = 11$, still 11 protons). \ce{Cl-} is still chlorine ($Z = 17$, still 17 protons). Gaining or losing electrons changes the charge, not the identity. Identity is determined by protons alone (PRIM-COM002).

\textbf{Polyatomic ions}: Some ions are groups of atoms bonded together that carry a collective charge. You do not need to memorize a long list, but here are five common ones that appear frequently in everyday products:

\begin{center}
\begin{tabular}{lll}
  \toprule
  \textbf{Ion} & \textbf{Formula} & \textbf{Found In} \\
  \midrule
  Hydroxide  & \ce{OH-}       & Cleaning products, drain cleaners \\
  Carbonate  & \ce{CO3^{2-}}  & Antacid tablets, baking soda \\
  Nitrate    & \ce{NO3-}      & Fertilizers, some food preservatives \\
  Sulfate    & \ce{SO4^{2-}}  & Epsom salts, some shampoos \\
  Phosphate  & \ce{PO4^{3-}}  & Detergents, fertilizers \\
  \bottomrule
\end{tabular}
\end{center}

\begin{hook}{Electrolytes in sports drinks}
  Sports drinks advertise ``electrolytes'' on the label. What are electrolytes? They are dissolved ions --- primarily \ce{Na+}, \ce{K+}, and \ce{Cl-}. These ions conduct electricity in solution (that is what ``electrolyte'' means) and regulate nerve signals and muscle contractions in your body. When you sweat, you lose these ions, and the sports drink replaces them.
\end{hook}

\begin{hook}{Calcium supplements}
  Calcium supplements often contain calcium carbonate (\ce{CaCO3}). In this compound, calcium is present as \ce{Ca^{2+}} ions and carbonate as \ce{CO3^{2-}} ions. The periodic table predicts calcium's +2 charge: calcium is in group~2, so it loses 2 electrons (PRIM-COM007 and DEF-COM002 working together).
\end{hook}

%% ---- Practice Questions ----
\begin{practicequestions}
  \practiceq{Oxygen-16 and oxygen-18 are two isotopes of oxygen. How many protons and neutrons does each have? Would they behave differently in a chemical reaction?}

  \practiceq{Predict the ion that magnesium (Mg, group~2) would form. What about bromine (Br, group~17)?}

  \practiceq{A sample contains \ce{Na+} ions and \ce{Cl-} ions. Is sodium still sodium even though it has lost an electron? Explain using the concept of elemental identity (PRIM-COM002).}

  \practiceq{What is the difference between an isotope and an ion? (Hint: one changes the neutron count, the other changes the electron count.)}

  \practiceq{Look at a sports drink label. What ions (electrolytes) are listed? Can you predict their charges from the periodic table?}
\end{practicequestions}

\medskip

\noindent\textit{You now have a complete toolkit for identifying what stuff is made of: atoms, their subatomic particles, the periodic table, chemical formulas, conservation of atoms, isotopes, and ions. But chemistry is not just about identifying substances --- it is about reasoning through claims. When someone tells you ``this chemical causes that effect,'' how do you evaluate the claim? That is what causal chain reasoning is for.}


%% ============================================================
\section{COM.5: How Do We Think in Causal Chains?}
\label{sec:com5}
%% ============================================================

%% ---- Reasoning Move: PRIM-COM008 ----
\begin{reasoningmove}{PRIM-COM008}{Causal Chain Reasoning}
  \textbf{Reasoning move}: Given a chemical claim (``X causes Y''), identify the causal chain: what molecular-level event leads to what macroscopic observation, through what intermediate steps, and whether the claimed causation is supported or merely correlational.
\end{reasoningmove}

\noindent\textit{This is a meta-reasoning skill --- a thinking tool that you will deploy across every remaining chapter in this course.}

\medskip

Chemistry is full of cause-and-effect claims. Advertisements, news articles, product labels, and even well-meaning friends make claims like:

\begin{itemize}[nosep]
  \item ``Fluoride in water strengthens your teeth.''
  \item ``Sulfates in shampoo damage your hair.''
  \item ``Antioxidants prevent aging.''
  \item ``BPA in plastic bottles is harmful to your health.''
\end{itemize}

Some of these claims are well-supported. Some are exaggerations. Some are flat-out wrong. How do you tell the difference?

\textbf{The skill is called causal chain reasoning}, and it works like this: take any claim that says ``X causes Y,'' and break it into a chain of intermediate steps. Each link in the chain should connect a molecular-level event to the next step, all the way from the submicroscopic cause to the macroscopic effect. Then ask: \textbf{is every link in the chain supported?}

\textbf{The template:}

\begin{enumerate}[nosep]
  \item \textbf{What is the molecular-level event?} (What is happening at the level of atoms and molecules?)
  \item \textbf{What is the mechanism?} (How does that event propagate --- through what chemical or biological pathway?)
  \item \textbf{What are the intermediate steps?} (What happens between the molecular event and the thing you can observe?)
  \item \textbf{What is the macroscopic observation?} (What do you actually see, measure, or experience?)
  \item \textbf{Is every link supported, or are steps missing?}
\end{enumerate}

\subsection*{Worked Example 1: ``Fluoride in water strengthens teeth.''}

\begin{enumerate}[nosep]
  \item \textbf{Molecular event}: Fluoride ions (\ce{F-}) in water contact tooth enamel, which is made of hydroxyapatite --- a calcium phosphate mineral.
  \item \textbf{Mechanism}: \ce{F-} replaces hydroxide ions (\ce{OH-}) in the enamel crystal, forming fluorapatite, which is harder and more resistant to acid.
  \item \textbf{Intermediate steps}: Acid produced by mouth bacteria attacks enamel. Fluorapatite resists this acid better than hydroxyapatite.
  \item \textbf{Macroscopic observation}: Fewer cavities over time.
  \item \textbf{Is the chain complete?} Yes --- each link is well-documented. The claim is supported. Note that this does not tell you anything about optimal fluoride concentration or whether excessive fluoride is safe; those are dose questions (a different reasoning move). But the basic causal chain holds.
\end{enumerate}

\subsection*{Worked Example 2: ``Sulfates in shampoo damage your hair.''}

\begin{enumerate}[nosep]
  \item \textbf{Molecular event}: Sodium lauryl sulfate (a surfactant) interacts with oils and proteins on hair.
  \item \textbf{Mechanism}: Surfactants strip oils from surfaces. They are very good at this --- it is how soap works.
  \item \textbf{Intermediate steps}: If too much oil is removed, hair can feel dry and brittle. For some people with sensitive skin or very dry hair, aggressive oil stripping causes irritation.
  \item \textbf{Macroscopic observation}: Dry, frizzy hair; possible scalp irritation.
  \item \textbf{Is the chain complete?} Partially. The chain holds for some individuals (those with dry or damaged hair), but for most people, sulfate-based shampoos work fine. The claim ``sulfates damage hair'' overgeneralizes --- it collapses a conditional causal chain (sulfates strip oils; excessive oil stripping causes dryness in some people) into an unconditional claim.
\end{enumerate}

\subsection*{Worked Example 3: ``Antioxidants prevent aging.''}

\begin{enumerate}[nosep]
  \item \textbf{Molecular event}: Reactive oxygen species (free radicals) can damage cell components by stealing electrons from other molecules.
  \item \textbf{Mechanism}: Antioxidant molecules donate electrons to free radicals, neutralizing them.
  \item \textbf{Intermediate steps}: This should mean less cellular damage, which should mean slower aging.
  \item \textbf{Macroscopic observation}: Slower visible aging? Longer lifespan? Fewer diseases?
  \item \textbf{Is the chain complete?} The first two links are solid chemistry. But the jump from ``neutralized some free radicals in a test tube'' to ``prevents aging in a living human'' is enormous. The body has its own antioxidant systems. Consuming extra antioxidants does not necessarily mean they reach the right cells at the right time. Clinical trials of antioxidant supplements have generally not shown anti-aging effects. \textbf{The causal chain has missing links} --- the molecular mechanism is real, but the macroscopic claim is not supported by the evidence.
\end{enumerate}

\medskip

\textbf{Why is this skill in Chapter~1?} Because causal chain reasoning is the fundamental analytical move of chemical literacy. Every chapter going forward will ask you to trace a chain from molecules to macroscopic effects. When we discuss why ice floats (Chapter~\ref{ch:str}), why some reactions release heat (Chapter~\ref{ch:nrg}), why vinegar neutralizes baking soda (Chapter~\ref{ch:chg}), you will be building causal chains. Learning to name this skill now --- and to notice when chains have missing links --- makes you a better evaluator of every chemical claim you encounter for the rest of your life.

%% ---- Practice Questions ----
\begin{practicequestions}
  \practiceq{Pick a product in your kitchen. Read its label. Choose one ingredient and try to build a causal chain: what does that ingredient do at the molecular level, and how does it lead to the product's advertised effect?}

  \practiceq{A friend says ``I only drink alkaline water because it neutralizes acid in your body.'' What steps would you need to verify in the causal chain before accepting this claim?}

  \practiceq{When you hear ``chemical X is toxic,'' what is the first question a causal chain analysis would ask?}
\end{practicequestions}

\medskip

\noindent\textit{You now have the complete COM toolkit: atomic composition analysis, elemental identity, periodic trends, substance classification, formula reading, conservation of atoms, isotopes, ions, and causal chain reasoning. For most students, this is where the chapter ends. The next section is optional enrichment material that goes deeper into the relationship between chemical formulas and mass measurements.}


%% ============================================================
\section{COM.E: How Do Formulas Connect to Mass Measurements?}
\label{sec:come}
%% ============================================================

\begin{enrichment}{Formulas and Mass Measurements}

This material extends the Core concepts from COM.3 and is not required for subsequent chapters. It provides useful quantitative skills for students who want to go deeper into formula analysis. Your instructor will tell you whether this section is assigned.

\bigskip

%% ---- Reasoning Move: DEF-COM003 ----
\begin{reasoningmove}{DEF-COM003}{Molecular vs.\ Empirical Formula}
  \textbf{Reasoning move}: Given composition data for a compound, distinguish between the molecular formula (actual atom count per molecule) and the empirical formula (simplest whole-number ratio of atoms).
\end{reasoningmove}

\depends{PRIM-COM005}{chemical formula reading --- must read and extract atom counts}{1}

Throughout this chapter, we have been reading chemical formulas and counting atoms. But there is a subtlety we glossed over: sometimes two different formulas describe the same ratio of atoms.

Consider these three compounds:

\begin{center}
\begin{tabular}{lll}
  \toprule
  \textbf{Compound} & \textbf{Molecular Formula} & \textbf{Simplest Ratio} \\
  \midrule
  Formaldehyde & \ce{CH2O}       & 1 C : 2 H : 1 O \\
  Acetic acid  & \ce{C2H4O2}     & 1 C : 2 H : 1 O \\
  Glucose      & \ce{C6H12O6}    & 1 C : 2 H : 1 O \\
  \bottomrule
\end{tabular}
\end{center}

All three have the same simplest whole-number ratio: one carbon atom for every two hydrogen atoms for every one oxygen atom. But they are vastly different substances. Formaldehyde is a toxic preservative. Acetic acid is the tangy component of vinegar. Glucose is the sugar your body uses for energy.

This leads to two definitions:

\begin{itemize}[nosep]
  \item \textbf{Molecular formula}: The actual number of each type of atom in one molecule. Glucose's molecular formula is \ce{C6H12O6} --- one molecule of glucose genuinely contains 6 carbon atoms, 12 hydrogen atoms, and 6 oxygen atoms.

  \item \textbf{Empirical formula}: The simplest whole-number ratio of atoms. Glucose's empirical formula is \ce{CH2O}. This tells you the ratio but not the actual count.
\end{itemize}

\textbf{How to find the empirical formula from a molecular formula:}

Divide all subscripts by their greatest common divisor (GCD).

For \ce{C6H12O6}: GCD of 6, 12, and 6 is 6. Divide each: \ce{C1H2O1} = \ce{CH2O}.

For \ce{C2H6}: GCD of 2 and 6 is 2. Divide each: \ce{CH3}. The empirical formula of ethane (\ce{C2H6}) is \ce{CH3}.

For \ce{NaCl}: GCD of 1 and 1 is 1. The molecular formula and empirical formula are the same.

\textbf{The relationship:}

\begin{quote}
  Molecular formula = (Empirical formula) $\times$ $n$
\end{quote}

where $n$ is a whole number. For glucose: \ce{CH2O} $\times$ 6 = \ce{C6H12O6}, so $n = 6$.

\textbf{Why does this matter?} Some laboratory methods --- like combustion analysis --- can only determine the ratio of atoms in a compound. They give you the empirical formula. To get the molecular formula, you need an additional piece of information: the compound's molar mass (which tells you how heavy one molecule actually is). This is important in research and forensic chemistry, where identifying an unknown substance requires distinguishing between compounds that share the same empirical formula.

\textbf{Worked Example}: A compound has the molecular formula \ce{C4H8O2}. What is its empirical formula?

\begin{enumerate}[nosep]
  \item Find the GCD of 4, 8, and 2. GCD = 2.
  \item Divide all subscripts by 2: \ce{C2H4O1} = \ce{C2H4O}.
  \item The empirical formula is \ce{C2H4O}.
\end{enumerate}

\bigskip

%% ---- Reasoning Move: DEF-COM004 ----
\begin{reasoningmove}{DEF-COM004}{Percent Composition}
  \textbf{Reasoning move}: Given a chemical formula, calculate the mass fraction (percent by mass) of each element in the compound.
\end{reasoningmove}

\depends{PRIM-COM005}{chemical formula reading --- extract atom counts}{1}
\depends{DEF-COM003}{molecular vs.\ empirical formula}{1}

Percent composition answers the question: \textbf{of the total mass of this compound, what fraction comes from each element?}

This is the bridge from the microscopic world (atom counts, which we cannot directly measure on a scale) to the macroscopic world (mass, which we can measure on a scale).

\textbf{The calculation:}

For each element in the formula:

\begin{quote}
  Percent by mass = $\dfrac{\text{number of atoms of that element} \times \text{atomic mass of that element}}{\text{total formula mass}} \times 100\%$
\end{quote}

\textbf{Worked Example}: What is the percent composition of water (\ce{H2O})?

\begin{enumerate}[nosep]
  \item \textbf{Identify atom counts} (PRIM-COM005): 2~H atoms and 1~O atom.
  \item \textbf{Look up atomic masses} (PRIM-COM002, periodic table):
    \begin{itemize}[nosep]
      \item Hydrogen: 1.008~amu
      \item Oxygen: 16.00~amu
    \end{itemize}
  \item \textbf{Calculate each element's mass contribution}:
    \begin{itemize}[nosep]
      \item Hydrogen: $2 \times 1.008 = 2.016$~amu
      \item Oxygen: $1 \times 16.00 = 16.00$~amu
    \end{itemize}
  \item \textbf{Calculate total formula mass}: $2.016 + 16.00 = 18.016$~amu
  \item \textbf{Calculate percent by mass}:
    \begin{itemize}[nosep]
      \item \%H = $(2.016 / 18.016) \times 100\% = 11.19\%$
      \item \%O = $(16.00 / 18.016) \times 100\% = 88.81\%$
    \end{itemize}
  \item \textbf{Verify}: $11.19\% + 88.81\% = 100.00\%$. Check.
\end{enumerate}

So water is about 11\% hydrogen and 89\% oxygen by mass. Even though there are twice as many hydrogen atoms as oxygen atoms, oxygen is so much heavier per atom that it dominates the mass.

\begin{hook}{Calcium supplements}
  A common supplement form is calcium carbonate (\ce{CaCO3}). What percent of its mass is actually calcium?

  \begin{enumerate}[nosep]
    \item Atom counts: 1~Ca, 1~C, 3~O
    \item Atomic masses: Ca = 40.08, C = 12.01, O = 16.00
    \item Mass contributions: Ca = 40.08, C = 12.01, O = $3 \times 16.00 = 48.00$
    \item Total formula mass: $40.08 + 12.01 + 48.00 = 100.09$
    \item \%Ca = $(40.08 / 100.09) \times 100\% = 40.04\%$
  \end{enumerate}

  Only about 40\% of calcium carbonate is calcium by mass. The rest is carbon and oxygen. So when a supplement label says ``\SI{500}{mg} of calcium carbonate,'' you are getting about \SI{200}{mg} of actual calcium. An alternative form, calcium citrate (\ce{Ca3(C6H5O7)2}), is only about 21\% calcium by mass, so you need a larger pill for the same amount of calcium. Percent composition tells you how much of the active element you are actually getting.
\end{hook}

\end{enrichment}

%% ---- Practice Questions ----
\begin{practicequestions}
  \practiceq{What is the empirical formula of hydrogen peroxide (\ce{H2O2})? What is $n$ (the multiplier relating the empirical formula to the molecular formula)?}

  \practiceq{Calculate the percent composition of carbon dioxide (\ce{CO2}). Which element contributes more to the total mass?}

  \practiceq{Ribose (a sugar found in RNA) has the molecular formula \ce{C5H10O5}. What is its empirical formula? Does it share this empirical formula with glucose (\ce{C6H12O6})?}

  \practiceq{A calcium supplement contains calcium citrate: \ce{Ca3(C6H5O7)2}. Without doing a full calculation, would you expect the percent calcium to be higher or lower than in calcium carbonate (\ce{CaCO3})? Why?}
\end{practicequestions}


%% ============================================================
%% Chapter Summary
%% ============================================================

\begin{chaptersummary}
\noindent This chapter established the \textbf{composition toolkit} --- the foundational reasoning moves for identifying what stuff is made of. Here is what you can now do:

\medskip

\begin{tabular}{llp{6.5cm}}
  \toprule
  \textbf{ID} & \textbf{Reasoning Move} & \textbf{What It Lets You Do} \\
  \midrule
  PRIM-COM001 & Atomic composition analysis & Break any substance down to its atomic inventory \\
  PRIM-COM002 & Elemental identity & Look up any element from its proton count on the periodic table \\
  PRIM-COM003 & Periodic position reasoning & Predict size, electronegativity, and reactivity from table position \\
  DEF-COM005  & Electronegativity & Compare how strongly two atoms attract electrons \\
  PRIM-COM007 & Valence electron reasoning & Count valence electrons and predict bonding from group number \\
  PRIM-COM004 & Substance classification & Sort any sample into element, compound, or mixture \\
  PRIM-COM005 & Chemical formula reading & Decode any formula into atom identities and counts \\
  PRIM-COM006 & Conservation of atoms & Verify that atoms are neither created nor destroyed \\
  DEF-COM001  & Isotope & Recognize same-element, different-mass variants \\
  DEF-COM002  & Ion & Recognize charged atoms and predict charges from group \\
  PRIM-COM008 & Causal chain reasoning & Evaluate cause-and-effect claims by tracing molecular-to-macroscopic chains \\
  DEF-COM003  & Molecular vs.\ empirical formula & Distinguish actual counts from simplest ratios (Enrichment) \\
  DEF-COM004  & Percent composition & Calculate mass fractions from formulas (Enrichment) \\
  \bottomrule
\end{tabular}
\end{chaptersummary}

\medskip

Every concept in the rest of this book builds on these 13 items. Chapter~\ref{ch:str} (Structure) will take the atoms you identified here and ask: how are they connected? In what three-dimensional arrangement? With what consequences for the substance's properties? To answer those questions, you will need the valence electron counts (PRIM-COM007) and electronegativity values (DEF-COM005) you learned here --- those are the bridges from composition to structure.

But composition always comes first. Before you ask how atoms are arranged, you must know what atoms are present. Before you ask how energy flows, you must know what substances are involved. Before you ask how a reaction proceeds, you must know what went in and what came out. \textbf{What is stuff made of?} is the root question. Everything else grows from the answer.
