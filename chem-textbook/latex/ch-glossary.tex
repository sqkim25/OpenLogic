% ch-glossary.tex — Glossary of All 62 Taxonomy Items
% Note: \chapter{} command is in the main chem-textbook.tex file.
% This file contains the chapter body only.

\noindent All 62 reasoning moves are listed below, organized alphabetically within each domain. Each entry provides the taxonomy ID, name, a one-sentence description of the reasoning move, the chapter where it is introduced, and whether it is Core~(C) or Enrichment~(E).

%% ============================================================
\section*{COM Domain --- Composition (13 items: 8 PRIM + 5 DEF)}
%% ============================================================

\begin{description}[style=nextline, leftmargin=1.5cm, font=\normalfont]

\item[\textbf{DEF-COM001: Isotope}]
Given two atoms of the same element (same~$Z$) with different mass numbers, recognize them as isotopes: same chemical behavior, different mass, potentially different nuclear stability. (Chapter~1, C)

\item[\textbf{DEF-COM002: Ion}]
Given an atom or group of atoms with a net electric charge (from gaining or losing electrons), recognize it as an ion and predict its charge from periodic position. (Chapter~1, C)

\item[\textbf{DEF-COM003: Molecular vs.\ empirical formula}]
Given composition data for a compound, distinguish between the molecular formula (actual atom count per molecule) and the empirical formula (simplest whole-number ratio of atoms). (Chapter~1, E)

\item[\textbf{DEF-COM004: Molar mass / Percent composition}]
Given a chemical formula, calculate the mass fraction (percent by mass) of each element in the compound to connect formula-level composition to mass-level measurements. (Chapter~1, E)

\item[\textbf{DEF-COM005: Electronegativity}]
Given two elements' periodic table positions, compare their electronegativities to predict which atom will attract shared electrons more strongly in a bond between them. (Chapter~1, C)

\item[\textbf{PRIM-COM001: Atomic composition analysis}]
Given a substance or sample, decompose it into its constituent atoms and identify their subatomic particles (protons, neutrons, electrons) to determine what the substance is made of at the most fundamental chemical level. (Chapter~1, C)

\item[\textbf{PRIM-COM002: Elemental identity}]
Given an atom's proton count (atomic number~$Z$), identify the element it belongs to and locate it on the periodic table to access that element's characteristic properties. (Chapter~1, C)

\item[\textbf{PRIM-COM003: Periodic position reasoning}]
Given an element's row (period) and column (group) on the periodic table, predict its relative atomic size, electronegativity, and typical reactivity compared to neighboring elements. (Chapter~1, C)

\item[\textbf{PRIM-COM004: Substance classification}]
Given a sample description or chemical formula, classify it as an element (one type of atom), compound (two or more types of atoms chemically bonded), or mixture (two or more substances physically combined). (Chapter~1, C)

\item[\textbf{PRIM-COM005: Chemical formula reading}]
Given a chemical formula (molecular, ionic, or empirical), extract which atoms are present, how many of each, and their ratios to determine the quantitative composition of the substance. (Chapter~1, C)

\item[\textbf{PRIM-COM006: Conservation of atoms}]
Given a before-and-after scenario (a chemical reaction or a physical change), verify that atom counts are preserved: atoms rearrange but are neither created nor destroyed. (Chapter~1, C)

\item[\textbf{PRIM-COM007: Valence electron reasoning}]
Given an element's group number on the periodic table, determine its valence electron count and predict its typical bonding behavior (how many bonds it tends to form or how many electrons it tends to gain or lose). (Chapter~1, C)

\item[\textbf{PRIM-COM008: Causal chain reasoning}]
Given a chemical claim (``X causes Y''), identify the causal chain: what molecular-level event leads to what macroscopic observation, through what intermediate steps, and whether the claimed causation is supported or merely correlational. (Chapter~1, C)

\end{description}

%% ============================================================
\section*{STR Domain --- Structure (15 items: 5 PRIM + 10 DEF)}
%% ============================================================

\begin{description}[style=nextline, leftmargin=1.5cm, font=\normalfont]

\item[\textbf{DEF-STR001: Lewis structure}]
Given a molecular formula and the valence electron count for each atom, distribute electrons as bonding pairs and lone pairs to satisfy the octet rule, producing a 2D diagram that shows how atoms are connected and where unshared electrons reside. (Chapter~2, C)

\item[\textbf{DEF-STR002: Molecular polarity}]
Given a molecule's shape and bond dipoles, determine the vector sum to classify the whole molecule as polar or nonpolar. (Chapter~2, C)

\item[\textbf{DEF-STR003: Hydrogen bond}]
Given a molecule containing H bonded to N, O, or F, identify the potential for hydrogen bonding with nearby molecules bearing lone pairs on N, O, or F. (Chapter~2, C)

\item[\textbf{DEF-STR004: ``Like dissolves like''}]
Given a solute and solvent, compare their polarities to predict whether the solute will dissolve: polar substances dissolve in polar solvents, and nonpolar substances dissolve in nonpolar solvents. (Chapter~2, C)

\item[\textbf{DEF-STR005: Isomer recognition}]
Given two molecules with the same molecular formula, determine if they are isomers (different structural arrangements) and predict that they will have different properties. (Chapter~2, C)

\item[\textbf{DEF-STR006: Phase as IMF balance}]
Given a substance's dominant intermolecular force strength and the temperature, predict whether the substance is solid, liquid, or gas by assessing the competition between IMFs (favoring condensed phases) and kinetic energy (favoring gas). (Chapter~2, C)

\item[\textbf{DEF-STR007: Carbon backbone reasoning}]
Given carbon's 4 valence electrons and 4 bonds, reason about chain, branch, and ring diversity, and explain how chain length and functional groups determine molecular properties. (Chapter~2, C)

\item[\textbf{DEF-STR008: Polymer reasoning}]
Given a monomer, predict how it links into a polymer chain, and explain how chain length, branching, cross-linking, and monomer polarity determine material properties. (Chapter~2, C)

\item[\textbf{DEF-STR009: Metallic structure} (E)]
Given a metallic element, reason about the electron sea model (delocalized valence electrons surrounding a lattice of cations) to predict metallic properties such as conductivity, malleability, and luster. (Chapter~2, E)

\item[\textbf{DEF-STR010: Water as universal solvent}]
Given water's bent molecular shape, strong dipole, and four-fold hydrogen bonding capacity, reason about why water is an exceptionally effective solvent for ionic and polar substances. (Chapter~2, C)

\item[\textbf{PRIM-STR001: Bond type classification}]
Given two elements and their electronegativities, calculate the electronegativity difference to classify the bond as nonpolar covalent (shared equally), polar covalent (shared unequally), or ionic (transferred). (Chapter~2, C)

\item[\textbf{PRIM-STR002: Bond polarity reasoning}]
Given a bond between two atoms, use their electronegativity difference to determine the direction and magnitude of partial charge separation (which end is delta-minus, which is delta-plus). (Chapter~2, C)

\item[\textbf{PRIM-STR003: Molecular shape reasoning (VSEPR)}]
Given a Lewis structure, count the electron groups around the central atom and apply VSEPR to predict the three-dimensional molecular geometry. (Chapter~2, C)

\item[\textbf{PRIM-STR004: Intermolecular force hierarchy}]
Given a molecule's structure, identify which intermolecular forces are present (London dispersion, dipole-dipole, hydrogen bonding, ionic) and rank them from weakest to strongest. (Chapter~2, C)

\item[\textbf{PRIM-STR005: Structure-to-property inference}]
Given a molecule's structural features (polarity, IMF type, chain length), predict the direction of a macroscopic property (boiling point, solubility, evaporation rate) relative to a comparison molecule. (Chapter~2, C)

\end{description}

%% ============================================================
\section*{NRG Domain --- Energy (11 items: 6 PRIM + 5 DEF)}
%% ============================================================

\begin{description}[style=nextline, leftmargin=1.5cm, font=\normalfont]

\item[\textbf{DEF-NRG001: Heat vs.\ temperature}]
Given two objects at different temperatures in contact, distinguish heat (energy transferred) from temperature (average molecular kinetic energy), and predict that heat flows from hot to cold until thermal equilibrium is reached. (Chapter~3, C)

\item[\textbf{DEF-NRG002: Specific heat capacity}]
Given two substances receiving the same amount of heat, use specific heat capacity to predict which substance's temperature rises more: higher specific heat means the substance is ``harder to heat up.'' (Chapter~3, C)

\item[\textbf{DEF-NRG003: Enthalpy change ($\DH$)}]
Given a reaction at constant pressure, read the sign and magnitude of $\DH$: negative means exothermic (energy released), positive means endothermic (energy absorbed). (Chapter~3, C)

\item[\textbf{DEF-NRG004: Free energy (conceptual)} (E)]
Given conflicting energy and entropy drives, evaluate whether the combined ``free energy scorecard'' is net favorable (spontaneous) or unfavorable, with temperature as the weighting factor. (Chapter~3, E)

\item[\textbf{DEF-NRG005: Calorie/joule}]
Given an energy value in calories, kilocalories (food Calories), or joules, convert between units and connect chemistry to nutrition labels ($1~\text{food Calorie} = 1~\text{kcal} = \SI{4184}{J}$). (Chapter~3, C)

\item[\textbf{PRIM-NRG001: Energy tracking}]
Given a process (chemical reaction, physical change, or energy transfer), trace energy through the system: identify input energy forms, energy transformations within the system, and output energy forms, then verify that total energy is conserved. (Chapter~3, C)

\item[\textbf{PRIM-NRG002: Bond energy reasoning}]
Given a chemical reaction, compare the total energy required to break all bonds in reactants with the total energy released when forming all bonds in products; the difference determines net energy release or absorption. (Chapter~3, C)

\item[\textbf{PRIM-NRG003: Exothermic/endothermic classification}]
Given an energy diagram or description, classify the process as exothermic (energy released to surroundings, products at lower energy) or endothermic (energy absorbed from surroundings, products at higher energy). (Chapter~3, C)

\item[\textbf{PRIM-NRG004: Entropy reasoning}]
Given a before-and-after description, determine whether the number of possible molecular arrangements (microstates) increased or decreased; more arrangements means higher entropy and greater probability. (Chapter~3, C)

\item[\textbf{PRIM-NRG005: Spontaneity reasoning}]
Given both the energy change (exo or endo) and the entropy change (increase or decrease), determine whether the process will occur spontaneously, using the four-case matrix where temperature acts as the tiebreaker. (Chapter~3, C)

\item[\textbf{PRIM-NRG006: Activation energy reasoning}]
Given a thermodynamically favorable but non-occurring process, identify the activation energy barrier that must be overcome to initiate the reaction, and recognize three ways to overcome it: heat, spark, or catalyst. (Chapter~3, C)

\end{description}

%% ============================================================
\section*{SCL Domain --- Scale (11 items: 6 PRIM + 5 DEF)}
%% ============================================================

\begin{description}[style=nextline, leftmargin=1.5cm, font=\normalfont]

\item[\textbf{DEF-SCL001: Molarity}]
Given a solution, calculate the number of moles of solute per liter of solution (mol/L), and use the dilution relationship $M_1V_1 = M_2V_2$ to solve dilution problems. (Chapter~4, C)

\item[\textbf{DEF-SCL002: Parts per million/billion}]
Given a trace-level concentration, express it in ppm (mg/L) or ppb ($\mu$g/L) and compare it against regulatory thresholds to assess safety. (Chapter~4, C)

\item[\textbf{DEF-SCL003: Ideal gas reasoning} (E)]
Given a gas sample, use $PV = nRT$ qualitatively to predict how changing one variable (pressure, volume, temperature, or moles) affects the others when the remaining variables are held constant. (Chapter~4, E)

\item[\textbf{DEF-SCL004: Colligative properties} (E)]
Given dissolved particles in a solvent, predict that the freezing point is lowered and the boiling point is raised, with the magnitude depending on particle count (not identity). (Chapter~4, E)

\item[\textbf{DEF-SCL005: Safety and risk reasoning}]
Given hazard information (GHS pictograms, LD50, PEL, exposure route, duration), distinguish hazard (intrinsic property of a substance) from risk (depends on concentration, route, and duration of exposure) to make informed safety judgments. (Chapter~4, C)

\item[\textbf{PRIM-SCL001: Macro-to-submicro translation}]
Given an observable macroscopic property or a molecular-level behavior, translate between the two levels: explain the macroscopic in terms of the molecular, or predict the macroscopic from the molecular. (Chapter~4, C)

\item[\textbf{PRIM-SCL002: Mole concept / Amount reasoning}]
Given atoms or molecules, convert to moles (divide by $6.02 \times 10^{23}$); given moles, convert to mass using molar mass; the mole is the quantitative bridge between submicroscopic particle counts and macroscopic mass. (Chapter~4, C)

\item[\textbf{PRIM-SCL003: Concentration reasoning}]
Given a solution, determine the amount of solute per volume of solution, and predict how concentration affects rate, toxicity, or any concentration-dependent property. (Chapter~4, C)

\item[\textbf{PRIM-SCL004: Emergent property reasoning}]
Given molecular-level features, explain why the bulk property (boiling point, color, viscosity) is not a simple sum of individual-molecule properties but arises only from the collective behavior of many molecules. (Chapter~4, C)

\item[\textbf{PRIM-SCL005: Proportional reasoning}]
Given a ratio (from a balanced equation, a recipe, or a dosage), scale it up or down to connect molecular-level ratios to lab-scale or everyday quantities. (Chapter~4, C)

\item[\textbf{PRIM-SCL006: Unit analysis}]
Given a multi-step calculation involving physical quantities, use dimensional analysis (tracking units through each step) to verify correctness, catch conversion errors, and ensure the final answer has the expected units. (Chapter~4, C)

\end{description}

%% ============================================================
\section*{CHG Domain --- Change (12 items: 7 PRIM + 5 DEF)}
%% ============================================================

\begin{description}[style=nextline, leftmargin=1.5cm, font=\normalfont]

\item[\textbf{DEF-CHG001: Catalyst}]
A catalyst speeds a reaction by providing a lower-activation-energy pathway without being consumed in the process; it does not change the equilibrium position ($K$) or the overall energy change ($\DH$). (Chapter~5, C)

\item[\textbf{DEF-CHG002: pH scale}]
Given a solution, classify it as acidic (pH below~7), neutral (pH equal to~7), or basic (pH above~7), recognizing that each pH unit represents a tenfold change in \ce{H+} concentration. (Chapter~5, C)

\item[\textbf{DEF-CHG003: Le Chatelier's principle}]
Given a system at equilibrium plus a stress (change in concentration, temperature, or pressure), predict the direction the equilibrium shifts to partially counteract the stress. (Chapter~5, C)

\item[\textbf{DEF-CHG004: Half-life} (E)]
Given a radioactive isotope and its half-life, predict the amount of radioactive material remaining after a whole number of half-lives: after $n$ half-lives, the fraction remaining is $(1/2)^n$. (Chapter~5, E)

\item[\textbf{DEF-CHG005: Precipitation} (E)]
Given two ionic solutions, predict whether mixing them will produce an insoluble solid (precipitate) by identifying the ion combinations and checking them against solubility rules. (Chapter~5, E)

\item[\textbf{PRIM-CHG001: Equation reading and balancing}]
Given a chemical equation with reactants and products, balance it so that atoms of each element are conserved on both sides, and extract the quantitative mole ratios (coefficients). (Chapter~5, C)

\item[\textbf{PRIM-CHG002: Reaction type recognition}]
Given reactants, classify the transformation type (synthesis, decomposition, single replacement, double replacement, or combustion) and use the classification to predict the general form of the products. (Chapter~5, C)

\item[\textbf{PRIM-CHG003: Equilibrium reasoning}]
Given a reversible process, explain why macroscopic properties stabilize even though molecular-level activity continues: at equilibrium, the forward and reverse reaction rates are equal, producing a dynamic steady state. (Chapter~5, C)

\item[\textbf{PRIM-CHG004: Rate reasoning}]
Given conditions (temperature, concentration, surface area, catalyst), predict whether the reaction rate will increase or decrease, using collision theory as the explanatory framework. (Chapter~5, C)

\item[\textbf{PRIM-CHG005: Acid-base reasoning}]
Given two substances, identify the proton donor (Br\o nsted acid) and proton acceptor (Br\o nsted base), predict the products of the proton transfer, and distinguish strong from weak acids and bases. (Chapter~5, C)

\item[\textbf{PRIM-CHG006: Oxidation-reduction reasoning}]
Given a process, identify the electron transfer: which species is oxidized (loses electrons) and which is reduced (gains electrons), using oxidation numbers as the bookkeeping tool. (Chapter~5, C)

\item[\textbf{PRIM-CHG007: Nuclear change reasoning} (E)]
Given an unstable nucleus, predict the type of radiation emitted (alpha, beta, or gamma), write the nuclear equation conserving mass number and atomic number, and distinguish nuclear change from chemical change. (Chapter~5, E)

\end{description}
