% ch-answers.tex — Selected Answers to Practice Questions
% Note: \chapter{} command is in the main chem-textbook.tex file.
% This file contains the chapter body only.

\noindent One representative practice question from each of the six chapters, with a model answer demonstrating the reasoning-move approach.

%% ============================================================
\section*{Chapter 1 (COM): Practice Questions COM.3, Question 3}
%% ============================================================

\begin{modelanswer}{1}{3}{COM.3}
\textbf{Question}: A log weighing \SI{5}{kg} is burned completely, leaving \SI{0.3}{kg} of ash. Has matter been destroyed? Use conservation of atoms (PRIM-COM006) to explain where the ``missing'' mass went.

\medskip

\textbf{Model Answer}:

Matter has not been destroyed. \textbf{Conservation of atoms (PRIM-COM006)} tells us that in any chemical reaction, every atom present before the reaction must be present after it --- atoms rearrange but are never created or destroyed.

The log is made primarily of cellulose and other carbon-based compounds. When it burns, the combustion reaction combines the carbon and hydrogen atoms in the wood with oxygen from the air:

\reaction{Cellulose + O2 -> CO2 + H2O + ash}

The \SI{5}{kg} starting mass accounts for the wood, but the reaction also consumes oxygen from the air --- oxygen that we did not weigh on the scale. The products include carbon dioxide gas (\ce{CO2}) and water vapor (\ce{H2O}), both of which escape into the atmosphere and are invisible. The \SI{0.3}{kg} of ash represents the small fraction of atoms (minerals, metals) in the wood that did not form gaseous products.

If we could collect all the \ce{CO2} gas and water vapor produced and weigh them together with the ash, the total mass of products would equal the total mass of wood plus the oxygen consumed. The ``missing'' \SI{4.7}{kg} did not vanish --- it became invisible gases. \textbf{PRIM-COM001} (atomic composition analysis) confirms that the carbon atoms are now in \ce{CO2}, the hydrogen atoms are now in \ce{H2O}, and the oxygen atoms are distributed between both products. Every atom is accounted for.
\end{modelanswer}

%% ============================================================
\section*{Chapter 2 (STR): Practice Questions STR.3, Question 2}
%% ============================================================

\begin{modelanswer}{2}{2}{STR.3}
\textbf{Question}: Explain why water (\ce{H2O}, molar mass 18) has a much higher boiling point than methane (\ce{CH4}, molar mass 16), even though they are similar in size.

\medskip

\textbf{Model Answer}:

The answer lies in the \textbf{intermolecular force hierarchy (PRIM-STR004)}.

First, identify the dominant IMFs for each molecule:

\begin{itemize}[nosep]
  \item \textbf{Methane (\ce{CH4})}: Methane is nonpolar (the C--H bond has a very small electronegativity difference of 0.4, and the tetrahedral shape is symmetric, so any small dipoles cancel). The only IMF present is \textbf{London dispersion forces} --- the weakest type.

  \item \textbf{Water (\ce{H2O})}: Water is polar (the O--H bonds are strongly polar with an electronegativity difference of 1.2, and the bent shape means the dipoles do not cancel). Water has hydrogen bonded to oxygen, meeting the criteria for \textbf{hydrogen bonding} --- a much stronger IMF than London dispersion. Each water molecule can form up to four hydrogen bonds with its neighbors.
\end{itemize}

Now apply \textbf{structure-to-property inference (PRIM-STR005)}: stronger IMFs mean more energy is required to pull molecules apart and send them into the gas phase. Water's extensive hydrogen bonding network holds molecules together much more stubbornly than methane's weak London forces. Therefore, water requires a much higher temperature (\SI{100}{\degreeCelsius}) to boil, while methane boils at \SI{-161}{\degreeCelsius}.

The key insight is that molecular mass alone does not determine boiling point. IMF strength does. Water and methane have nearly the same mass, but their IMFs differ by roughly an order of magnitude in strength.
\end{modelanswer}

%% ============================================================
\section*{Chapter 3 (NRG): Practice Questions NRG.2, Question 1}
%% ============================================================

\begin{modelanswer}{3}{1}{NRG.2}
\textbf{Question}: A student says, ``When you burn wood, the bonds in wood break apart and release energy.'' What is wrong with this statement? Correct it using bond energy reasoning (PRIM-NRG002).

\medskip

\textbf{Model Answer}:

The statement contains a fundamental misconception. \textbf{Bond energy reasoning (PRIM-NRG002)} establishes that:

\begin{itemize}[nosep]
  \item \textbf{Breaking bonds ALWAYS requires energy} (it costs energy to pull atoms apart).
  \item \textbf{Forming bonds ALWAYS releases energy} (atoms snap together and release the excess energy as heat).
\end{itemize}

The student's error is thinking that breaking the bonds in wood is what produces the heat. In reality, burning wood involves two steps:

\begin{enumerate}[nosep]
  \item \textbf{Energy input}: Bonds in the wood (C--C, C--H, C--O) and in oxygen (\ce{O=O}) are broken. This step \textbf{consumes} energy.
  \item \textbf{Energy output}: New bonds in the products (\ce{C=O} bonds in \ce{CO2} and O--H bonds in \ce{H2O}) are formed. This step \textbf{releases} energy.
\end{enumerate}

The corrected statement is: ``When you burn wood, the bonds in wood and oxygen break (requiring energy), and new, stronger bonds form in \ce{CO2} and \ce{H2O} (releasing energy). The energy released by forming the product bonds is greater than the energy consumed by breaking the reactant bonds. The surplus energy is what you feel as heat.''

Using \textbf{PRIM-NRG003} (exothermic classification): since more energy is released than consumed, the net process is exothermic, and the surroundings get warmer.
\end{modelanswer}

%% ============================================================
\section*{Chapter 4 (SCL): Practice Questions SCL.3, Question 5}
%% ============================================================

\begin{modelanswer}{4}{5}{SCL.3}
\textbf{Question}: A student says, ``10 ppb is basically zero --- it cannot matter.'' Respond to this claim using the Flint water crisis as evidence.

\medskip

\textbf{Model Answer}:

This claim confuses ``small'' with ``insignificant.'' \textbf{Concentration reasoning (PRIM-SCL003)} and \textbf{parts-per-billion interpretation (DEF-SCL002)} show that at the molecular level, even trace concentrations can have significant biological effects.

Ten ppb means 10 micrograms of a substance per liter of water. While this seems vanishingly small, the biological impact depends on the substance and the duration of exposure, not just the amount:

\begin{itemize}[nosep]
  \item In the Flint, Michigan, water crisis, lead levels in some homes reached \SI{100}{ppb} --- only ten times the student's ``basically zero.'' The EPA action level for lead is \SI{15}{ppb}. At concentrations above this threshold, lead accumulates in the body over time, causing irreversible neurological damage in children, including reduced IQ and developmental delays.

  \item Lead at \SI{10}{ppb} is below the EPA action level, but this does not mean it is harmless. The CDC states there is no known safe level of lead exposure for children. The effects are cumulative --- even \SI{10}{ppb}, consumed daily for months or years, deposits significant quantities of lead in growing bones and developing brains.
\end{itemize}

Using \textbf{PRIM-SCL005} (proportional reasoning): if a person drinks 2 liters of water per day at \SI{10}{ppb}, they ingest 20 micrograms of lead daily, or about 7,300 micrograms per year. Over several years, this cumulative exposure can cause measurable health effects.

The lesson: concentration reasoning requires comparing measured values against known thresholds and considering exposure duration, not simply judging whether a number ``sounds small.''
\end{modelanswer}

%% ============================================================
\section*{Chapter 5 (CHG): Practice Questions CHG.2, Question 3}
%% ============================================================

\begin{modelanswer}{5}{3}{CHG.2}
\textbf{Question}: Explain, using Le Chatelier's principle, why a sealed bottle of soda remains fizzy in the refrigerator but goes flat quickly once opened and left on a warm countertop. (Hint: consider both temperature and pressure changes.)

\medskip

\textbf{Model Answer}:

The relevant equilibrium is the dissolution of carbon dioxide in water:

\reaction{CO2 (g) <=> CO2 (aq)}

The forward direction (dissolving) is slightly exothermic --- it releases a small amount of heat.

\textbf{Two stresses are applied when you open the bottle and leave it on a warm counter:}

\textbf{Stress 1: Pressure decrease (opening the bottle).} In the sealed bottle, \ce{CO2} gas above the liquid is at higher-than-atmospheric pressure. Opening the bottle releases this pressurized gas, lowering the pressure above the liquid. By \textbf{Le Chatelier's principle (DEF-CHG003)}, the system shifts to counteract the pressure drop by producing more gas-phase \ce{CO2}. The equilibrium shifts to the left: dissolved \ce{CO2} escapes from the liquid. You see this as bubbles.

\textbf{Stress 2: Temperature increase (warm countertop).} The dissolution of \ce{CO2} in water is exothermic (the forward direction releases heat). Raising the temperature is like ``adding heat'' to the system. By Le Chatelier's principle, the equilibrium shifts in the direction that absorbs heat --- the reverse direction. More \ce{CO2} leaves the solution and enters the gas phase. Warm soda goes flat faster than cold soda.

\textbf{In the sealed refrigerator}, neither stress is applied: the pressure is maintained by the sealed cap, and the cold temperature favors the exothermic forward direction (more \ce{CO2} stays dissolved). The equilibrium strongly favors dissolved \ce{CO2} at low temperature and high pressure.

This analysis uses \textbf{PRIM-CHG003} (equilibrium reasoning) to establish the dynamic balance and \textbf{DEF-CHG003} (Le Chatelier's principle) to predict the direction of shift under two independent stresses.
\end{modelanswer}

%% ============================================================
\section*{Chapter 6 (MULTI): Practice Questions, Question 1}
%% ============================================================

\begin{modelanswer}{6}{1}{MULTI}
\textbf{Question}: When you boil an egg, the clear, liquid egg white becomes solid and opaque. Using PRIM-STR004 (IMF hierarchy) and PRIM-CHG004 (rate reasoning), explain why the egg cooks faster in boiling water than at room temperature. Why is the process irreversible?

\medskip

\textbf{Model Answer}:

The egg white is a solution of proteins --- primarily albumin --- dissolved in water. Each protein molecule is folded into a specific three-dimensional shape held together by a network of intermolecular forces: \textbf{hydrogen bonds} between N--H and \ce{C=O} groups, \textbf{London dispersion forces} between nonpolar side chains, and \textbf{ionic interactions} between charged side chains (\textbf{PRIM-STR004}, IMF hierarchy).

\textbf{Why it cooks faster at higher temperature}: Using \textbf{PRIM-CHG004} (rate reasoning), heating the egg increases the kinetic energy of the protein molecules. At boiling temperature (\SI{100}{\degreeCelsius}), the molecules vibrate so vigorously that the hydrogen bonds, London forces, and ionic interactions holding the folded shape are disrupted. This process --- protein denaturation --- is a set of chemical changes whose rate increases with temperature according to collision theory: more energetic molecular motions mean more of the intramolecular IMFs are broken per unit time. At room temperature, the proteins retain their folded shape because the molecular motions are not energetic enough to overcome the cumulative IMF network.

\textbf{Why it is irreversible}: Once the proteins unfold, their exposed hydrophobic side chains (which were previously buried in the protein interior, away from water) come into contact with each other. These unfolded chains aggregate --- tangling together through new London dispersion forces and forming a dense, solid network. Using \textbf{PRIM-STR005} (structure-to-property), this aggregated network is the opaque solid you see as cooked egg white. The original folded shapes required a very specific sequence of IMF contacts. The denatured, aggregated mass has settled into a different, thermodynamically stable arrangement with many new intermolecular contacts. Reversing this would require untangling billions of protein chains and refolding each one into its original precise configuration --- an astronomically improbable event without biological machinery (chaperone proteins) to guide the process. Cooling the egg does not restore the original arrangement; it merely solidifies the tangled mass in place.
\end{modelanswer}
