\chapter{Formal Set Theory} \label{ch:set}

%% ===================================================================
%% SET.1: The Language of Set Theory
%% Sources: sth/z/sep (partial, language intro), FORMALIZE from domain spec
%% ===================================================================

\section{The Language of Set Theory} \label{SET.1}

In Chapter~\ref{ch:bst} we developed set-theoretic foundations
informally, treating sets as intuitively given collections of objects.
We now turn to \emph{formal} set theory: a first-order theory in which
every concept is defined from a single binary relation symbol, and
every existence claim is justified by explicit axioms.

\begin{defn}[Set (Formal)] % PRIM-SET001
\label{PRIM-SET001}
In ZFC, the variables of $\Lang{L}_\in$ range over \emph{sets}.
Unlike the naive sets of Chapter~\ref{ch:bst}, formal sets are objects
within a first-order theory: they exist only to the extent that the
ZFC axioms guarantee their existence.
\end{defn}

The formal language of set theory is remarkably austere:

\begin{defn}[Membership] % PRIM-SET002
\label{PRIM-SET002}
The language $\Lang{L}_\in = \{\in\}$ has a single binary relation
symbol~$\in$. All set-theoretic concepts---subset, union, power set,
ordinal, cardinal, function---are defined in terms of $\in$ and the
logical connectives of first-order logic (see SYN.1--SYN.2,
Chapter~\ref{ch:syn}).
\end{defn}

The notation $(\forall x \in A)\phi$ abbreviates $\forall x(x \in A
\lif \phi)$, and $(\exists x \in A)\phi$ abbreviates $\exists x(x \in
A \land \phi)$.

\begin{rem} % PRIM-SET003
\label{PRIM-SET003}
It is sometimes useful to speak of \emph{classes}---collections that
may be ``too large'' to be sets (for instance, the collection of all
ordinals). Proper classes will be introduced in SET.3, where the
Burali-Forti paradox motivates the distinction between sets and
proper classes.
\end{rem}


%% ===================================================================
%% SET.2: ZFC Axioms
%% Sources: sth/story/extensionality (KEEP), sth/z/sep (KEEP),
%%          sth/z/pairs (KEEP), sth/z/union (KEEP),
%%          sth/z/powerset (KEEP), sth/z/infinity-again (KEEP),
%%          sth/ordinals/replacement (KEEP),
%%          sth/spine/foundation (CONDENSE),
%%          sth/z/story (CONDENSE), sth/z/milestone (CONDENSE),
%%          sth/ordinals/zfm (CONDENSE), sth/spine/zf (CONDENSE),
%%          sth/z/arbintersections (CONDENSE),
%%          sth/cardinals/zfc (CONDENSE)
%% ===================================================================

\section{ZFC Axioms} \label{SET.2}

The axioms of ZFC are motivated by the \emph{cumulative-iterative
conception} of set, which rests on three informal principles:
\begin{enumerate}
\item[] \stageshier. Every set is formed at some stage.
\item[] \stagesord. Stages are ordered: some come \emph{before} others.
\item[] \stagesacc. For any stage $S$, and for any sets which were
formed \emph{before} stage~$S$: a set is formed at stage~$S$ whose
members are exactly those sets. Nothing else is formed at stage~$S$.
\end{enumerate}
These informal principles do not constitute a formal theory, but they
serve to justify each axiom we adopt. We will introduce additional
stage-theoretic principles as needed.


%%% -----------------------------------------------------------------
%%% SET.2.1  Extensionality
%%% -----------------------------------------------------------------

\subsection{Extensionality}

The very first thing to say is that sets are individuated by their
elements. More precisely:

\begin{axiom}[Extensionality] % AX-SET001
\label{AX-SET001}
If sets $A$ and $B$ have the same elements, then $A$ and $B$ are
the same set.
\[
  \lforall[A][\lforall[B][(\lforall[x][(x \in A \liff x \in B)] \lif
  \eq[A][B])]]
\]
\end{axiom}

The Axiom of Extensionality expresses the basic idea that a set is
determined by its elements. (So sets might be contrasted with
\emph{concepts}, where precisely the same objects might fall under many
different concepts.)

Why embrace this principle? Well, it is plausible to say that any
denial of Extensionality is a decision to abandon anything which might
even be called \emph{set theory}. Set theory is no more nor less than
the theory of extensional collections.

The real challenge, though, is to lay down principles which tell us
\emph{which sets exist}. And it turns out that the only truly
``obvious'' answer to this question is provably wrong.


%%% -----------------------------------------------------------------
%%% SET.2.2  Separation
%%% -----------------------------------------------------------------

\subsection{Separation}

We start with a principle to replace Naive Comprehension:

\begin{axiom}[Scheme of Separation] % AX-SET006
\label{AX-SET006}
For every formula $\phi(x)$, this is an axiom: for any $A$, the set
$\Setabs{x \in A}{\phi(x)}$ exists.
\end{axiom}

Note that this is not a single axiom. It is a \emph{scheme} of axioms.
There are \emph{infinitely many} Separation axioms; one for every
formula $\phi(x)$. The scheme can equally well be (and normally is)
written down as follows:

\begin{defish}
For any formula $\phi(x)$ which does not contain ``$S$'', this is an
axiom:
\[
	\forall A \exists S \forall x(x \in S \liff (\phi(x) \land x \in A)).
\]
\end{defish}

The formulas~$\phi$ in the Separation axioms may have parameters.

Separation is immediately justified by the cumulative-iterative
conception. To see why, let $A$ be a set. So $A$ is formed by some
stage~$S$ (by \stageshier). Since $A$ was formed at stage~$S$, all of
$A$'s members were formed before stage $S$ (by \stagesacc). Now in
particular, consider all the sets which are members of $A$ and which
also satisfy $\phi$; clearly all of these sets, too, were formed
before stage~$S$. So they are formed into a set $\Setabs{x \in
A}{\phi(x)}$ at stage~$S$ too (by \stagesacc).

Unlike Naive Comprehension, this avoids Russell's Paradox. For we
cannot simply assert the existence of the set $\Setabs{x}{x \notin
x}$. Rather, \emph{given} some set~$A$, we can assert the existence of
the set $R_A = \Setabs{x \in A}{x \notin x}$. But all this proves is
that $R_A \notin R_A$ and $R_A \notin A$, none of which is very
worrying.

However, Separation has an immediate and striking consequence: there is
no \emph{universal} set, i.e., $\Setabs{x}{x = x}$ does not exist.
For if $V$ were a universal set, then by Separation, $R = \Setabs{x
\in V}{x \notin x} = \Setabs{x}{x \notin x}$ would exist,
contradicting Russell's Paradox.

The absence of a universal set---indeed, the open-endedness of the
hierarchy of sets---is one of the most fundamental ideas behind the
cumulative-iterative conception.

Here are a few more consequences of Separation and Extensionality.

\begin{prop} % AX-SET002
\label{AX-SET002}
If any set exists, then $\emptyset$ exists.
\end{prop}

\begin{proof}
If $A$ is a set, $\emptyset = \Setabs{x \in A}{x \neq x}$ exists by Separation.
\end{proof}

\begin{rem}[Empty Set] \label{AX-SET002:rem}
The existence of the empty set is thus \emph{derived} from Separation
(given the existence of at least one set). In many formulations of ZFC,
the Empty Set axiom is listed separately; in our development it follows
from AX-SET006 (Separation, \S\ref{AX-SET006}).
\end{rem}

\begin{rem}[Intersection existence] \label{SET.2:arbint}
Given Separation, if $A \neq \emptyset$, then $\bigcap A = \Setabs{x}{
(\forall y \in A)\, x \in y}$ exists. For let $c \in A$; then $\bigcap
A = \Setabs{x \in c}{(\forall y \in A)\, x \in y}$, which exists by
Separation. Note that $\bigcap \emptyset$ would be the universal set
(vacuously), so the condition $A \neq \emptyset$ is essential. More
generally, definitions of the form $C = \bigcap \Setabs{X}{\phi(X)}$
are justified whenever some witness $S$ with $\phi(S)$ can be exhibited:
one simply defines $C = \Setabs{x \in S}{\forall X(\phi(X) \lif x \in
X)}$ using Separation.
\end{rem}


%%% -----------------------------------------------------------------
%%% SET.2.3  Pairing
%%% -----------------------------------------------------------------

\subsection{Pairing}

\begin{axiom}[Pairs] % AX-SET003
\label{AX-SET003}
For any sets $a, b$, the set $\{a, b\}$ exists.
\[
	\forall a \forall b \exists P \forall x (x \in P \liff (x = a \lor x = b))
\]
\end{axiom}

Here is how to justify this axiom, using the iterative conception.
Suppose $a$ is available at stage $S$, and $b$ is available at stage
$T$. Let $M$ be whichever of stages $S$ and $T$ comes later. Then
since $a$ and $b$ are both available at stage $M$, the set $\{a,b\}$
is a possible collection available at any stage after $M$.

But why assume that there \emph{are} any stages after $M$? If there are
none, then our justification will fail. So, to justify Pairs, we add
another principle to the story:
\begin{enumerate}
	\item[] \stagessucc. There is no last stage.
\end{enumerate}
Even if this principle was not stated explicitly in the story of stages,
it fits well with the basic idea that sets are formed in stages. We
accept it in what follows, and with it, the Axiom of Pairs.

\begin{rem}[Consequences of Pairing] \label{AX-SET003:consequences}
For any sets $a$ and $b$, the following sets exist:
\begin{enumerate}
\item $\{a\}$ (the singleton): by Pairs, $\{a, a\}$ exists, which is
$\{a\}$ by Extensionality.
\item $a \cup b$ (binary union): by Pairs, $\{a, b\}$ exists; now $a
\cup b = \bigcup \{a, b\}$ exists by Union (see below).
\item $\tuple{a, b}$ (the ordered pair): $\{a\}$ exists by~(1); $\{a,
b\}$ by Pairs; so $\{\{a\}, \{a, b\}\} = \tuple{a, b}$ exists by Pairs
again.
\end{enumerate}
\end{rem}


%%% -----------------------------------------------------------------
%%% SET.2.4  Union
%%% -----------------------------------------------------------------

\subsection{Union}

\begin{axiom}[Union] % AX-SET004
\label{AX-SET004}
For any set $A$, the set $\bigcup A = \Setabs{x}{(\exists b \in A)\, x
\in b}$ exists.
\[
	\forall A \exists U \forall x(x \in U \liff (\exists b \in A)x \in b)
\]
\end{axiom}

This axiom is also justified by the cumulative-iterative conception.
Let $A$ be a set, so $A$ is formed at some stage $S$ (by \stageshier).
Every member of $A$ was formed \emph{before} $S$ (by \stagesacc); so,
reasoning similarly, every member of every member of $A$ was formed
before $S$. Thus all of \emph{those} sets are available before $S$, to
be formed into a set at $S$. And that set is just $\bigcup A$.


%%% -----------------------------------------------------------------
%%% SET.2.5  Power Set
%%% -----------------------------------------------------------------

\subsection{Power Set}

\begin{axiom}[Power Set] % AX-SET005
\label{AX-SET005}
For any set $A$, the set $\Pow{A} = \Setabs{x}{x \subseteq A}$ exists.
\[
	\forall A \exists P \forall x(x \in P \liff (\forall z \in x)z \in A)
\]
\end{axiom}

Our justification is straightforward. Suppose $A$ is formed at stage
$S$. Then all of $A$'s members were available before $S$ (by
\stagesacc). So, reasoning as in our justification for Separation,
every subset of $A$ is formed by stage $S$. So they are all available,
to be formed into a single set, at any stage after $S$. And we know
that there is some such stage, since $S$ is not the last stage (by
\stagessucc). So $\Pow{A}$ exists.

\begin{rem}[Cartesian products] \label{AX-SET005:products}
Given any sets $A, B$, their Cartesian product $A \times B$ exists.
This follows because $\Pow{\Pow{A \cup B}}$ exists by Power Set, and
$A \times B$ can be carved out by Separation: $A \times B = \Setabs{z
\in \Pow{\Pow{A \cup B}}}{(\exists x \in A)(\exists y \in B)\, z =
\tuple{x, y}}$.
\end{rem}


%%% -----------------------------------------------------------------
%%% SET.2.6  Infinity
%%% -----------------------------------------------------------------

\subsection{Infinity}

We already have enough axioms to ensure that there are infinitely many
sets (if there are any). For suppose some set exists, and so
$\emptyset$ exists (by Proposition~\ref{AX-SET002}). Now for any
set~$x$, the set $x \cup \{x\}$ exists by
Remark~\ref{AX-SET003:consequences}. So, applying this a few times, we
obtain sets as follows:
\begin{enumerate}
	\item[0.] $\emptyset$
	\item[1.] $\{\emptyset\}$
	\item[2.] $\{\emptyset, \{\emptyset\}\}$
	\item[3.] $\{\emptyset, \{\emptyset\}, \{\emptyset, \{\emptyset\}\}\}$
	\item[4.] $\{\emptyset, \{\emptyset\}, \{\emptyset, \{\emptyset\}\},
	\{\emptyset, \{\emptyset\}, \{\emptyset, \{\emptyset\}\}\}\}$
\end{enumerate}
and we can check that each of these sets is distinct. It is not hard to
verify that the set labelled ``$n$'' has exactly $n$ members, and
(intuitively) is formed at the $n$th stage.

But this gives us \emph{infinitely many} sets without guaranteeing that
there is an \emph{infinite set}, i.e., a set with infinitely many
members. And this really matters: unless we can find a (Dedekind)
infinite set, we cannot construct a Dedekind algebra to serve as the
natural numbers (compare PRIM-BST012, Dedekind algebra,
Chapter~\ref{ch:bst}).

The axioms we have laid down so far do \emph{not} guarantee the
existence of any infinite set. So we lay down a new axiom:

\begin{axiom}[Infinity] % Infinity axiom (no separate AX-SET ID in lean-outline; cf.\ AX-SET009 for Choice)
\label{SET.2:infinity}
There is a set, $I$, such that $\emptyset \in I$ and $x \cup \{x\} \in
I$ whenever $x \in I$.
\begin{align*}
	\exists I( & (\exists o \in I)\forall x\ x \notin o \land {}\\
	& (\forall x \in I)(\exists s \in I)\forall z(z \in s \liff (z \in x \lor z = x)))
\end{align*}
\end{axiom}

It is easy to see that the set $I$ given to us by the Axiom of Infinity
is Dedekind infinite. Its distinguished element is $\emptyset$, and the
injection on $I$ is given by $s(x) = x \cup \{x\}$.

\begin{defn}[$\omega$] % defines omega
\label{SET.2:defnomega}
Let $I$ be any set given to us by the Axiom of Infinity. Let $s$ be the
function $s(x) = x \cup \{x\}$. Let $\omega =
\closureofunder{s}{\emptyset}$. We call the members of $\omega$ the
\emph{natural numbers}, and say that $n$ is the result of $n$-many
applications of $s$ to $\emptyset$.
\end{defn}

To justify the Axiom of Infinity, we add another principle:
\begin{enumerate}
	\item[] \stagesinf. There is an infinite stage. That is, there is a
	stage which (a) is not the first stage, (b) has some stages before
	it, but (c) has no immediate predecessor.
\end{enumerate}
The Axiom of Infinity follows: natural number $n$ is formed at stage
$n$, so $\omega$ is formed at the first infinite stage. Unlike
\stagessucc, the principle \stagesinf{} is not ``forced upon us'' by
the iterative conception---it seems perfectly coherent to think that the
stages are ordered like the natural numbers. We simply accept
\stagesinf{} in what follows.


%%% -----------------------------------------------------------------
%%% Milestone: Z^-
%%% -----------------------------------------------------------------

\begin{rem}[$\Zminus$: A Milestone] \label{SET.2:zminus}
The theory $\Zminus$ has these axioms: Extensionality, Union, Pairs,
Power Set, Infinity, and all instances of the Separation scheme.
The name stands for \emph{Zermelo} set theory (minus Foundation, which
we will come to below). Zermelo essentially formulated this theory in
1908. $\Zminus$ is powerful enough to carry out an enormous amount of
mathematics; in particular, the naive set-theoretic constructions of
Chapter~\ref{ch:bst} can be made rigorous within~$\Zminus$.
\end{rem}


%%% -----------------------------------------------------------------
%%% SET.2.7  Replacement
%%% -----------------------------------------------------------------

\subsection{Replacement}

In order to prove that every well-ordering is isomorphic to some ordinal
(a key result for ordinal theory in SET.3), we need a new axiom that
goes beyond the power of~$\Zminus$.

\begin{axiom}[Scheme of Replacement] % AX-SET007
\label{AX-SET007}
For any formula $\phi(x, y)$, the following is an axiom:
\begin{quote}
	for any $A$, if $(\forall x \in A)\lexists![y][\phi(x,y)]$, then
	$\Setabs{y}{(\exists x \in A)\phi(x,y)}$ exists.
\end{quote}
\end{axiom}
\noindent
As with Separation, this is a scheme: it yields infinitely many axioms,
for each of the infinitely many different $\phi$'s. It can equally well
be written thus:

\begin{defish}
For any formula $\phi(x,y)$ which does not contain ``$B$'', the
following is an axiom:
\[
\forall A[(\forall x \in A)\lexists![y][\phi(x,y)] \lif \exists B\forall y (y \in B \liff (\exists x \in A)\phi(x,y))]
\]
\end{defish}

On first encounter this is quite a tangled formula. The following quick
consequence gives a \emph{clearer} expression to the intuitive idea:
for any term $\tau(x)$ and any set $A$, the set
$\Setabs{\tau(x)}{x \in A} = \Setabs{y}{(\exists x \in A)\, y =
\tau(x)}$ exists. This is because $\tau$ is a term, so $\forall x
\lexists![y][\tau(x) = y]$. Thus ``Replacement'' is a good name: given
a set $A$, you can form a new set $\Setabs{\tau(x)}{x \in A}$ by
replacing every member of $A$ with its image under~$\tau$. Following
the notation for the image of a set under a function, we might write
$\funimage{\tau}{A}$ for $\Setabs{\tau(x)}{x \in A}$.

Crucially, $\tau$ is a \emph{term}. It need not be a \emph{function}
in the sense of a set of ordered pairs. If $f$ is a function (in that
sense), then $\funimage{f}{A}$ is just a subset of $\ran{f}$, already
guaranteed to exist by the axioms of~$\Zminus$. Replacement, by
contrast, is a \emph{powerful} addition to our axioms.


%%% -----------------------------------------------------------------
%%% Milestone: ZF^-
%%% -----------------------------------------------------------------

\begin{rem}[$\ZFminus$: A Milestone] \label{SET.2:zfminus}
The theory $\ZFminus$ adds all instances of the Replacement scheme
to~$\Zminus$. The name stands for \emph{Zermelo--Fraenkel} set theory
(minus Foundation). Fraenkel is credited with the formulation of
Replacement in 1922, although the first precise formulation was due to
Skolem in the same year.
\end{rem}


%%% -----------------------------------------------------------------
%%% SET.2.8  Foundation
%%% -----------------------------------------------------------------

\subsection{Foundation}

We are \emph{almost} done---but not \emph{quite}---because nothing in
$\ZFminus$ guarantees that \emph{every} set is in some $V_\alpha$,
i.e., that every set is formed at some stage.

There is a fairly straightforward sense in which we don't \emph{care}
whether there are sets outside the hierarchy (if there are any, we can
simply ignore them). But we have motivated our \emph{concept} of set
with the thought that every set is formed at some stage (see
\stageshier). So we preclude the possibility of sets falling outside the
hierarchy by adding a new axiom.

Since the $V_\alpha$s are our stages, we might simply consider adding
Regularity as an axiom:

\begin{defish}
\emph{Regularity.} $\forall A \exists \alpha\, A \subseteq V_\alpha$
\end{defish}

This would be perfectly reasonable. However, for technical reasons we
instead adopt an alternative formulation:

\begin{axiom}[Foundation] % AX-SET008
\label{AX-SET008}
$(\forall A \neq \emptyset)(\exists B \in A)\, A \cap B = \emptyset$.
\end{axiom}

The connection between Foundation and Regularity requires some work.
The key notion is the \emph{transitive closure}:

\begin{defn}[Transitive Closure] % transitive closure trcl(A)
\label{SET.2:trcl}
For each set $A$, let:
\begin{align*}
	\text{cl}_0(A) &= A,\\
	\text{cl}_{n+1}(A) &= \bigcup \text{cl}_n(A),\\
	\trcl{A} &= \bigcup_{n < \omega} \text{cl}_{n}(A).
\end{align*}
We call $\trcl{A}$ the \emph{transitive closure} of $A$.
\end{defn}

One can show that $A \subseteq \trcl{A}$ and that $\trcl{A}$ is a
transitive set. Using Foundation and the transitive closure, one proves:

\begin{thm}[Foundation entails Regularity] % THM: zfentailsregularity
\label{SET.2:zfentailsregularity}
Regularity holds in $\ZFminus + \text{Foundation}$.
\end{thm}

\begin{proof}[Proof sketch]
Fix $A$. Since $A \subseteq \trcl{A}$ and $\trcl{A}$ is transitive, it
suffices to show that every transitive set is contained in some
$V_\alpha$. Let $A$ be transitive, and suppose for contradiction that
the set $D = \Setabs{x \in A}{\forall \delta\; x \nsubseteq V_\delta}$
is non-empty. By Foundation, there is some $B \in D$ with $D \cap B =
\emptyset$. Since $A$ is transitive, every element of $B$ is in $A$ but
not in $D$, so every element of $B$ is contained in some $V_\delta$.
Collecting these witnesses yields $B \subseteq V_\beta$ for some
ordinal~$\beta$, contradicting $B \in D$.
\end{proof}

\begin{rem}[Foundation--Regularity equivalence] \label{SET.2:found-reg}
In $\ZFminus$, Foundation and Regularity are equivalent. The converse
direction (Regularity implies Foundation) is established using the
notion of the rank of a set (see \S\ref{SET.5}). Given $\ZFminus$, we can
justify Foundation by noting that it is equivalent to Regularity, and
Regularity follows immediately from \stageshier.

The reason we take Foundation rather than Regularity as our official
axiom is that Foundation can be stated without using the
$V_\alpha$-hierarchy. The definition of the $V_\alpha$s relies on
Transfinite Recursion, whose proof employs Replacement. So while
Foundation and Regularity are equivalent modulo $\ZFminus$, they are
\emph{not} equivalent modulo $\Zminus$; indeed, both $\Zminus$ and $\Z$
are too weak to define the $V_\alpha$s, so Regularity (as formulated
above) does not even make \emph{sense} in~$\Z$.
\end{rem}


%%% -----------------------------------------------------------------
%%% Milestones: Z, ZF
%%% -----------------------------------------------------------------

\begin{rem}[$\Z$ and $\ZF$: A Milestone] \label{SET.2:z-zf}
The theory $\Z$ adds Foundation to $\Zminus$. Its axioms are:
Extensionality, Union, Pairs, Power Set, Infinity, Foundation, and all
instances of the Separation scheme. The theory $\ZF$ adds Foundation to
$\ZFminus$; equivalently, $\ZF$ adds all instances of Replacement
to~$\Z$. From now on we work in $\ZF$ (unless otherwise stated).
\end{rem}


%%% -----------------------------------------------------------------
%%% Milestone: ZFC
%%% -----------------------------------------------------------------

\begin{rem}[$\ZFC$: Final Milestone] \label{SET.2:zfc}
The theory $\ZFC$ adds Well-Ordering to $\ZF$. Its axioms are:
Extensionality, Union, Pairs, Power Set, Infinity, Foundation,
Well-Ordering, and all instances of the Separation and Replacement
schemes. The name stands for \emph{Zermelo--Fraenkel with Choice},
because Well-Ordering turns out to be equivalent (modulo $\ZF$) to the
Axiom of Choice. The Well-Ordering principle (AX-SET009) is stated in
SET.4 after the ordinal theory needed to define cardinals, and proven
equivalent to Choice in SET.6.
\end{rem}


%% ===================================================================
%% SET.3: Ordinals
%% Sources: sth/ordinals/wo (KEEP), sth/ordinals/iso (CONDENSE),
%%          sth/ordinals/vn (KEEP), sth/ordinals/basic (KEEP),
%%          sth/ordinals/ordtype (KEEP), sth/ordinals/opps (KEEP),
%%          sth/spine/recursion (KEEP), sth/choice/hartogs (CONDENSE)
%% ===================================================================

\section{Ordinals} \label{SET.3}

In SET.2 we postulated that there is an infinite stage of the
hierarchy (\stagesinf). Given \stagessucc, the stages do not stop
there: at the next stage after the first infinite stage, we form all
possible collections of sets available at the first infinite stage; and
repeat; and repeat. Implicitly, we have invoked a notion of number
that extends \emph{beyond} the natural numbers---the notion of a
\emph{transfinite ordinal}. The aim of this section is to make that
idea rigorous: we define well-orderings, introduce von Neumann's
ordinals, prove their fundamental properties (including transfinite
induction and the Burali-Forti paradox), and develop the machinery of
transfinite recursion.


%%% -----------------------------------------------------------------
%%% SET.3.1  Well-Orderings
%%% -----------------------------------------------------------------

\subsection{Well-Orderings}

\begin{defn}[Well-Ordering] % DEF-SET009
\label{DEF-SET009}
The relation $<$ \emph{well-orders} $A$ iff it meets these two
conditions:
\begin{enumerate}
	\item $<$ is connected, i.e., for all $a, b \in A$, either $a < b$
	or $a = b$ or $b < a$;
	\item every non-empty subset of $A$ has a $<$-minimal element,
	i.e., if $\emptyset \neq X \subseteq A$ then $(\exists m \in
	X)(\forall z \in X)\, z \nless m$.
\end{enumerate}
\end{defn}

\begin{prop} \label{SET.3:wo:strictorder}
If $<$ well-orders $A$, then every non-empty subset of $A$ has a unique
$<$-least member, and $<$ is irreflexive, asymmetric and transitive.
\end{prop}

\begin{proof}
If $X$ is a non-empty subset of $A$, it has a $<$-minimal element
$m$, i.e., $(\forall z \in X)\, z \nless m$. Since $<$ is connected,
$(\forall z \in X)\, m \leq z$. So $m$ is the $<$-least element of $X$.

For irreflexivity, fix $a \in A$; the $<$-least element of $\{a\}$ is
$a$, so $a \nless a$. For transitivity, if $a < b < c$, then since
$\{a, b, c\}$ has a $<$-least element, $a < c$. Asymmetry follows from
irreflexivity and transitivity.
\end{proof}

\begin{prop}[Well-Ordering Induction] \label{SET.3:propwoinduction}
If $<$ well-orders $A$, then for any formula $\phi(x)$:
\[
	\text{if }(\forall a \in A)((\forall b < a)\phi(b) \lif
		\phi(a))\text{, then }(\forall a \in A)\phi(a).
\]
\end{prop}

\begin{proof}
Suppose $\lnot(\forall a \in A)\phi(a)$, i.e., $X = \Setabs{x \in
A}{\lnot\phi(x)} \neq \emptyset$. Then $X$ has a $<$-minimal element,
$a$. So $(\forall b < a)\phi(b)$ but $\lnot \phi(a)$.
\end{proof}

This last property should remind the reader of the principle of strong
induction on the naturals: if $(\forall n \in \omega)((\forall m <
n)\phi(m) \lif \phi(n))$, then $(\forall n \in \omega)\phi(n)$. It is
this property that makes well-ordering such a \emph{robust} notion.


%%% -----------------------------------------------------------------
%%% SET.3.2  Order-Isomorphisms
%%% -----------------------------------------------------------------

\subsection{Order-Isomorphisms}

\begin{defn}[Order-Isomorphism] \label{SET.3:deforderiso}
A \emph{well-ordering} is a pair $\tuple{A, <}$ such that $<$
well-orders $A$. The well-orderings $\tuple{A, <}$ and $\tuple{B,
\lessdot}$ are \emph{order-isomorphic} iff there is a bijection $f
\colon A \to B$ such that $x < y$ iff $f(x) \lessdot f(y)$. In this
case, we write $\ordeq{\tuple{A, <}}{\tuple{B, \lessdot}}$, and say
that $f$ is an \emph{order-isomorphism} (or simply an
\emph{isomorphism}).
\end{defn}

\begin{defn}[Initial Segment] \label{SET.3:definitseg}
When $\tuple{A, <}$ is a well-ordering with $a \in A$, let $A_a =
\Setabs{x \in A}{x < a}$. We say that $A_a$ is a proper \emph{initial
segment} of $A$. Let $<_a$ be the restriction of $<$ to $A_a^2$.
\end{defn}

\begin{lem} \label{SET.3:wellordnotinitial}
If $\tuple{A, <}$ is a well-ordering with $a \in A$, then
$\ordneq{\tuple{A, <}}{\tuple{A_a, <_a}}$.
\end{lem}

\begin{proof}
For reductio, suppose $f \colon A \to A_a$ is an isomorphism. Since $f$
is a bijection and $A_a \subsetneq A$, let $b \in A$ be the $<$-least
element such that $b \neq f(b)$. One shows that $(\forall x \in A)(x <
b \liff x < f(b))$, from which $b = f(b)$ by the extensionality of
strict linear orders, completing the reductio.
\end{proof}

\begin{lem} \label{SET.3:wellordinitialsegment}
Let $\tuple{A, <}$ and $\tuple{B, \lessdot}$ be well-orderings. If $f
\colon A \to B$ is an isomorphism and $a \in A$, then
$\funrestrictionto{f}{A_{a}} : A_a \to B_{f(a)}$ is an isomorphism.
\end{lem}

\begin{lem} \label{SET.3:lemordsegments}
Let $\tuple{A, <}$ and $\tuple{B, \lessdot}$ be well-orderings. If
$\ordeq{\tuple{A_{a_1}, <_{a_1}}}{\tuple{B_{b_1}, \lessdot_{b_1}}}$
and $\ordeq{\tuple{A_{{a_2}}, <_{a_2}}}{\tuple{B_{{b_2}},
\lessdot_{b_2}}}$, then ${a_1} < {a_2}$ iff ${b_1} \lessdot {b_2}$.
\end{lem}

\begin{proof}[Proof sketch]
If $a_1 < a_2$, then $A_{a_1} \subsetneq A_{a_2}$.
The isomorphism $A_{a_2} \cong B_{b_2}$ restricts
(by Lemma~\ref{SET.3:wellordinitialsegment}) to an isomorphism
$A_{a_1} \cong (B_{b_2})_{b'}$ for some $b' \lessdot b_2$.
Since isomorphisms between well-orderings are unique
(by Lemma~\ref{SET.3:wellordnotinitial}), $b' = b_1$,
giving $b_1 \lessdot b_2$.  The converse is symmetric.
\end{proof}

\begin{thm}[Comparability of Well-Orderings] \label{SET.3:woalwayscomparable}
Given any two well-orderings, one is isomorphic to an initial segment
(not necessarily proper) of the other.
\end{thm}

\begin{proof}[Proof sketch]
Let $\tuple{A, <}$ and $\tuple{B, \lessdot}$ be well-orderings. Using
Separation, let
\[
	f = \Setabs{\tuple{a, b} \in A \times B}{
		\ordeq{\tuple{A_a, <_a}}{\tuple{B_b, \lessdot_b}}}.
\]
By Lemma~\ref{SET.3:lemordsegments}, $f$ preserves order. One shows
that $\dom{f}$ is an initial segment of $A$ and $\ran{f}$ is an initial
segment of $B$. If both were \emph{proper} initial segments, say
$\dom{f} = A_a$ and $\ran{f} = B_b$, then $f \colon A_a \to B_b$ would
be an isomorphism, forcing $\tuple{a, b} \in f$---a contradiction.
\end{proof}


%%% -----------------------------------------------------------------
%%% SET.3.3  Von Neumann's Ordinals
%%% -----------------------------------------------------------------

\subsection{Von Neumann's Ordinals}

Theorem~\ref{SET.3:woalwayscomparable} gives rise to a thought: we could
introduce certain objects, called \emph{order types}, to go proxy for
the well-orderings. We would hope to secure:
\begin{align*}
	\ordtype{A, <} = \ordtype{B, \lessdot} &
	\text{ iff } \ordeq{\tuple{A, <}}{\tuple{B, \lessdot}}\\
	\ordtype{A, <} < \ordtype{B, \lessdot} &
	\text{ iff }\ordeq{\tuple{A, <}}{\tuple{B_b, \lessdot_b}}\text{ for some }b \in B
\end{align*}
The most common way to achieve this---and the approach we follow---is
to define order types via certain \emph{canonical} well-ordered sets,
first introduced by von Neumann:

\begin{defn}[Transitive Set] % DEF-SET002
\label{DEF-SET002}
The set $A$ is \emph{transitive} iff $(\forall x \in A)\, x \subseteq A$.
\end{defn}

\begin{defn}[Ordinal] % DEF-SET001
\label{DEF-SET001}
$A$ is an \emph{ordinal} iff $A$ is transitive and well-ordered by
$\in$.
\end{defn}

In what follows, we use Greek letters for ordinals. It follows
immediately from the definition that if $\alpha$ is an ordinal, then
$\tuple{\alpha, \in_\alpha}$ is a well-ordering, where $\in_\alpha =
\Setabs{\tuple{x, y} \in \alpha^2}{x \in y}$. So, abusing notation, we
can say that $\alpha$ \emph{itself} is a well-ordering.

Here are our first few ordinals:
\[
	\emptyset, \quad \{\emptyset\}, \quad
	\{\emptyset, \{\emptyset\}\}, \quad
	\{\emptyset, \{\emptyset\}, \{\emptyset, \{\emptyset\}\}\}, \quad \ldots
\]
These are exactly the sets that appeared in our Axiom of Infinity, i.e.,
in the definition of $\omega$ (Definition~\ref{SET.2:defnomega}). This
is no coincidence: von Neumann's construction treats natural numbers as
ordinals, but allows for transfinite ordinals too.


%%% -----------------------------------------------------------------
%%% SET.3.4  Basic Properties of the Ordinals
%%% -----------------------------------------------------------------

\subsection{Basic Properties of the Ordinals}

\begin{lem} \label{SET.3:ordmemberord}
Every element of an ordinal is an ordinal.
\end{lem}

\begin{proof}
Let $\alpha$ be an ordinal with $b \in \alpha$. Since $\alpha$ is
transitive, $b \subseteq \alpha$. So $\in$ well-orders $b$ as $\in$
well-orders $\alpha$.

To see that $b$ is transitive, suppose $x \in c \in b$. So $c \in
\alpha$ as $b \subseteq \alpha$. Again, as $\alpha$ is transitive, $c
\subseteq \alpha$, so that $x \in \alpha$. So $x, c, b \in \alpha$.
Since $\in$ well-orders $\alpha$, $\in$ is transitive on $\alpha$ by
Proposition~\ref{SET.3:wo:strictorder}. Hence $x \in c \in b$ gives $x
\in b$. Generalising, $c \subseteq b$.
\end{proof}

\begin{thm}[Transfinite Induction] % DEF-SET005
\label{DEF-SET005}
For any formula $\phi(x)$:
\[
	\text{if }\exists \alpha \phi(\alpha)\text{, then }\exists \alpha(\phi(\alpha)
	\land  (\forall \beta \in \alpha) \lnot \phi(\beta))
\]
where the displayed quantifiers are implicitly restricted to ordinals.
\end{thm}

\begin{proof}
Suppose $\phi(\alpha)$, for some ordinal $\alpha$. If $(\forall \beta
\in \alpha) \lnot \phi(\beta)$, then we are done. Otherwise, as
$\alpha$ is an ordinal, it has some $\in$-least element which is
$\phi$, and this is an ordinal by Lemma~\ref{SET.3:ordmemberord}.
\end{proof}

We can equally express Transfinite Induction as the scheme:
\[
\text{if }\forall \alpha((\forall \beta \in \alpha)\phi(\beta) \lif
\phi(\alpha))\text{, then }\forall \alpha\phi(\alpha).
\]

\begin{thm}[Trichotomy] \label{SET.3:ordtrichotomy}
$\alpha \in \beta \lor \alpha = \beta \lor \beta \in \alpha$, for any
ordinals $\alpha$ and $\beta$.
\end{thm}

\begin{proof}
The proof is by double induction, using Transfinite Induction
(Theorem~\ref{DEF-SET005}) twice. Say that $x$ is \emph{comparable}
with $y$ iff $x \in y \lor x = y \lor y \in x$.

For induction, suppose that every ordinal in $\alpha$ is comparable with
\emph{every} ordinal. For further induction, suppose that $\alpha$ is
comparable with every ordinal in $\beta$. We show that $\alpha$ is
comparable with $\beta$. By induction on $\beta$, it follows that
$\alpha$ is comparable with every ordinal; and by induction on $\alpha$,
\emph{every} ordinal is comparable with \emph{every} ordinal. It
suffices to assume $\alpha \notin \beta$ and $\beta \notin \alpha$, and
show $\alpha = \beta$.

To show $\alpha \subseteq \beta$: fix $\gamma \in \alpha$; this is an
ordinal by Lemma~\ref{SET.3:ordmemberord}. By the first induction
hypothesis, $\gamma$ is comparable with $\beta$. But if $\gamma = \beta$
or $\beta \in \gamma$, then $\beta \in \alpha$ (using transitivity of
$\alpha$ if necessary), contrary to assumption; so $\gamma \in \beta$.
Similar reasoning shows $\beta \subseteq \alpha$. So $\alpha = \beta$.
\end{proof}

We sometimes write $\alpha < \beta$ rather than $\alpha \in \beta$,
since $\in$ behaves as an ordering relation on the ordinals.

\begin{cor} \label{SET.3:corordtransitiveord}
$A$ is an ordinal iff $A$ is a transitive set of ordinals.
\end{cor}

\begin{proof}
\emph{Left-to-right.} By Lemma~\ref{SET.3:ordmemberord}.
\emph{Right-to-left.} If $A$ is a transitive set of ordinals, then
$\in$ well-orders $A$ by Transfinite Induction
(Theorem~\ref{DEF-SET005}) and Trichotomy
(Theorem~\ref{SET.3:ordtrichotomy}).
\end{proof}

Now, we glossed Theorems~\ref{DEF-SET005} and~\ref{SET.3:ordtrichotomy}
as telling us that $\in$ well-orders the ordinals. However, we must be
cautious, thanks to the following result:

\begin{thm}[Burali-Forti Paradox] % PRIM-SET003
\label{SET.3:buraliforti}
There is no set of all the ordinals.
\end{thm}

\begin{proof}
For reductio, suppose $O$ is the set of all ordinals. If $\alpha \in
\beta \in O$, then $\alpha$ is an ordinal by
Lemma~\ref{SET.3:ordmemberord}, so $\alpha \in O$. So $O$ is
transitive, and hence $O$ is an ordinal by
Corollary~\ref{SET.3:corordtransitiveord}. Hence $O \in O$,
contradicting irreflexivity (Proposition~\ref{SET.3:wo:strictorder}).
\end{proof}

The Burali-Forti paradox shows that the ordinals form a \emph{proper
class} (see Remark~\ref{PRIM-SET003}). Ordinals are sets which are
individually well-ordered by membership, and collectively well-ordered
by membership, without collectively constituting a set.


%%% -----------------------------------------------------------------
%%% SET.3.5  Ordinals as Order-Types
%%% -----------------------------------------------------------------

\subsection{Ordinals as Order-Types}

Armed with Replacement (AX-SET007, \S\ref{AX-SET007}), and so now
working in $\ZFminus$, we can prove the key representation theorem:

\begin{thm}[Ordinal Representation] \label{SET.3:thmOrdinalRepresentation}
Every well-ordering is isomorphic to a unique ordinal.
\end{thm}

\begin{proof}
Let $\tuple{A, <}$ be a well-ordering. By
Theorem~\ref{SET.3:ordtrichotomy} and
Lemma~\ref{SET.3:wellordnotinitial}, it is isomorphic to at most one
ordinal. For reductio, suppose it is not isomorphic to \emph{any}
ordinal. ``Make $\tuple{A, <}$ as small as possible'': if some proper
initial segment $\tuple{A_a, <_a}$ is not isomorphic to any ordinal,
let $a$ be least with that property and set $B = A_a$; otherwise let $B
= A$.

By construction, every proper initial segment of $B$ is isomorphic to
some (unique) ordinal. By Replacement, the following set exists and is a
function:
\[
	f = \Setabs{\tuple{\beta, b}}{b \in B\text{ and }
	\ordeq{\beta}{\tuple{B_b, \lessdot_b}}}
\]
Clearly $\ran{f} = B$. By Lemma~\ref{SET.3:lemordsegments}, $f$
preserves ordering. To show $\dom{f}$ is an ordinal, by
Corollary~\ref{SET.3:corordtransitiveord} it suffices to show that
$\dom{f}$ is transitive: if $\beta \in \dom{f}$, i.e.,
$\ordeq{\beta}{\tuple{B_b, \lessdot_b}}$ for some $b$, and $\gamma \in
\beta$, then $\gamma \in \dom{f}$ by
Lemma~\ref{SET.3:wellordinitialsegment}; so $\beta \subseteq \dom{f}$.
This is a contradiction.
\end{proof}

This licenses the following definition:

\begin{defn}[Order Type] \label{SET.3:defordtype}
If $\tuple{A, <}$ is a well-ordering, then its \emph{order type},
$\ordtype{A, <}$, is the unique ordinal $\alpha$ such that
$\ordeq{\tuple{A, <}}{\alpha}$.
\end{defn}

\begin{cor} \label{SET.3:ordtypesworklikeyouwant}
Where $\tuple{A, <}$ and $\tuple{B, \lessdot}$ are well-orderings:
\begin{align*}
	\ordtype{A, <} = \ordtype{B, \lessdot}&\text{ iff }\ordeq{\tuple{A, <}}{\tuple{B, \lessdot}}\\
	\ordtype{A, <} \in \ordtype{B, \lessdot}&\text{ iff }\ordeq{\tuple{A, <}}{\tuple{B_b, \lessdot_b}}\text{ for some }b \in B
\end{align*}
\end{cor}

\begin{proof}
The first claim holds by Trichotomy and
Lemma~\ref{SET.3:wellordnotinitial}. For the second, let $\ordtype{A,
<} = \alpha$ and $\ordtype{B, \lessdot} = \beta$, and let $f \colon
\beta \to \tuple{B, \lessdot}$ be an isomorphism. Then:
\begin{align*}
	\alpha \in \beta &\text{ iff }\funrestrictionto{f}{\alpha} \colon
	\alpha \to B_{f(\alpha)}\text{ is an isomorphism}\\
	&\text{ iff }\ordeq{\tuple{A, <}}{\tuple{B_{f(\alpha)},
	\lessdot_{f(\alpha)}}}\\
	&\text{ iff }\ordeq{\tuple{A, <}}{\tuple{B_b, \lessdot_b}}\text{
	for some $b \in B$}
\end{align*}
by Lemmas~\ref{SET.3:wellordinitialsegment}
and~\ref{SET.3:lemordsegments}.
\end{proof}


%%% -----------------------------------------------------------------
%%% SET.3.6  Successor and Limit Ordinals
%%% -----------------------------------------------------------------

\subsection{Successor and Limit Ordinals}

\begin{defn}[Successor and Limit Ordinal] % DEF-SET003, DEF-SET004
\label{DEF-SET003}
\label{DEF-SET004}
For any ordinal $\alpha$, its \emph{successor} is $\ordsucc{\alpha} =
\alpha \cup \{\alpha\}$. We say that $\alpha$ is a \emph{successor}
ordinal if $\ordsucc{\beta} = \alpha$ for some ordinal $\beta$. We say
that $\alpha$ is a \emph{limit} ordinal iff $\alpha$ is neither empty
nor a successor ordinal.
\end{defn}

\begin{prop} \label{SET.3:succprops}
For any ordinal $\alpha$: (1) $\alpha \in \ordsucc{\alpha}$;
(2) $\ordsucc{\alpha}$ is an ordinal; (3) there is no ordinal $\beta$
such that $\alpha \in \beta \in \ordsucc{\alpha}$.
\end{prop}

\begin{proof}
Trivially, $\alpha \in \alpha \cup \{\alpha\} = \ordsucc{\alpha}$.
Equally, $\ordsucc{\alpha}$ is a transitive set of ordinals, hence an
ordinal by Corollary~\ref{SET.3:corordtransitiveord}. And $\alpha \in
\beta \in \ordsucc{\alpha}$ is impossible, since then either $\beta \in
\alpha$ or $\beta = \alpha$, contradicting irreflexivity.
\end{proof}

\begin{thm}[Simple Transfinite Induction] \label{SET.3:simpletransrecursion}
Let $\phi(x)$ be a formula such that: (1) $\phi(\emptyset)$;
(2) for any ordinal $\alpha$, if $\phi(\alpha)$ then
$\phi(\ordsucc{\alpha})$; and (3) if $\alpha$ is a limit ordinal and
$(\forall \beta \in \alpha)\phi(\beta)$, then $\phi(\alpha)$. Then
$\forall \alpha\, \phi(\alpha)$.
\end{thm}

\begin{proof}
Suppose there is some ordinal which is $\lnot\phi$; let $\gamma$ be the
least such ordinal. Then either $\gamma = \emptyset$, or $\gamma =
\ordsucc{\alpha}$ for some $\alpha$ such that $\phi(\alpha)$, or
$\gamma$ is a limit ordinal and $(\forall \beta \in
\gamma)\phi(\beta)$. In each case, we obtain a contradiction.
\end{proof}

\begin{defn}[Least Strict Upper Bound] \label{SET.3:defsupstrict}
If $X$ is a set of ordinals, then $\supstrict(X) = \bigcup_{\alpha \in
X} \ordsucc{\alpha}$.
\end{defn}

\begin{prop}
If $X$ is a set of ordinals, $\supstrict(X)$ is the least ordinal
greater than every ordinal in $X$.
\end{prop}

\begin{proof}
Let $Y = \Setabs{\ordsucc{\alpha}}{\alpha \in X}$, so $\supstrict(X) =
\bigcup Y$. Since ordinals are transitive and every element of an
ordinal is an ordinal, $\supstrict(X)$ is a transitive set of ordinals,
hence an ordinal by Corollary~\ref{SET.3:corordtransitiveord}.

If $\alpha \in X$, then $\ordsucc{\alpha} \in Y$, so $\ordsucc{\alpha}
\subseteq \bigcup Y = \supstrict(X)$, hence $\alpha \in \supstrict(X)$.
So $\supstrict(X)$ is strictly greater than every ordinal in $X$.
Conversely, if $\alpha \in \supstrict(X)$, then $\alpha \in
\ordsucc{\beta} \in Y$ for some $\beta \in X$, so $\alpha \leq \beta
\in X$. So $\supstrict(X)$ is the \emph{least} strict upper bound on
$X$.
\end{proof}


%%% -----------------------------------------------------------------
%%% SET.3.7  Transfinite Recursion
%%% -----------------------------------------------------------------

\subsection{Transfinite Recursion}

The overarching moral of this subsection is that Transfinite Induction
plus Replacement guarantee the legitimacy of several versions of
transfinite recursion.

\begin{defn}[$\alpha$-Approximation] \label{SET.3:defapprox}
Let $\tau(x)$ be a term; let $f$ be a function; let $\alpha$ be an
ordinal. We say that $f$ is an \emph{$\alpha$-approximation} for $\tau$
iff $\dom{f} = \alpha$ and $(\forall \beta \in \alpha)\, f(\beta) =
\tau(\funrestrictionto{f}{\beta})$.
\end{defn}

\begin{lem}[Bounded Recursion] \label{SET.3:transrecursionfun}
For any term $\tau(x)$ and any ordinal $\alpha$, there is a unique
$\alpha$-approximation for $\tau$.
\end{lem}

\begin{proof}
We first establish uniqueness. Let $g$ and $h$ be $\gamma$- and
$\delta$-approximations, respectively. A transfinite induction on their
arguments shows $g(\beta) = h(\beta)$ for any $\beta \in \dom{g} \cap
\dom{h} = \min(\gamma, \delta)$. So approximations are unique (if they
exist) and agree on all values.

To establish existence, we use Simple Transfinite Induction
(Theorem~\ref{SET.3:simpletransrecursion}) on ordinals $\delta \leq
\alpha$.

The empty function is trivially an $\emptyset$-approximation.

If $g$ is a $\gamma$-approximation, then $g \cup \{\tuple{\gamma,
\tau(g)}\}$ is a $\ordsucc{\gamma}$-approximation.

If $\gamma$ is a limit ordinal and $g_\delta$ is a
$\delta$-approximation for all $\delta < \gamma$, let $g =
\bigcup_{\delta \in \gamma} g_\delta$. This is a function since the
various $g_\delta$s agree on all values. And if $\delta \in \gamma$ then
$g(\delta) = g_{\ordsucc{\delta}}(\delta) =
\tau(\funrestrictionto{g_{\ordsucc{\delta}}}{\delta}) =
\tau(\funrestrictionto{g}{\delta})$.
\end{proof}

\begin{thm}[General Recursion] % DEF-SET006, THM-SET002
\label{DEF-SET006}
\label{THM-SET002}
For any term $\tau(x)$, we can explicitly define a term $\sigma(x)$
such that $\sigma(\alpha) = \tau(\funrestrictionto{\sigma}{\alpha})$
for any ordinal $\alpha$.
\end{thm}

\begin{proof}
For each $\alpha$, by Lemma~\ref{SET.3:transrecursionfun} there is a
unique $\alpha$-approximation, $f_\alpha$, for $\tau$. Define
$\sigma(\alpha)$ as $f_{\ordsucc{\alpha}}(\alpha)$. Then:
\begin{align*}
	\sigma(\alpha) &=
	f_{\ordsucc{\alpha}}(\alpha) \\&=
	\tau(\funrestrictionto{f_{\ordsucc{\alpha}}}{\alpha}) \\&=
	\tau(\Setabs{\tuple{\beta, f_{\ordsucc{\alpha}}(\beta)}}{\beta \in \alpha}) \\&=
	\tau(\Setabs{\tuple{\beta, f_{\ordsucc{\beta}}(\beta)}}{\beta \in \alpha}) \\&=
	\tau(\funrestrictionto{\sigma}{\alpha})
\end{align*}
noting that $f_{\ordsucc{\beta}}(\beta) =
f_{\ordsucc{\alpha}}(\beta)$ for all $\beta < \alpha$, as in
Lemma~\ref{SET.3:transrecursionfun}.
\end{proof}

Note that Theorem~\ref{DEF-SET006} is a \emph{schema}. Crucially, we
cannot expect $\sigma$ to define a function (i.e., a set), since then
$\dom{\sigma}$ would be the set of all ordinals, contradicting
Burali-Forti (Theorem~\ref{SET.3:buraliforti}).

\begin{thm}[Simple Recursion] \label{SET.3:simplerecursionschema}
For any terms $\tau(x)$ and $\theta(x)$ and any set $A$, we can
explicitly define a term $\sigma(x)$ such that:
\begin{align*}
	\sigma(\emptyset) &= A\\
	\sigma(\ordsucc{\alpha}) &= \tau(\sigma(\alpha)) &&
		\text{for any ordinal }\alpha\\
	\sigma(\alpha) &= \theta(\ran{\funrestrictionto{\sigma}{\alpha}})&&
	\text{when }\alpha\text{ is a limit ordinal}
\end{align*}
\end{thm}

\begin{proof}
Define a term $\xi(x)$ by:
\[
	\xi(x) =
	\begin{cases}
		A & \text{if $x$ is not a function whose}\\
		  & \text{\quad domain is an ordinal; otherwise:}\\
		\tau(x(\alpha)) & \text{if $\dom{x} = \ordsucc{\alpha}$}\\
		\theta(\ran{x}) & \text{if $\dom{x}$ is a limit ordinal}
	\end{cases}
\]
By Theorem~\ref{DEF-SET006}, there is a term $\sigma(x)$ such that
$\sigma(\alpha) = \xi(\funrestrictionto{\sigma}{\alpha})$ for every
ordinal $\alpha$; moreover, $\funrestrictionto{\sigma}{\alpha}$ is a
function with domain $\alpha$. A Simple Transfinite Induction
(Theorem~\ref{SET.3:simpletransrecursion}) confirms:
$\sigma(\emptyset) = \xi(\emptyset) = A$;
$\sigma(\ordsucc{\alpha}) =
\xi(\funrestrictionto{\sigma}{\ordsucc{\alpha}}) =
\tau(\sigma(\alpha))$; and when $\alpha$ is a limit,
$\sigma(\alpha) = \xi(\funrestrictionto{\sigma}{\alpha}) =
\theta(\ran{\funrestrictionto{\sigma}{\alpha}})$.
\end{proof}


%%% -----------------------------------------------------------------
%%% SET.3.8  Hartogs' Lemma
%%% -----------------------------------------------------------------

\subsection{Hartogs' Lemma}

\begin{lem}[Hartogs' Lemma, in $\ZF$] \label{SET.3:HartogsLemma}
For any set $A$, there is an ordinal $\alpha$ such that
$\cardnless{\alpha}{A}$.
\end{lem}

\begin{proof}[Proof sketch]
Using Separation, consider:
\[
	C = \Setabs{\tuple{B, R}}{B \subseteq A
	\text{ and $\tuple{B, R}$ is a well-ordering}}.
\]
Using Replacement and
Theorem~\ref{SET.3:thmOrdinalRepresentation}, form
$\alpha = \Setabs{\ordtype{B, R}}{\tuple{B, R} \in C}$.
By Corollary~\ref{SET.3:corordtransitiveord}, $\alpha$ is an ordinal
(it is a transitive set of ordinals). If there were an injection $f
\colon \alpha \to A$, then $\alpha = \ordtype{\ran{f}, R}$ for a
suitable $R$, giving $\alpha \in \alpha$---a contradiction.
\end{proof}

\begin{rem}[Hartogs' Number] \label{SET.3:hartogsrem}
The ordinal $\alpha$ produced in the proof is called the \emph{Hartogs
number} of $A$, sometimes written $\aleph(A)$. It is the least ordinal
that does not inject into $A$.
\end{rem}


%% ===================================================================
%% SET.4: Cardinals
%% Sources: sth/cardinals/cp (CONDENSE), sth/cardinals/cardsasords (KEEP),
%%          sth/cardinals/classing (CONDENSE),
%%          sth/choice/tarskiscott (CONDENSE),
%%          sth/cardinals/alephs (CONDENSE if exists)
%% ===================================================================

\section{Cardinals} \label{SET.4}

In SET.3, we introduced ordinals to calibrate \emph{well-orderings}.
Two well-orderings have the same order type iff they are isomorphic. We
now turn to a simpler notion: the \emph{size} of a set. Two sets have
the same size iff they are equinumerous, i.e., iff there is a bijection
between them (see Chapter~\ref{ch:bst}). Just as we introduced ordinals
to calibrate order types, we now introduce \emph{cardinals} to
calibrate size. Writing $\card{X}$ for the cardinality of $X$, we want
cardinals to satisfy \emph{Cantor's Principle}:
\[
	\card{A} = \card{B} \text{ iff } \cardeq{A}{B}.
\]


%%% -----------------------------------------------------------------
%%% SET.4.1  Cardinals as Ordinals
%%% -----------------------------------------------------------------

\subsection{Cardinals as Ordinals}

Our theory of cardinals makes shameless use of our theory of ordinals:
we define cardinals as certain specific ordinals.

\begin{defn}[Cardinal Number] % DEF-SET007, DEF-SET008
\label{DEF-SET007}
\label{DEF-SET008}
If $A$ can be well-ordered, then $\card{A}$ is the least ordinal
$\gamma$ such that $\cardeq{A}{\gamma}$. For any ordinal $\gamma$, we
say that $\gamma$ is a \emph{cardinal} iff $\gamma = \card{\gamma}$.
\end{defn}

There is a snag with Definition~\ref{DEF-SET007}: we would like
$\card{A}$ to exist for \emph{every} set $A$, but the definition begins
with a conditional---``if $A$ can be well-ordered''. If some set cannot
be well-ordered, the definition fails to define $\card{A}$. So we need a
guarantee that every set can be well-ordered:

\begin{axiom}[Well-Ordering] % AX-SET009
\label{AX-SET009}
Every set can be well-ordered.
\end{axiom}

This guarantee is unavailable in $\ZF$ alone. The Well-Ordering axiom
is the final axiom of $\ZFC$ (see Remark~\ref{SET.2:zfc}). Its
equivalence to the Axiom of Choice will be established in SET.6
(Theorem~\ref{THM-SET001}).

Using Well-Ordering, it is straightforward to show that cardinals exist
and behave well:

\begin{lem} \label{SET.4:CardinalsExist}
For every set $A$: (1) $\card{A}$ exists and is unique;
(2) $\cardeq{\card{A}}{A}$; (3) $\card{A}$ is a cardinal, i.e.,
$\card{A} = \card{\card{A}}$.
\end{lem}

\begin{proof}
Fix $A$. By Well-Ordering (AX-SET009), there is a well-ordering
$\tuple{A, R}$. By Theorem~\ref{SET.3:thmOrdinalRepresentation},
$\tuple{A, R}$ is isomorphic to a unique ordinal $\beta$, so
$\cardeq{A}{\beta}$. By Transfinite Induction, there is a uniquely
least ordinal $\gamma$ such that $\cardeq{A}{\gamma}$. So $\card{A} =
\gamma$, establishing (1) and (2). For (3), if $\delta \in \gamma$ then
$\cardless{\delta}{A}$ by choice of $\gamma$, so also
$\cardless{\delta}{\gamma}$ since equinumerosity is an equivalence
relation. So $\gamma = \card{\gamma}$.
\end{proof}

\begin{lem} \label{SET.4:CardinalsBehaveRight}
For any sets $A$ and $B$:
\begin{align*}
	\cardeq{A}{B} &\text{ iff } \card{A} = \card{B}\\
	\cardle{A}{B} &\text{ iff } \card{A} \leq \card{B}\\
	\cardless{A}{B}&\text{ iff } \card{A} < \card{B}
\end{align*}
\end{lem}

\begin{proof}
We prove left-to-right of the second claim; the other cases are
similar. If $\cardle{A}{B}$, there is an injection $A \to B$. By
Lemma~\ref{SET.4:CardinalsExist}, there are bijections $\card{A} \to A$
and $B \to \card{B}$. Composing, we obtain an injection $\card{A} \to
\card{B}$, giving $\card{A} \leq \card{B}$.
\end{proof}

\begin{rem}[Cantor's Principle] \label{SET.4:cantorprinciple}
Lemma~\ref{SET.4:CardinalsBehaveRight} guarantees Cantor's Principle:
$\card{A} = \card{B}$ iff $\cardeq{A}{B}$. It also yields a quick
re-proof of Schr\"oder--Bernstein: if $\cardle{A}{B}$ and
$\cardle{B}{A}$ then $\card{A} \leq \card{B}$ and $\card{B} \leq
\card{A}$, so $\card{A} = \card{B}$ by Trichotomy. (This implicitly
uses Replacement and Well-Ordering; the proof in Chapter~\ref{ch:bst}
works in $\Zminus$.)
\end{rem}


%%% -----------------------------------------------------------------
%%% SET.4.2  Finite, Infinite, and Uncountable Cardinals
%%% -----------------------------------------------------------------

\subsection{Finite, Infinite, and Uncountable Cardinals}

\begin{defn}[Finite and Infinite Sets] \label{SET.4:defnfinite}
We say that $A$ is \emph{finite} iff $\card{A} \in \omega$, i.e.,
$\card{A}$ is a natural number. Otherwise, $A$ is \emph{infinite}.
\end{defn}

Note that this definition assumes $\ZFC$, since we need Well-Ordering
to guarantee $\card{A}$ exists. Without Well-Ordering, there can be
sets that are neither finite nor Dedekind infinite (see SET.6 for the
role of Choice).

\begin{cor} \label{SET.4:omegaisacardinal}
$\omega$ is the least infinite cardinal.
\end{cor}

\begin{proof}
$\omega$ is a cardinal: it is Dedekind infinite, and if
$\cardeq{\omega}{n}$ for any $n \in \omega$, then $n$ would be Dedekind
infinite, a contradiction. Now $\omega$ is the least infinite cardinal
by definition, since the finite cardinals are exactly the natural
numbers.
\end{proof}

\begin{thm} \label{SET.4:NoLargestCardinal}
There is no largest cardinal.
\end{thm}

\begin{proof}[Proof sketch]
For any cardinal $\cardfont{a}$, Cantor's Theorem (THM-BST001,
Chapter~\ref{ch:bst}) gives $\cardless{\cardfont{a}}{\Pow{\cardfont{a}}}$.
By Lemma~\ref{SET.4:CardinalsExist}, $\cardfont{a} <
\card{\Pow{\cardfont{a}}}$.
\end{proof}

\begin{prop} \label{SET.4:unioncardinalscardinal}
If every member of $X$ is a cardinal, then $\bigcup X$ is a cardinal.
\end{prop}

\begin{proof}
It is easy to check that $\bigcup X$ is an ordinal. If $\alpha \in
\bigcup X$, then $\alpha \in \cardfont{b} \in X$ for some cardinal
$\cardfont{b}$. Since $\cardfont{b}$ is a cardinal,
$\cardless{\alpha}{\cardfont{b}}$. Since $\cardfont{b} \subseteq
\bigcup X$, we have $\cardle{\cardfont{b}}{\bigcup X}$, and so
$\cardneq{\alpha}{\bigcup X}$. Generalising, $\bigcup X$ is a cardinal.
\end{proof}


%%% -----------------------------------------------------------------
%%% SET.4.3  Tarski--Scott Cardinals (Without Well-Ordering)
%%% -----------------------------------------------------------------

\subsection{Cardinals without Well-Ordering}

\begin{rem}[Tarski--Scott Trick] \label{SET.4:tarskiscott}
Cardinals can be developed \emph{without} Well-Ordering using the
\emph{Tarski--Scott trick}: for any formula $\phi(x)$, let $[x :
\phi(x)]$ be the set of all $x$ of least possible rank such that
$\phi(x)$, where the \emph{rank} of a set is its position in the
cumulative hierarchy $V_0 \subset V_1 \subset \cdots$
(see \S\ref{SET.5}).
Working in $\ZF$, one defines the \textsc{ts}-cardinality of
$A$ as $\text{tsc}(A) = [x : \cardeq{A}{x}]$. This yields a
well-behaved notion of cardinality without assuming Well-Ordering, at
the cost of defining cardinals as \emph{sets of sets} rather than
ordinals.
\end{rem}


%% ===================================================================
%% SET.5: Advanced Topics
%% Sources: sth/spine/valpha (KEEP), sth/spine/recursion (KEEP),
%%          sth/spine/Valphabasic, rank, separation, height (CONDENSE),
%%          sth/ord-arithmetic/addition, mult, expo, using-addition (CONDENSE),
%%          sth/card-arithmetic/opps (KEEP), ch (KEEP),
%%          sth/card-arithmetic/simp, expotough, fix (CONDENSE),
%%          sth/replacement/strength, ref, refproofs, finiteaxiom (CONDENSE),
%%          sth/choice/countablechoice, justifications (CONDENSE)
%% ===================================================================

\section{Advanced Topics} \label{SET.5}

We now develop the major advanced topics of formal set theory: the von
Neumann hierarchy $V_\alpha$, ordinal arithmetic, cardinal arithmetic,
the Continuum Hypothesis, and the role of the Axiom of Choice.


%%% -----------------------------------------------------------------
%%% SET.5.1  The Cumulative Hierarchy
%%% -----------------------------------------------------------------

\subsection{The Cumulative Hierarchy}

With the machinery of transfinite recursion in hand, we can give a
formal, \emph{internal} characterisation of the stages of the hierarchy.

\begin{defn}[Von Neumann Hierarchy] % DEF-SET012
\label{DEF-SET012}
\begin{align*}
	V_\emptyset &\defis \emptyset\\
	V_{\ordsucc{\alpha}} &\defis \Pow{V_\alpha} &&
	\text{for any ordinal }\alpha\\
	V_{\alpha} &\defis \bigcup_{\gamma < \alpha} V_\gamma &&
	\text{when }\alpha\text{ is a limit ordinal}
\end{align*}
\end{defn}

This definition is legitimate by Simple Recursion
(Theorem~\ref{SET.3:simplerecursionschema}): take $A = \emptyset$,
$\tau(x) = \Pow{x}$, and $\theta(x) = \bigcup x$.

\begin{lem} \label{SET.5:Valphabasicprops}
For each ordinal $\alpha$: (1) $V_\alpha$ is transitive; (2) $V_\alpha$
is potent (if $x \subseteq y \in V_\alpha$ then $x \in V_\alpha$);
(3) if $\gamma \in \alpha$, then $V_\gamma \in V_\alpha$ (and hence
$V_\gamma \subseteq V_\alpha$).
\end{lem}

\begin{proof}
By simultaneous transfinite induction on $\alpha$. The case $\alpha =
\emptyset$ is trivial. For successor $\alpha = \ordsucc{\beta}$:
(3) if $\gamma \in \alpha$ then $V_\gamma \subseteq V_\beta$ by
hypothesis, so $V_\gamma \in \Pow{V_\beta} = V_\alpha$; (2) if $A
\subseteq B \in V_\alpha$ then $A \subseteq V_\beta$, so $A \in
V_\alpha$; (1) if $x \in A \in V_\alpha$ then $x \in V_\beta$ and $x
\subseteq V_\beta$ by the induction hypothesis, so $x \in V_\alpha$.
For limit $\alpha$: (3) if $\gamma \in \alpha$ then $V_\gamma \in
V_{\ordsucc{\gamma}} \subseteq V_\alpha$; (1) and (2) hold because a
union of transitive (resp.\ potent) sets is transitive (resp.\ potent).
\end{proof}

\begin{defn}[Rank] \label{SET.5:defnsetrank}
For each set $A$, $\setrank{A}$ is the least ordinal $\alpha$ such that
$A \subseteq V_\alpha$.
\end{defn}

Ranks exist by Foundation (AX-SET008, \S\ref{AX-SET008}) and
Regularity (Theorem~\ref{SET.2:zfentailsregularity}).

\begin{prop} \label{SET.5:rankmemberslower}
If $B \in A$, then $\setrank{B} \in \setrank{A}$.
\end{prop}

\begin{thm}[$\in$-Induction Scheme] \label{SET.5:eininduction}
For any formula $\phi$:
\[
	\forall A((\forall x \in A)\phi(x) \lif \phi(A)) \lif \forall A\, \phi(A).
\]
\end{thm}

\begin{proof}
Suppose $\lnot\forall A\, \phi(A)$. By Transfinite Induction, there is
some $A$ of least rank with $\lnot\phi(A)$. If $x \in A$ then
$\setrank{x} \in \setrank{A}$ by Proposition~\ref{SET.5:rankmemberslower},
so $\phi(x)$. Hence $(\forall x \in A)\phi(x)$ and $\lnot\phi(A)$.
\end{proof}

\begin{prop} \label{SET.5:ordsetrankalpha}
$\setrank{\alpha} = \alpha$ for any ordinal $\alpha$.
\end{prop}


%%% -----------------------------------------------------------------
%%% SET.5.2  Ordinal Arithmetic
%%% -----------------------------------------------------------------

\subsection{Ordinal Arithmetic}

We define addition, multiplication, and exponentiation on ordinals. Each
can be defined synthetically (via an explicit well-ordered set and its
order type) or recursively (via transfinite recursion equations). Both
approaches yield the same result by
Theorem~\ref{SET.3:thmOrdinalRepresentation}.

\begin{defn}[Ordinal Addition] \label{SET.5:defordplus}
The \emph{disjoint sum} of $A$ and $B$ is $A \disjointsum B = (A \times
\{0\}) \cup (B \times \{1\})$. Define the \emph{reverse lexicographic
ordering} $\rlexless$ on $\alpha \disjointsum \beta$ by:
$\tuple{\alpha_1, \alpha_2} \rlexless \tuple{\beta_1, \beta_2}$ iff
either $\alpha_2 \in \beta_2$, or both $\alpha_2 = \beta_2$ and
$\alpha_1 \in \beta_1$. Then $\alpha \ordplus \beta =
\ordtype{\alpha \disjointsum \beta, \rlexless}$.
\end{defn}

Ordinal addition satisfies the following recursion equations:
\begin{align*}
	\alpha \ordplus 0 &= \alpha\\
	\alpha \ordplus (\beta \ordplus 1) &= (\alpha \ordplus \beta) \ordplus 1\\
	\alpha \ordplus \beta &= \supstrict_{\delta < \beta}(\alpha \ordplus \delta)
	&& \text{if $\beta$ is a limit ordinal}
\end{align*}

Addition is associative: $\alpha \ordplus (\beta \ordplus \gamma) =
(\alpha \ordplus \beta) \ordplus \gamma$. It is \emph{not} commutative:
$1 \ordplus \omega = \omega < \omega \ordplus 1$. Intuitively, placing
one element \emph{before} an $\omega$-sequence does not change the order
type (by a Hilbert's Hotel argument), but placing one element
\emph{after} it does.

\begin{defn}[Ordinal Multiplication] \label{SET.5:defordtimes}
$\alpha \ordtimes \beta = \ordtype{\alpha \times \beta, \rlexless}$.
Equivalently, by transfinite recursion:
\begin{align*}
	\alpha \ordtimes 0 &= 0\\
	\alpha \ordtimes (\beta \ordplus 1) &=
		(\alpha \ordtimes \beta) \ordplus \alpha\\
	\alpha \ordtimes \beta &=
		\supstrict_{\delta < \beta}(\alpha \ordtimes \delta) &&
		\text{when $\beta$ is a limit ordinal}
\end{align*}
\end{defn}

Multiplication is associative but \emph{not} commutative: $2 \ordtimes
\omega = \omega < \omega \ordtimes 2$. It distributes over addition from
the right: $\alpha \ordtimes (\beta \ordplus \gamma) = (\alpha
\ordtimes \beta) \ordplus (\alpha \ordtimes \gamma)$.

\begin{defn}[Ordinal Exponentiation] \label{SET.5:defordexpo}
By transfinite recursion:
\begin{align*}
	\ordexpo{\alpha}{0} &= 1\\
	\ordexpo{\alpha}{\beta \ordplus 1} &= \ordexpo{\alpha}{\beta}
	\ordtimes \alpha\\
	\ordexpo{\alpha}{\beta} &= \bigcup_{\delta < \beta}
	\ordexpo{\alpha}{\delta} && \text{when $\beta$ is a limit ordinal}
\end{align*}
\end{defn}

Ordinal exponentiation does not commute: $\ordexpo{2}{\omega} =
\bigcup_{n < \omega} \ordexpo{2}{n} = \omega$, whereas
$\ordexpo{\omega}{2} = \omega \ordtimes \omega$.

\begin{lem}[Characterisation of Infinite Ordinals]
\label{SET.5:ordinfinitycharacter}
For any ordinal $\alpha$, the following are equivalent:
(1) $\alpha \notin \omega$; (2) $\omega \leq \alpha$;
(3) $1 \ordplus \alpha = \alpha$; (4) $\alpha$ and $\alpha \ordplus 1$
are equinumerous; (5) $\alpha$ is Dedekind infinite.
\end{lem}

\begin{proof}
$(1) \Rightarrow (2)$: By Trichotomy.
$(2) \Rightarrow (3)$: Write $\alpha = \beta \ordplus \gamma$ where
$\beta$ is a limit ordinal and $\gamma$ is least. Then $1 \ordplus
\alpha = (1 \ordplus \beta) \ordplus \gamma = \beta \ordplus \gamma =
\alpha$, using the fact that $1 \ordplus \beta =
\supstrict_{\delta < \beta}(1 \ordplus \delta) = \beta$.
$(3) \Rightarrow (4)$: There is a bijection $\alpha \disjointsum 1 \to
1 \disjointsum \alpha$; compose with the isomorphism $1 \disjointsum
\alpha \to \alpha$.
$(4) \Rightarrow (5)$: A bijection $\alpha \disjointsum 1 \to \alpha$
restricts to an injection $\alpha \to \alpha$ that is not surjective.
$(5) \Rightarrow (1)$: No natural number is Dedekind infinite.
\end{proof}


%%% -----------------------------------------------------------------
%%% SET.5.3  Cardinal Arithmetic
%%% -----------------------------------------------------------------

\subsection{Cardinal Arithmetic}

Since we do not need to keep track of order, cardinal arithmetic is
rather easier to define than ordinal arithmetic.

\begin{defn}[Cardinal Operations] \label{SET.5:defcardops}
When $\cardfont{a}$ and $\cardfont{b}$ are cardinals:
\begin{align*}
	\cardfont{a} \cardplus \cardfont{b} &\defis
	\card{\cardfont{a} \disjointsum \cardfont{b}}\\
	\cardfont{a} \cardtimes \cardfont{b} &\defis
	\card{\cardfont{a} \times \cardfont{b}}\\
	\cardexpo{\cardfont{a}}{\cardfont{b}} &\defis
	\card{\funfromto{\cardfont{b}}{\cardfont{a}}}
\end{align*}
where $\funfromto{X}{Y} = \Setabs{f}{f\text{ is a function }X \to Y}$.
\end{defn}

Cardinal addition and multiplication are commutative and associative
(unlike their ordinal counterparts).

\begin{lem} \label{SET.5:SizePowerset2Exp}
$\card{\Pow{A}} = \cardexpo{2}{\card{A}}$, for any $A$.
\end{lem}

\begin{proof}
For each $B \subseteq A$, define the characteristic function $\chi_B
\in \funfromto{A}{2}$ by $\chi_B(x) = 1$ if $x \in B$, and
$\chi_B(x) = 0$ otherwise. The map $B \mapsto \chi_B$ is a bijection
$\Pow{A} \to \funfromto{A}{2}$.
\end{proof}

\begin{cor}[Cantor's Theorem in Cardinal Arithmetic] % THM-SET003, DEF-SET011
\label{THM-SET003}\label{DEF-SET011}
$\cardfont{a} < \cardexpo{2}{\cardfont{a}}$ for any
cardinal~$\cardfont{a}$.
\end{cor}

\begin{proof}
From Cantor's Theorem (THM-BST001, Chapter~\ref{ch:bst}) and
Lemma~\ref{SET.5:SizePowerset2Exp}.
\end{proof}

\begin{thm} \label{SET.5:continuumis2aleph0}
$\card{\Real} = \cardexpo{2}{\omega}$.
\end{thm}

\begin{proof}[Proof skeleton]
Show $\cardle{\Pow{\omega}}{\Real}$ and $\cardle{\Real}{\Pow{\omega}}$;
apply Schr\"oder--Bernstein to get $\cardeq{\Real}{\Pow{\omega}}$; then
$\card{\Real} = \card{\Pow{\omega}} = \cardexpo{2}{\omega}$ by
Lemma~\ref{SET.5:SizePowerset2Exp}.
\end{proof}

\begin{thm}[Simplification of Addition and Multiplication]
\label{SET.5:cardplustimesmax}
If $\cardfont{a}, \cardfont{b}$ are infinite cardinals, then
$\cardfont{a} \cardtimes \cardfont{b} = \cardfont{a} \cardplus
\cardfont{b} = \max(\cardfont{a}, \cardfont{b})$.
\end{thm}

\begin{proof}[Proof sketch]
The key step is showing $\cardeq{\alpha}{\alpha \times \alpha}$ for any
infinite ordinal $\alpha$. This uses the \emph{canonical ordering}
$\canonord$ on pairs: $\tuple{\alpha_1, \alpha_2} \canonord
\tuple{\beta_1, \beta_2}$ iff either $\max(\alpha_1, \alpha_2) <
\max(\beta_1, \beta_2)$, or they have the same max and $\alpha_1 <
\beta_1$, or same max and first coordinate and $\alpha_2 < \beta_2$.
This is a well-ordering of $\alpha \times \alpha$. For the least
infinite ordinal $\alpha$ for which $\cardeq{\alpha}{\alpha \times
\alpha}$ fails, one shows $\alpha$ must be a cardinal, and each segment
$\text{Seg}(\gamma_1, \gamma_2)$ in the canonical ordering has
cardinality $< \alpha$, giving $\ordtype{\alpha \times \alpha,
\canonord} \leq \alpha$, hence $\cardle{\alpha \times \alpha}{\alpha}$.
The reverse injection is obvious, so Schr\"oder--Bernstein applies.

Given this, without loss of generality let $\cardfont{a} =
\max(\cardfont{a}, \cardfont{b})$. Then $\cardfont{a} \cardtimes
\cardfont{a} = \cardfont{a} \leq \cardfont{a} \cardplus \cardfont{b}
\leq \cardfont{a} \cardplus \cardfont{a} \leq \cardfont{a} \cardtimes
\cardfont{a}$.
\end{proof}

\begin{prop} \label{SET.5:kappaunionkappasize}
Let $\cardfont{a}$ be an infinite cardinal. For each $\beta \in
\cardfont{a}$, let $X_\beta$ be a set with $\card{X_\beta} \leq
\cardfont{a}$. Then $\card{\bigcup_{\beta \in \cardfont{a}} X_\beta}
\leq \cardfont{a}$.
\end{prop}

\begin{proof}[Proof sketch]
Each $X_\beta$ injects into $\cardfont{a}$, so $\bigcup_\beta X_\beta$
injects into $\cardfont{a} \times \cardfont{a}$; by the preceding
theorem, $\card{\cardfont{a} \times \cardfont{a}} = \cardfont{a}$.
\end{proof}


%%% -----------------------------------------------------------------
%%% SET.5.4  Aleph and Beth Numbers
%%% -----------------------------------------------------------------

\subsection{Aleph and Beth Numbers}

\begin{defn}[Aleph and Beth Numbers] % DEF-SET013
\label{DEF-SET013}
Where $\cardsucc{\cardfont{a}}$ is the least cardinal strictly greater
than $\cardfont{a}$, we define two sequences by transfinite recursion:
\begin{align*}
	\aleph_{0} &\defis \omega &
	\beth_{0} &\defis \omega\\
	\aleph_{\alpha \ordplus 1} &\defis \cardsucc{(\aleph_{\alpha})} &
	\beth_{\alpha+1} &\defis \cardexpo{2}{\beth_{\alpha}}\\
	\aleph_{\alpha} &\defis \bigcup_{\beta < \alpha} \aleph_{\beta} &
	\beth_{\alpha} &\defis \bigcup_{\beta < \alpha}\beth_{\beta}
	& \text{when $\alpha$ is a limit ordinal}
\end{align*}
\end{defn}

The definition of $\cardsucc{\cardfont{a}}$ is in order: for each
cardinal $\cardfont{a}$, there is some cardinal greater than
$\cardfont{a}$ (Theorem~\ref{SET.4:NoLargestCardinal}), and Transfinite
Induction gives the \emph{least} such. The ``$\aleph$'' notation is due
to Cantor; ``$\beth$'' is due to Peirce.

\begin{prop}
$\aleph_\alpha$ and $\beth_\alpha$ are cardinals for every ordinal
$\alpha$. Moreover, every infinite cardinal is an $\aleph$: if
$\cardfont{a}$ is an infinite cardinal, then $\cardfont{a} =
\aleph_\gamma$ for some unique $\gamma$.
\end{prop}

\begin{proof}
By transfinite induction. $\aleph_0 = \beth_0 = \omega$ is a cardinal
by Corollary~\ref{SET.4:omegaisacardinal}; successors are cardinals by
definition; limits are cardinals by
Proposition~\ref{SET.4:unioncardinalscardinal}.

For the second claim, induct on cardinals. If $\cardfont{a}$ is the
successor of some $\cardfont{b} = \aleph_\gamma$, then $\cardfont{a} =
\aleph_{\gamma+1}$. If $\cardfont{a}$ is not a cardinal successor, then
$\cardfont{a} = \bigcup_{\cardfont{b} < \cardfont{a}} \cardfont{b} =
\bigcup_{\cardfont{b} < \cardfont{a}} \aleph_{\gamma_\cardfont{b}} =
\aleph_\gamma$ for a suitable limit $\gamma$.
\end{proof}


%%% -----------------------------------------------------------------
%%% SET.5.5  The Continuum Hypothesis
%%% -----------------------------------------------------------------

\subsection{The Continuum Hypothesis}

Since every infinite cardinal is an $\aleph$, we ask: is every infinite
cardinal a $\beth$? If so, infinite cardinals would ``play
straightforwardly'' with powersets:

\begin{defish}[Generalised Continuum Hypothesis] % DEF-SET015
\label{DEF-SET015}
GCH: $\aleph_\alpha = \beth_\alpha$, for all $\alpha$.
\end{defish}

If GCH held, cardinal exponentiation could be completely determined: for
$\cardfont{b} < \cardfont{a}$, the value $\cardexpo{\cardfont{a}}{\cardfont{b}}$
would be trapped by $\cardfont{a} \leq
\cardexpo{\cardfont{a}}{\cardfont{b}} \leq \cardsucc{\cardfont{a}}$.

But GCH is a \emph{hypothesis}, not a theorem. G\"odel (1938) proved
that if $\ZFC$ is consistent, then so is $\ZFC + \text{GCH}$. The
simplest non-trivial instance is:

\begin{defish}[Continuum Hypothesis] \label{SET.5:CH}
CH: $\aleph_1 = \beth_1$.
\end{defish}

Cohen (1963) proved that if $\ZFC$ is consistent, then so is $\ZFC +
\lnot\text{CH}$. So the Continuum Hypothesis is \emph{independent} from
$\ZFC$.

The Continuum Hypothesis is so-called because ``the continuum'' is
another name for the real line $\Real$.
Theorem~\ref{SET.5:continuumis2aleph0} tells us $\card{\Real} = \beth_1$.
So CH states there is no cardinal between $\aleph_0 = \beth_0$ (the
cardinality of the natural numbers) and $\beth_1$ (the cardinality of
the continuum).

Two observations are worth emphasising. First, it does not immediately
follow from independence that CH is indeterminate in truth value: perhaps
additional natural axioms will settle it. G\"odel himself suggested this
response. Second, the independence of CH is striking but not incredible:
for all $\ZFC$ tells us, moving from a cardinal to its successor may
involve a more refined tool than simply taking powersets.


%%% -----------------------------------------------------------------
%%% SET.5.6  The Strength of Replacement
%%% -----------------------------------------------------------------

\subsection{The Strength of Replacement}

We briefly note the strength of the Replacement scheme.

\begin{thm} \label{SET.5:Znotomegaomega}
$\Z$ is consistent with the non-existence of $\omega + \omega$.
\end{thm}

\begin{proof}[Proof sketch]
Working in $\ZF$, consider $V_{\omega + \omega}$. One can show that for
every axiom $\phi$ of $\Z$, we have $\ZF \vdash \phi^{V_{\omega +
\omega}}$ (where $\phi^M$ restricts all quantifiers to $M$). But $\omega
+ \omega \notin V_{\omega + \omega}$ (since $\setrank{\omega + \omega} =
\omega + \omega$ by Proposition~\ref{SET.5:ordsetrankalpha}). So $\Z$
does not prove the existence of $\omega + \omega$.
\end{proof}

This is why Theorem~\ref{SET.3:thmOrdinalRepresentation} cannot be
proved without Replacement: within $\Z$ one can define a well-ordering
of order type $\omega + \omega$, but if $\omega + \omega$ does not
exist, this well-ordering is not isomorphic to any ordinal.

More broadly, Replacement forces the hierarchy to be very tall.

\begin{thm}[Reflection Schema] \label{SET.5:reflectionschema}
For any formula $\phi$:
\[
\forall \alpha \exists \beta > \alpha\, (\forall x_1, \ldots, x_n \in
V_\beta)(\phi(x_1, \ldots, x_n) \liff \phi^{V_\beta}(x_1, \ldots, x_n))
\]
\end{thm}

Montague (1961) and L\'evy (1960) showed that Replacement and
Reflection are equivalent modulo $\Z$, so adding either gives $\ZF$.

\begin{thm} \label{SET.5:zfnotfinitely}
$\ZF$ is not finitely axiomatizable: if $\mathcal{T}$ is finite and
$\mathcal{T} \vdash \ZF$, then $\mathcal{T}$ is inconsistent.
\end{thm}

\begin{proof}[Proof sketch]
By Reflection, for any finite set of sentences $\mathcal{T} \subseteq
\ZF$ there is a $\beta$ such that $V_\beta \models \mathcal{T}$.
If $\mathcal{T}$ axiomatised all of $\ZF$, then $\ZF$ would prove
``there exists a set model of $\mathcal{T}$,'' hence
$\ZF \vdash \mathrm{Con}(\mathcal{T}) = \mathrm{Con}(\ZF)$.
By the Second Incompleteness Theorem, $\ZF$ is then inconsistent.
\end{proof}


%%% -----------------------------------------------------------------
%%% SET.5.7  Aleph-Fixed Points
%%% -----------------------------------------------------------------

\subsection{Fixed Points}

The following results illustrate just how tall Replacement forces the
hierarchy to be.

\begin{prop}[$\aleph$-Fixed Point] \label{SET.5:alephfixed}
There is a cardinal $\kappa$ such that $\kappa = \aleph_\kappa$.
\end{prop}

\begin{proof}
Define by recursion: $\kappa_0 = 0$; $\kappa_{n+1} =
\aleph_{\kappa_n}$; $\kappa = \bigcup_{n < \omega} \kappa_n$. Then
$\kappa$ is a cardinal by Proposition~\ref{SET.4:unioncardinalscardinal},
and $\kappa = \bigcup_{n < \omega} \kappa_{n+1} = \bigcup_{n < \omega}
\aleph_{\kappa_n} = \bigcup_{\alpha < \kappa} \aleph_\alpha =
\aleph_\kappa$.
\end{proof}

\begin{prop}[$\beth$-Fixed Point] \label{SET.5:bethfixed}
There is a $\kappa$ such that $\kappa = \beth_\kappa$.
\end{prop}

\begin{proof}
As in Proposition~\ref{SET.5:alephfixed}, using ``$\beth$'' in place of
``$\aleph$''.
\end{proof}

\begin{prop} \label{SET.5:stagesize}
$\card{V_{\omega + \alpha}} = \beth_\alpha$. If $\omega \ordtimes \omega
\leq \alpha$, then $\card{V_\alpha} = \beth_\alpha$.
\end{prop}

\begin{proof}[Proof sketch]
By transfinite induction on $\alpha$.  For the base,
$V_\omega$ is countable, so $\card{V_\omega} = \aleph_0 = \beth_0$.
At successor stages,
$\card{V_{\omega+\alpha+1}} = 2^{\card{V_{\omega+\alpha}}}
= 2^{\beth_\alpha} = \beth_{\alpha+1}$.
At limits, take the union.
\end{proof}

\begin{cor}
There is a $\kappa$ such that $\card{V_\kappa} = \kappa$.
\end{cor}

\begin{proof}
Let $\kappa$ be a $\beth$-fixed point (Proposition~\ref{SET.5:bethfixed}).
Then $\card{V_\kappa} = \beth_\kappa = \kappa$ by
Proposition~\ref{SET.5:stagesize}.
\end{proof}

Intuitively, $V_\kappa$ is ``as wide as it is tall'': there are as many
stages beneath it as there are elements of it. This is a
Tristram-Shandy phenomenon: we move from one stage to the next by taking
powersets, making the hierarchy much wider with each step, yet ``in the
end'' the width catches up with the height.


%%% -----------------------------------------------------------------
%%% SET.5.8  Choice and Countable Choice
%%% -----------------------------------------------------------------

\subsection{The Axiom of Choice}

The Axiom of Well-Ordering (AX-SET009, \S\ref{AX-SET009}) has been
essential to our development of cardinal arithmetic. We now discuss its
justification and its relationship to other principles.

\begin{defn}[Choice Function] \label{SET.5:defchoicefun}
A function $f$ is a \emph{choice function} iff $f(x) \in x$ for all $x
\in \dom{f}$. We say $f$ is a choice function \emph{for} $A$ iff
$\dom{f} = A \setminus \{\emptyset\}$.
\end{defn}

\begin{axiom}[Choice] \label{SET.5:axiomchoice}
Every set has a choice function.
\end{axiom}

The equivalence of Choice and Well-Ordering is established in SET.6
(Theorem~\ref{THM-SET001}).

\begin{defn}[Zorn's Lemma] % DEF-SET010
\label{DEF-SET010}
A \emph{chain} in a partially ordered set $P$ is a totally ordered
subset of~$P$.
\emph{Zorn's Lemma} states: if every chain in a partially ordered set
$P$ has an upper bound in $P$, then $P$ has a maximal element.
\end{defn}

Zorn's Lemma is equivalent (in $\ZF$) to both Choice and
Well-Ordering (Theorem~\ref{THM-SET001}).

\begin{rem}[Justification of Choice] \label{SET.5:choicejust}
One intrinsic justification appeals to the cumulative-iterative
conception. Let $A$'s elements be disjoint and non-empty. By
\stageshier, $A$ is formed at some stage $S$. All elements of $\bigcup
A$ are available before $S$. By \stagesacc, every possible collection of
earlier-available sets exists at $S$. It is certainly \emph{possible} to
select one element from each member of $A$---a choice set for $A$. So
such a choice set exists. (For a careful development of this argument,
see Potter (\S14.8).)
\end{rem}

\begin{rem}[Countable Choice] \label{SET.5:countablechoice}
\emph{Countable Choice} states that every countable set has a choice
function. This special case was used frequently by 19th-century
mathematicians (including Dedekind and Cantor) without awareness of its
role. Two notable consequences requiring Countable Choice: (1) every
set is either finite or contains a countably infinite subset (Dedekind);
(2) a countable union of countable sets is countable (Cantor). Both fail
in $\ZF$ alone: Cohen proved it is consistent with $\ZF$ that some sets
are incomparable with $\omega$; Feferman and Levy proved it is
consistent with $\ZF$ that a countable union of countable sets has
cardinality $\beth_1$.
\end{rem}


%% ===================================================================
%% SET.6: Theorems
%% Sources: sth/choice/wellorderingproblem (KEEP),
%%          sth/choice/hartogs (CONDENSE)
%% ===================================================================

\section{Theorems} \label{SET.6}

We close this chapter by establishing the fundamental equivalences that
unify the theory.


%%% -----------------------------------------------------------------
%%% SET.6.1  Well-Ordering iff Choice
%%% -----------------------------------------------------------------

\subsection{Well-Ordering iff Choice}

\begin{thm}[in $\ZF$] % THM-SET001
\label{THM-SET001}
The following are equivalent:
\begin{enumerate}
	\item Well-Ordering (AX-SET009): every set can be well-ordered.
	\item Choice: every set has a choice function.
	\item Zorn's Lemma (DEF-SET010): if every chain in a partially
	ordered set $P$ has an upper bound in $P$, then $P$ has a maximal
	element.
\end{enumerate}
\end{thm}

\begin{proof}
\emph{Well-Ordering $\Rightarrow$ Choice.} Let $A$ be a set of sets.
Then $\bigcup A$ exists by Union, and by Well-Ordering there is some
$<$ which well-orders $\bigcup A$. Define $f(x) = \text{the $<$-least
member of }x$. This is a choice function for $A$.

\emph{Choice $\Rightarrow$ Well-Ordering.} Fix $A$. By Choice, there is
a choice function $f$ for $\Pow{A} \setminus \{\emptyset\}$. Using
Transfinite Recursion (Theorem~\ref{DEF-SET006}), define:
\begin{align*}
	g(0) &= f(A)\\
	g(\alpha) &=
		\begin{cases}
			\text{stop} & \text{if }A = \funimage{g}{\alpha}\\
			f(A \setminus \funimage{g}{\alpha}) & \text{otherwise}
		\end{cases}
\end{align*}
Since $f$ is a choice function, for each $\alpha$ (when defined) we have
$g(\alpha) = f(A \setminus \funimage{g}{\alpha}) \in A \setminus
\funimage{g}{\alpha}$, so $g(\alpha) \notin \funimage{g}{\alpha}$. If
$g(\alpha) = g(\beta)$ then $g(\beta) \notin \funimage{g}{\alpha}$,
i.e., $\beta \notin \alpha$, and similarly $\alpha \notin \beta$. So
$\alpha = \beta$ by Trichotomy, hence $g$ is injective.

We must stop: i.e., there is some least $\alpha$ with $A =
\funimage{g}{\alpha}$. For if not, then as $g$ is injective we would
have $\cardless{\alpha}{\Pow{A} \setminus \{\emptyset\}}$ for every
ordinal $\alpha$, contradicting Hartogs' Lemma
(Lemma~\ref{SET.3:HartogsLemma}).

Assembling these facts, $g$ is a bijection from some ordinal to $A$, and
can be used to well-order $A$.

\emph{Choice $\Leftrightarrow$ Zorn's Lemma.} This equivalence is a
standard result; the proof that Choice implies Zorn's Lemma uses
transfinite recursion to build a maximal chain. The converse constructs
a choice function by applying Zorn's Lemma to a suitable partial order
of partial choice functions.
\end{proof}


%%% -----------------------------------------------------------------
%%% SET.6.2  Comparability of Sets
%%% -----------------------------------------------------------------

\subsection{Comparability of Sets}

\begin{thm}[in $\ZF$] \label{SET.6:WOiffComparability}
The following are equivalent:
\begin{enumerate}
	\item Well-Ordering.
	\item Either $\cardle{A}{B}$ or $\cardle{B}{A}$, for any sets $A$
	and $B$.
\end{enumerate}
\end{thm}

\begin{proof}
\emph{(1) $\Rightarrow$ (2).} Fix $A$ and $B$. By Well-Ordering, there
are well-orderings $\tuple{A, R}$ and $\tuple{B, S}$. Let $f \colon
\alpha \to \tuple{A, R}$ and $g \colon \beta \to \tuple{B, S}$ be
isomorphisms. By Trichotomy, either $\alpha \subseteq \beta$ or $\beta
\subseteq \alpha$. If $\alpha \subseteq \beta$, then $g^{-1} \circ
(\funrestrictionto{f}{\alpha})$ is an injection $A \to B$, so
$\cardle{A}{B}$; similarly in the other case.

\emph{(2) $\Rightarrow$ (1).} Fix $A$; by Hartogs' Lemma
(Lemma~\ref{SET.3:HartogsLemma}) there is an ordinal $\beta$ with
$\cardnless{\beta}{A}$. By (2), $\cardle{A}{\beta}$. An injection $f
\colon A \to \beta$ induces a well-ordering on $A$ via $a_1 < a_2$ iff
$f(a_1) \in f(a_2)$.
\end{proof}

As an immediate consequence: if Well-Ordering fails, then some sets are
\emph{literally incomparable} with regard to their size, and transfinite
cardinal arithmetic becomes significantly more complex.


%%% -----------------------------------------------------------------
%%% Summary
%%% -----------------------------------------------------------------

\begin{rem}[Summary of SET.3--SET.6] \label{SET.6:summary}
Working in $\ZF$, we developed the ordinals (von Neumann's construction)
and proved the key structural theorems: Transfinite Induction
(Theorem~\ref{DEF-SET005}), Trichotomy
(Theorem~\ref{SET.3:ordtrichotomy}), Burali-Forti
(Theorem~\ref{SET.3:buraliforti}), and Ordinal Representation
(Theorem~\ref{SET.3:thmOrdinalRepresentation}). The machinery of
transfinite recursion (Theorems~\ref{DEF-SET006}
and~\ref{SET.3:simplerecursionschema}) enabled the definition of the
cumulative hierarchy $V_\alpha$ (Definition~\ref{DEF-SET012}) and
ordinal arithmetic. Adding the Well-Ordering axiom (AX-SET009,
\S\ref{AX-SET009}) yielded $\ZFC$ and a robust theory of cardinals.
Cardinal arithmetic
(Theorem~\ref{SET.5:cardplustimesmax}) simplifies dramatically for
infinite cardinals; but the Continuum Hypothesis
(Definition~\ref{DEF-SET015}) remains independent of $\ZFC$. The
equivalence of Well-Ordering, Choice, and Zorn's Lemma
(Theorem~\ref{THM-SET001}) unifies the theory.
\end{rem}
