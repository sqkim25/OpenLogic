\chapter{Deduction} \label{ch:ded}

%% ===================================================================
%% DED.1: Generic Proof Theory
%% Sources: axd/rul (GENERALIZE), axd/ptn (GENERALIZE+ABSORB),
%%          axd/prv (ABSORB), axd/ppr (CONDENSE), axd/qpr (CONDENSE),
%%          ntd/rul (EXTRACT), seq/rul (EXTRACT), seq/srl (EXTRACT),
%%          com/mcs (CONDENSE)
%% ===================================================================

\section{Generic Proof Theory} \label{DED.1}

This section develops proof-theoretic concepts at the \emph{generic}
level, independent of any particular proof system.  Each concept
defined here---axiom schema, rule of inference, derivation,
provability, consistency, and so on---is instantiated differently by
the concrete proof systems presented in DED.2--DED.5.  The structural
properties established below (reflexivity, monotonicity, transitivity,
compactness) hold in each system, though the proofs differ in
detail. Canonical proofs are given here for the Hilbert-style
(axiomatic) case; remarks indicate how they adapt to other systems.

%%% -----------------------------------------------------------------
%%% DED.1.1  Axiom Schemas and Rules of Inference
%%% -----------------------------------------------------------------

\subsection{Axiom Schemas and Rules of Inference}

\begin{defn}[Axiom Schema] % PRIM-DED001
\label{PRIM-DED001}
An \emph{axiom schema} is a schematic expression involving metavariables
that stands for a (typically infinite) collection of formulas: every
formula obtained by uniformly replacing the metavariables with concrete
formulas is an \emph{instance} of the schema and counts as a logical
axiom.
\end{defn}

For instance, the schema $!A \lif (!B \lif !A)$ generates infinitely
many axioms---one for each choice of formulas~$!A$ and~$!B$.  Different
proof systems employ different collections of axiom schemas (or none at
all, as in the case of natural deduction).

\begin{defn}[Non-Logical Axiom] % PRIM-DED002
\label{PRIM-DED002}
A \emph{non-logical axiom} is a sentence adopted as a premise of
a theory but not derivable from the logical axioms alone.  A set of
non-logical axioms is sometimes called a set of \emph{proper axioms}
or simply the \emph{axioms of the theory}.  A theory is
\emph{axiomatized} by a set~$\Gamma_0$ of non-logical axioms when the
theory consists of all sentences that follow (semantically or
proof-theoretically) from~$\Gamma_0$.
\end{defn}

\begin{defn}[Rule of Inference] % PRIM-DED003
\label{PRIM-DED003}
A \emph{rule of inference} gives a sufficient condition for what counts
as a correct inference step in a derivation from a set of
assumptions~$\Gamma$.  More precisely, a rule specifies one or more
\emph{premises} and a \emph{conclusion}; a step is correct when each
premise either appears earlier in the derivation, is an axiom, or is
an element of~$\Gamma$.
\end{defn}

The simplest and most ubiquitous rule of inference is \emph{modus ponens}:

\begin{quote}
If $!B \lif !A$ and $!B$ both occur (as axioms, assumptions, or
earlier conclusions) in a derivation, then $!A$ is a correct inference
step.
\end{quote}

What counts as a correct derivation depends on which rules of inference
are admitted and on what is taken as an axiom.  Different proof systems
make different choices:
\begin{itemize}
\item \emph{Axiomatic (Hilbert-style) deduction} (see DED.2) uses
  many axiom schemas and few rules (typically just modus ponens and
  a quantifier rule).
\item \emph{Natural deduction} (see DED.3) uses no logical axioms,
  but has introduction and elimination rules for each connective and
  quantifier.
\item \emph{Sequent calculus} (see DED.4) operates on sequents and
  employs left/right introduction rules together with structural rules.
\item \emph{Tableaux} (see DED.5) work by attempted refutation,
  applying branch-extension rules to signed formulas.
\end{itemize}

%%% -----------------------------------------------------------------
%%% DED.1.2  Proof Systems and Derivations
%%% -----------------------------------------------------------------

\subsection{Proof Systems and Derivations}

\begin{defn}[Proof System] % PRIM-DED004
\label{PRIM-DED004}
A \emph{proof system} is a specification of:
\begin{enumerate}
\item A set of logical axioms (possibly empty, possibly given by schemas);
\item A set of rules of inference;
\item A definition of what constitutes a \emph{derivation}.
\end{enumerate}
A proof system determines a derivability relation $\Gamma \Proves !A$
on formulas.
\end{defn}

\begin{defn}[Derivation] % PRIM-DED005
\label{PRIM-DED005}
Let $\Gamma$ be a set of formulas of~$\Lang L$.  A \emph{derivation}
from~$\Gamma$ is a finite combinatorial object---a sequence of
formulas, a tree of formulas, or a tree of sequents, depending on the
proof system---in which every step is justified either as:
\begin{enumerate}
\item an element of~$\Gamma$ (an assumption); or
\item an instance of a logical axiom; or
\item the conclusion of a correct application of a rule of inference
  to earlier steps.
\end{enumerate}
A derivation \emph{derives} its final formula (its last element, its
root, or the succedent of its root sequent, depending on the system).
\end{defn}

\begin{rem}
The shape of derivations differs across proof systems.  In axiomatic
deduction, a derivation is a finite \emph{sequence} of formulas.  In
natural deduction and the sequent calculus, a derivation is a finite
\emph{tree}.  In tableaux, a derivation is a finitely branching tree
of signed formulas.  Despite these differences, the abstract notion
of derivability is uniform: $\Gamma \Proves !A$ means that there
exists a derivation of~$!A$ from~$\Gamma$ in the given system.
\end{rem}

%%% -----------------------------------------------------------------
%%% DED.1.3  Sequents and Structural Rules
%%% -----------------------------------------------------------------

\subsection{Sequents and Structural Rules}

\begin{defn}[Sequent] % PRIM-DED008
\label{PRIM-DED008}
A \emph{sequent} is an expression of the form
\[
\Gamma \Sequent \Delta
\]
where $\Gamma$ and $\Delta$ are finite (possibly empty) sequences of
sentences of the language~$\Lang L$.  $\Gamma$ is called the
\emph{antecedent} and $\Delta$ is called the \emph{succedent}.
\end{defn}

The intuitive reading of a sequent $\Gamma \Sequent \Delta$ is: if all
of the sentences in the antecedent hold, then at least one of the
sentences in the succedent holds.  That is, if $\Gamma = \tuple{!A_1,
\dots, !A_m}$ and $\Delta = \tuple{!B_1, \dots, !B_n}$, then $\Gamma
\Sequent \Delta$ holds iff
\[
(!A_1 \land \cdots \land !A_m) \lif (!B_1 \lor \cdots \lor !B_n)
\]
holds.  When $\Gamma$ is empty, $\Sequent \Delta$ asserts that
$!B_1 \lor \dots \lor !B_n$ holds.  When $\Delta$ is empty, $\Gamma
\Sequent$ asserts that $\lnot(!A_1 \land \dots \land !A_m)$.

\begin{defn}[Structural Rules] % PRIM-DED007
\label{PRIM-DED007}
The \emph{structural rules} of a proof system govern the manipulation
of the context (the set or sequence of assumptions or side formulas)
without introducing or eliminating any logical connective.  The
principal structural rules are:
\begin{enumerate}
\item \textbf{Weakening} ($\Weakening$): one may add a formula to the
  context without affecting derivability.
  \begin{defish}
  \Axiom$ \Gamma \fCenter \Delta $
  \RightLabel{\LeftR{\Weakening}}
  \UnaryInf$ !A, \Gamma \fCenter \Delta$
  \DisplayProof
  \hfill
  \Axiom$ \Gamma \fCenter \Delta$
  \RightLabel{\RightR{\Weakening}}
  \UnaryInf$ \Gamma \fCenter \Delta, !A$
  \DisplayProof
  \end{defish}

\item \textbf{Contraction} ($\Contraction$): two copies of the same
  formula may be collapsed into one.
  \begin{defish}
  \Axiom$ !A, !A, \Gamma \fCenter \Delta $
  \RightLabel{\LeftR{\Contraction}}
  \UnaryInf$ !A, \Gamma \fCenter \Delta$
  \DisplayProof
  \hfill
  \Axiom$ \Gamma \fCenter \Delta, !A, !A$
  \RightLabel{\RightR{\Contraction}}
  \UnaryInf$ \Gamma \fCenter \Delta, !A$
  \DisplayProof
  \end{defish}

\item \textbf{Exchange} ($\Exchange$): the order of formulas in the
  context may be permuted.
  \begin{defish}
  \Axiom$ \Gamma, !A, !B, \Pi \fCenter \Delta $
  \RightLabel{\LeftR{\Exchange}}
  \UnaryInf$ \Gamma, !B, !A, \Pi \fCenter \Delta$
  \DisplayProof
  \hfill
  \Axiom$ \Gamma \fCenter \Delta, !A, !B, \Lambda$
  \RightLabel{\RightR{\Exchange}}
  \UnaryInf$ \Gamma \fCenter \Delta, !B, !A, \Lambda$
  \DisplayProof
  \end{defish}

\item \textbf{Cut} ($\Cut$): if a formula can be derived on the right
  and consumed on the left, it may be eliminated.
  \begin{defish}
  \[
  \Axiom$ \Gamma \fCenter \Delta, !A$
  \Axiom$ !A, \Pi \fCenter \Lambda $
  \RightLabel{\Cut}
  \BinaryInf$ \Gamma, \Pi \fCenter \Delta, \Lambda$
  \DisplayProof
  \]
  \end{defish}
\end{enumerate}
\end{defn}

\begin{rem}
The structural rules are stated here in their sequent-calculus form,
but analogous phenomena occur in every proof system.  In axiomatic
deduction, weakening corresponds to monotonicity of derivability
(adding unused hypotheses), and cut corresponds to transitivity
(chaining derivations).  In natural deduction, weakening is built into
the assumption mechanism, contraction is implicit in re-use of
assumptions, and cut corresponds to substituting a derivation for an
assumption.  Substructural logics arise by restricting or removing
structural rules: e.g., linear logic drops weakening and contraction;
relevant logic restricts weakening.
\end{rem}

%%% -----------------------------------------------------------------
%%% DED.1.4  Assumption Discharge
%%% -----------------------------------------------------------------

\subsection{Assumption Discharge}

\begin{defn}[Assumption Discharge] % PRIM-DED009
\label{PRIM-DED009}
In natural deduction, an \emph{assumption} is any sentence occupying a
topmost (leaf) position in a derivation tree. Certain rules of
inference---notably $\Intro{\lif}$, $\Intro{\lnot}$, $\Elim{\lor}$,
and $\Elim{\lexists}$---permit one to \emph{discharge} assumptions:
the discharged assumption is cancelled and no longer counts among the
open (undischarged) assumptions of the derivation. The label notation
$\Discharge{!A}{n}$ indicates that the assumption~$!A$ bearing
label~$n$ is discharged by the corresponding inference step.
\end{defn}

Discharging is a permission, not a requirement: one may apply a
discharging rule even when the assumption to be discharged does not
actually occur in the derivation above.  The set of
\emph{undischarged} assumptions of a derivation is the set of
assumptions that have not been cancelled by any rule application.  A
derivation with no undischarged assumptions is a \emph{proof} (of a
theorem).

%%% -----------------------------------------------------------------
%%% DED.1.5  Provability and Theorems
%%% -----------------------------------------------------------------

\subsection{Provability and Theorems}

\begin{defn}[Provability] % PRIM-DED006
\label{PRIM-DED006}
A formula~$!A$ is \emph{derivable} from $\Gamma$, written
$\Gamma \Proves !A$, if there is a derivation from~$\Gamma$ ending
in~$!A$ (in whatever proof system is in force).
\end{defn}

\begin{defn}[Theorem] % PRIM-DED010
\label{PRIM-DED010}
A formula~$!A$ is a \emph{theorem} if there is a derivation
of~$!A$ from the empty set.  We write $\Proves !A$ if $!A$ is a
theorem and $\Proves/ !A$ if it is not.
\end{defn}

%%% -----------------------------------------------------------------
%%% DED.1.6  Structural Properties of Derivability
%%% -----------------------------------------------------------------

\subsection{Structural Properties of Derivability}

Just as we have defined semantic notions (validity, entailment,
satisfiability), we now establish corresponding \emph{proof-theoretic
properties}. These are not defined by appeal to satisfaction in
structures, but by appeal to the derivability or non-derivability of
formulas. It was an important discovery, the content of the
\emph{soundness} and \emph{completeness} theorems (see
CP-001, Soundness, \S\ref{META.1} and CP-002, Completeness, \S\ref{META.2}), that these notions coincide.

\begin{prop}[Reflexivity] % structural property
\label{DED-prop:reflexivity}
If $!A \in \Gamma$, then $\Gamma \Proves !A$.
\end{prop}

\begin{proof}
The formula~$!A$ by itself constitutes a (trivial) derivation of~$!A$
from~$\Gamma$: in axiomatic deduction, it is a one-element sequence
whose sole entry is justified as an element of~$\Gamma$; in natural
deduction, it is a single-node tree (an assumption); in the sequent
calculus, the initial sequent $!A \Sequent !A$ suffices.
\end{proof}

\begin{prop}[Monotonicity] % structural property
\label{DED-prop:monotonicity}
If $\Gamma \subseteq \Delta$ and $\Gamma \Proves !A$, then $\Delta
\Proves !A$.
\end{prop}

\begin{proof}
Any derivation of~$!A$ from~$\Gamma$ is also a derivation of~$!A$
from~$\Delta$, since every element of~$\Gamma$ used in the derivation
is also an element of~$\Delta$.
\end{proof}

\begin{prop}[Transitivity] % structural property
\label{DED-prop:transitivity}
If $\Gamma \Proves !A$ and $\{!A\} \cup \Delta \Proves
!B$, then $\Gamma \cup \Delta \Proves !B$.
\end{prop}

\begin{proof}
Suppose $\{!A\} \cup \Delta \Proves !B$.  Then there is
a derivation $!B_1, \dots, !B_l = !B$ from~$\{!A\} \cup
\Delta$. Some of the steps in that derivation will be correct
because of a rule which refers to a prior line~$!B_i = !A$. By
hypothesis, there is a derivation of~$!A$ from~$\Gamma$, i.e.,
a derivation~$!A_1, \dots, !A_k = !A$ where every $!A_i$ is an
axiom, an element of~$\Gamma$, or correct by a rule of
inference. Now consider the sequence
\[
!A_1, \dots, !A_k = !A, !B_1, \dots, !B_l = !B.
\]
This is a correct derivation of~$!B$ from $\Gamma \cup \Delta$
since every $!B_i = !A$ is now justified by the same rule which
justifies~$!A_k = !A$.
\end{proof}

\begin{rem}
The proof above is stated for axiomatic deduction, where derivations
are sequences and transitivity amounts to concatenation. In natural
deduction, transitivity is realized by substituting the derivation
of~$!A$ for the assumption~$!A$ in the derivation of~$!B$. In the
sequent calculus, transitivity is an instance of the cut rule.  Each
proof system instantiates these concepts differently; see DED.2--DED.5.
\end{rem}

Note that transitivity implies in particular: if $\Gamma \Proves !A$
and $!A \Proves !B$, then $\Gamma \Proves !B$.  It follows also that
if $!A_1, \dots, !A_n \Proves !B$ and $\Gamma \Proves !A_i$ for
each~$i$, then $\Gamma \Proves !B$.

\begin{prop}[Inconsistency Characterization] % structural property
\label{DED-prop:incons}
$\Gamma$ is inconsistent iff $\Gamma \Proves !A$ for every~$!A$.
\end{prop}

\begin{proof}
If $\Gamma$ is inconsistent, then $\Gamma \Proves \lfalse$.  From
$\lfalse$ any formula~$!A$ follows (by the logical axiom or rule
$\lfalse \lif !A$ in all standard proof systems).  Conversely, if
$\Gamma \Proves !A$ for every~$!A$, then in particular $\Gamma
\Proves \lfalse$, so $\Gamma$ is inconsistent.
\end{proof}

\begin{prop}[Compactness] % structural property
\label{DED-prop:proves-compact}
\begin{enumerate}
\item If $\Gamma \Proves !A$ then there is a finite subset $\Gamma_0
  \subseteq \Gamma$ such that $\Gamma_0 \Proves !A$.
\item If every finite subset of~$\Gamma$ is consistent, then $\Gamma$
  is consistent.
\end{enumerate}
\end{prop}

\begin{proof}
\begin{enumerate}
  \item If $\Gamma \Proves !A$, then there is a derivation---a finite
    object. Let $\Gamma_0$ be the set of elements of~$\Gamma$ that
    actually appear in the derivation.  Since the derivation is finite,
    $\Gamma_0$ is finite, and the derivation is equally a derivation
    from~$\Gamma_0$.  So $\Gamma_0 \Proves !A$.
  \item This is the contrapositive of~(1) for the special case $!A
    \ident \lfalse$.
\end{enumerate}
\end{proof}

%%% -----------------------------------------------------------------
%%% DED.1.7  Consistency and Derivability
%%% -----------------------------------------------------------------

\subsection{Consistency and Derivability}

We now establish a number of properties of the derivability relation.
They are independently interesting, and each plays a role in the proof
of the completeness theorem (see CP-002, Completeness, \S\ref{META.2}).

\begin{defn}[Syntactic Consistency] % DEF-DED001
\label{DEF-DED001}
A set $\Gamma$ of formulas is \emph{consistent} if and only if
$\Gamma \Proves/ \lfalse$; it is \emph{inconsistent} otherwise.
\end{defn}

\begin{prop} % provability-contr
\label{DED-prop:provability-contr}
If $\Gamma \Proves !A$ and $\Gamma \cup \{!A\}$ is inconsistent,
then $\Gamma$ is inconsistent.
\end{prop}

\begin{proof}
If $\Gamma \cup \{!A\}$ is inconsistent, then $\Gamma \cup \{!A\}
\Proves \lfalse$.  By reflexivity, $\Gamma \Proves !B$ for every
$!B \in \Gamma$.  Since also $\Gamma \Proves !A$ by hypothesis,
$\Gamma \Proves !B$ for every $!B \in \Gamma \cup \{!A\}$.  By
transitivity, $\Gamma \Proves \lfalse$, i.e., $\Gamma$ is
inconsistent.
\end{proof}

\begin{prop} % prov-incons
\label{DED-prop:prov-incons}
$\Gamma \Proves !A$ iff $\Gamma \cup \{\lnot !A\}$ is inconsistent.
\end{prop}

\begin{proof}
First suppose $\Gamma \Proves !A$.  Then $\Gamma \cup \{\lnot !A\}
\Proves !A$ by monotonicity, and $\Gamma \cup \{\lnot !A\} \Proves
\lnot !A$ by reflexivity.  Since from~$!A$ and~$\lnot !A$ together
we can derive~$\lfalse$ (via the axiom or rule $\lnot !A \lif (!A
\lif \lfalse)$ and modus ponens, in all standard proof systems),
$\Gamma \cup \{\lnot !A\} \Proves \lfalse$.

Now assume $\Gamma \cup \{\lnot !A\}$ is inconsistent.  By the
deduction theorem (THM-DED001, proved for each system in \S\ref{DED.7} below), $\Gamma \Proves \lnot
!A \lif \lfalse$.  Since $(\lnot !A \lif \lfalse) \lif \lnot\lnot !A$
and $\lnot\lnot !A \lif !A$ are derivable, we obtain $\Gamma \Proves
!A$ by modus ponens.
\end{proof}

\begin{prop} % explicit-inc
\label{DED-prop:explicit-inc}
If $\Gamma \Proves !A$ and $\lnot !A \in \Gamma$, then $\Gamma$ is
inconsistent.
\end{prop}

\begin{proof}
Since $\lnot !A \in \Gamma$, by reflexivity $\Gamma \Proves \lnot
!A$.  Together with $\Gamma \Proves !A$ and the derivability of
$\lnot !A \lif (!A \lif \lfalse)$, two applications of modus ponens
yield $\Gamma \Proves \lfalse$.
\end{proof}

\begin{prop} % provability-exhaustive
\label{DED-prop:provability-exhaustive}
If $\Gamma \cup \{!A\}$ and $\Gamma \cup \{\lnot !A\}$ are both
inconsistent, then $\Gamma$ is inconsistent.
\end{prop}

\begin{proof}
By \cref{DED-prop:prov-incons}, $\Gamma \cup \{\lnot !A\}$ inconsistent
implies $\Gamma \Proves !A$.  Since $\Gamma \cup \{!A\}$ is also
inconsistent, \cref{DED-prop:provability-contr} gives that $\Gamma$
is inconsistent.
\end{proof}

%%% -----------------------------------------------------------------
%%% DED.1.8  Derived and Admissible Rules
%%% -----------------------------------------------------------------

\subsection{Derived and Admissible Rules}

\begin{defn}[Derived Rule] % DEF-DED009
\label{DEF-DED009}
A rule $\frac{!A_1 \cdots !A_n}{!B}$ is a \emph{derived rule} of a
proof system if there exists a derivation of~$!B$ from assumptions
$!A_1, \ldots, !A_n$ using only the primitive rules. Derived rules
serve as shortcuts that do not extend the system's deductive power.
\end{defn}

\begin{defn}[Admissible Rule] % DEF-DED010
\label{DEF-DED010}
A rule $\frac{!A_1 \cdots !A_n}{!B}$ is \emph{admissible} if:
whenever $\Proves !A_1, \ldots, \Proves !A_n$ are all provable (as
theorems), then $\Proves !B$ is also provable. Unlike derived rules,
admissible rules need not yield a derivation using the premises
directly.
\end{defn}

\begin{rem}
Every derived rule is admissible, but not conversely.  For example,
the cut rule is admissible in the sequent calculus~$\Log{LK}$ (this is
the content of the cut-elimination theorem, see CP-010, Cut
Elimination, \S\ref{DED.4}), but in a system with cut it is a primitive
(hence trivially derived) rule.  The distinction matters for proof
search: derived rules can always be ``compiled away'' by inlining
their justifying derivation, while admissible rules may require a
global transformation of the proof.
\end{rem}

%%% -----------------------------------------------------------------
%%% DED.1.9  Deductive Closure and Conservative Extension
%%% -----------------------------------------------------------------

\subsection{Deductive Closure and Conservative Extension}

\begin{defn}[Deductive Closure] % DEF-DED003
\label{DEF-DED003}
The \emph{deductive closure} of a set of formulas~$\Gamma$ is the set
\[
\Thms{\Gamma} = \Setabs{!A}{\Gamma \Proves !A}.
\]
A set $\Gamma$ is \emph{deductively closed} if $\Gamma = \Thms{\Gamma}$.
\end{defn}

\begin{defn}[Conservative Extension] % DEF-DED004
\label{DEF-DED004}
A theory $T'$ in language $\Lang{L'} \supseteq \Lang{L}$ is a
\emph{conservative extension} of a theory $T$ in $\Lang{L}$ if for
every $\Lang{L}$-sentence $!A$: $T' \Proves !A$ implies $T \Proves
!A$.
\end{defn}

\begin{rem}
Conservative extensions are important in mathematical logic because
they guarantee that expanding a theory with new symbols and axioms does
not prove new theorems in the original language.  This notion appears
throughout the metatheory: definitional extensions are conservative,
and the method of Henkin constants used in the completeness proof (see
\S\ref{META.3}) produces a conservative extension of the original theory.
\end{rem}

%%% -----------------------------------------------------------------
%%% DED.1.10  Maximally Consistent Sets
%%% -----------------------------------------------------------------

\subsection{Maximally Consistent Sets}

\begin{defn}[Maximally Consistent Set] % DEF-DED002
\label{DEF-DED002}
A set~$\Gamma$ of sentences is \emph{maximally consistent} iff
\begin{enumerate}
\item $\Gamma$ is consistent, and
\item if $\Gamma \subsetneq \Gamma'$, then $\Gamma'$ is inconsistent.
\end{enumerate}
Equivalently, $\Gamma$ is maximally consistent iff $\Gamma$ is
consistent and for every sentence~$!A$: if $\Gamma \cup \{!A\}$ is
consistent, then $!A \in \Gamma$.
\end{defn}

Maximally consistent sets are central to the completeness proof.
Every consistent set of sentences is contained in a maximally
consistent set (by Lindenbaum's Lemma, see THM-DED005, \S\ref{DED.7}).
A maximally consistent set~$\Gamma$ contains, for each sentence~$!A$,
either $!A$ or~$\lnot !A$.  This property is what allows us to
construct a structure satisfying~$\Gamma$ in the completeness argument.

\begin{prop}
\label{DED-prop:mcs}
Suppose $\Gamma$ is maximally consistent. Then:
\begin{enumerate}
\item \label{DED-prop:mcs-prov-in} If $\Gamma \Proves !A$, then
  $!A \in \Gamma$.
\item \label{DED-prop:mcs-either-or} For any $!A$, either $!A \in
  \Gamma$ or $\lnot !A \in \Gamma$.
\item $(!A \land !B) \in \Gamma$ iff both $!A \in \Gamma$ and
  $!B \in \Gamma$.
\item $(!A \lor !B) \in \Gamma$ iff either $!A \in \Gamma$ or
  $!B \in \Gamma$.
\item $(!A \lif !B) \in \Gamma$ iff either $!A \notin \Gamma$ or
  $!B \in \Gamma$.
\end{enumerate}
\end{prop}

\begin{proof}
Let $\Gamma$ be maximally consistent throughout.
\begin{enumerate}
\item If $\Gamma \Proves !A$ and $!A \notin \Gamma$, then since
  $\Gamma$ is maximally consistent, $\Gamma \cup \{!A\}$ is
  inconsistent. By \cref{DED-prop:provability-contr}, $\Gamma$ is
  inconsistent, contradicting the hypothesis.  Hence $!A \in \Gamma$.

\item Suppose both $!A \notin \Gamma$ and $\lnot !A \notin \Gamma$.
  Then $\Gamma \cup \{!A\}$ and $\Gamma \cup \{\lnot !A\}$ are both
  inconsistent. By \cref{DED-prop:provability-exhaustive}, $\Gamma$ is
  inconsistent---a contradiction.

\item For the forward direction: if $(!A \land !B) \in \Gamma$ then
  $\Gamma \Proves !A \land !B$.  Since $!A \land !B \Proves !A$ and
  $!A \land !B \Proves !B$, we get $\Gamma \Proves !A$ and $\Gamma
  \Proves !B$ by transitivity. By~(1), $!A \in \Gamma$ and $!B \in
  \Gamma$.  For the reverse: if $!A, !B \in \Gamma$ then $\Gamma
  \Proves !A$ and $\Gamma \Proves !B$, so $\Gamma \Proves !A \land
  !B$, and by~(1), $(!A \land !B) \in \Gamma$.

\item Analogous, using the derivability properties of~$\lor$.

\item Analogous, using the derivability properties of~$\lif$.
\end{enumerate}
\end{proof}

%%% -----------------------------------------------------------------
%%% DED.1.11  Generic Connective and Quantifier Derivability
%%% -----------------------------------------------------------------

\subsection{Generic Connective and Quantifier Derivability}

The following propositions state basic derivability facts for the
propositional connectives and quantifiers that hold in all standard
proof systems.  They are used in the proof of the completeness theorem.
System-specific derivations establishing these facts are given in
DED.2--DED.5.

\begin{prop} % generic connective derivability
\label{DED-prop:provability-land}
\begin{enumerate}
\item Both $!A \land !B \Proves !A$ and $!A \land !B \Proves !B$.
\item $!A, !B \Proves !A \land !B$.
\end{enumerate}
\end{prop}

\begin{prop}
\label{DED-prop:provability-lor}
\begin{enumerate}
\item $!A \lor !B, \lnot !A, \lnot !B$ is inconsistent.
\item Both $!A \Proves !A \lor !B$ and $!B \Proves !A \lor !B$.
\end{enumerate}
\end{prop}

\begin{prop}
\label{DED-prop:provability-lif}
\begin{enumerate}
\item $!A, !A \lif !B \Proves !B$.
\item Both $\lnot !A \Proves !A \lif !B$ and $!B \Proves !A \lif !B$.
\end{enumerate}
\end{prop}

\begin{proof}[Proof sketch]
(1) In axiomatic deduction, modus ponens applied to $!A$ and the
axiom $!A \lif !B$ immediately yields $!B$.  In natural deduction,
$\lif$-elimination serves the same role.  In the sequent calculus,
Left-$\lif$ gives $!B$ from $!A \lif !B$ and $!A$.
(2) From $!B$ we derive $!A \lif !B$ by axiom~\eqref{ax:lif1} (or
$\lif$-introduction / Right-$\lif$); from $\lnot !A$ we derive
$!A \lif !B$ by axiom~\eqref{ax:lnot2} (or analogous rules).
Detailed derivations appear in DED.2 (axiomatic), DED.3 (natural
deduction), DED.4 (sequent calculus), and DED.5 (tableaux).
\end{proof}

\begin{thm}[Strong Generalization]
\label{DED-thm:strong-generalization}
If $c$ is a constant symbol not occurring in $\Gamma$ or $!A(x)$ and
$\Gamma \Proves !A(c)$, then $\Gamma \Proves \lforall[x][!A(x)]$.
\end{thm}

\begin{proof}
By the deduction theorem (THM-DED001), $\Gamma \Proves \ltrue \lif !A(c)$.  Since
$c$ does not occur in $\Gamma$ or~$\ltrue$, the quantifier rule (or
its equivalent in other systems) gives $\Gamma \Proves \ltrue \lif
\lforall[x][!A(x)]$.  By the deduction theorem again, $\Gamma \Proves
\lforall[x][!A(x)]$.
\end{proof}

\begin{prop}
\label{DED-prop:provability-quantifiers}
\begin{enumerate}
\item $!A(t) \Proves \lexists[x][!A(x)]$.
\item $\lforall[x][!A(x)] \Proves !A(t)$.
\end{enumerate}
\end{prop}

\begin{proof}
Both follow from the quantifier axioms (or quantifier rules, in
natural deduction and the sequent calculus) and the deduction theorem
(or the corresponding introduction/elimination rules).  For detailed
derivations, see DED.2--DED.5.
\end{proof}


%% ===================================================================
%% DED.2: Axiomatic (Hilbert) Systems
%% Sources: prf/axd (CONDENSE), axd/prp (KEEP), axd/qua (KEEP),
%%          axd/ded (KEEP+MERGE ddq), axd/ppr (CONDENSE),
%%          axd/qpr (CONDENSE), axd/sou (KEEP)
%% ===================================================================

\section{Axiomatic (Hilbert) Systems} \label{DED.2}

The axiomatic system instantiates the generic proof-theoretic framework
of \S\ref{DED.1} as follows.  A derivation (\ref{PRIM-DED005}) is a
finite \emph{sequence} of sentences; the system has many axiom schemas
(\ref{PRIM-DED001}) and only two rules of inference
(\ref{PRIM-DED003}): modus ponens and a quantifier rule.  Provability
(\ref{PRIM-DED006}) and consistency (\ref{DEF-DED001}) are defined
exactly as in \S\ref{DED.1}; all structural properties established
there (reflexivity, monotonicity, transitivity, compactness) hold with
derivations understood as sequences.

Axiomatic derivation systems were introduced by Gottlob Frege in 1879,
refined by Whitehead and Russell, and perfected by Hilbert and his
students in the 1920s.  Because derivations have a very simple
structure, it is relatively easy to prove things \emph{about} them
(e.g., the deduction theorem), though finding derivations in practice
is difficult.

%%% -----------------------------------------------------------------
%%% DED.2.1  Propositional Axioms and Modus Ponens
%%% -----------------------------------------------------------------

\subsection{Propositional Axioms and Modus Ponens}

\begin{defn}[Axiomatic System] % DEF-DED005
\label{DEF-DED005}
The \emph{axiomatic (Hilbert-style) deduction system} is defined by
the propositional axiom schemas of \ref{AX-DED003} below, the
quantifier axioms and quantifier rule of \S\ref{DED.2}.2, and the
rule of modus ponens (\ref{AX-DED001}).  A derivation from a set of
sentences~$\Gamma$ is a finite sequence $!B_1, \dots, !B_n$ in which
every~$!B_i$ is either
\begin{enumerate}
\item an element of~$\Gamma$, or
\item an instance of one of the axiom schemas, or
\item justified by a rule of inference applied to earlier items in the
  sequence.
\end{enumerate}
We write $\Gamma \Proves !A$ if there exists such a sequence ending
in~$!A$.
\end{defn}

\begin{defn}[Propositional Axioms] % AX-DED003 (propositional part)
\label{AX-DED003}
The set $\PAx$ of \emph{axioms} for the propositional connectives
comprises all formulas of the following forms:
\begin{align}
  & (!A \land !B) \lif !A \tag{A1}\label{ax:land1}\\
  & (!A \land !B) \lif !B \tag{A2}\label{ax:land2}\\
  & !A \lif (!B \lif (!A \land !B)) \tag{A3}\label{ax:land3}\\
  & !A \lif (!A \lor !B) \tag{A4}\label{ax:lor1}\\
  & !A \lif (!B \lor !A) \tag{A5}\label{ax:lor2}\\
  & (!A \lif !C) \lif ((!B \lif !C) \lif ((!A \lor !B) \lif !C))
    \tag{A6}\label{ax:lor3}\\
  & !A \lif (!B \lif !A) \tag{A7}\label{ax:lif1}\\
  & (!A \lif (!B \lif !C)) \lif ((!A \lif !B) \lif (!A \lif !C))
    \tag{A8}\label{ax:lif2}\\
  & (!A \lif !B) \lif ((!A \lif \lnot !B) \lif \lnot !A)
    \tag{A9}\label{ax:lnot1}\\
  & \lnot !A \lif (!A \lif !B) \tag{A10}\label{ax:lnot2}\\
  & \ltrue \tag{A11}\label{ax:ltrue}\\
  & \lfalse \lif !A \tag{A12}\label{ax:lfalse1}\\
  & (!A \lif \lfalse) \lif \lnot !A \tag{A13}\label{ax:lfalse2}\\
  & \lnot\lnot !A \lif !A \tag{A14}\label{ax:dne}
\end{align}
\end{defn}

\begin{defn}[Modus Ponens] % AX-DED001
\label{AX-DED001}
If $!B$ and $!B \lif !A$ already occur in a derivation, then $!A$ is
a correct inference step. We abbreviate this rule as~$\MP$.
\end{defn}

%%% -----------------------------------------------------------------
%%% DED.2.2  Quantifier Axioms and Rules
%%% -----------------------------------------------------------------

\subsection{Quantifier Axioms and Rules}

\begin{defn}[Quantifier Axioms] % AX-DED002 (part of AX-DED003)
\label{AX-DED002}
The \emph{axioms} governing quantifiers are all instances of:
\begin{align}
  & \lforall[x][!B] \lif !B(t), \tag{Q1}\label{ax:q1}\\
  & !B(t) \lif \lexists[x][!B], \tag{Q2}\label{ax:q2}
\end{align}
for any closed term~$t$.
\end{defn}

\begin{defn}[Quantifier Rule] % AX-DED002 (continued)
The quantifier rule $\QR$ has two forms:
\begin{enumerate}
\item If $!B \lif !A(a)$ already occurs in the derivation and $a$
  does not occur in~$\Gamma$ or~$!B$, then $!B \lif
  \lforall[x][!A(x)]$ is a correct inference step.
\item If $!A(a) \lif !B$ already occurs in the derivation and $a$
  does not occur in~$\Gamma$ or~$!B$, then $\lexists[x][!A(x)] \lif
  !B$ is a correct inference step.
\end{enumerate}
\end{defn}

%%% -----------------------------------------------------------------
%%% DED.2.3  The Deduction Theorem
%%% -----------------------------------------------------------------

\subsection{The Deduction Theorem (Axiomatic Proof)}

The deduction theorem (\ref{THM-DED001}) is the central metatheorem for
axiomatic systems.  Below we give its full proof, including the
quantifier-rule case.

\begin{prop}[Meta-Modus Ponens]
\label{DED2-prop:mp}
If $\Gamma \Proves !A$ and $\Gamma \Proves !A \lif !B$, then
$\Gamma \Proves !B$.
\end{prop}

\begin{proof}
We have that $\{!A, !A \lif !B\} \Proves !B$:
\begin{derivation}
  1. & $!A$ & Hyp.\\
  2. & $!A \lif !B$ & Hyp.\\
  3. & $!B$ & 1, 2, \MP
\end{derivation}
By transitivity (\ref{DED-prop:transitivity}), $\Gamma \Proves !B$.
\end{proof}

\begin{thm}[Deduction Theorem --- Axiomatic Proof] % THM-DED001, CP-009
\label{DED2-thm:deduction-thm}
$\Gamma \cup \{!A\} \Proves !B$ if and only if $\Gamma \Proves !A
\lif !B$.
\end{thm}

\begin{proof}
The ``if'' direction is immediate: if $\Gamma \Proves !A \lif !B$
then $\Gamma \cup \{!A\} \Proves !A \lif !B$ by monotonicity
(\ref{DED-prop:monotonicity}), $\Gamma \cup \{!A\} \Proves !A$ by
reflexivity (\ref{DED-prop:reflexivity}), and
\cref{DED2-prop:mp} gives $\Gamma \cup \{!A\} \Proves !B$.

For the ``only if'' direction, we proceed by induction on the length
of the derivation of~$!B$ from~$\Gamma \cup \{!A\}$.

\emph{Base case.}  A derivation of length~$1$ consists of $!B$ alone,
justified because $!B \in \Gamma \cup \{!A\}$ or $!B$ is an axiom.
If $!B \in \Gamma$ or $!B$ is an axiom, then $\Gamma \Proves !B$.
Since $\Gamma \Proves !B \lif (!A \lif !B)$ by axiom~\eqref{ax:lif1},
\cref{DED2-prop:mp} gives $\Gamma \Proves !A \lif !B$.  If $!B
\ident !A$, then $\Gamma \Proves !A \lif !A$ is derivable using
axioms \eqref{ax:lif1} and~\eqref{ax:lif2} and two applications
of~\MP.

\emph{Inductive step (modus ponens).}  Suppose the derivation of~$!B$
from $\Gamma \cup \{!A\}$ ends with a step justified by modus ponens
from earlier lines $!C \lif !B$ and~$!C$.  Then $\Gamma \cup \{!A\}
\Proves !C \lif !B$ and $\Gamma \cup \{!A\} \Proves !C$, and both
derivations are shorter. By induction hypothesis:
\begin{align*}
  & \Gamma \Proves !A \lif (!C \lif !B); \\
  & \Gamma \Proves !A \lif !C.
\end{align*}
By axiom~\eqref{ax:lif2},
\[
\Gamma \Proves (!A \lif (!C \lif !B)) \lif
((!A\lif !C)  \lif (!A \lif !B)),
\]
and two applications of \cref{DED2-prop:mp} give
$\Gamma \Proves !A \lif !B$.

\emph{Inductive step ($\QR$, universal case).}  Suppose $!B \ident !C
\lif \lforall[x][!D(x)]$ and a formula $!C \lif !D(a)$ appears
earlier in the derivation, where $a$ does not occur in~$!C$, $!A$,
or~$\Gamma$.  By induction hypothesis, $\Gamma \Proves !A \lif (!C
\lif !D(a))$.  From
\[
\Proves (!A \lif (!C \lif !D(a))) \lif ((!A \land !C) \lif !D(a))
\]
and modus ponens we get $\Gamma \Proves (!A \land !C) \lif !D(a)$.
Since the eigenvariable condition still holds, a step justified
by~$\QR$ gives $\Gamma \Proves (!A \land !C) \lif
\lforall[x][!D(x)]$. From
\[
\Proves ((!A \land !C) \lif \lforall[x][!D(x)]) \lif (!A \lif (!C
\lif \lforall[x][!D(x)])),
\]
modus ponens yields $\Gamma \Proves !A \lif (!C \lif
\lforall[x][!D(x)])$, i.e., $\Gamma \Proves !A \lif !B$.

The existential case ($!B \ident \lexists[x][!D(x)] \lif !C$) is
symmetric: one replaces $!C \lif !D(a)$ with $!D(a) \lif !C$
throughout and applies the second form of~$\QR$.
\end{proof}

Notice how axioms \eqref{ax:lif1} and~\eqref{ax:lif2} were chosen
precisely so that the Deduction Theorem would hold.

%%% -----------------------------------------------------------------
%%% DED.2.4  Derivability Properties
%%% -----------------------------------------------------------------

\subsection{Derivability Properties}

The following propositions provide the derivability facts stated
generically in \S\ref{DED.1}, now with explicit axiomatic proofs.

\begin{prop}\label{DED2-prop:provability-land}
\begin{enumerate}
\item Both $!A \land !B \Proves !A$ and $!A \land !B \Proves !B$.
\item $!A, !B \Proves !A \land !B$.
\end{enumerate}
\end{prop}

\begin{proof}
(1) From axioms \eqref{ax:land1} and~\eqref{ax:land2} by modus ponens.
(2) From axiom~\eqref{ax:land3} by two applications of modus ponens.
\end{proof}

\begin{prop}\label{DED2-prop:provability-lor}
\begin{enumerate}
\item $!A \lor !B, \lnot !A, \lnot !B$ is inconsistent.
\item Both $!A \Proves !A \lor !B$ and $!B \Proves !A \lor !B$.
\end{enumerate}
\end{prop}

\begin{proof}
(1) From axiom~\eqref{ax:lnot2} we derive $\{\lnot !A\} \Proves !A
\lif \lfalse$ and $\{\lnot !B\} \Proves !B \lif \lfalse$.  By
axiom~\eqref{ax:lor3} and the deduction theorem, $\{!A \lor !B, \lnot
!A, \lnot !B\} \Proves \lfalse$.
(2) From axioms \eqref{ax:lor1} and~\eqref{ax:lor2} by modus ponens.
\end{proof}

\begin{prop}\label{DED2-prop:provability-lif}
\begin{enumerate}
\item $!A, !A \lif !B \Proves !B$.
\item Both $\lnot !A \Proves !A \lif !B$ and $!B \Proves !A \lif !B$.
\end{enumerate}
\end{prop}

\begin{proof}
(1) Immediate from modus ponens.
(2) By axiom~\eqref{ax:lnot2} and axiom~\eqref{ax:lif1},
respectively, together with the deduction theorem.
\end{proof}

\begin{thm}[Strong Generalization]
\label{DED2-thm:strong-generalization}
If $c$ is a constant symbol not occurring in~$\Gamma$ or $!A(x)$ and
$\Gamma \Proves !A(c)$, then $\Gamma \Proves \lforall[x][!A(x)]$.
\end{thm}

\begin{proof}
By the deduction theorem, $\Gamma \Proves \ltrue \lif !A(c)$.  Since
$c$ does not occur in~$\Gamma$ or~$\ltrue$, the quantifier rule gives
$\Gamma \Proves \ltrue \lif \lforall[x][!A(x)]$. By the deduction
theorem again, $\Gamma \Proves \lforall[x][!A(x)]$.
\end{proof}

\begin{prop}\label{DED2-prop:provability-quantifiers}
\begin{enumerate}
\item $!A(t) \Proves \lexists[x][!A(x)]$.
\item $\lforall[x][!A(x)] \Proves !A(t)$.
\end{enumerate}
\end{prop}

\begin{proof}
(1) By axiom~\eqref{ax:q2} and modus ponens.
(2) By axiom~\eqref{ax:q1} and modus ponens.
\end{proof}

%%% -----------------------------------------------------------------
%%% DED.2.5  Soundness (Axiomatic)
%%% -----------------------------------------------------------------

\subsection{Soundness} \label{DED.2.sou}

\begin{prop}\label{DED2-prop:axioms-valid}
If $!A$ is an axiom (propositional, quantifier, or identity), then
$\Sat{M}{!A}[s]$ for each structure~$\Struct{M}$ and assignment~$s$.
\end{prop}

\begin{proof}
We verify that all the axioms are valid. For instance, here is the
case for axiom~\eqref{ax:q1}: suppose $t$ is free for~$x$ in~$!A$,
and assume $\Sat{M}{\lforall[x][!A]}[s]$. Then for each
$\varAssign{s'}{s}{x}$, also $\Sat{M}{!A}[s']$, and in particular
this holds when $s'(x) = \Value{t}{M}[s]$. By the substitution lemma
(see SYN.4), $\Sat{M}{\Subst{!A}{t}{x}}[s]$.  This shows that
$\Sat{M}{(\lforall[x][!A] \lif \Subst{!A}{t}{x})}[s]$. The remaining
propositional axioms are verified by truth-value analysis.

For the identity axioms: $\eq[t][t]$ is valid since $\Value{t}{M} =
\Value{t}{M}$. The axiom $\eq[t_1][t_2] \lif (!B(t_1) \lif !B(t_2))$
is valid because if $\Value{t_1}{M} = \Value{t_2}{M}$, then by the
substitution lemma $\Sat{M}{!B(t_1)}$ iff $\Sat{M}{!B(t_2)}$.
\end{proof}

\begin{thm}[Soundness] % CP-001(AX)
\label{DED2-thm:soundness}
If $\Gamma \Proves !A$ then $\Gamma \Entails !A$.
\end{thm}

\begin{proof}
By induction on the length of the derivation of~$!A$ from~$\Gamma$.

If there are no steps justified by inferences, then all formulas in
the derivation are either axiom instances or in~$\Gamma$. By
\cref{DED2-prop:axioms-valid}, all axioms are valid, so if $!A$ is an
axiom then $\Gamma \Entails !A$. If $!A \in \Gamma$, then trivially
$\Gamma \Entails !A$.

If the last step is justified by modus ponens, then there are formulas
$!B$ and $!B \lif !A$ in the derivation, and the induction hypothesis
applies to the parts of the derivation ending in those formulas (since
they contain at least one fewer inference step). So by induction
hypothesis, $\Gamma \Entails !B$ and $\Gamma \Entails !B \lif !A$.
Then $\Gamma \Entails !A$ by the semantic deduction theorem (see
THM-SEM, Semantic Deduction, \S SEM).

If the last step is justified by~$\QR$ and has the form $!C \lif
\lforall[x][!B(x)]$, then there is a preceding step $!C \lif !B(c)$
with $c$ not in~$\Gamma$, $!C$, or $\lforall[x][!B(x)]$. By
induction hypothesis, $\Gamma \Entails !C \lif !B(c)$. By the
semantic deduction theorem, $\Gamma \cup \{!C\} \Entails !B(c)$.
Consider a structure~$\Struct{M}$ with $\Sat{M}{\Gamma \cup \{!C\}}$.
We must show $\Sat{M}{\lforall[x][!B(x)]}$, i.e., for every variable
assignment~$s$, $\Sat{M}{!B(x)}[s]$. Since $\Gamma \cup \{!C\}$
consists of sentences, $\Sat{M}{!D}[s]$ for all $!D \in \Gamma \cup
\{!C\}$. Let $\Struct{M'}$ be like~$\Struct{M}$ except $\Assign{c}{M'}
= s(x)$. Since $c$ does not occur in~$\Gamma$ or~$!C$,
$\Sat{M'}{\Gamma \cup \{!C\}}$. Since $\Gamma \cup \{!C\} \Entails
!B(c)$, $\Sat{M'}{!B(c)}$. Since $!B(c)$ is a sentence,
$\Sat{M'}{!B(c)}[s]$. By the substitution lemma, $\Sat{M'}{!B(x)}[s]$.
Since $c$ does not occur in~$!B(x)$, $\Sat{M}{!B(x)}[s]$. Since $s$
was arbitrary, $\Sat{M}{\lforall[x][!B(x)]}$, and by the semantic
deduction theorem, $\Gamma \Entails !C \lif \lforall[x][!B(x)]$.

The case where the last step is $\lexists[x][!B(x)] \lif !C$ is
symmetric.

For the identity axioms, \cref{DED2-prop:axioms-valid} ensures they are
valid, so the base case of the induction applies.
\end{proof}

\begin{cor}[Weak Soundness]
\label{DED2-cor:weak-soundness}
If $\Proves !A$, then $!A$ is valid.
\end{cor}

\begin{cor}
\label{DED2-cor:consistency-soundness}
If $\Gamma$ is satisfiable, then it is consistent.
\end{cor}

\begin{proof}
We prove the contrapositive. If $\Gamma$ is inconsistent, then
$\Gamma \Proves \lfalse$. By \cref{DED2-thm:soundness}, any
structure~$\Struct{M}$ satisfying~$\Gamma$ must satisfy~$\lfalse$.
Since $\Sat/{M}{\lfalse}$ for every~$\Struct{M}$, no structure can
satisfy~$\Gamma$, so $\Gamma$ is unsatisfiable.
\end{proof}


%% ===================================================================
%% DED.3: Natural Deduction
%% Sources: prf/ntd (CONDENSE), ntd/rul (KEEP), ntd/prl (KEEP),
%%          ntd/qrl (KEEP), ntd/der (CONDENSE), ntd/sou (KEEP+ABSORB sid)
%% ===================================================================

\section{Natural Deduction} \label{DED.3}

Natural deduction instantiates the generic framework of \S\ref{DED.1}
differently from axiomatic deduction.  There are no logical axioms;
instead, each connective and quantifier has introduction and
elimination rules, corresponding to natural inference patterns (e.g.,
conditional proof, proof by cases, indirect proof).  A derivation
(\ref{PRIM-DED005}) is a finite \emph{tree} of sentences rather than
a sequence.  Provability (\ref{PRIM-DED006}) and consistency
(\ref{DEF-DED001}) are defined exactly as in \S\ref{DED.1}.

A distinguishing feature of natural deduction is the mechanism of
\emph{assumption discharge} (\ref{PRIM-DED009}): certain rules cancel
(\emph{discharge}) hypothetical assumptions, so that they no longer
count among the open assumptions of the derivation.  The undischarged
assumptions of a completed derivation are the set~$\Gamma$ from which
the conclusion is derived.

Natural deduction systems were developed by Gentzen and Ja\'skowski in
the 1930s, and later refined by Prawitz and Fitch.  In the philosophy
of logic, the rules of natural deduction have sometimes been taken to
define the meanings of the logical operators (``proof-theoretic
semantics'').

%%% -----------------------------------------------------------------
%%% DED.3.1  Rules and Derivations
%%% -----------------------------------------------------------------

\subsection{Rules and Derivations}

\begin{defn}[Assumption]
An \emph{assumption} is any sentence
in the topmost position of any branch of a derivation tree.
\end{defn}

Derivations in natural deduction are certain trees of sentences, where
the topmost sentences are assumptions, and if a sentence stands below
one, two, or three other sentences, it must follow correctly by a rule
of inference. The sentences at the top of the inference are the
\emph{premises} and the sentence below the \emph{conclusion}. The
rules come in pairs---an introduction and an elimination rule for each
operator---and some rules allow an assumption to be
\emph{discharged}. To indicate which assumption is discharged by which
inference, both receive a matching label: the assumption is written
``$\Discharge{!A}{n}$.''

It is customary to consider rules for all the operators $\land$,
$\lor$, $\lif$, $\lnot$, and $\lfalse$, even if some of those are
defined.

%%% -----------------------------------------------------------------
%%% DED.3.2  Propositional Rules
%%% -----------------------------------------------------------------

\subsection{Propositional Rules} % DEF-DED006
\label{DEF-DED006}

\subsubsection*{Rules for $\land$}

\begin{defish}
\AxiomC{$!A$}
\AxiomC{$!B$}
\RightLabel{\Intro{\land}}
\BinaryInfC{$!A \land !B$}
\DisplayProof
\hfill
\begin{tabular}{r}
\AxiomC{$!A \land !B$}
\RightLabel{\Elim{\land}}
\UnaryInfC{$!A$}
\DisplayProof
\\[3ex]
\AxiomC{$!A \land !B$}
\RightLabel{\Elim{\land}}
\UnaryInfC{$!B$}
\DisplayProof
\end{tabular}
\end{defish}

\subsubsection*{Rules for $\lor$}

\begin{defish}
\begin{tabular}{r}
\AxiomC{$!A$}
\RightLabel{\Intro{\lor}}
\UnaryInfC{$!A \lor !B$}
\DisplayProof
\\[3ex]
\AxiomC{$!B$}
\RightLabel{\Intro{\lor}}
\UnaryInfC{$!A \lor !B$}
\DisplayProof
\end{tabular}
\hfill
\AxiomC{$!A \lor !B$}
\AxiomC{$\Discharge{!A}{n}$}
\DeduceC{$!C$}
\AxiomC{$\Discharge{!B}{n}$}
\DeduceC{$!C$}
\DischargeRule{\Elim{\lor}}{n}
\TrinaryInfC{$!C$}
\DisplayProof
\end{defish}

\subsubsection*{Rules for $\lif$}

\begin{defish}
\AxiomC{$\Discharge{!A}{n}$}
\DeduceC{$!B$}
\DischargeRule{\Intro{\lif}}{n}
\UnaryInfC{$!A \lif !B$}
\DisplayProof
\hfill
\AxiomC{$!A \lif !B$}
\AxiomC{$!A$}
\RightLabel{\Elim{\lif}}
\BinaryInfC{$!B$}
\DisplayProof
\end{defish}

\subsubsection*{Rules for $\lnot$}

\begin{defish}
\AxiomC{$\Discharge{!A}{n}$}
\noLine
\DeduceC{$\lfalse$}
\DischargeRule{\Intro{\lnot}}{n}
\UnaryInfC{$\lnot !A$}
\DisplayProof
\hfill
\AxiomC{$\lnot !A$}
\AxiomC{$!A$}
\RightLabel{\Elim{\lnot}}
\BinaryInfC{$\lfalse$}
\DisplayProof
\end{defish}

\subsubsection*{Rules for $\lfalse$}

\begin{defish}
\AxiomC{$\lfalse$}
\RightLabel{\FalseInt}
\UnaryInfC{$!A$}
\DisplayProof
\hfill
\AxiomC{$\Discharge{\lnot !A}{n}$}
\DeduceC{$\lfalse$}
\DischargeRule{\FalseCl}{n}
\UnaryInfC{$!A$}
\DisplayProof
\end{defish}

Note that $\Intro{\lnot}$ and $\FalseCl$ are very similar: the
difference is that $\Intro{\lnot}$ derives a negated sentence~$\lnot
!A$ while $\FalseCl$ derives a positive sentence~$!A$.

Whenever a rule indicates that some assumption may be discharged, this
is a permission, not a requirement.  In the $\Intro{\lif}$ rule, for
instance, we may discharge any number of assumptions of the
form~$!A$, including zero.

%%% -----------------------------------------------------------------
%%% DED.3.3  Quantifier Rules
%%% -----------------------------------------------------------------

\subsection{Quantifier Rules}

\subsubsection*{Rules for $\lforall$}

\begin{defish}
\AxiomC{$!A(a)$}
\RightLabel{\Intro{\lforall}}
\UnaryInfC{$\lforall[x][\Atom{!A}{x}]$}
\DisplayProof
\hfill
\AxiomC{$\lforall[x][\Atom{!A}{x}]$}
\RightLabel{\Elim{\lforall}}
\UnaryInfC{$!A(t)$}
\DisplayProof
\end{defish}

In $\Elim{\lforall}$, $t$ is a closed term. In $\Intro{\lforall}$,
$a$ is a constant symbol which does not occur in the
conclusion~$\lforall[x][!A(x)]$ or in any undischarged assumption. We
call $a$ the \emph{eigenvariable} of the $\Intro{\lforall}$ inference.

\subsubsection*{Rules for $\lexists$}

\begin{defish}
\AxiomC{$\Atom{!A}{t}$}
\RightLabel{\Intro{\lexists}}
\UnaryInfC{$\lexists[x][\Atom{!A}{x}]$}
\DisplayProof
\hfill
\AxiomC{$\lexists[x][\Atom{!A}{x}]$}
\AxiomC{[$\Atom{!A}{a}$]$^n$}
\DeduceC{$!C$}
\DischargeRule{\Elim{\lexists}}{n}
\BinaryInfC{$!C$}
\DisplayProof
\end{defish}

Again, $t$ is a closed term, and $a$ is a constant symbol which does
not occur in $\lexists[x][!A(x)]$, in~$!C$, or in any undischarged
assumption (other than~$!A(a)$). We call $a$ the \emph{eigenvariable}
of the $\Elim{\lexists}$ inference.

The condition that an eigenvariable not occur in the premises or in
any undischarged assumption is the \emph{eigenvariable condition}.  It
is necessary to ensure soundness.

\subsubsection*{Rules for Identity}

\begin{defish}
\AxiomC{}
\RightLabel{\Intro{\eq}}
\UnaryInfC{$\eq[t][t]$}
\DisplayProof
\hfill
\begin{tabular}{r}
\AxiomC{$\eq[t_1][t_2]$}
\AxiomC{$!A(t_1)$}
\RightLabel{\Elim{\eq}}
\BinaryInfC{$!A(t_2)$}
\DisplayProof
\\[3ex]
\AxiomC{$\eq[t_1][t_2]$}
\AxiomC{$!A(t_2)$}
\RightLabel{\Elim{\eq}}
\BinaryInfC{$!A(t_1)$}
\DisplayProof
\end{tabular}
\end{defish}

In the above rules, $t$, $t_1$, and $t_2$ are closed terms. The
$\Intro{\eq}$ rule derives $\eq[t][t]$ from no assumptions.

%%% -----------------------------------------------------------------
%%% DED.3.4  Derivations
%%% -----------------------------------------------------------------

\subsection{Derivations}

\begin{defn}[Derivation] % ND instantiation of PRIM-DED005
\label{DED3-defn:derivation}
A \emph{derivation} of a sentence~$!A$ from assumptions~$\Gamma$ is a
finite tree of sentences satisfying:
\begin{enumerate}
\item The topmost sentences are either in~$\Gamma$ or are discharged
  by an inference in the tree.
\item The bottommost sentence is~$!A$.
\item Every sentence except~$!A$ is a premise of a correct application
  of an inference rule whose conclusion stands directly below it.
\end{enumerate}
We call $!A$ the \emph{conclusion} and $\Gamma$ the
\emph{undischarged assumptions}. If such a derivation exists, we write
$\Gamma \Proves !A$. If every assumption is discharged, we
write~$\Proves !A$.
\end{defn}

%%% -----------------------------------------------------------------
%%% DED.3.5  Soundness (Natural Deduction)
%%% -----------------------------------------------------------------

\subsection{Soundness} \label{DED.3.sou}

\begin{thm}[Soundness] % CP-001(ND)
\label{DED3-thm:soundness}
If $!A$ is derivable from the undischarged assumptions~$\Gamma$, then
$\Gamma \Entails !A$.
\end{thm}

\begin{proof}
Let $\delta$ be a derivation of~$!A$. We proceed by induction on the
number of inferences in~$\delta$.

\emph{Base case.} If there are no inferences, $\delta$ consists of a
single sentence~$!A$ that is an undischarged assumption. Any
structure satisfying~$\Gamma$ satisfies~$!A$ since $!A \in \Gamma$.

\emph{Inductive step.} Suppose $\delta$ contains~$n$ inferences.  We
assume the induction hypothesis for each sub-derivation (which has
fewer than~$n$ inferences) and distinguish cases by the last
inference.

\begin{enumerate}
\item \emph{$\Intro{\lnot}$:} The derivation ends in
\begin{prooftree}
  \AxiomC{$\Gamma, \Discharge{!A}{n}$}
  \RightLabel{$\delta_1$}
  \DeduceC{$\lfalse$}
  \DischargeRule{\Intro{\lnot}}{n}
  \UnaryInfC{$\lnot !A$}
\end{prooftree}
By induction hypothesis, $\Gamma \cup \{!A\} \Entails \lfalse$.
Suppose $\Sat{M}{\Gamma}$ but $\Sat/{M}{\lnot !A}$, i.e.,
$\Sat{M}{!A}$. Then $\Sat{M}{\Gamma \cup \{!A\}}$, so
$\Sat{M}{\lfalse}$, a contradiction. Hence $\Sat{M}{\lnot !A}$.

\item \emph{$\Elim{\land}$:} By induction hypothesis $\Gamma \Entails
!A \land !B$; by definition of satisfaction, $\Sat{M}{!A}$ (or
$\Sat{M}{!B}$, for the other variant).

\item \emph{$\Intro{\lor}$:} By induction hypothesis $\Gamma \Entails
!A$; since $\Sat{M}{!A}$ implies $\Sat{M}{!A \lor !B}$.

\item \emph{$\Intro{\lif}$:} By induction hypothesis $\Gamma \cup
\{!A\} \Entails !B$. Suppose $\Sat{M}{\Gamma}$ but
$\Sat/{M}{!A \lif !B}$. Then $\Sat{M}{!A}$ and $\Sat/{M}{!B}$, so
$\Sat{M}{\Gamma \cup \{!A\}}$, giving $\Sat{M}{!B}$, contradiction.

\item \emph{$\Elim{\lif}$:} By induction hypothesis $\Gamma_1
\Entails !A \lif !B$ and $\Gamma_2 \Entails !A$. If
$\Sat{M}{\Gamma_1 \cup \Gamma_2}$, then $\Sat{M}{!A \lif !B}$ and
$\Sat{M}{!A}$, so $\Sat{M}{!B}$.

\item \emph{$\Elim{\lnot}$:} By induction hypothesis $\Gamma_1
\Entails \lnot !A$ and $\Gamma_2 \Entails !A$. If
$\Sat{M}{\Gamma_1 \cup \Gamma_2}$ then both $\Sat{M}{\lnot !A}$ and
$\Sat{M}{!A}$, giving $\Sat{M}{\lfalse}$.

\item \emph{$\FalseInt$:} By induction hypothesis $\Gamma \Entails
\lfalse$. Since $\Sat/{M}{\lfalse}$ for every~$\Struct{M}$, no
structure satisfies~$\Gamma$, so $\Gamma \Entails !A$ vacuously.

\item \emph{$\FalseCl$:} By induction hypothesis $\Gamma \cup \{\lnot
!A\} \Entails \lfalse$. Suppose $\Sat{M}{\Gamma}$ but
$\Sat/{M}{!A}$, i.e., $\Sat{M}{\lnot !A}$. Then $\Sat{M}{\Gamma \cup
\{\lnot !A\}}$, so $\Sat{M}{\lfalse}$, contradiction.

\item \emph{$\Intro{\land}$:} By induction hypothesis $\Gamma_1
\Entails !A$ and $\Gamma_2 \Entails !B$. If $\Sat{M}{\Gamma_1 \cup
\Gamma_2}$, then $\Sat{M}{!A}$ and $\Sat{M}{!B}$, so $\Sat{M}{!A
\land !B}$.

\item \emph{$\Elim{\lor}$:} By induction hypothesis $\Gamma_1
\Entails !A \lor !B$, $\Gamma_2 \cup \{!A\} \Entails !C$, and
$\Gamma_3 \cup \{!B\} \Entails !C$. If $\Sat{M}{\Gamma_1 \cup
\Gamma_2 \cup \Gamma_3}$, then $\Sat{M}{!A \lor !B}$, so either
$\Sat{M}{!A}$ or $\Sat{M}{!B}$; in either case $\Sat{M}{!C}$.

\item \emph{$\Intro{\lforall}$:} By induction hypothesis $\Gamma
\Entails !A(a)$ where $a$ does not occur in~$\Gamma$ or
$\lforall[x][!A(x)]$. Suppose $\Sat{M}{\Gamma}$. We must show
$\Sat{M}{\lforall[x][!A(x)]}$, i.e., for every variable
assignment~$s$, $\Sat{M}{!A(x)}[s]$. Let $\Struct{M'}$ be
like~$\Struct{M}$ except $\Assign{a}{M'} = s(x)$. Since $a$ does not
occur in~$\Gamma$, $\Sat{M'}{\Gamma}$, so $\Sat{M'}{!A(a)}$, whence
$\Sat{M'}{!A(a)}[s]$. By the substitution lemma,
$\Sat{M'}{!A(x)}[s]$. Since $a$ does not occur in~$!A(x)$,
$\Sat{M}{!A(x)}[s]$.

\item \emph{$\Intro{\lexists}$:} By induction hypothesis $\Gamma
\Entails !A(t)$. By the substitution lemma, $\Sat{M}{!A(t)}$ implies
$\Sat{M}{\lexists[x][!A(x)]}$.

\item \emph{$\Elim{\lforall}$:} By induction hypothesis $\Gamma
\Entails \lforall[x][!A(x)]$. If $\Sat{M}{\lforall[x][!A(x)]}$, by
the substitution lemma $\Sat{M}{!A(t)}$.

\item \emph{$\Elim{\lexists}$:} By induction hypothesis $\Gamma_1
\Entails \lexists[x][!A(x)]$ and $\Gamma_2 \cup \{!A(a)\} \Entails
!C$, where $a$ does not occur in $\lexists[x][!A(x)]$, $!C$, or the
undischarged assumptions. The argument is analogous to the
$\Intro{\lforall}$ case, constructing a suitable~$\Struct{M'}$.

\item \emph{$\Intro{\eq}$:} $\eq[t][t]$ is valid since
$\Value{t}{M} = \Value{t}{M}$ for every~$\Struct{M}$.

\item \emph{$\Elim{\eq}$:} By induction hypothesis $\Gamma_1
\Entails \eq[t_1][t_2]$ and $\Gamma_2 \Entails !A(t_1)$. If
$\Sat{M}{\Gamma_1 \cup \Gamma_2}$, then $\Value{t_1}{M} =
\Value{t_2}{M}$. By the substitution lemma, $\Sat{M}{!A(t_1)}$
iff $\Sat{M}{!A(t_2)}$, so $\Sat{M}{!A(t_2)}$.
\end{enumerate}
\end{proof}

\begin{cor}[Weak Soundness]
\label{DED3-cor:weak-soundness}
If $\Proves !A$, then $!A$ is valid.
\end{cor}

\begin{cor}
\label{DED3-cor:consistency-soundness}
If $\Gamma$ is satisfiable, then it is consistent.
\end{cor}

\begin{proof}
Contrapositive: if $\Gamma$ is inconsistent, then $\Gamma \Proves
\lfalse$. By \cref{DED3-thm:soundness}, any structure satisfying
$\Gamma$ must satisfy~$\lfalse$. Since $\Sat/{M}{\lfalse}$ for
every~$\Struct{M}$, $\Gamma$ is unsatisfiable.
\end{proof}

\begin{rem}
The structural properties of \S\ref{DED.1} (reflexivity, monotonicity,
transitivity, compactness) hold for natural deduction. Transitivity
uses implication-introduction and elimination rather than sequence
concatenation.
\end{rem}


%% ===================================================================
%% DED.4: Sequent Calculus
%% Sources: seq/rul (KEEP), seq/prl (KEEP), seq/srl (KEEP),
%%          seq/qrl (KEEP), seq/der (CONDENSE), seq/sou (KEEP+ABSORB sid),
%%          seq/ide (CONDENSE)
%% ===================================================================

\section{Sequent Calculus} \label{DED.4}

The sequent calculus (Gentzen's $\Log{LK}$) instantiates the generic
framework of \S\ref{DED.1} with derivations that are trees of
\emph{sequents} (\ref{PRIM-DED008}) rather than trees of single
formulas.  The system has no axiom schemas in the Hilbert sense;
instead, it employs initial sequents, logical rules (one left and one
right rule per connective), and the structural rules of
\S\ref{DED.1}.3 (\ref{PRIM-DED007}).  Provability
(\ref{PRIM-DED006}) and consistency (\ref{DEF-DED001}) are defined as
before: $\Gamma \Proves !A$ iff for some finite $\Gamma_0 \subseteq
\Gamma$, the sequent $\Gamma_0 \Sequent !A$ has a derivation.

%%% -----------------------------------------------------------------
%%% DED.4.1  Sequents and Initial Sequents
%%% -----------------------------------------------------------------

\subsection{Sequents and Initial Sequents} % DEF-DED007
\label{DEF-DED007}

For the following, let $\Gamma, \Delta, \Pi, \Lambda$ represent finite
sequences of sentences.

\begin{defn}[Sequent]
A \emph{sequent} is an expression $\Gamma \Sequent \Delta$ where
$\Gamma$ (the \emph{antecedent}) and $\Delta$ (the \emph{succedent})
are finite (possibly empty) sequences of sentences.
\end{defn}

The intuitive reading of $\Gamma \Sequent \Delta$ is: if all
sentences in the antecedent hold, then at least one of the sentences
in the succedent holds.  That is, if $\Gamma = \tuple{!A_1, \dots,
!A_m}$ and $\Delta = \tuple{!B_1, \dots, !B_n}$, then $\Gamma
\Sequent \Delta$ corresponds to
\[
(!A_1 \land \cdots \land !A_m) \lif (!B_1 \lor \cdots \lor !B_n).
\]
When $\Gamma$ is empty, $\Sequent \Delta$ asserts $!B_1 \lor \dots
\lor !B_n$. When $\Delta$ is empty, $\Gamma \Sequent$ asserts
$\lnot(!A_1 \land \dots \land !A_m)$.

If $\Gamma$ is a sequence, we write $\Gamma, !A$ for the result of
appending $!A$ to the right end of~$\Gamma$ (and $!A, \Gamma$ for
appending to the left). $\Gamma, \Delta$ denotes concatenation.

\begin{defn}[Initial Sequent]
An \emph{initial sequent} is a sequent of one of the following forms:
\begin{enumerate}
\item $!A \Sequent !A$ for any sentence~$!A$;
\item $\quad \Sequent \ltrue$;
\item $\lfalse \Sequent \quad$.
\end{enumerate}
\end{defn}

Derivations in the sequent calculus are trees of sequents, where the
topmost sequents are initial sequents, and the logical rules are named
for the main operator of the sentence they introduce. Each comes in
two forms: a left rule (introducing the operator in the antecedent)
and a right rule (introducing it in the succedent).

%%% -----------------------------------------------------------------
%%% DED.4.2  Propositional Rules
%%% -----------------------------------------------------------------

\subsection{Propositional Rules}

\subsubsection*{Rules for $\lnot$}

\begin{defish}
\Axiom$ \Gamma \fCenter \Delta, !A $
\RightLabel{\LeftR{\lnot}}
\UnaryInf$ \lnot !A, \Gamma \fCenter \Delta$
\DisplayProof
\hfill
\Axiom$!A, \Gamma \fCenter \Delta$
\RightLabel{\RightR{\lnot}}
\UnaryInf$ \Gamma \fCenter \Delta, \lnot !A $
\DisplayProof
\end{defish}

\subsubsection*{Rules for $\land$}

\begin{defish}\noindent
\begin{tabular}{l}
\Axiom$ !A, \Gamma \fCenter \Delta$
\RightLabel{\LeftR{\land}}
\UnaryInf$ !A \land !B, \Gamma \fCenter \Delta$
\DisplayProof
\\[3ex]
\Axiom$!B, \Gamma \fCenter \Delta$
\RightLabel{\LeftR{\land}}
\UnaryInf$!A \land !B, \Gamma \fCenter \Delta$
\DisplayProof
\end{tabular}
\hfill
\Axiom$\Gamma \fCenter \Delta, !A$
\Axiom$ \Gamma \fCenter \Delta, !B$
\RightLabel{\RightR{\land}}
\BinaryInf$ \Gamma \fCenter \Delta, !A \land !B $
\DisplayProof
\end{defish}

\subsubsection*{Rules for $\lor$}

\begin{defish}
\Axiom$!A, \Gamma \fCenter \Delta$
\Axiom$!B, \Gamma \fCenter \Delta$
\RightLabel{\LeftR{\lor}}
\BinaryInf$!A \lor !B, \Gamma \fCenter \Delta$
\DisplayProof
\hfill
\begin{tabular}{r}
\Axiom$\Gamma \fCenter \Delta, !A$
\RightLabel{\RightR{\lor}}
\UnaryInf$ \Gamma \fCenter \Delta, !A \lor !B$
\DisplayProof
\\[3ex]
\Axiom$ \Gamma \fCenter \Delta, !B$
\RightLabel{\RightR{\lor}}
\UnaryInf$ \Gamma \fCenter \Delta, !A \lor !B$
\DisplayProof
\end{tabular}
\end{defish}

\subsubsection*{Rules for $\lif$}

\begin{defish}
\Axiom$ \Gamma \fCenter \Delta, !A$
\Axiom$ !B, \Pi \fCenter \Lambda$
\RightLabel{\LeftR{\lif}}
\BinaryInf$ !A \lif !B, \Gamma, \Pi \fCenter \Delta, \Lambda$
\DisplayProof
\hfill
\Axiom$ !A, \Gamma \fCenter \Delta, !B$
\RightLabel{\RightR{\lif}}
\UnaryInf$ \Gamma \fCenter \Delta, !A \lif !B $
\DisplayProof
\end{defish}

%%% -----------------------------------------------------------------
%%% DED.4.3  Structural Rules
%%% -----------------------------------------------------------------

\subsection{Structural Rules}

The structural rules (\ref{PRIM-DED007}) of $\Log{LK}$ allow
rearranging sentences in the antecedent and succedent. Because the
logical rules require that the principal sentence stand at a specific
position, exchange is needed to move it there; weakening and
contraction handle unused or duplicated formulas.

\subsubsection*{Weakening}

\begin{defish}
\Axiom$ \Gamma \fCenter \Delta $
\RightLabel{\LeftR{\Weakening}}
\UnaryInf$ !A, \Gamma \fCenter \Delta$
\DisplayProof
\hfill
\Axiom$ \Gamma \fCenter \Delta$
\RightLabel{\RightR{\Weakening}}
\UnaryInf$ \Gamma \fCenter \Delta, !A$
\DisplayProof
\end{defish}

\subsubsection*{Contraction}

\begin{defish}
\Axiom$ !A, !A, \Gamma \fCenter \Delta $
\RightLabel{\LeftR{\Contraction}}
\UnaryInf$ !A, \Gamma \fCenter \Delta$
\DisplayProof
\hfill
\Axiom$ \Gamma \fCenter \Delta, !A, !A$
\RightLabel{\RightR{\Contraction}}
\UnaryInf$ \Gamma \fCenter \Delta, !A$
\DisplayProof
\end{defish}

\subsubsection*{Exchange}

\begin{defish}
\Axiom$ \Gamma, !A, !B, \Pi \fCenter \Delta $
\RightLabel{\LeftR{\Exchange}}
\UnaryInf$ \Gamma, !B, !A, \Pi \fCenter \Delta$
\DisplayProof
\hfill
\Axiom$ \Gamma \fCenter \Delta, !A, !B, \Lambda$
\RightLabel{\RightR{\Exchange}}
\UnaryInf$ \Gamma \fCenter \Delta, !B, !A, \Lambda$
\DisplayProof
\end{defish}

A series of weakening, contraction, and exchange inferences is often
indicated by double inference lines.

\subsubsection*{Cut}

\begin{defish}
\[
\Axiom$ \Gamma \fCenter \Delta, !A$
\Axiom$ !A, \Pi \fCenter \Lambda $
\RightLabel{\Cut}
\BinaryInf$ \Gamma, \Pi \fCenter \Delta, \Lambda$
\DisplayProof
\]
\end{defish}

The cut rule is not strictly necessary, but makes it considerably
easier to reuse and combine derivations.

\begin{rem}[Cut Elimination] % CP-010
\label{CP-010}
\emph{Cut Elimination (Gentzen's Hauptsatz):} The $\Cut$ rule is
\emph{admissible} in~$\Log{LK}$---every $\Log{LK}$-derivation using
$\Cut$ can be transformed into one without~$\Cut$. The proof is a
detailed structural induction on the complexity of the cut formula and
the height of the derivation; it is beyond our scope here. Cut
elimination has profound consequences: it implies the subformula
property (every formula in a cut-free derivation is a subformula of
the end-sequent), which in turn yields consistency proofs, decidability
results, and interpolation theorems.
\end{rem}

%%% -----------------------------------------------------------------
%%% DED.4.4  Quantifier Rules
%%% -----------------------------------------------------------------

\subsection{Quantifier Rules}

\subsubsection*{Rules for $\lforall$}

\begin{defish}
\Axiom$ !A(t), \Gamma \fCenter \Delta$
\RightLabel{\LeftR{\lforall}}
\UnaryInf$ \lforall[x][!A(x)],\Gamma \fCenter \Delta$
\DisplayProof
\hfill
\Axiom$ \Gamma \fCenter \Delta, !A(a) $
\RightLabel{\RightR{\lforall}}
\UnaryInf$ \Gamma \fCenter \Delta, \lforall[x][!A(x)]$
\DisplayProof
\end{defish}

In $\LeftR{\lforall}$, $t$ is a closed term. In $\RightR{\lforall}$,
$a$ is a constant symbol which must not occur in the lower sequent.
We call $a$ the \emph{eigenvariable} of the $\RightR{\lforall}$
inference.

\subsubsection*{Rules for $\lexists$}

\begin{defish}
\Axiom$ !A(a), \Gamma \fCenter \Delta $
\RightLabel{\LeftR{\lexists}}
\UnaryInf$ \lexists[x][!A(x)], \Gamma \fCenter \Delta$
\DisplayProof
\hfill
\Axiom$ \Gamma \fCenter \Delta, !A(t) $
\RightLabel{\RightR{\lexists}}
\UnaryInf$ \Gamma \fCenter \Delta, \lexists[x][!A(x)]$
\DisplayProof
\end{defish}

Again, $t$ is a closed term, and $a$ is a constant symbol not
occurring in the lower sequent. The \emph{eigenvariable condition}
requires that the eigenvariable~$a$ not occur anywhere in the lower
sequent of the $\RightR{\lforall}$ or $\LeftR{\lexists}$ inference.

\subsubsection*{Identity Rules}

\begin{defn}[Initial Sequents for $\eq$]
If $t$ is a closed term, then ${} \Sequent \eq[t][t]$ is an initial
sequent.
\end{defn}

The rules for~$\eq$ are ($t_1$ and $t_2$ are closed terms):

\begin{defish}
\Axiom$ \eq[t_1][t_2], \Gamma \fCenter \Delta, !A(t_1) $
\RightLabel{$\eq$}
\UnaryInf$\eq[t_1][t_2], \Gamma \fCenter \Delta, !A(t_2)$
\DisplayProof
\hfill
\Axiom$\eq[t_1][t_2], \Gamma \fCenter \Delta, !A(t_2) $
\RightLabel{$\eq$}
\UnaryInf$\eq[t_1][t_2], \Gamma  \fCenter \Delta, !A(t_1)$
\DisplayProof
\end{defish}

%%% -----------------------------------------------------------------
%%% DED.4.5  Derivations
%%% -----------------------------------------------------------------

\subsection{Derivations}

\begin{defn}[$\Log{LK}$-Derivation]
\label{DED4-defn:derivation}
An \emph{$\Log{LK}$-derivation} of a sequent~$S$ is a finite tree of
sequents satisfying:
\begin{enumerate}
\item The topmost sequents are initial sequents.
\item The bottommost sequent is~$S$.
\item Every sequent except $S$ is a premise of a correct application
  of an inference rule whose conclusion stands directly below it.
\end{enumerate}
We say $S$ is the \emph{end-sequent} and that $S$ is
\emph{$\Log{LK}$-derivable}.
\end{defn}

%%% -----------------------------------------------------------------
%%% DED.4.6  Soundness (Sequent Calculus)
%%% -----------------------------------------------------------------

\subsection{Soundness} \label{DED.4.sou}

\begin{defn}[Valid Sequent]
\label{DED4-defn:valid-sequent}
A structure~$\Struct{M}$ \emph{satisfies} a sequent $\Gamma \Sequent
\Delta$ iff either $\Sat/{M}{!A}$ for some $!A \in \Gamma$ or
$\Sat{M}{!A}$ for some $!A \in \Delta$.  A sequent is \emph{valid}
iff every structure satisfies it.
\end{defn}

\begin{thm}[Soundness] % CP-001(SC)
\label{DED4-thm:sequent-soundness}
If $\Log{LK}$ derives $\Theta \Sequent \Xi$, then $\Theta \Sequent
\Xi$ is valid.
\end{thm}

\begin{proof}
Let $\pi$ be a derivation of $\Theta \Sequent \Xi$. We proceed by
induction on the number of inferences~$n$ in~$\pi$.

If $n = 0$, then $\pi$ is an initial sequent. Every initial sequent
$!A \Sequent !A$ is valid, since for every~$\Struct{M}$, either
$\Sat/{M}{!A}$ or $\Sat{M}{!A}$.  The sequents $\Sequent \ltrue$ and
$\lfalse \Sequent$ are also valid.  Identity initial sequents
$\Sequent \eq[t][t]$ are valid since $\Value{t}{M} = \Value{t}{M}$.

If $n > 0$, we distinguish cases by the last inference.  By induction
hypothesis the premise(s) are valid.

\begin{enumerate}
\item \emph{Weakening:} If the premise $\Gamma \Sequent \Delta$ is
valid, then so is $!A, \Gamma \Sequent \Delta$ (and $\Gamma \Sequent
\Delta, !A$), since any witness to the validity of $\Gamma \Sequent
\Delta$ also witnesses the validity of the weakened sequent.

\item \emph{$\LeftR{\lnot}$:} The premise is $\Gamma \Sequent \Delta,
!A$ and the conclusion is $\lnot !A, \Gamma \Sequent \Delta$. Given
$\Struct{M}$, if $\Struct{M}$ falsifies some $!C \in \Gamma$ or
satisfies some $!C \in \Delta$, the conclusion is satisfied. Otherwise,
the validity of the premise forces $\Sat{M}{!A}$, whence
$\Sat/{M}{\lnot !A}$, and $\lnot !A \in \Theta$ is falsified.

\item \emph{$\RightR{\lnot}$:} Symmetric to the $\LeftR{\lnot}$ case.

\item \emph{$\LeftR{\land}$:} The premise is $!A, \Gamma \Sequent
\Delta$ and the conclusion is $!A \land !B, \Gamma \Sequent \Delta$.
If $\Sat/{M}{!A}$, then $\Sat/{M}{!A \land !B}$, and the conclusion
is satisfied. Otherwise $\Sat{M}{!A}$, and the validity of the
premise gives the result. The case with $!B$ is analogous.

\item \emph{$\RightR{\lor}$:} The premise is $\Gamma \Sequent \Delta,
!A$ and the conclusion is $\Gamma \Sequent \Delta, !A \lor !B$. If
$\Sat{M}{!A}$ then $\Sat{M}{!A \lor !B}$. Otherwise the validity of
the premise gives a witness in $\Gamma$ or~$\Delta$.

\item \emph{$\RightR{\lif}$:} The premise is $!A, \Gamma \Sequent
\Delta, !B$ and the conclusion is $\Gamma \Sequent \Delta, !A \lif
!B$. If $\Sat/{M}{!A}$ or $\Sat{M}{!B}$, then $\Sat{M}{!A \lif !B}$.
Otherwise the validity of the premise gives a witness.

\item \emph{$\RightR{\land}$:} (Two premises.) If $\Struct{M}$ does
not satisfy $\Gamma \Sequent \Delta$, then the validity of the first
premise forces $\Sat{M}{!A}$ and the validity of the second forces
$\Sat{M}{!B}$, whence $\Sat{M}{!A \land !B}$.

\item \emph{$\LeftR{\lor}$:} (Two premises.) If $\Sat{M}{!A \lor !B}$
then either $\Sat{M}{!A}$ or $\Sat{M}{!B}$. In the former case the
validity of the left premise gives the result; in the latter, the
right premise.

\item \emph{$\LeftR{\lif}$:} (Two premises.) Suppose $\Struct{M}$
does not satisfy $\Gamma, \Pi \Sequent \Delta, \Lambda$. Then
$\Struct{M}$ satisfies neither $\Gamma \Sequent \Delta$ nor $\Pi
\Sequent \Lambda$. The validity of $\Gamma \Sequent \Delta, !A$
forces $\Sat{M}{!A}$, and the validity of $!B, \Pi \Sequent \Lambda$
forces $\Sat/{M}{!B}$. Hence $\Sat/{M}{!A \lif !B}$.

\item \emph{$\Cut$:} (Two premises.) Either $\Sat/{M}{!A}$ or
$\Sat{M}{!A}$. In the first case, $\Struct{M}$ must satisfy $\Gamma
\Sequent \Delta$ by the validity of the left premise. In the second,
$\Struct{M}$ must satisfy $\Pi \Sequent \Lambda$ by the validity of
the right premise.

\item \emph{$\LeftR{\lforall}$:} The premise is $!A(t), \Gamma
\Sequent \Delta$ and the conclusion is $\lforall[x][!A(x)], \Gamma
\Sequent \Delta$. If $\Sat{M}{\lforall[x][!A(x)]}$, then by the
substitution lemma $\Sat{M}{!A(t)}$. The validity of the premise then
gives the result. If $\Sat/{M}{\lforall[x][!A(x)]}$, the conclusion
is satisfied since $\lforall[x][!A(x)]$ is in the antecedent.

\item \emph{$\RightR{\lforall}$:} The premise is $\Gamma \Sequent
\Delta, !A(a)$ where the eigenvariable condition holds. Suppose
$\Sat{M}{\Gamma}$ and $\Sat/{M}{!C}$ for all $!C \in \Delta$. We
must show $\Sat{M}{\lforall[x][!A(x)]}$, i.e., for all~$s$,
$\Sat{M}{!A(x)}[s]$. Let $\Struct{M'}$ be like~$\Struct{M}$ except
$\Assign{a}{M'} = s(x)$. Since $a$ does not occur in $\Gamma$ or
$\Delta$, the same truth values hold in~$\Struct{M'}$. The validity
of the premise then gives $\Sat{M'}{!A(a)}$. By the substitution
lemma, $\Sat{M'}{!A(x)}[s]$, and since $a$ does not occur
in~$!A(x)$, $\Sat{M}{!A(x)}[s]$.

\item \emph{$\LeftR{\lexists}$:} Symmetric to the $\RightR{\lforall}$
case.

\item \emph{$\RightR{\lexists}$:} Symmetric to the $\LeftR{\lforall}$
case.

\item \emph{Identity rule~$\eq$:} The premise is $\eq[t_1][t_2],
\Gamma \Sequent \Delta, !A(t_1)$ and the conclusion is
$\eq[t_1][t_2], \Gamma \Sequent \Delta, !A(t_2)$. By induction
hypothesis the premise is valid. If $\Sat{M}{\eq[t_1][t_2]}$ and
$\Sat{M}{!A(t_1)}$, then $\Value{t_1}{M} = \Value{t_2}{M}$, and by
the substitution lemma $\Sat{M}{!A(t_2)}$.
\end{enumerate}
\end{proof}

\begin{cor}[Weak Soundness]
\label{DED4-cor:weak-soundness}
If $\Proves !A$ then $!A$ is valid.
\end{cor}

\begin{cor}
\label{DED4-cor:entailment-soundness}
If $\Gamma \Proves !A$ then $\Gamma \Entails !A$.
\end{cor}

\begin{proof}
If $\Gamma \Proves !A$ then for some finite $\Gamma_0 \subseteq
\Gamma$, there is a derivation of $\Gamma_0 \Sequent !A$. By
\cref{DED4-thm:sequent-soundness}, every structure~$\Struct{M}$
either falsifies some $!B \in \Gamma_0$ or satisfies~$!A$. Hence, if
$\Sat{M}{\Gamma}$ then $\Sat{M}{!A}$.
\end{proof}

\begin{cor}
\label{DED4-cor:consistency-soundness}
If $\Gamma$ is satisfiable, then it is consistent.
\end{cor}

\begin{proof}
Contrapositive: if $\Gamma$ is inconsistent, then there is a finite
$\Gamma_0 \subseteq \Gamma$ and a derivation of $\Gamma_0 \Sequent$.
By \cref{DED4-thm:sequent-soundness}, $\Gamma_0 \Sequent$ is valid,
i.e., for every $\Struct{M}$ there is $!C \in \Gamma_0$ with
$\Sat/{M}{!C}$. Hence no structure satisfies~$\Gamma$.
\end{proof}

\begin{rem}
In the sequent calculus, derivability is expressed via $\Gamma_0
\Sequent !A$. Transitivity corresponds to the $\Cut$ rule.
\end{rem}


%% ===================================================================
%% DED.5: Tableaux
%% Sources: tab/rul (KEEP), tab/prl (KEEP), tab/qrl (KEEP),
%%          tab/der (CONDENSE), tab/sou (KEEP+ABSORB sid),
%%          tab/ide (CONDENSE)
%% ===================================================================

\section{Tableaux} \label{DED.5}

Tableaux instantiate the generic framework of \S\ref{DED.1} by
working \emph{refutationally}: to show $\Gamma \Proves !A$, one
attempts to build a systematic survey of all ways the assumptions
in~$\Gamma$ could be true and~$!A$ false, and demonstrates that every
such possibility leads to a contradiction. Derivations
(\ref{PRIM-DED005}) are finitely branching trees of \emph{signed
formulas} rather than trees of plain formulas or sequents. Provability
(\ref{PRIM-DED006}) and consistency (\ref{DEF-DED001}) are defined as
in \S\ref{DED.1}: $\Gamma \Proves !A$ iff $\{\sFmla{\True}{!B_1},
\dots, \sFmla{\True}{!B_n}, \sFmla{\False}{!A}\}$ has a closed
tableau for some $!B_1, \dots, !B_n \in \Gamma$.

%%% -----------------------------------------------------------------
%%% DED.5.1  Signed Formulas and Tableau Rules
%%% -----------------------------------------------------------------

\subsection{Signed Formulas and Tableau Rules} % DEF-DED008
\label{DEF-DED008}

\begin{defn}[Signed Formula]
A \emph{signed formula} is a pair consisting of a truth value sign and
a sentence:
\[
\sFmla{\True}{!A} \quad\text{or}\quad \sFmla{\False}{!A}.
\]
\end{defn}

Intuitively, $\sFmla{\True}{!A}$ means ``$!A$ might be true'' and
$\sFmla{\False}{!A}$ means ``$!A$ might be false'' in some structure.

Each signed formula in a tableau is either an \emph{assumption} (at
the top) or obtained from a signed formula above it by a rule of
inference. There are two rules per main operator---one for sign~$\True$
and one for sign~$\False$---and some rules branch the tree.

A branch is \emph{closed} when it contains both $\sFmla{\True}{!A}$
and $\sFmla{\False}{!A}$. A \emph{closed tableau} is one where every
branch is closed. A closed tableau \emph{for $!A$} is a closed
tableau with root~$\sFmla{\False}{!A}$. If such a closed tableau
exists, all possibilities for~$!A$ being false have been ruled out.

%%% -----------------------------------------------------------------
%%% DED.5.2  Propositional Rules
%%% -----------------------------------------------------------------

\subsection{Propositional Rules}

\subsubsection*{Rules for $\lnot$}

\begin{defish}
\AxiomC{\sFmla{\True}{\lnot !A}}
\RightLabel{\TRule{\True}{\lnot}}
\UnaryInfC{\sFmla{\False}{!A}}
\DisplayProof
\hfill
\AxiomC{\sFmla{\False}{\lnot !A}}
\RightLabel{\TRule{\False}{\lnot}}
\UnaryInfC{\sFmla{\True}{!A}}
\DisplayProof
\end{defish}

\subsubsection*{Rules for $\land$}

\begin{defish}\noindent
\AxiomC{\sFmla{\True}{!A \land !B}}
\RightLabel{\TRule{\True}{\land}}
\UnaryInfC{\sFmla{\True}{!A}}
\noLine
\UnaryInfC{\sFmla{\True}{!B}}
\DisplayProof
\hfill
\AxiomC{\sFmla{\False}{!A \land !B}}
\RightLabel{\TRule{\False}{\land}}
\UnaryInfC{$\sFmla{\False}{!A} \quad \mid \quad \sFmla{\False}{!B}$}
\DisplayProof
\end{defish}

\subsubsection*{Rules for $\lor$}

\begin{defish}
\AxiomC{\sFmla{\True}{!A \lor !B}}
\RightLabel{\TRule{\True}{\lor}}
\UnaryInfC{$\sFmla{\True}{!A} \quad \mid \quad \sFmla{\True}{!B}$}
\DisplayProof
\hfill
\AxiomC{\sFmla{\False}{!A \lor !B}}
\RightLabel{\TRule{\False}{\lor}}
\UnaryInfC{\sFmla{\False}{!A}}
\noLine
\UnaryInfC{\sFmla{\False}{!B}}
\DisplayProof
\end{defish}

\subsubsection*{Rules for $\lif$}

\begin{defish}
\AxiomC{\sFmla{\True}{!A \lif !B}}
\RightLabel{\TRule{\True}{\lif}}
\UnaryInfC{$\sFmla{\False}{!A} \quad \mid \quad \sFmla{\True}{!B}$}
\DisplayProof
\hfill
\AxiomC{\sFmla{\False}{!A \lif !B}}
\RightLabel{\TRule{\False}{\lif}}
\UnaryInfC{\sFmla{\True}{!A}}
\noLine
\UnaryInfC{\sFmla{\False}{!B}}
\DisplayProof
\end{defish}

\subsubsection*{The Cut Rule}

\begin{defish}
\AxiomC{}
\RightLabel{\Cut}
\UnaryInfC{$\sFmla{\True}{!A} \quad \mid \quad \sFmla{\False}{!A}$}
\DisplayProof
\end{defish}

The $\Cut$ rule splits every branch in two. It is not necessary (any
set of signed formulas with a closed tableau has one not using $\Cut$),
but it allows convenient combination of tableaux.

%%% -----------------------------------------------------------------
%%% DED.5.3  Quantifier Rules
%%% -----------------------------------------------------------------

\subsection{Quantifier Rules}

\subsubsection*{Rules for $\lforall$}

\begin{defish}
\AxiomC{\sFmla{\True}{\lforall[x][!A(x)]}}
\RightLabel{\TRule{\True}{\forall}}
\UnaryInfC{\sFmla{\True}{!A(t)}}
\DisplayProof
\hfill
\AxiomC{\sFmla{\False}{\lforall[x][!A(x)]}}
\RightLabel{\TRule{\False}{\lforall}}
\UnaryInfC{\sFmla{\False}{!A(a)}}
\DisplayProof
\end{defish}

In $\TRule{\True}{\lforall}$, $t$ is a closed term. In
$\TRule{\False}{\lforall}$, $a$ is a constant symbol not occurring
anywhere in the branch above. We call $a$ the \emph{eigenvariable}.

\subsubsection*{Rules for $\lexists$}

\begin{defish}
\AxiomC{\sFmla{\True}{\lexists[x][!A(x)]}}
\RightLabel{\TRule{\True}{\lexists}}
\UnaryInfC{\sFmla{\True}{!A(a)}}
\DisplayProof
\hfill
\AxiomC{\sFmla{\False}{\lexists[x][!A(x)]}}
\RightLabel{\TRule{\False}{\lexists}}
\UnaryInfC{\sFmla{\False}{!A(t)}}
\DisplayProof
\end{defish}

Again, $t$ is a closed term, and $a$ is a constant symbol not
occurring in the branch above. The \emph{eigenvariable condition}
requires that $a$ not occur in the branch above the
$\TRule{\False}{\lforall}$ or $\TRule{\True}{\lexists}$ inference.

\subsubsection*{Identity Rules}

\begin{defish}
\AxiomC{}
\RightLabel{$\eq$}
\UnaryInfC{\sFmla{\True}{\eq[t][t]}}
\DisplayProof
\hfill
\AxiomC{\sFmla{\True}{\eq[t_1][t_2]}}
\noLine
\UnaryInfC{\sFmla{\True}{!A(t_1)}}
\RightLabel{$\TRule{\True}{\eq}$}
\UnaryInfC{\sFmla{\True}{!A(t_2)}}
\DisplayProof
\hfill
\AxiomC{\sFmla{\True}{\eq[t_1][t_2]}}
\noLine
\UnaryInfC{\sFmla{\False}{!A(t_1)}}
\RightLabel{$\TRule{\False}{\eq}$}
\UnaryInfC{\sFmla{\False}{!A(t_2)}}
\DisplayProof
\end{defish}

In contrast to the other rules, $\TRule{\True}{\eq}$ and
$\TRule{\False}{\eq}$ require two signed formulas already on the
branch: both $\sFmla{\True}{\eq[t_1][t_2]}$ and $\sFmla{S}{!A(t_1)}$.

%%% -----------------------------------------------------------------
%%% DED.5.4  Tableaux
%%% -----------------------------------------------------------------

\subsection{Tableaux}

\begin{defn}[Tableau]
\label{DED5-defn:tableau}
A \emph{tableau} for assumptions $\sFmla{S_1}{!A_1}$, \dots,
$\sFmla{S_n}{!A_n}$ (where each $S_i$ is $\True$ or~$\False$) is a
finite tree of signed formulas satisfying:
\begin{enumerate}
\item The $n$ topmost signed formulas are $\sFmla{S_i}{!A_i}$, one
  below the other.
\item Every signed formula not among the assumptions results from a
  correct application of an inference rule to a signed formula in the
  branch above it.
\end{enumerate}
A branch is \emph{closed} iff it contains both $\sFmla{\True}{!A}$
and~$\sFmla{\False}{!A}$, and \emph{open} otherwise. A tableau with
every branch closed is a \emph{closed tableau}; otherwise it is
\emph{open}.
\end{defn}

%%% -----------------------------------------------------------------
%%% DED.5.5  Soundness (Tableaux)
%%% -----------------------------------------------------------------

\subsection{Soundness} \label{DED.5.sou}

\begin{defn}[Satisfaction of Signed Formulas]
\label{DED5-defn:satisfies-signed}
A structure~$\Struct{M}$ \emph{satisfies} $\sFmla{\True}{!A}$ iff
$\Sat{M}{!A}$, and satisfies $\sFmla{\False}{!A}$ iff $\Sat/{M}{!A}$.
$\Struct{M}$ satisfies a set~$\Gamma$ of signed formulas iff it
satisfies every member.  $\Gamma$ is \emph{satisfiable} if some
structure satisfies it, and \emph{unsatisfiable} otherwise.
\end{defn}

\begin{thm}[Soundness] % CP-001(Tab)
\label{DED5-thm:tableau-soundness}
If $\Gamma$ has a closed tableau, $\Gamma$ is unsatisfiable.
\end{thm}

\begin{proof}
Call a branch \emph{satisfiable} iff the set of signed formulas on it
is satisfiable, and call a tableau \emph{satisfiable} if it has at
least one satisfiable branch.

We show: extending a satisfiable tableau by one rule of inference
always results in a satisfiable tableau. This proves the theorem: a
closed tableau results from applying rules to the tableau consisting
only of the assumptions from~$\Gamma$. If $\Gamma$ were satisfiable,
the initial tableau would be satisfiable, and hence every extension
would be satisfiable. But a closed tableau is clearly not satisfiable:
every branch contains both $\sFmla{\True}{!A}$ and
$\sFmla{\False}{!A}$.

Suppose we have a satisfiable tableau with a satisfiable branch to
which a rule is applied. Let $\Gamma$ be the set of signed formulas on
that branch, and let $\sFmla{S}{!A} \in \Gamma$ be the signed formula
to which the rule is applied. If the rule does not split, we show that
the extended branch is satisfiable. If the rule splits, we show at
least one resulting branch is satisfiable.

\emph{Non-splitting rules:}
\begin{enumerate}
\item \emph{$\TRule{\True}{\lnot}$ applied to
$\sFmla{\True}{\lnot !B}$:} The extended branch contains $\Gamma \cup
\{\sFmla{\False}{!B}\}$. If $\Sat{M}{\Gamma}$, then
$\Sat{M}{\lnot !B}$, so $\Sat/{M}{!B}$, i.e., $\Struct{M}$ satisfies
$\sFmla{\False}{!B}$.

\item \emph{$\TRule{\False}{\lnot}$ applied to
$\sFmla{\False}{\lnot !B}$:} If $\Sat/{M}{\lnot !B}$, then
$\Sat{M}{!B}$, so $\Struct{M}$ satisfies $\sFmla{\True}{!B}$.

\item \emph{$\TRule{\True}{\land}$ applied to
$\sFmla{\True}{!B \land !C}$:} If $\Sat{M}{!B \land !C}$, then
$\Sat{M}{!B}$ and $\Sat{M}{!C}$, so $\Struct{M}$ satisfies both new
signed formulas.

\item \emph{$\TRule{\False}{\lor}$ applied to
$\sFmla{\False}{!B \lor !C}$:} If $\Sat/{M}{!B \lor !C}$, then
$\Sat/{M}{!B}$ and $\Sat/{M}{!C}$.

\item \emph{$\TRule{\False}{\lif}$ applied to
$\sFmla{\False}{!B \lif !C}$:} If $\Sat/{M}{!B \lif !C}$, then
$\Sat{M}{!B}$ and $\Sat/{M}{!C}$.

\item \emph{$\TRule{\True}{\lforall}$ applied to
$\sFmla{\True}{\lforall[x][!A(x)]}$:} This adds
$\sFmla{\True}{!A(t)}$. If $\Sat{M}{\lforall[x][!A(x)]}$, by the
substitution lemma $\Sat{M}{!A(t)}$.

\item \emph{$\TRule{\False}{\lforall}$ applied to
$\sFmla{\False}{\lforall[x][!A(x)]}$:} This adds
$\sFmla{\False}{!A(a)}$ where $a$ does not occur in~$\Gamma$. Since
$\Sat/{M}{\lforall[x][!A(x)]}$, for some variable assignment~$s$,
$\Sat/{M}{!A(x)}[s]$. Let $\Struct{M'}$ be like~$\Struct{M}$ except
$\Assign{a}{M'} = s(x)$. Since $a$ does not occur in~$\Gamma$,
$\Struct{M'}$ still satisfies~$\Gamma$. By the substitution lemma,
$\Sat/{M'}{!A(a)}$, so $\Struct{M'}$ satisfies
$\sFmla{\False}{!A(a)}$.

\item \emph{$\TRule{\True}{\lexists}$ applied to
$\sFmla{\True}{\lexists[x][!A(x)]}$:} Symmetric to the
$\TRule{\False}{\lforall}$ case, constructing $\Struct{M'}$ with
$\Assign{a}{M'} = s(x)$ for a suitable~$s$.

\item \emph{$\TRule{\False}{\lexists}$ applied to
$\sFmla{\False}{\lexists[x][!A(x)]}$:} Symmetric to the
$\TRule{\True}{\lforall}$ case.

\item \emph{Identity rule~$\eq$} (adding $\sFmla{\True}{\eq[t][t]}$):
Trivially $\Sat{M}{\eq[t][t]}$.

\item \emph{$\TRule{\True}{\eq}$} (adding $\sFmla{\True}{!A(t_2)}$
from $\sFmla{\True}{\eq[t_1][t_2]}$ and $\sFmla{\True}{!A(t_1)}$):
Since $\Value{t_1}{M} = \Value{t_2}{M}$, the substitution lemma gives
$\Sat{M}{!A(t_2)}$. The $\TRule{\False}{\eq}$ case is similar.
\end{enumerate}

\emph{Splitting rules:}
\begin{enumerate}
\item \emph{$\TRule{\False}{\land}$ applied to
$\sFmla{\False}{!B \land !C}$:} Splits into $\sFmla{\False}{!B}$ and
$\sFmla{\False}{!C}$. If $\Sat/{M}{!B \land !C}$, then
$\Sat/{M}{!B}$ or $\Sat/{M}{!C}$; accordingly $\Struct{M}$ satisfies
the left or right branch.

\item \emph{$\TRule{\True}{\lor}$ applied to
$\sFmla{\True}{!B \lor !C}$:} Splits into $\sFmla{\True}{!B}$ and
$\sFmla{\True}{!C}$. Since $\Sat{M}{!B}$ or $\Sat{M}{!C}$, at
least one branch is satisfiable.

\item \emph{$\TRule{\True}{\lif}$ applied to
$\sFmla{\True}{!B \lif !C}$:} Splits into $\sFmla{\False}{!B}$ and
$\sFmla{\True}{!C}$. Since either $\Sat/{M}{!B}$ or $\Sat{M}{!C}$,
at least one branch is satisfiable.

\item \emph{$\Cut$:} Splits into $\sFmla{\True}{!B}$ and
$\sFmla{\False}{!B}$. Since either $\Sat{M}{!B}$ or $\Sat/{M}{!B}$,
at least one branch is satisfiable.
\end{enumerate}
\end{proof}

\begin{cor}[Weak Soundness]
\label{DED5-cor:weak-soundness}
If $\Proves !A$ then $!A$ is valid.
\end{cor}

\begin{cor}
\label{DED5-cor:entailment-soundness}
If $\Gamma \Proves !A$ then $\Gamma \Entails !A$.
\end{cor}

\begin{proof}
If $\Gamma \Proves !A$ then for some $!B_1, \dots, !B_n \in \Gamma$,
$\{\sFmla{\False}{!A}, \sFmla{\True}{!B_1}, \dots,
\sFmla{\True}{!B_n}\}$ has a closed tableau. By
\cref{DED5-thm:tableau-soundness}, every structure~$\Struct{M}$ either
falsifies some~$!B_i$ or satisfies~$!A$. Hence if
$\Sat{M}{\Gamma}$ then $\Sat{M}{!A}$.
\end{proof}

\begin{cor}
\label{DED5-cor:consistency-soundness}
If $\Gamma$ is satisfiable, then it is consistent.
\end{cor}

\begin{proof}
Contrapositive: if $\Gamma$ is inconsistent, then there are $!B_1,
\dots, !B_n \in \Gamma$ with a closed tableau for
$\{\sFmla{\True}{!B_1}, \dots, \sFmla{\True}{!B_n}\}$. By
\cref{DED5-thm:tableau-soundness}, no structure satisfies all~$!B_i$,
so $\Gamma$ is unsatisfiable.
\end{proof}

\begin{rem}
Tableau consistency---the absence of a closed tableau from $\True$-signed
assumptions---is a reformulation of the generic consistency notion of
\S\ref{DED.1}. Transitivity uses the $\Cut$ rule.
\end{rem}


%% ===================================================================
%% DED.6: Theories and Arithmetic
%% Sources: inc/int/def (DISTRIBUTE: Q, PA), inc/tcp/oqn (DEF only),
%%          inc/inp/prc (DEF-DED014)
%% ===================================================================

\section{Theories and Arithmetic} \label{DED.6}

This section introduces the formal theories of arithmetic that serve
as the principal objects of study in the incompleteness theorems (see
CH-BST, Boundedness). We define Robinson's~$\Th{Q}$, Peano
Arithmetic~$\Th{PA}$, the notion of $\omega$-consistency, and the
derivability conditions that any sufficiently strong theory must
satisfy for the incompleteness theorems to apply.

%%% -----------------------------------------------------------------
%%% DED.6.1  Robinson Arithmetic Q
%%% -----------------------------------------------------------------

\subsection{Robinson Arithmetic $\Th{Q}$}

The natural language in which to express facts of arithmetic is~$\Lang
L_A$.  $\Lang L_A$ contains a single two-place predicate symbol~$<$, a
single constant symbol~$\Obj 0$, one one-place function symbol~$\prime$,
and two two-place function symbols~$+$ and~$\times$ (see
PRIM-SYN009, Language, \S\ref{SYN.1} for the general notion of a
first-order language).

\begin{defn}[Robinson Arithmetic $\Th{Q}$] % DEF-DED011
\label{DEF-DED011}
The theory $\Th{Q}$ axiomatized by the following sentences is known
as ``Robinson's $\Th{Q}$'' and is a very simple theory of arithmetic.
\begin{align*}
& \lforall[x][\lforall[y][(\eq[x'][y'] \lif \eq[x][y])]]
  \tag{$!Q_1$}\\
& \lforall[x][\eq/[\Obj 0][x']] \tag{$!Q_2$}\\
& \lforall[x][(\eq[x][\Obj 0] \lor \lexists[y][\eq[x][y']])]
  \tag{$!Q_3$}\\
& \lforall[x][\eq[(x + \Obj 0)][x]] \tag{$!Q_4$}\\
& \lforall[x][\lforall[y][\eq[(x + y')][(x + y)']]] \tag{$!Q_5$}\\
& \lforall[x][\eq[(x \times \Obj 0)][\Obj 0]] \tag{$!Q_6$}\\
& \lforall[x][\lforall[y][\eq[(x \times y')][((x \times y) + x)]]]
  \tag{$!Q_7$}\\
& \lforall[x][\lforall[y][(x < y \liff
  \lexists[z][\eq[(z' + x)][y]])]] \tag{$!Q_8$}
\end{align*}
The sentences $\{!Q_1, \dots, !Q_8\}$ are the axioms of~$\Th{Q}$, so
\[
\Th{Q} = \Setabs{!A}{\{!Q_1, \dots, !Q_8\} \Entails !A}.
\]
\end{defn}

The axioms~$!Q_1$ and~$!Q_2$ express that the successor function is
injective with $0$ not in its range. Axiom~$!Q_3$ says every number is
either~$0$ or a successor.  Axioms~$!Q_4$--$!Q_7$ give the recursive
definitions of addition and multiplication.  Axiom~$!Q_8$ defines the
ordering in terms of addition.

$\Th{Q}$ is weak: it cannot even prove the commutativity of addition.
Its importance lies in the fact that $\Th{Q}$ is strong enough to
represent all computable functions and all decidable relations (see
\S\ref{CMP.4}), which is the key hypothesis of the incompleteness theorems.
Since any theory that extends~$\Th{Q}$ inherits this representability,
the incompleteness theorems apply to a wide class of theories.

%%% -----------------------------------------------------------------
%%% DED.6.2  Peano Arithmetic PA
%%% -----------------------------------------------------------------

\subsection{Peano Arithmetic $\Th{PA}$}

\begin{defn}[Peano Arithmetic $\Th{PA}$] % DEF-DED012
\label{DEF-DED012}
Suppose $!A(x)$ is a formula in $\Lang L_A$ with free variables~$x$
and $y_1$, \dots, $y_n$.  Then any sentence of the form
\[
\lforall[y_1][\dots\lforall[y_n][((!A(\Obj 0) \land \lforall[x][(!A(x)
\lif !A(x'))]) \lif \lforall[x][!A(x)])]]
\]
is an instance of the \emph{induction schema}.

\emph{Peano Arithmetic}~$\Th{PA}$ is the theory axiomatized by the
axioms of $\Th{Q}$ together with all instances of the induction schema.
\end{defn}

Every instance of the induction schema is true in the standard model
of arithmetic~$\Struct{N}$ (see DEF-SEM012, Standard Model, \S\ref{SEM.5}).
If $!A(x)$ defines a subset~$X_{!A}$ of~$\Nat$ in~$\Struct{N}$, then the
induction schema asserts that if $0 \in X_{!A}$ and $X_{!A}$ is closed
under the successor function, then $X_{!A} = \Nat$.

The induction schema is genuinely a \emph{schema}: it generates
infinitely many axioms, and $\Th{PA}$ is not finitely axiomatizable.
However, since one can effectively determine whether a string of
symbols is an instance of an induction axiom, the set of axioms
for~$\Th{PA}$ is decidable, and $\Th{PA}$ is an axiomatizable theory
in the sense of PRIM-DED002.

$\Th{PA}$ is a much more robust theory than~$\Th{Q}$: one can prove
the commutativity and associativity of addition and multiplication, and
in fact most finitary number-theoretic and combinatorial arguments can
be carried out in~$\Th{PA}$.

%%% -----------------------------------------------------------------
%%% DED.6.3  Omega-Consistency
%%% -----------------------------------------------------------------

\subsection{$\omega$-Consistency}

\begin{defn}[$\omega$-Consistency] % DEF-DED013
\label{DEF-DED013}
A theory $\Th{T}$ is \emph{$\omega$-consistent} if the following
holds: if $\lexists[x][!A(x)]$ is any sentence and $\Th{T}$ proves
$\lnot !A(\num 0)$, $\lnot !A(\num 1)$, $\lnot !A(\num 2)$, \dots,
then $\Th{T}$ does not prove $\lexists[x][!A(x)]$.
\end{defn}

$\omega$-consistency is strictly stronger than ordinary consistency:
every $\omega$-consistent theory is consistent, but the converse fails.
G\"odel's original 1931 proof of the first incompleteness theorem
assumed $\omega$-consistency.  Rosser subsequently strengthened the
result by replacing $\omega$-consistency with ordinary consistency (see
Rosser's Theorem, \S\ref{META.5}).

%%% -----------------------------------------------------------------
%%% DED.6.4  Derivability Conditions
%%% -----------------------------------------------------------------

\subsection{Derivability Conditions}

\begin{defn}[Derivability Conditions] % DEF-DED014 (authoritative: \S META.6)
Let $\Th{T}$ be an axiomatizable theory extending~$\Th{Q}$, and let
$\OProv[\Th{T}](y)$ be a formula representing the derivability
predicate for~$\Th{T}$ (i.e., $\OProv[\Th{T}](y) \defis
\lexists[x][\OPrf[\Th{T}](x,y)]$, where $\OPrf[\Th{T}](x,y)$
represents the proof relation).  The \emph{derivability conditions} (also
called Hilbert--Bernays--L\"ob conditions) are:
\begin{enumerate}
\item[\textbf{P1.}] If $\Th{T} \Proves !A$, then $\Th{T} \Proves
  \OProv[\Th{T}](\gn{!A})$.
\item[\textbf{P2.}] For all formulas $!A$ and $!B$,
  \[
  \Th{T} \Proves \OProv[\Th{T}](\gn{!A \lif !B}) \lif
  (\OProv[\Th{T}](\gn{!A}) \lif \OProv[\Th{T}](\gn{!B})).
  \]
\item[\textbf{P3.}] For every formula~$!A$,
  \[
  \Th{T} \Proves \OProv[\Th{T}](\gn{!A})
  \lif \OProv[\Th{T}](\gn{\OProv[\Th{T}](\gn{!A})}).
  \]
\end{enumerate}
\end{defn}

Condition P1 says that $\Th{T}$ is aware of its own theorems: if it
proves~$!A$, it proves that it proves~$!A$.  Condition P2 says that
the provability predicate distributes over the conditional, so that
$\Th{T}$ can reason internally about modus ponens.  Condition P3 is a
form of positive introspection: $\Th{T}$ can verify its own provability.

All three conditions hold for $\Th{PA}$ (and, more generally, for any
axiomatizable extension of~$\Th{Q}$ with a suitably chosen provability
predicate).  Conditions P1 and P2 are relatively easy to verify; P3
requires substantial formalization of proof theory inside~$\Th{T}$
itself.

The derivability conditions are the essential hypotheses for L\"ob's
Theorem (see THM-DED007, \S\ref{DED.7}) and the Second Incompleteness
Theorem (see CP-006, \S\ref{META.6}).


%% ===================================================================
%% DED.7: Theorems
%% Sources: axd/ded (THM-DED001 cross-ref), com/lin (THM-DED005),
%%          inc/inp/fix (THM-DED006), inc/inp/lob (THM-DED007)
%% ===================================================================

\section{Theorems} \label{DED.7}

This section collects the principal theorems of deduction theory.

%%% -----------------------------------------------------------------
%%% DED.7.1  The Deduction Theorem
%%% -----------------------------------------------------------------

\subsection{The Deduction Theorem}

\begin{thm}[Deduction Theorem] % THM-DED001
\label{THM-DED001}
$\Gamma \cup \{!A\} \Proves !B$ if and only if $\Gamma \Proves !A
\lif !B$.
\end{thm}

The deduction theorem is a fundamental metatheorem relating assumption
introduction to the conditional connective.  Its proof depends on the
proof system:
\begin{itemize}
\item In axiomatic deduction, the proof is by induction on the length
  of the derivation and uses the logical axiom $!A \lif (!B \lif !A)$
  and the distribution axiom $(!A \lif (!B \lif !C)) \lif ((!A \lif
  !B) \lif (!A \lif !C))$.  See DED.2 for the full proof.
\item In natural deduction, the deduction theorem is immediate from
  the $\Intro{\lif}$ rule, which allows one to discharge an
  assumption~$!A$ and conclude $!A \lif !B$.  See DED.3.
\item In the sequent calculus, the deduction theorem corresponds to
  the right conditional rule $\RightR{\lif}$, which moves a formula
  from the antecedent to the succedent.  See DED.4.
\end{itemize}

%%% -----------------------------------------------------------------
%%% DED.7.2  Lindenbaum's Lemma
%%% -----------------------------------------------------------------

\subsection{Lindenbaum's Lemma}

\begin{lem}[Lindenbaum's Lemma] % THM-DED005 (authoritative: \S META.2)
Every consistent set~$\Gamma$ in a language~$\Lang{L}$ can be
extended to a complete and consistent set~$\Gamma^*$.
\end{lem}

\begin{proof}
Let $\Gamma$ be consistent.  Let $!A_0, !A_1, \dots$ be an
enumeration of all the sentences of~$\Lang L$.  Define $\Gamma_0 =
\Gamma$, and
\[
\Gamma_{n+1} =
\begin{cases}
\Gamma_n \cup \{ !A_n \} & \text{if $\Gamma_n \cup \{!A_n\}$ is
  consistent;} \\
\Gamma_n \cup \{ \lnot !A_n \} & \text{otherwise.}
\end{cases}
\]
Let $\Gamma^* = \bigcup_{n \geq 0} \Gamma_n$.

Each $\Gamma_n$ is consistent: $\Gamma_0$ is consistent by definition.
If $\Gamma_{n+1} = \Gamma_n \cup \{!A_n\}$, this is because the latter
is consistent.  If it is not, $\Gamma_{n+1} = \Gamma_n \cup \{\lnot
!A_n\}$. We verify that $\Gamma_n \cup \{\lnot !A_n\}$ is consistent.
If it were not, then \emph{both} $\Gamma_n \cup \{!A_n\}$ and $\Gamma_n
\cup \{\lnot !A_n\}$ would be inconsistent. By
\cref{DED-prop:provability-exhaustive}, $\Gamma_n$ would be
inconsistent, contradicting the induction hypothesis.

For every~$n$ and every $i < n$, $\Gamma_i \subseteq \Gamma_n$. This
follows by a simple induction on~$n$.

From this it follows that $\Gamma^*$ is consistent: let $\Gamma'
\subseteq \Gamma^*$ be finite. Each $!B \in \Gamma'$ is also
in~$\Gamma_i$ for some~$i$.  Let $n$ be the largest of these. Since
$\Gamma_i \subseteq \Gamma_n$ if $i \le n$, every $!B \in \Gamma'$ is
also $\in \Gamma_n$, i.e., $\Gamma' \subseteq \Gamma_n$, and
$\Gamma_n$~is consistent.  So every finite subset $\Gamma' \subseteq
\Gamma^*$ is consistent.  By compactness
(\cref{DED-prop:proves-compact}), $\Gamma^*$ is consistent.

Every sentence of $\Frm[L]$ appears on the list used to
define~$\Gamma^*$.  If $!A_n \notin \Gamma^*$, then that is because
$\Gamma_n \cup \{!A_n\}$ was inconsistent.  But then $\lnot !A_n
\in \Gamma^*$, so $\Gamma^*$ is complete.
\end{proof}

%%% -----------------------------------------------------------------
%%% DED.7.3  The Fixed-Point Lemma
%%% -----------------------------------------------------------------

\subsection{The Fixed-Point Lemma}

The fixed-point lemma (also called the diagonal lemma or
self-referential lemma) is the engine behind the incompleteness
theorems.  It guarantees that any expressible property can be asserted
of its own G\"odel number.

\begin{lem}[Fixed-Point Lemma] % THM-DED006 (authoritative: \S META.5)
Let $!B(x)$ be any formula with one free variable~$x$. Then there is a
sentence~$!A$ such that $\Th{Q} \Proves !A \liff !B(\gn{!A})$.
\end{lem}

\begin{proof}
Given $!B(x)$, let $!E(x)$ be the formula
$\lexists[y][(!D_{\fn{diag}}(x,y) \land !B(y))]$, where
$!D_{\fn{diag}}(x,y)$ is a formula representing the primitive
recursive diagonalization function~$\fn{diag}$ in~$\Th{Q}$. The
function $\fn{diag}(n)$ computes, given the G\"odel number~$n$ of a
formula~$!E(x)$, the G\"odel number of its diagonalization
$!E(\gn{!E(x)})$.

Let $!A$ be the diagonalization of~$!E(x)$, i.e., $!A$ is
$!E(\gn{!E(x)})$.

Since $!D_{\fn{diag}}$ represents $\fn{diag}$, and
$\fn{diag}(\Gn{!E(x)}) = \Gn{!A}$, $\Th{Q}$ can derive:
\begin{align}
  & !D_{\fn{diag}}(\gn{!E(x)}, \gn{!A}) \label{repdiag1} \\
  & \lforall[y][(!D_{\fn{diag}}(\gn{!E(x)},y) \lif
  \eq[y][\gn{!A}])]. \label{repdiag2}
\end{align}
We show that $\Th{Q} \Proves !A \liff !B(\gn{!A})$, arguing informally
using logic and facts derivable in~$\Th{Q}$.

\emph{Forward direction.} Suppose~$!A$, i.e., $!E(\gn{!E(x)})$, which
by definition of~$!E(x)$ is
\[
\lexists[y][(!D_{\fn{diag}}(\gn{!E(x)},y) \land !B(y))].
\]
Consider such a~$y$. Since $!D_{\fn{diag}}(\gn{!E(x)},y)$, by
\eqref{repdiag2}, $y = \gn{!A}$. So from $!B(y)$ we
have~$!B(\gn{!A})$.

\emph{Reverse direction.} Suppose $!B(\gn{!A})$. By
\eqref{repdiag1}, we have
\[
!D_{\fn{diag}}(\gn{!E(x)}, \gn{!A}) \land !B(\gn{!A}).
\]
It follows that
\[
\lexists[y][(!D_{\fn{diag}}(\gn{!E(x)},y) \land !B(y))],
\]
which is $!E(\gn{!E(x)})$, i.e.,~$!A$.
\end{proof}

%%% -----------------------------------------------------------------
%%% DED.7.4  L\"ob's Theorem
%%% -----------------------------------------------------------------

\subsection{L\"ob's Theorem}

\begin{thm}[L\"ob's Theorem] % THM-DED007 (authoritative: \S META.6)
Let $\Th{T}$ be an axiomatizable theory extending $\Th{Q}$, and
suppose $\OProv[\Th{T}](y)$ is a formula satisfying the derivability
conditions P1--P3 (see DEF-DED014, \S\ref{DED.6}). If $\Th{T}$ derives
$\OProv[\Th{T}](\gn{!A}) \lif !A$, then in fact $\Th{T}$ derives $!A$.
\end{thm}

Equivalently: if $\Th{T} \Proves/ !A$, then $\Th{T} \Proves/
\OProv[\Th{T}](\gn{!A}) \lif !A$.  The schema
$\OProv[\Th{T}](\gn{!A}) \lif !A$ is called the \emph{reflection
principle}; L\"ob's theorem says that a consistent theory can only
derive those instances of the reflection principle where~$!A$ is
already a theorem.

\begin{proof}
Suppose $!A$ is a sentence such that $\Th{T}$ derives
$\OProv[\Th{T}](\gn{!A}) \lif !A$.  Let $!B(y)$ be the
formula~$\OProv[\Th{T}](y) \lif !A$, and use the fixed-point lemma
(THM-DED006) to find a sentence~$!D$ such that $\Th{T}$
derives $!D \liff !B(\gn{!D})$. Then each of the following is
derivable in $\Th{T}$:
\begin{align}
  & !D \liff (\OProv[\Th{T}](\gn{!D}) \lif !A) \label{L-1}\\
  & \qquad \text{$!D$ is a fixed point of~$!B(y)$}\notag \\
  & !D \lif (\OProv[\Th{T}](\gn{!D}) \lif !A) \label{L-2}\\
  & \qquad\text{from \eqref{L-1}}\notag\\
  & \OProv[\Th{T}](\gn{!D \lif (\OProv[\Th{T}](\gn{!D}) \lif !A)})
    \label{L-3}\\
  & \qquad \text{from \eqref{L-2} by condition P1}\notag \\
  & \OProv[\Th{T}](\gn{!D}) \lif
    \OProv[\Th{T}](\gn{\OProv[\Th{T}](\gn{!D}) \lif !A})
    \label{L-4}\\
  &\qquad \text{from \eqref{L-3} using condition P2}\notag \\
  & \OProv[\Th{T}](\gn{!D}) \lif
    (\OProv[\Th{T}](\gn{\OProv[\Th{T}](\gn{!D})}) \lif
    \OProv[\Th{T}](\gn{!A})) \label{L-5}\\
  &\qquad \text{from \eqref{L-4} using P2 again} \notag\\
  & \OProv[\Th{T}](\gn{!D}) \lif
    \OProv[\Th{T}](\gn{\OProv[\Th{T}](\gn{!D})}) \label{L-6}\\
  & \qquad\text{by derivability condition P3} \notag\\
  & \OProv[\Th{T}](\gn{!D}) \lif \OProv[\Th{T}](\gn{!A}) \label{L-7}\\
  &\qquad\text{from \eqref{L-5} and \eqref{L-6}}\notag\\
  & \OProv[\Th{T}](\gn{!A}) \lif !A \label{L-8}\\
  &\qquad\text{by assumption of the theorem} \notag\\
  & \OProv[\Th{T}](\gn{!D}) \lif !A \label{L-9}\\
  &\qquad\text{from \eqref{L-7} and \eqref{L-8}}\notag\\
  & (\OProv[\Th{T}](\gn{!D}) \lif !A) \lif !D \label{L-10}\\
  & \qquad \text{from \eqref{L-1}}\notag \\
  & !D \label{L-11}\\
  & \qquad\text{from \eqref{L-9} and \eqref{L-10}}\notag \\
  & \OProv[\Th{T}](\gn{!D}) \label{L-12}\\
  & \qquad\text{from \eqref{L-11} by condition P1}\notag \\
  & !A \qquad\qquad\text{from \eqref{L-8} and \eqref{L-12}}\notag
\end{align}
\end{proof}

\begin{rem}
With L\"ob's theorem in hand, there is a short proof of the second
incompleteness theorem (see CP-006, Second Incompleteness Theorem,
\S\ref{META.6}).  Take $!A \defis \lfalse$.  If $\Th{T} \Proves
\OProv[\Th{T}](\gn{\lfalse}) \lif \lfalse$, then by L\"ob's theorem
$\Th{T} \Proves \lfalse$.  Contrapositively, if $\Th{T}$ is
consistent, then $\Th{T} \Proves/ \OProv[\Th{T}](\gn{\lfalse}) \lif
\lfalse$, i.e., $\Th{T} \Proves/ \OCon[\Th{T}]$.

L\"ob's theorem also settles the status of the fixed point~$!H$ of
$\OProv[\Th{T}](x)$, i.e., a sentence~$!H$ such that $\Th{T} \Proves
\OProv[\Th{T}](\gn{!H}) \liff !H$.  Since in particular $\Th{T}
\Proves \OProv[\Th{T}](\gn{!H}) \lif !H$, L\"ob's theorem gives
$\Th{T} \Proves !H$.
\end{rem}
