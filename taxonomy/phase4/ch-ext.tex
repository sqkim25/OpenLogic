\chapter{Extensions} \label{ch:ext}

The preceding seven chapters establish a self-contained development of
classical first-order logic: from set-theoretic foundations (CH-BST)
through syntax, semantics, deduction, computation, metatheory, and formal
set theory. This final chapter catalogues the principal directions in
which the core framework can be extended. Each topic area is the subject
of extensive treatment in the OpenLogic Project source material; we
provide brief orientations and pointers.


%% ===================================================================
%% EXT.1: Modal Logic
%% Sources: 69 normal-modal-logic/ + 16 applied-modal-logic/ sections
%% ===================================================================

\section{Modal Logic} \label{EXT.1}

Modal logic extends propositional and first-order logic with operators
expressing necessity~($\Box$) and possibility~($\Diamond$). The
semantics is given by Kripke structures (relational models): a set of
worlds, an accessibility relation, and a valuation at each world. A
formula~$\Box\varphi$ holds at a world~$w$ iff $\varphi$ holds at every
world accessible from~$w$. By varying the properties of the
accessibility relation---reflexive, transitive, symmetric, Euclidean---one
obtains the standard normal modal systems K, T, S4, and S5.
Completeness, correspondence theory (frame conditions vs.\ axiom
schemas), and filtration-based decidability proofs constitute the core
metatheory. Applied extensions include epistemic logic (knowledge and
belief), temporal logic (operators for future and past), and deontic
logic (obligation and permission).

\emph{Source material}: OpenLogic \texttt{content/normal-modal-logic/}
(69~sections) and \texttt{content/applied-modal-logic/} (16~sections).


%% ===================================================================
%% EXT.2: Intuitionistic Logic
%% Sources: 34 intuitionistic-logic/ sections
%% ===================================================================

\section{Intuitionistic Logic} \label{EXT.2}

Intuitionistic logic rejects the law of excluded middle
$\varphi \lor \lnot\varphi$ and the double-negation elimination rule
$\lnot\lnot\varphi \to \varphi$, restricting provability to
constructive reasoning. Its semantics can be given via Kripke models
(with a partial order of information states), Heyting algebras, or the
Brouwer--Heyting--Kolmogorov interpretation (proofs as constructions).
Natural deduction and sequent calculus formulations are obtained by
restricting the classical rules (e.g., allowing at most one formula in
the succedent of sequent calculus sequents, or restricting
\emph{reductio ad absurdum} in ND). The propositions-as-types correspondence (Curry--Howard
isomorphism) connects intuitionistic proofs to typed lambda terms,
establishing deep links between logic and computation.

\emph{Source material}: OpenLogic \texttt{content/intuitionistic-logic/}
(34~sections).


%% ===================================================================
%% EXT.3: Many-Valued Logic
%% Sources: 25 many-valued-logic/ sections
%% ===================================================================

\section{Many-Valued Logic} \label{EXT.3}

Many-valued logics generalize classical logic by allowing truth values
beyond $\{\True, \False\}$. Three-valued logics
({\L}ukasiewicz, Kleene, Priest) introduce a third value for
``unknown,'' ``undefined,'' or ``both true and false.'' Continuous-valued
logics ({\L}ukasiewicz infinite-valued, G\"odel logic, product logic)
use the real interval $[0,1]$ as the truth-value set. The algebraic
semantics is given by MV-algebras, Heyting algebras, or BL-algebras.
Key metatheoretic results include completeness theorems for various
axiomatic systems and the McNaughton theorem characterizing
{\L}ukasiewicz logic functions as piecewise-linear functions on~$[0,1]$.

\emph{Source material}: OpenLogic \texttt{content/many-valued-logic/}
(25~sections).


%% ===================================================================
%% EXT.4: Second-Order Logic
%% Sources: 20 second-order-logic/ sections
%% ===================================================================

\section{Second-Order Logic} \label{EXT.4}

Second-order logic extends first-order logic by allowing quantification
over predicate and function variables in addition to individual
variables. Under the standard (or ``full'') semantics, second-order
quantifiers range over all subsets (or all functions) on the domain.
This dramatically increases expressive power: the natural numbers,
the real numbers, and well-orderings are categorically axiomatizable
in second-order logic. However, this power comes at a metatheoretic
cost: the Completeness Theorem fails for the standard semantics
(second-order validity is not axiomatizable), and the
L\"owenheim--Skolem theorem fails. Under Henkin semantics (where
second-order quantifiers range over a specified subcollection),
completeness and compactness are restored, but categorical
axiomatizability is lost.

\emph{Source material}: OpenLogic \texttt{content/second-order-logic/}
(20~sections).


%% ===================================================================
%% EXT.5: Lambda Calculus
%% Sources: 44 lambda-calculus/ sections
%% ===================================================================

\section{Lambda Calculus} \label{EXT.5}

The lambda calculus, introduced by Church in the 1930s, provides a
formal system for defining and applying functions. The untyped lambda
calculus uses three term constructors---variables, abstraction
($\lambda x.\, M$), and application ($M\, N$)---with
$\beta$-reduction ($(\lambda x.\, M)\, N \to M[N/x]$) as the
fundamental computation rule. The Church--Rosser theorem guarantees
confluence: if a term reduces in two different ways, both reduction
paths can be completed to a common reduct. The simply-typed lambda
calculus adds type annotations ($\lambda x{:}\alpha.\, M$) and
guarantees strong normalization (every reduction sequence terminates),
establishing a correspondence with intuitionistic propositional logic
via the Curry--Howard isomorphism. Extensions to System~F
(polymorphism), dependent types, and the calculus of constructions
connect to modern proof assistants and programming language theory.

\emph{Source material}: OpenLogic \texttt{content/lambda-calculus/}
(44~sections).


%% ===================================================================
%% EXT.6: Other Extensions
%% Sources: counterfactuals/ + methods/ + history/ + reference/
%% ===================================================================

\section{Other Extensions} \label{EXT.6}

Several additional topics in the OpenLogic Project fall outside the
core development but are of independent interest:

\begin{itemize}
\item \textbf{Counterfactual conditionals} (13~sections): Stalnaker
  and Lewis semantics for counterfactuals using sphere models and
  minimal-change semantics, addressing the failure of antecedent
  strengthening, contraposition, and transitivity for the
  counterfactual conditional.

\item \textbf{Methods} (19~sections): Proof techniques and problem
  sets designed for introductory logic courses, including induction
  templates, proof strategies, and worked examples.

\item \textbf{History of logic} (20~sections): Biographies of key
  figures (Cantor, G\"odel, Turing, Tarski, Church, Gentzen, Robinson,
  Noether, Peter, Russell, Zermelo) and historical developments in set
  theory (Cantor on the line and the plane, space-filling curves,
  infinitesimals, pathological sets).

\item \textbf{Reference material} (3~sections): Notation summaries
  and symbol tables.
\end{itemize}

\emph{Source material}: OpenLogic \texttt{content/counterfactuals/},
\texttt{content/methods/}, \texttt{content/history/}, and
\texttt{content/reference/}.
