%% further-reading.tex
%% Further Reading for "A Lean Systematization of Mathematical Logic"

\chapter*{Further Reading}
\addcontentsline{toc}{chapter}{Further Reading}

\noindent
This text is compiled from the Open Logic Project
(\texttt{openlogicproject.org}).  The following sources provide deeper
treatment of each domain.

\subsection*{General Logic Textbooks}

\begin{itemize}
\item H.~B.\ Enderton, \textit{A Mathematical Introduction to Logic},
  2nd ed., Academic Press, 2001.  Covers propositional and
  first-order logic through completeness with careful attention to
  detail.

\item E.\ Mendelson, \textit{Elements of Mathematical Logic},
  6th ed., CRC Press, 2015.  A classic treatment including
  axiomatics, first-order logic, and incompleteness.

\item D.\ Marker, \textit{Model Theory: An Introduction}, Springer,
  2002.  Graduate-level model theory; excellent for the semantic and
  metatheoretic material of CH-SEM and CH-META.
\end{itemize}

\subsection*{Set Theory (CH-BST, CH-SET)}

\begin{itemize}
\item P.~R.\ Halmos, \textit{Naive Set Theory}, Springer, 1960.
  The foundational informal set theory underlying CH-BST.

\item K.\ Kunen, \textit{Set Theory: An Introduction to Independence
  Proofs}, North-Holland, 1980.  Standard graduate reference for the
  formal set theory of CH-SET.

\item T.\ Jech, \textit{Set Theory}, 3rd millennium ed., Springer,
  2003.  Comprehensive reference for ordinals, cardinals, and the
  cumulative hierarchy.
\end{itemize}

\subsection*{Syntax and Semantics (CH-SYN, CH-SEM)}

\begin{itemize}
\item W.\ Hodges, \textit{A Shorter Model Theory}, Cambridge
  University Press, 1997.  Concise introduction to structures,
  satisfaction, and elementary equivalence.

\item C.~C.\ Chang and H.~J.\ Keisler, \textit{Model Theory},
  3rd ed., North-Holland, 1990.  Definitive reference for
  L\"owenheim--Skolem, compactness, and ultraproducts.
\end{itemize}

\subsection*{Deduction (CH-DED)}

\begin{itemize}
\item A.~S.\ Troelstra and H.\ Schwichtenberg, \textit{Basic Proof
  Theory}, 2nd ed., Cambridge University Press, 2000.  The standard
  reference for natural deduction and sequent calculus, including
  cut elimination.

\item R.~M.\ Smullyan, \textit{First-Order Logic}, Dover, 1995
  (reprint).  The original systematic treatment of analytic tableaux.

\item S.~R.\ Buss (ed.), \textit{Handbook of Proof Theory},
  North-Holland, 1998.  Comprehensive survey of all major proof
  systems.
\end{itemize}

\subsection*{Computability (CH-CMP)}

\begin{itemize}
\item N.~J.\ Cutland, \textit{Computability: An Introduction to
  Recursive Function Theory}, Cambridge University Press, 1980.
  Clear development of primitive and general recursive functions,
  Turing machines, and undecidability.

\item R.~I.\ Soare, \textit{Recursively Enumerable Sets and
  Degrees}, Springer, 1987.  Advanced treatment of the computability
  theory underlying representability and Church's thesis.

\item H.\ Rogers, \textit{Theory of Recursive Functions and
  Effective Computability}, MIT Press, 1987.  The classic reference
  for recursion theory.
\end{itemize}

\subsection*{Metatheory and Incompleteness (CH-META)}

\begin{itemize}
\item G.~S.\ Boolos, J.~P.\ Burgess, and R.~C.\ Jeffrey,
  \textit{Computability and Logic}, 5th ed., Cambridge University
  Press, 2007.  Excellent coverage of G\"odel's theorems,
  representability, and Tarski's theorem.

\item C.\ Smorynski, \textit{Self-Reference and Modal Logic},
  Springer, 1985.  Deep treatment of the derivability conditions,
  L\"ob's theorem, and the provability logic GL.

\item P.\ Lindstr\"om, ``On extensions of elementary logic,''
  \textit{Theoria}, 35:1--11, 1969.  The original paper on
  Lindstr\"om's theorem characterizing first-order logic.
\end{itemize}

\subsection*{The Open Logic Project}

\begin{itemize}
\item Open Logic Project, \textit{Open Logic: A Complete Text},
  available at \texttt{openlogicproject.org}.  The upstream source
  of all material in this volume, released under a Creative Commons
  license.

\item The taxonomy and dependency analysis underlying this
  systematization are documented in the \texttt{taxonomy/} directory
  of the project repository.
\end{itemize}
