\chapter{Semantics} \label{ch:sem}

%% ===================================================================
%% SEM.1: Structures
%% Sources: fol/syn/str (KEEP), fol/syn/cov (CONDENSE)
%% ===================================================================

\section{Structures} \label{SEM.1}

First-order languages are, by themselves, \emph{uninterpreted}: the
constant symbols, function symbols, and predicate symbols have no specific
meaning attached to them.  Meanings are given by specifying
a \emph{structure}. It specifies the
\emph{domain}, i.e., the objects which the constant symbols pick out, the
function symbols operate on, and the quantifiers range over. In addition,
it specifies which constant symbols pick out which objects, how
a function symbol maps objects to objects, and which objects the
predicate symbols apply to.  Structures are the basis for
\emph{semantic} notions in logic, e.g., the notion of consequence,
validity, satisfiability. They are variously called ``structures,''
``interpretations,'' or ``models'' in the literature.

\begin{defn}[Structures] % PRIM-SEM001, PRIM-SEM002, PRIM-SEM003
\label{PRIM-SEM001}
A \emph{structure}~$\Struct M$ for a language
$\Lang{L}$ of first-order logic consists of the following elements:
\begin{enumerate}
\item \emph{Domain:} a non-empty set, $\Domain M$. \label{PRIM-SEM002}
\item \emph{Interpretation of constant symbols:} for each constant symbol~$c$ of
  $\Lang{L}$, an element $\Assign{c}{M} \in \Domain M$.
  \label{PRIM-SEM003}
\item \emph{Interpretation of predicate symbols:} for each $n$-place
  predicate symbol~$R$ of $\Lang{L}$ (other than $\eq$), an $n$-place
  relation $\Assign{R}{M} \subseteq \Domain{M}^n$.
\item \emph{Interpretation of function symbols:} for each $n$-place
  function symbol~$f$ of $\Lang{L}$, an $n$-place function $\Assign{f}{M}
  \colon \Domain{M}^n \to \Domain{M}$.
\end{enumerate}
\end{defn}

\begin{ex}
A structure~$\Struct M$ for the language of arithmetic consists of a
set, an element of $\Domain M$, $\Assign{\Obj 0}{M}$, as
interpretation of the constant symbol~$\Obj 0$, a one-place function
$\Assign{\Obj \prime}{M} \colon \Domain{M} \to \Domain M$, two
two-place functions $\Assign{\Obj +}{M}$ and $\Assign{\Obj
  \times}{M}$, both $\Domain M^2 \to \Domain M$, and a two-place
relation $\Assign{\Obj <}{M} \subseteq \Domain{M}^2$.

An obvious example of such a structure is the following:
\begin{enumerate}
\item $\Domain N = \Nat$
\item $\Assign{\Obj 0}{N} = 0$
\item $\Assign{\Obj \prime}{N}(n) = n + 1$ for all $n \in \Nat$
\item $\Assign{\Obj +}{N}(n, m) = n + m$ for all $n, m \in \Nat$
\item $\Assign{\Obj \times}{N}(n, m) = n\cdot m$ for all $n, m \in \Nat$
\item $\Assign{\Obj <}{N} = \Setabs{\tuple{n, m}}{n \in \Nat, m \in
  \Nat, n < m}$
\end{enumerate}
The structure~$\Struct N$ for $\Lang L_A$ so defined is called the
\emph{standard model of arithmetic}, because it interprets the
non-logical constants of~$\Lang L_A$ exactly how you would expect.
\end{ex}

The stipulations we make as to what counts as a structure impact
our logic. For example, the choice to prevent empty domains ensures
that $\lexists[x][(!A(x) \lor \lnot !A(x))]$ is
valid. Allowing empty domains or names that do not refer leads to
\emph{free logic}.\footnote{In free logic, existential generalization
requires an additional premise: $!A(a)$ and $\lexists[x][\eq[x][a]]$,
therefore $\lexists[x][!A(x)]$.}

%%% Value of Closed Terms (from fol/syn/cov — CONDENSE)

We can assign values to closed terms (terms containing no variables)
using only the structure, without needing a variable assignment.

\begin{defn}[Value of closed terms] % PRIM-SEM006 (closed-term case)
\label{PRIM-SEM006:closed}
If $t$ is a closed term of the language~$\Lang L$ and $\Struct M$ is a
structure for~$\Lang L$, the \emph{value}~$\Value{t}{M}$ is
defined as follows:
\begin{enumerate}
\item If $t$ is just the constant symbol~$c$, then $\Value{c}{M} = \Assign{c}{M}$.
\item If $t$ is of the form $\Atom{f}{t_1, \ldots, t_n}$, then
  \[
  \Value{t}{M} = \Assign{f}{M}(\Value{t_1}{M}, \ldots,
  \Value{t_n}{M}).
  \]
\end{enumerate}
\end{defn}

A structure is \emph{covered} if every element of the domain is the
value of some closed term.


%% ===================================================================
%% SEM.2: Satisfaction and Truth
%% Sources: fol/syn/sat (ABSORB:pl/syn/val), fol/syn/ass (CONDENSE)
%% ===================================================================

\section{Satisfaction and Truth} \label{SEM.2}

The basic notion that relates expressions such as terms and
formulas, on the one hand, and structures on the other, are
those of \emph{value} of a term and \emph{satisfaction} of
a formula.  Informally, the value of a term is an element of
a structure---if the term is just a constant, its value is the
object assigned to the constant by the structure, and if it is
built up using function symbols, the value is computed from the
values of constants and the functions assigned to the functions in
the term.  A formula is \emph{satisfied} in a structure if the
interpretation given to the predicates makes the formula true in
the domain of the structure. This notion of satisfaction is
specified inductively: the specification of the structure directly
states when atomic formulas are satisfied, and we define when a
complex formula is satisfied depending on the main connective or
quantifier and whether or not the immediate subformulas are
satisfied.

The case of the quantifiers is a bit tricky, as the
immediate subformula of a quantified formula has a free
variable, and structures don't specify the values of
variables.  In order to deal with this difficulty, we
introduce \emph{variable assignments} and define satisfaction not with
respect to a structure alone, but with respect to a structure
plus a variable assignment.

\begin{defn}[Variable Assignment] % PRIM-SEM004
\label{PRIM-SEM004}
A \emph{variable assignment}~$s$ for a structure~$\Struct{M}$ is a
function which maps each variable to an element of~$\Domain M$,
i.e., $s\colon \Var \to \Domain M$.
\end{defn}

A structure assigns a value to each constant symbol, and a
variable assignment to each variable.  But we want to use terms built
up from them to also name elements of the domain.  For this we
define the value of terms inductively. For constant symbols and
variables the value is just as the structure or the variable
assignment specifies it; for more complex terms it is computed
recursively using the functions the structure assigns to the
function symbols.

\begin{defn}[Value of Terms] % PRIM-SEM006
\label{PRIM-SEM006}
If $t$ is a term of the language~$\Lang L$, $\Struct M$ is a
structure for~$\Lang L$, and $s$ is a variable assignment
for~$\Struct M$, the \emph{value}~$\Value{t}{M}[s]$ is defined as
follows:
\begin{enumerate}
\item If $t$ is a constant symbol~$c$, then $\Value{c}{M}[s] = \Assign{c}{M}$.
\item If $t$ is a variable~$x$, then $\Value{x}{M}[s] = s(x)$.
\item If $t = \Atom{f}{t_1, \ldots, t_n}$, then
\[
\Value{t}{M}[s] = \Assign{f}{M}(\Value{t_1}{M}[s], \ldots,
\Value{t_n}{M}[s]).
\]
\end{enumerate}
\end{defn}

\begin{defn}[$x$-Variant] % PRIM-SEM005
\label{PRIM-SEM005}
If $s$ is a variable assignment for a structure~$\Struct M$, then any
variable assignment~$s'$ for~$\Struct M$ which differs from~$s$ at most
in what it assigns to~$x$ is called an \emph{$x$-variant} of~$s$.  If
$s'$ is an $x$-variant of~$s$ we write $\varAssign{s'}{s}{x}$.
\end{defn}

Note that an $x$-variant of an assignment~$s$ does not \emph{have} to
assign something different to~$x$.  In fact, every assignment counts
as an $x$-variant of itself.

\begin{defn}
  If $s$ is a variable assignment for a structure~$\Struct M$
  and $m \in \Domain{M}$, then the assignment~$\Subst{s}{m}{x}$ is the
  variable assignment defined by
  \[\Subst{s}{m}{x}(y) = \begin{cases}
    m & \text{if } y \ident x\\
    s(y) & \text{otherwise}.
  \end{cases}\]
\end{defn}

In other words, $\Subst{s}{m}{x}$ is the particular $x$-variant of~$s$
which assigns the domain element~$m$ to~$x$, and assigns the same
things to variables other than~$x$ that $s$ does.

\begin{defn}[Satisfaction] % PRIM-SEM007, DEF-SEM001
\label{PRIM-SEM007}
\label{DEF-SEM001}
Satisfaction of a formula~$!A$ in a structure~$\Struct M$
relative to a variable assignment~$s$, in symbols:
$\Sat{M}{!A}[s]$, is defined recursively as follows. (We write
$\Sat/{M}{!A}[s]$ to mean ``not $\Sat{M}{!A}[s]$.'')
\begin{enumerate}
\item $\Sat{M}{\Atom{R}{t_1, \dots, t_n}}[s]$
  iff $\langle \Value{t_1}{M}[s], \dots, \Value{t_n}{M}[s] \rangle \in
  \Assign{R}{M}$.

\item $\Sat{M}{\eq[t_1][t_2]}[s]$ iff
  $\Value{t_1}{M}[s] = \Value{t_2}{M}[s]$.

\item $\Sat{M}{\lnot !B}[s]$ iff
    $\Sat/{M}{!B}[s]$.

\item $\Sat{M}{(!B \land !C)}[s]$ iff $\Sat{M}{!B}[s]$
    and $\Sat{M}{!C}[s]$.

\item $\Sat{M}{(!B \lor !C)}[s]$ iff
    $\Sat{M}{!B}[s]$ or $\Sat{M}{!C}[s]$ (or both).

\item $\Sat{M}{(!B \lif !C)}[s]$ iff $\Sat/{M}{!B}[s]$
    or $\Sat{M}{!C}[s]$ (or both).

\item $\Sat{M}{(!B \liff !C)}[s]$ iff either both
    $\Sat{M}{!B}[s]$ and $\Sat{M}{!C}[s]$, or neither $\Sat{M}{!B}[s]$
    nor $\Sat{M}{!C}[s]$.

\item $\Sat{M}{\lforall[x][!B]}[s]$ iff for every
    element~$m \in \Domain M$, $\Sat{M}{!B}[\Subst{s}{m}{x}]$.

\item $\Sat{M}{\lexists[x][!B]}[s]$ iff for at least
  one element~$m \in \Domain M$, $\Sat{M}{!B}[\Subst{s}{m}{x}]$.
\end{enumerate}
\end{defn}

The variable assignments are important in the quantifier clauses. We
cannot define satisfaction of $\lforall[x][!B(x)]$ by ``for all $m \in
\Domain{M}$, $\Sat{M}{!B(m)}$,'' because if $m \in
\Domain M$, it is not a symbol of the language, and so $!B(m)$~is not
a formula (that is, $\Subst{!B}{m}{x}$ is undefined).  We also
cannot assume that we have constant symbols or terms available that name
every element of~$\Struct{M}$, since there is nothing in the
definition of structures that requires it.  Variable assignments
allow us to link variables directly with elements of the domain,
resolving this difficulty.

\begin{rem}[PL Specialization] % PRIM-SEM015
\label{PRIM-SEM015}
For propositional logic, a \emph{truth-value assignment} (or
\emph{valuation}) $v \colon \mathrm{PropVar} \to \{0,1\}$ replaces
the structure-plus-variable-assignment pair. Satisfaction reduces to:
$v \vDash p$ iff $v(p) = 1$; the connective clauses are the same as
above (without quantifier clauses). Propositional satisfaction is
thus the quantifier-free fragment of first-order satisfaction.
\end{rem}

%%% Satisfaction depends only on free variables (from fol/syn/ass — CONDENSE)

The value of a term~$t$, and whether or not a formula~$!A$ is
satisfied in a structure with respect to~$s$, only depend on the
assignments~$s$ makes to the variables in~$t$ and the free
variables of~$!A$.

The value of a term depends only on the values of its variables:
if $s_1$ and $s_2$ agree on all variables occurring in~$t$, then
$\Value{t}{M}[s_1] = \Value{t}{M}[s_2]$.

\begin{prop}
\label{prop:satindep}
If the free variables in $!A$ are among $x_1$, \dots,~$x_n$, and
$s_1(x_i) = s_2(x_i)$ for $i = 1$, \dots,~$n$, then $\Sat{M}{!A}[s_1]$
iff $\Sat{M}{!A}[s_2]$.
\end{prop}

\begin{proof}[Proof sketch]
By induction on the complexity of $!A$. The base case uses the
corresponding result for terms. For connectives, apply the induction
hypothesis to subformulas (whose free variables are among those
of~$!A$). For the quantifier case $\lexists[x][!B]$: if
$\Sat{M}{!B}[\Subst{s_1}{m}{x}]$ for some~$m$, then
$\Subst{s_1}{m}{x}$ and $\Subst{s_2}{m}{x}$ agree on the free
variables of~$!B$ (which are among $x_1, \ldots, x_n, x$), so the
induction hypothesis gives
$\Sat{M}{!B}[\Subst{s_2}{m}{x}]$. Similarly for $\lforall[x][!B]$.
\end{proof}

Sentences have no free variables, so any two variable assignments
assign the same things to all the (zero) free variables of any
sentence. Therefore:

\begin{cor}
\label{cor:sat-sentence}
If $!A$ is a sentence and $s$ a variable assignment, then
$\Sat{M}{!A}[s]$ iff $\Sat{M}{!A}[s']$ for every variable
assignment~$s'$.
\end{cor}

This justifies the following definition.

\begin{defn}[Truth in a Structure] % PRIM-SEM008
\label{PRIM-SEM008}
If $!A$ is a sentence, we say that a structure~$\Struct M$
\emph{satisfies}~$!A$, $\Sat{M}{!A}$, iff $\Sat{M}{!A}[s]$ for all
variable assignments~$s$.
\end{defn}

If $\Sat{M}{!A}$, we also simply say that \emph{$!A$ is true
in~$\Struct{M}$.}

\begin{defn}[Satisfaction for sets of sentences] % PRIM-SEM011 (alternative formulation)
\label{defn:sat-set}
If $\Gamma$ is a set of sentences, we say that
a structure~$\Struct M$ \emph{satisfies}~$\Gamma$,
$\Sat{M}{\Gamma}$, iff $\Sat{M}{!A}$ for all $!A \in \Gamma$.
\end{defn}

\begin{prop}
\label{prop:sentence-sat-true}
  Let $\Struct{M}$ be a structure, $!A$ be a sentence, and $s$ a
  variable assignment.  $\Sat{M}{!A}$ iff $\Sat{M}{!A}[s]$.
\end{prop}


%% ===================================================================
%% SEM.3: Validity and Consequence
%% Sources: fol/syn/sem (ABSORB:pl/syn/sem)
%% ===================================================================

\section{Validity and Consequence} \label{SEM.3}

Given the definition of structures for first-order languages, we can
define some basic semantic properties of and relationships between
sentences.  The simplest of these is the notion of \emph{validity} of
a sentence.  A sentence is valid if it is satisfied in every
structure.  Valid sentences are those that are satisfied regardless of
how the non-logical symbols in it are interpreted.  Valid sentences
are therefore also called \emph{logical truths}---they are true, i.e.,
satisfied, in any structure and hence their truth depends only on the
logical symbols occurring in them and their syntactic structure, but not
on the non-logical symbols or their interpretation.

\begin{defn}[Validity] % PRIM-SEM009
\label{PRIM-SEM009}
A sentence $!A$ is \emph{valid}, $\Entails !A$, iff $\Sat{M}{!A}$ for every
structure~$\Struct M$.
\end{defn}

\begin{defn}[Entailment] % PRIM-SEM010
\label{PRIM-SEM010}
A set of sentences~$\Gamma$ \emph{entails} a sentence~$!A$, $\Gamma
\Entails !A$, iff for every structure~$\Struct M$ with
$\Sat{M}{\Gamma}$, $\Sat{M}{!A}$.
\end{defn}

\begin{defn}[Satisfiability] % DEF-SEM002
\label{DEF-SEM002}
A set of sentences~$\Gamma$ is \emph{satisfiable} if $\Sat{M}{\Gamma}$
for some structure~$\Struct M$.  If $\Gamma$ is not satisfiable it is
called \emph{unsatisfiable}.
\end{defn}

\begin{defn}[Model] % PRIM-SEM011
\label{PRIM-SEM011}
Let $\Gamma$ be a set of sentences in a language~$\Lang L$.  We
say that a structure~$\Struct M$ \emph{is a model of}~$\Gamma$ if
$\Sat{M}{!A}$ for all $!A \in \Gamma$.
\end{defn}

\begin{defn}[Semantic Consistency] % DEF-SEM004
\label{DEF-SEM004}
A set of sentences~$\Gamma$ is \emph{semantically consistent} if it is
satisfiable, i.e., if it has at least one model. This is the semantic
counterpart of syntactic consistency (see \S\ref{DED.1}).
\end{defn}

We record some basic relationships:

\begin{itemize}
\item $\Gamma \Entails !A$ iff $\Gamma \cup \{\lnot !A\}$ is unsatisfiable.
\item (Monotonicity) If $\Gamma \subseteq \Gamma'$ and $\Gamma \Entails !A$,
  then $\Gamma' \Entails !A$.
\item (Semantic Deduction Theorem) $\Gamma \cup \{!A\} \Entails !B$ iff
  $\Gamma \Entails !A \lif !B$.
\end{itemize}

\begin{rem}[PL Specialization: Tautology] % DEF-SEM009
\label{DEF-SEM009}
A propositional formula is a \emph{tautology} if it is true under
every truth-value assignment. This specializes first-order validity to
propositional structures: a tautology is a formula valid under all
valuations $v \colon \mathrm{PropVar} \to \{0,1\}$.
\end{rem}

\begin{rem}[PL Specialization: PL Consequence] % DEF-SEM010
\label{DEF-SEM010}
PL entailment $\Gamma \vDash_{\mathrm{PL}} !A$ specializes first-order
entailment to propositional structures: every truth-value assignment
making all formulas in~$\Gamma$ true also makes~$!A$ true.
\end{rem}


%% ===================================================================
%% SEM.4: Models and Theories
%% Sources: fol/mat/exs (CONDENSE), fol/mat/int (DISTRIBUTE — DEF-SEM007)
%% ===================================================================

\section{Models and Theories} \label{SEM.4}

\begin{defn}[Semantic Completeness of a Theory] % DEF-SEM005
\label{DEF-SEM005}
A theory $T$ is \emph{semantically complete} iff for every
sentence~$!A$ in its language, either $T \Entails !A$ or
$T \Entails \lnot !A$.
\end{defn}

Note that semantic completeness of a theory is \emph{not} the same as
the Completeness Theorem (see \S\ref{META.2}), which states that
$\Gamma \Entails !A$ implies $\Gamma \Proves !A$.

\begin{defn}[Theory of a Structure] % DEF-SEM006
\label{DEF-SEM006}
Given a structure~$\Struct M$, the \emph{theory} of
$\Struct{M}$ is the set $\Theory{M}$ of sentences
that are true in $\Struct{M}$, i.e., $\Theory{M} =
\Setabs{!A}{\Sat{M}{!A}}$.
\end{defn}

We also use the term ``theory'' informally to refer to sets
of sentences having an intended interpretation, whether deductively
closed or not.

\begin{prop}
\label{prop:ThM-complete}
For any $\Struct{M}$, $\Theory{M}$ is semantically complete.
\end{prop}

\begin{proof}
For any sentence~$!A$ either $\Sat{M}{!A}$ or $\Sat{M}{\lnot !A}$,
so either $!A \in \Theory{M}$ or $\lnot !A \in \Theory{M}$.
\end{proof}

\begin{defn}[Definable Set] % DEF-SEM007
\label{DEF-SEM007}
A subset of $\Domain{M}^n$ is \emph{definable} (in $\Struct{M}$)
if there is a formula $!A(x_1, \ldots, x_n)$ with free variables
$x_1, \ldots, x_n$ such that the subset equals
$\Setabs{\langle a_1, \ldots, a_n \rangle \in
\Domain{M}^n}{\Sat{M}{!A}[\Subst{s}{a_1}{x_1}, \ldots,
\Subst{s}{a_n}{x_n}]}$.
\end{defn}

\begin{defn}[Elementary Equivalence] % DEF-SEM008
\label{DEF-SEM008}
Given two structures $\Struct{M}$ and $\Struct M'$ for the same
language~$\Lang{L}$, we say that $\Struct{M}$ is \emph{elementarily
  equivalent to} $\Struct M'$, written $\Struct{M} \equiv \Struct M'$,
if and only if for every sentence~$!A$ of~$\Lang{L}$,
$\Sat{M}{!A}$ iff $\Sat{M'}{!A}$.
\end{defn}

\begin{prop}
\label{prop:equiv}
  If $\Sat{N}{!A}$ for every $!A \in \Theory{M}$, then
  $\Struct{M} \equiv \Struct{N}$.
\end{prop}

\begin{proof}
Since $\Sat{N}{!A}$ for all $!A \in \Theory{M}$, $\Theory{M} \subseteq
\Theory{N}$. If $\Sat{N}{!A}$, then $\Sat/{N}{\lnot !A}$, so $\lnot !A
\notin \Theory{M}$. Since $\Theory{M}$ is complete, $!A \in
\Theory{M}$. So, $\Theory{N} \subseteq \Theory{M}$, and we have
$\Struct{M} \equiv \Struct{N}$.
\end{proof}

\begin{rem}
\label{rem:R}
  Consider $\Struct{R} = \langle\Real, <\rangle$, the structure
  whose domain is the set $\Real$ of the real numbers, in the language
  comprising only a 2-place predicate symbol interpreted as the $<$
  relation over the reals. Clearly $\Struct{R}$ is uncountable;
  however, since $\Theory{R}$ is obviously consistent, by the
  L\"owenheim--Skolem theorem (see \S\ref{META.4}) it has a countable model, say
  $\Struct{S}$, and by the above proposition, $\Struct{R}
  \equiv \Struct{S}$. Moreover, since $\Struct{R}$ and $\Struct{S}$
  are not isomorphic, this shows that the converse of
  the Isomorphism Lemma (Theorem~\ref{THM-SEM001}) fails in general:
  elementary equivalence does not imply isomorphism.
\end{rem}

\begin{defn}[Axiomatized Theory]
\label{defn:axiomatized-theory}
A set of sentences~$\Gamma$ is \emph{closed} iff, whenever
$\Gamma \Entails !A$ then $!A \in \Gamma$.  The \emph{closure} of a set
of sentences~$\Gamma$ is $\Setabs{!A}{\Gamma \Entails !A}$.
We say that~$\Gamma$ is \emph{axiomatized by} a set of
sentences~$\Delta$ if $\Gamma$ is the closure of~$\Delta$.
\end{defn}


%% ===================================================================
%% SEM.5: Arithmetic Models
%% Sources: inc/int/def (DISTRIBUTE — DEF-SEM017, DEF-SEM018),
%%          inc/tcp/itp (CONDENSE — DEF-SEM019 remark),
%%          mod/mar/stm, mod/mar/nst, mod/mar/mpa, mod/mar/cmp
%% ===================================================================

\section{Arithmetic Models} \label{SEM.5}

The structures studied in this section are models of the first-order
theories of arithmetic ($\Th{Q}$, $\Th{PA}$) introduced in
\S\ref{DED.6}.

\begin{defn}[Standard Model of Arithmetic] % DEF-SEM017
\label{DEF-SEM017}
The \emph{standard model of arithmetic} is the structure~$\Struct
N$ defined as follows:
\begin{enumerate}
\item $\Domain N = \Nat$
\item $\Assign{\Obj 0}{N} = 0$
\item $\Assign{\Obj \prime}{N}(n) = n + 1$ for all $n \in \Nat$
\item $\Assign{\Obj +}{N}(n, m) = n + m$ for all $n, m \in \Nat$
\item $\Assign{\Obj \times}{N}(n, m) = n\cdot m$ for all $n, m \in \Nat$
\item $\Assign{\Obj <}{N} = \Setabs{\tuple{n, m}}{n \in \Nat, m \in
  \Nat, n < m}$
\end{enumerate}
\end{defn}

\begin{defn}[True Arithmetic] % DEF-SEM018
\label{DEF-SEM018}
The theory of \emph{true arithmetic} is the set of sentences
satisfied in the standard model of arithmetic, i.e.,
\[
\Th{TA} = \Setabs{!A}{\Sat{N}{!A}}.
\]
\end{defn}

$\Th{TA}$ is a theory (closed under entailment), for whenever $\Th{TA} \Entails !A$, $!A$ is
satisfied in every structure which satisfies~$\Th{TA}$. Since
$\Sat{N}{\Th{TA}}$, we have that~$\Sat{N}{!A}$, and so $!A \in \Th{TA}$.

\begin{rem}[Interpretability] % DEF-SEM019
\label{DEF-SEM019}
Informally, an interpretation of a language $\Lang{L_1}$ in another
language $\Lang{L_2}$ involves defining the universe, relation symbols,
and function symbols of $\Lang{L_1}$ with formulas in $\Lang{L_2}$.
One can show: if a theory~$\Th{T}$ is consistent with the
interpretation of Robinson's $\Th{Q}$ (see \S\ref{DED.6}), then $\Th{T}$ is
undecidable, and no consistent extension of $\Th{T}$ is decidable. In
particular, there is no decidable or complete consistent axiomatizable extension of $\Th{ZFC}$.
\end{rem}


%% ===================================================================
%% SEM.6: Model-Theoretic Concepts
%% Sources: mod/bas/red, mod/bas/sub, mod/bas/iso, mod/bas/thm,
%%          mod/bas/dlo, mod/bas/ove
%% ===================================================================

\section{Model-Theoretic Concepts} \label{SEM.6}

%%% Reducts and Expansions (from mod/bas/red — CONDENSE)

\subsection*{Reducts and Expansions}

Often it is useful or necessary to compare languages which have
symbols in common, as well as structures for these languages.  The
most common case is when all the symbols in a language~$\Lang{L}$
are also part of a language~$\Lang{L'}$, i.e., $\Lang{L} \subseteq
\Lang{L'}$.

\begin{defn}[Reduct and Expansion]
\label{defn:reduct}
Suppose $\Lang L \subseteq \Lang L'$, $\Struct M$ is an
$\Lang L$-structure and $\Struct M'$ is an $\Lang L'$-structure.
$\Struct M$ is the \emph{reduct} of $\Struct M'$ to $\Lang L$, and
$\Struct M'$ is an \emph{expansion} of $\Struct M$ to $\Lang L'$ iff
\begin{enumerate}
\item $\Domain{M} = \Domain{M'}$;
\item For every constant symbol~$c \in \Lang L$, $\Assign{c}{M} =
  \Assign{c}{M'}$;
\item For every function symbol~$f \in \Lang L$, $\Assign{f}{M} =
  \Assign{f}{M'}$;
\item For every predicate symbol~$P \in \Lang L$, $\Assign{P}{M} =
  \Assign{P}{M'}$.
\end{enumerate}
\end{defn}

If $\Struct{M}$ is a reduct of $\Struct{M'}$, then for all
$\Lang{L}$-sentences~$!A$, $\Sat{M}{!A}$ iff $\Sat{M'}{!A}$.

When we have an $\Lang{L}$-structure $\Struct{M}$, and $\Lang{L'} =
\Lang{L} \cup \{P\}$ is the expansion of $\Lang{L}$ obtained by adding
a single $n$-place predicate symbol~$P$, and $R \subseteq \Domain{M}^n$
is an $n$-place relation, then we write $\Expan{M}{R}$ for the
expansion~$\Struct{M'}$ of~$\Struct{M}$ with $\Assign{P}{M'} = R$.

%%% Substructures (from mod/bas/sub — KEEP)

\subsection*{Substructures}

The domain of a structure~$\Struct{M}$ may be a subset of
another~$\Struct{M'}$.  But we should obviously only consider
$\Struct{M}$ a ``part'' of $\Struct{M'}$ if not only $\Domain{M}
\subseteq \Domain{M'}$, but $\Struct{M}$ and $\Struct{M'}$ ``agree''
in how they interpret the symbols of the language at least on the
shared part~$\Domain{M}$.

\begin{defn}[Substructure] % PRIM-SEM013
\label{PRIM-SEM013}
Given structures $\Struct M$ and $\Struct M'$ for the same
language~$\Lang L$, we say that $\Struct M$ is a \emph{substructure}
of $\Struct M'$, and $\Struct M'$ an \emph{extension} of $\Struct M$,
written $\Struct M \substruct \Struct M'$, iff
\begin{enumerate}
\item $\Domain{M} \subseteq \Domain{M'}$;
\item For each constant $c \in \Lang L$, $\Assign{c}{M} =
    \Assign{c}{M'}$;
\item For each $n$-place function symbol $f \in \Lang L$,
  $\Assign{f}{M}(a_1, \dots, a_n) = \Assign{f}{M'}(a_1, \dots, a_n)$
  for all $a_1$, \dots, $a_n \in \Domain{M}$;
\item For each $n$-place predicate symbol $R \in \Lang L$, $\langle
  a_1, \dots, a_n\rangle \in \Assign{R}{M}$ iff $\langle a_1, \dots,
  a_n\rangle \in \Assign{R}{M'}$ for all $a_1$, \dots, $a_n \in
  \Domain{M}$.
\end{enumerate}
\end{defn}

\begin{rem}
\label{rem:substructure}
If the language contains no constant or function symbols, then any
non-empty $N \subseteq \Domain{M}$ determines a substructure~$\Struct{N}$ of
$\Struct M$ with domain~$\Domain{N} = N$ by putting $\Assign{R}{N} =
\Assign{R}{M} \cap N^n$.
\end{rem}

%%% Homomorphisms (NEW-CONTENT from DOMAIN-SEMANTICS)

\subsection*{Homomorphisms}

\begin{defn}[Homomorphism] % PRIM-SEM014
\label{PRIM-SEM014}
A function $h \colon \Domain{M} \to \Domain{M'}$ between structures
$\Struct{M}$ and $\Struct{M'}$ for the same language~$\Lang{L}$ is a
\emph{homomorphism} if:
\begin{enumerate}
\item For every constant symbol $c$: $h(\Assign{c}{M}) = \Assign{c}{M'}$;
\item For every $n$-place function symbol $f$:
  $h(\Assign{f}{M}(a_1, \ldots, a_n)) = \Assign{f}{M'}(h(a_1), \ldots, h(a_n))$;
\item For every $n$-place predicate symbol $R$:
  $\langle a_1, \ldots, a_n \rangle \in \Assign{R}{M} \Rightarrow
  \langle h(a_1), \ldots, h(a_n) \rangle \in \Assign{R}{M'}$.
\end{enumerate}
A homomorphism is weaker than an isomorphism: it need not be bijective,
and it preserves relations in one direction only (it need not reflect them).
\end{defn}

%%% Embedding (NEW-CONTENT from DOMAIN-SEMANTICS)

\begin{defn}[Embedding] % DEF-SEM016
\label{DEF-SEM016}
An \emph{embedding} $h \colon \Domain{M} \hookrightarrow \Domain{M'}$
between structures $\Struct{M}$ and $\Struct{M'}$ for the same
language~$\Lang{L}$ is an injective homomorphism that also
\emph{reflects} relations: for every $n$-place predicate symbol $R$,
\[
\langle a_1, \ldots, a_n \rangle \in \Assign{R}{M} \quad\text{iff}\quad
\langle h(a_1), \ldots, h(a_n) \rangle \in \Assign{R}{M'}.
\]
Every isomorphism is a surjective embedding; every embedding is a
homomorphism that additionally reflects atomic formulas.
\end{defn}

%%% Isomorphism (from mod/bas/iso — KEEP)

\subsection*{Isomorphism}

First-order structures can be alike in one of two ways. One way in
which they can be alike is that they make the same sentences
true---we call such structures \emph{elementarily equivalent}
(Definition~\ref{DEF-SEM008}). But structures can be very different and still
make the same sentences true---for instance, one can be countable and
the other not.  So another, stricter, aspect in which structures can
be alike is if they are fundamentally the same, in the sense that they
only differ in the objects that make them up, but not in their
structural features.

\begin{defn}[Isomorphism] % PRIM-SEM012
\label{PRIM-SEM012}
Given two structures $\Struct{M}$ and
$\Struct M'$ for the same language~$\Lang L$, we say that
$\Struct{M}$ is \emph{isomorphic to}~$\Struct M'$, written $\Struct{M}
\simeq \Struct M'$, if and only if there is a function $h \colon
\Domain{M} \to \Domain{M'}$ such that:
\begin{enumerate}
\item $h$ is injective: if $h(x) = h(y)$ then $x = y$;
\item $h$ is surjective: for every $y \in \Domain{M'}$ there
  is $x \in \Domain{M}$ such that $h(x) = y$;
\item For every constant symbol $c$:
  $h(\Assign{c}{M}) = \Assign{c}{M'}$;
\item For every $n$-place predicate symbol~$P$:
  \[
  \tuple{a_1, \dots, a_n}\in \Assign{P}{M} \quad\text{iff}\quad
  \tuple{h(a_1), \dots, h(a_n)} \in \Assign{P}{M'};
  \]
\item For every $n$-place function symbol $f$:
  \[
  h(\Assign{f}{M}(a_1, \dots, a_n)) =
  \Assign{f}{M'}(h(a_1), \dots, h(a_n)).
  \]
\end{enumerate}
\end{defn}

\begin{thm}[Isomorphism Lemma] % THM-SEM001
\label{THM-SEM001}
If $\Struct{M} \iso \Struct M'$ then $\Struct{M} \equiv
\Struct{M'}$.
\end{thm}

\begin{proof}
Let $h$ be an isomorphism of $\Struct{M}$ onto $\Struct M'$. For any
assignment~$s$, $h \circ s$ is the composition of $h$ and $s$, i.e.,
the assignment in $\Struct{M'}$ such that $(h \circ s)(x) = h(s(x))$.
By induction on $t$ and $!A$ one proves the stronger claims:
\begin{enumerate}
  \item[a.] $h(\Value{t}{M}[s]) = \Value{t}{M'}[h\circ s]$.
  \item[b.] $\Sat{M}{!A}[s]$ iff $\Sat{M'}{!A}[h \circ s]$.
\end{enumerate}
Part (a) is proved by induction on the complexity of~$t$.
\begin{enumerate}
\item If $t \ident c$, then $h(\Value{c}{M}[s]) = h(\Assign{c}{M}) = \Assign{c}{M'} =
  \Value{c}{M'}[h \circ s]$.
\item If $t \ident x$, then $h(\Value{x}{M}[s]) =
  h(s(x)) = (h \circ s)(x) = \Value{x}{M'}[h \circ s]$.
\item If $t \ident f(t_1, \dots, t_n)$, then by induction hypothesis
  $h(\Value{t_i}{M}[s]) = \Value{t_i}{M'}[h\circ s]$ for each $i$, so
  \begin{align*}
    h(\Value{t}{M}[s])
    & = h(\Assign{f}{M}(\Value{t_1}{M}[s], \dots, \Value{t_n}{M}[s])) \\
    & = \Assign{f}{M'}(h(\Value{t_1}{M}[s]), \dots,
    h(\Value{t_n}{M}[s])) \\
    & = \Assign{f}{M'}(\Value{t_1}{M'}[h \circ s], \dots,
    \Value{t_n}{M'}[h \circ s]) \\
    & = \Value{t}{M'}[h\circ s].
  \end{align*}
\end{enumerate}
Part (b) is proved by induction on the complexity of~$!A$.
If $!A$ is a sentence, the assignments~$s$ and $h \circ s$ are
irrelevant, and we have $\Sat{M}{!A}$ iff $\Sat{M'}{!A}$.
\end{proof}

\begin{defn}
An \emph{automorphism} of a structure $\Struct{M}$ is an isomorphism
of $\Struct{M}$ onto itself.
\end{defn}

\begin{defn}[Elementary Substructure] % DEF-SEM011
\label{DEF-SEM011}
$\Struct{M}$ is an \emph{elementary substructure} of $\Struct{M'}$,
written $\Struct{M} \preccurlyeq \Struct{M'}$, if $\Struct{M}$ is a
substructure of $\Struct{M'}$ and for every formula
$!A(x_1, \ldots, x_n)$ and all $a_1, \ldots, a_n \in \Domain{M}$:
$\Sat{M}{!A}[a_1, \ldots, a_n]$ iff $\Sat{M'}{!A}[a_1, \ldots, a_n]$.
\end{defn}

%%% Diagram (NEW-CONTENT from DOMAIN-SEMANTICS)

\subsection*{Diagrams}

\begin{defn}[Diagram] % DEF-SEM012
\label{DEF-SEM012}
Let $\Struct{M}$ be a structure for $\Lang{L}$. Expand $\Lang{L}$ to
$\Lang{L}_M = \Lang{L} \cup \{c_a : a \in \Domain{M}\}$ by adding a
new constant symbol~$c_a$ for each element~$a$ of the domain. The
\emph{atomic diagram} of $\Struct{M}$ is the set
\[
\mathrm{Diag}(\Struct{M}) = \Setabs{!A}{!A \text{ is atomic or negated
    atomic in } \Lang{L}_M \text{ and } \Sat{M}{!A}}.
\]
The \emph{elementary diagram} of $\Struct{M}$ is the set
\[
\mathrm{ElDiag}(\Struct{M}) = \Setabs{!A}{!A \in
  \mathrm{Sent}(\Lang{L}_M) \text{ and } \Sat{M}{!A}}.
\]
Any model of $\mathrm{Diag}(\Struct{M})$ contains an isomorphic copy
of $\Struct{M}$ (via the map $a \mapsto c_a^{\Struct{M}}$), and any
model of $\mathrm{ElDiag}(\Struct{M})$ contains an elementary
extension of $\Struct{M}$.
\end{defn}

%%% Type (NEW-CONTENT from DOMAIN-SEMANTICS)

\begin{defn}[Complete Type] % DEF-SEM013
\label{DEF-SEM013}
Let $\Struct{M}$ be a structure for $\Lang{L}$ and $A \subseteq
\Domain{M}$. A \emph{complete $n$-type over $A$} is a maximal
consistent set~$p$ of formulas $!A(x_1, \ldots, x_n)$ with
parameters from $A$ that is finitely satisfiable in $\Struct{M}$.
The set of all complete $n$-types over $A$ is denoted
$S_n^{\Struct{M}}(A)$. Types classify the possible ``behaviors'' of
$n$-tuples in models extending~$\Struct{M}$.
\end{defn}

%%% Ultraproduct (NEW-CONTENT from DOMAIN-SEMANTICS)

\subsection*{Ultraproducts}

\begin{defn}[Ultraproduct] % DEF-SEM015
\label{DEF-SEM015}
Given a family of structures $\{\Struct{M}_i\}_{i \in I}$ for the
same language~$\Lang{L}$ and an ultrafilter $\mathcal{U}$ on $I$, the
\emph{ultraproduct} $\prod_{i \in I} \Struct{M}_i / \mathcal{U}$ is
the structure with domain
\[
\prod_{i \in I} \Domain{M_i} \big/ {\sim_{\mathcal{U}}},
\]
where $f \sim_{\mathcal{U}} g$ iff $\{i \in I : f(i) = g(i)\} \in
\mathcal{U}$. Constant, function, and predicate symbols are
interpreted componentwise modulo~$\mathcal{U}$. When all
$\Struct{M}_i = \Struct{M}$, the construction is called an
\emph{ultrapower} of $\Struct{M}$.
\end{defn}

Ultraproducts provide a purely model-theoretic proof of compactness
(see CP-003, \S\ref{META.3}): a set of sentences is satisfiable iff
every finite subset is, by taking an ultraproduct of the finite-subset
models. This avoids completeness entirely.

%%% Categoricity (from mod/bas/dlo — KEEP, with DEF-SEM014 added)

\subsection*{Categoricity and Dense Linear Orders}

\begin{defn}[Categoricity] % DEF-SEM014
\label{DEF-SEM014}
A theory~$T$ is \emph{$\kappa$-categorical} if all models of~$T$ of
cardinality~$\kappa$ are isomorphic. By the L\"owenheim--Skolem
theorem, no theory with infinite models is categorical in all
cardinalities.
\end{defn}

\begin{defn}[Dense Linear Ordering]
  A \emph{dense linear ordering without endpoints} is a structure
  $\Struct{M}$ for the language containing a single 2-place
  predicate symbol~$<$ satisfying the following sentences:
  \begin{enumerate}
  \item $\lforall[x][\lnot x < x]$ \hfill (irreflexivity)
  \item $\lforall[x][\lforall[y][\lforall[z][(x < y \lif (y < z \lif x
    <z ))]]]$ \hfill (transitivity)
  \item $\lforall[x][\lforall[y][(x< y \lor \eq[x][y] \lor y < x)]]$ \hfill (totality)
  \item $\lforall[x][\lexists[y][x < y]]$ \hfill (no greatest element)
  \item $\lforall[x][\lexists[y][y < x]]$ \hfill (no least element)
  \item $\lforall[x][\lforall[y][(x < y \lif \lexists[z][(x < z \land
        z < y)])]]$ \hfill (density)
 \end{enumerate}
\end{defn}

\begin{thm}[Cantor]
\label{thm:cantorQ}
  Any two countable dense linear orderings without
  endpoints are isomorphic.
\end{thm}

\begin{proof}
  Let $\Struct{M_1}$ and $\Struct{M_2}$ be countable dense linear
  orderings without endpoints, with ${<_1} = \Assign{<}{M_1}$ and ${<_2} =
  \Assign{<}{M_2}$, and let $\PIso{I}$ be the set of all partial
  isomorphisms between them. $\PIso{I}$ is not empty since at least
  $\emptyset \in \PIso{I}$. We show that $\PIso{I}$ satisfies the
  Back-and-Forth property.

  To show $\PIso{I}$ satisfies the Forth property, let $p \in
  \PIso{I}$ and let $p(a_i) = b_i$ for $i = 1$, \dots,~$n$, and
  without loss of generality suppose $a_1 <_1 a_2 <_1 \cdots <_1
  a_n$. Given $a \in \Domain{M_1}$, find $b \in \Domain{M_2}$ as
  follows:
  \begin{enumerate}
  \item if $a <_1 a_1$ let $b \in \Domain{M_2}$ be such that $b <_2
    b_1$;
  \item if $a_n <_1 a$ let $b \in \Domain{M_2}$ be such that $b_n <_2 b$;
 \item if $a_i <_1 a <_1 a_{i+1}$ for some $i$, then let $b \in
   \Domain{M_2}$ be such that $b_i <_2 b <_2 b_{i+1}$.
  \end{enumerate}
  It is always possible to find a $b$ with the desired property since
  $\Struct{M_2}$ is a dense linear ordering without endpoints. Define
  $q = p \cup \{ \langle a, b \rangle \}$ so that $q \in \PIso{I}$ is
  the desired extension of $p$. The Back property is similar. By
  the back-and-forth theorem (applied to countable structures),
  $\Struct{M_1} \iso \Struct{M_2}$.
\end{proof}

The theory of dense linear orders without endpoints is thus
$\aleph_0$-categorical.

\begin{rem}
  Let $\Struct{S}$ be any countable dense linear ordering without
  endpoints. Then by Cantor's theorem, $\Struct{S} \iso
  \Struct{Q}$, where $\Struct{Q} = (\Rat, <)$. Now consider the
  structure~$\Struct{R} =
  (\Real, <)$ from Remark~\ref{rem:R}. There is
  a countable structure~$\Struct{S}$ such that $\Struct{R}
  \equiv \Struct{S}$. But $\Struct{S}$ is a countable dense
  linear ordering without endpoints, and so it is isomorphic (and
  hence elementarily equivalent) to $\Struct{Q}$. By
  transitivity of elementary equivalence, $\Struct{R} \equiv
  \Struct{Q}$.
\end{rem}

%%% Standard and Non-Standard Models of Arithmetic
%%% (from mod/mar/stm, mod/mar/nst, mod/mar/mpa, mod/mar/cmp — CONDENSE)

\subsection*{Standard and Non-Standard Models of Arithmetic}

A structure for $\Lang{L_A}$ is \emph{standard} if it is
isomorphic to~$\Struct{N}$.

\begin{prop}
\label{prop:standard-domain}
If a structure~$\Struct{M}$ is standard,
then its domain is the set of values of the standard numerals, i.e.,
$\Domain{M} = \Setabs{\Value{\num{n}}{M}}{n \in \Nat}$.
\end{prop}

\begin{proof}[Proof sketch]
Since $\Struct{M}$ is standard, there is an isomorphism $g \colon \Nat
\to \Domain{M}$. Then $g(n) = g(\Value{\num{n}}{N}) =
\Value{\num{n}}{M}$, and $g$ is surjective.
\end{proof}

\begin{prop}
\label{prop:thq-standard}
If $\Sat{M}{\Th{Q}}$, and $\Domain{M} = \Setabs{\Value{\num{n}}{M}}{n
  \in \Nat}$, then $\Struct{M}$ is standard.
\end{prop}

\begin{proof}[Proof sketch]
The function $g(n) = \Value{\num{n}}{M}$ is surjective by hypothesis
and injective because $\Th{Q} \Proves \eq/[\num{n}][\num{m}]$ whenever
$n \neq m$, so $\Sat{M}{\eq/[\num{n}][\num{m}]}$. One verifies that
$g$ is an isomorphism by checking that $\Th{Q}$ proves the relevant
identities for $\Obj{0}$, $\prime$, $+$, $\times$, and $<$.
\end{proof}

A structure~$\Struct{M}$ for $\Lang{L_A}$ is \emph{non-standard}
if it is not isomorphic to~$\Struct{N}$. The elements $x \in
\Domain{M}$ which are equal to $\Value{\num{n}}{M}$ for some $n \in
\Nat$ are called \emph{standard numbers} (of $\Struct{M}$), and those
not, \emph{non-standard numbers}.

If a structure~$\Struct{M}$ for $\Lang{L_A}$ contains a
non-standard number, $\Struct{M}$ is non-standard.

\begin{prop}
$\Th{TA}$ has a countable non-standard model.
\end{prop}

\begin{proof}
Expand $\Lang{L_A}$ by a new constant symbol~$c$ and consider the set of
sentences
\[
\Gamma = \Th{TA} \cup \{\eq/[c][\num{0}], \eq/[c][\num{1}],
\eq/[c][\num{2}], \dots\}
\]
Any model~$\Struct{M^c}$ of~$\Gamma$ would contain an element~$x =
\Assign{c}{M}$ which is non-standard, since $x \neq
\Value{\num{n}}{M}$ for all $n \in \Nat$. Also, obviously,
$\Sat{M^c}{\Th{TA}}$, since $\Th{TA} \subseteq \Gamma$. If we turn
$\Struct{M^c}$ into a structure~$\Struct{M}$ for $\Lang{L_A}$
simply by forgetting about~$c$, its domain still contains the
non-standard~$x$, and also~$\Sat{M}{\Th{TA}}$.

We use the compactness theorem to show that~$\Gamma$ has a model. Consider any finite subset $\Gamma_0 \subseteq
\Gamma$. Suppose $k$ is
the largest number so that $\eq/[c][\num{k}] \in \Gamma_0$. Define
$\Struct{N_k}$ by expanding~$\Struct{N}$ to include the
interpretation~$\Assign{c}{N_k} = k+1$. Then $\Sat{N_k}{\Gamma_0}$,
since $c$ does not occur in~$\Th{TA}$, and $\Value{c}{N_k} = k+1 \neq n$
for $n \le k$. Thus every finite subset of~$\Gamma$ is
satisfiable, so by compactness, $\Gamma$ is satisfiable.
\end{proof}

%%% Block structure of non-standard models of PA

In a non-standard model~$\Struct{M}$ of $\Th{PA}$, the ordering
$\Assign{<}{M}$ is a linear strict order. The element
$\Assign{\Obj{0}}{M}$ is least, every element has a unique successor
and (except for $\Assign{\Obj{0}}{M}$) a unique predecessor, and all
standard elements are less than all non-standard elements.

Every non-standard element~$x$ is contained in a \emph{block}~$[x]$
consisting of all elements reachable from~$x$ by finitely many
applications of successor and predecessor. Each block has no least and
no greatest element. Distinct blocks are disjoint and respect the
ordering: if $x < y$ and $[x] \neq [y]$, then $u < v$ for all $u \in
[x]$ and $v \in [y]$.

\begin{prop}
\label{prop:blocks-dense}
The ordering of the non-standard blocks is dense: if $x < y$ and
$[x] \neq [y]$, then there is a block $[z]$ distinct from both that is
between them.
\end{prop}

\begin{proof}[Proof sketch]
$\Th{PA}$ proves that every element is divisible by~$2$ (possibly with
remainder). If $x$ is non-standard, the ``average'' $z$ of $x$
and~$y$ satisfies $x < z < y$ and $[z] \neq [x]$, $[z] \neq [y]$.
\end{proof}

The non-standard blocks are therefore ordered like the rationals: they
form a countable dense linear ordering without endpoints.  By
Cantor's theorem (Theorem~\ref{thm:cantorQ}), any two such orderings are isomorphic. It follows
that for any two countable non-standard models $\Struct{M}_1$ and
$\Struct{M_2}$ of true arithmetic, their reducts to the language
containing $<$ and $=$ only are isomorphic. However, they need not be
isomorphic in the full language of arithmetic, as there are
non-isomorphic ways to define addition and multiplication.

\begin{defn}[Computable Structure]
\label{defn:computable-structure}
A structure~$\Struct{M}$ for $\Lang{L_A}$ is \emph{computable} iff
  $\Domain{M} = \Nat$ and $\Assign{\prime}{M}$, $\Assign{+}{M}$,
  $\Assign{\times}{M}$ are computable functions and $\Assign{<}{M}$ is
  a decidable relation.
\end{defn}

\begin{thm}[Tennenbaum's Theorem]
\label{thm:tennenbaum}
$\Struct{N}$ is the only computable model of~$\Th{PA}$.
\end{thm}

%%% Overspill (from mod/bas/ove — CONDENSE)

\subsection*{Overspill}

A classical application of the compactness theorem is the overspill
principle.

\begin{thm}[Overspill]
\label{thm:overspill}
If a set $\Gamma$ of sentences has arbitrarily
large finite models, then it has an infinite model.
\end{thm}

\begin{proof}
Expand the language of $\Gamma$ by adding countably many new constants
$c_0$, $c_1$, \dots\ and consider $\Gamma \cup \{c_i \neq c_j :
i \neq j\}$. Since $\Gamma$ has arbitrarily large finite models,
every finite subset of this expanded set is satisfiable. By compactness, the whole set has a
model $\Struct M$ whose domain
must be infinite, since it satisfies all inequalities $c_i \neq c_j$.
\end{proof}

\begin{prop}
\label{prop:inf-not-fo}
There is no sentence $!A$ of any first-order language that is true in
a structure~$\Struct M$ if and only if the domain $\Domain{M}$ of
the structure is infinite.
\end{prop}

\begin{proof}
If there were such a $!A$, its negation $\lnot !A$ would be true in
all and only the finite structures, and it would therefore have
arbitrarily large finite models but lack an infinite model,
contradicting the overspill theorem.
\end{proof}


%% ===================================================================
%% SEM.7: Theorems
%% Sources: fol/syn/ext (KEEP)
%% ===================================================================

\section{Theorems} \label{SEM.7}

\subsection*{The Coincidence Lemma}

Extensionality, sometimes called relevance, can be expressed
informally as follows: the only factors that bear upon the
satisfaction of formula~$!A$ in a structure~$\Struct M$
relative to a variable assignment~$s$, are the size of the
domain and the assignments made by~$\Struct M$ and~$s$ to the
elements of the language that actually appear in~$!A$.

One immediate consequence is that where two
structures~$\Struct M$ and~$\Struct M'$ agree on all the elements
of the language appearing in a sentence~$!A$ and have the same
domain,~$\Struct M$ and~$\Struct M'$ must also agree on whether or not
$!A$ itself is true.

\begin{prop}[Extensionality / Coincidence Lemma] % THM-SEM002
\label{THM-SEM002}
  Let $!A$ be a formula, and $\Struct M_1$ and $\Struct M_2$ be
  structures with $\Domain{M_1} = \Domain{M_2}$, and $s$ a
  variable assignment on $\Domain{M_1} = \Domain{M_2}$.  If
  $\Assign{c}{M_1} = \Assign{c}{M_2}$, $\Assign{R}{M_1}=\Assign{R}{M_2}$,
  and $\Assign{f}{M_1} = \Assign{f}{M_2}$ for every constant symbol~$c$,
  relation symbol~$R$, and function symbol $f$ occurring in~$!A$, then
  $\Sat{M_1}{!A}[s]$ iff $\Sat{M_2}{!A}[s]$.
\end{prop}

\begin{proof}[Proof sketch]
  First prove (by induction on~$t$) that for every term,
  $\Value{t}{M_1}[s] = \Value{t}{M_2}[s]$.  Then prove the proposition
  by induction on~$!A$, making use of the claim just proved for the
  induction basis (where $!A$ is atomic).
\end{proof}

\begin{cor}[Extensionality for Sentences]
\label{cor:extensionality-sent}
  Let $!A$ be a sentence and $\Struct{M_1}$, $\Struct{M_2}$ as in
  the Coincidence Lemma. Then $\Sat{M_1}{!A}$ iff $\Sat{M_2}{!A}$.
\end{cor}

\begin{proof}
Follows from the Coincidence Lemma by Corollary~\ref{cor:sat-sentence}.
\end{proof}

\subsection*{The Substitution Lemma}

Moreover, the value of a term, and whether or not a structure
satisfies a formula, only depend on the values of its subterms.

\begin{prop}[Substitution Lemma for Terms] % THM-SYN003 (terms)
\label{THM-SYN003:terms}
Let $\Struct M$ be a structure, $t$ and $t'$ terms, and $s$ a
variable assignment. Then $\Value{\Subst{t}{t'}{x}}{M}[s] =
\Value{t}{M}[\Subst{s}{\Value{t'}{M}[s]}{x}]$.
\end{prop}

\begin{proof}
By induction on~$t$.
\begin{enumerate}
\item If $t$ is a constant, say, $t\ident c$, then $\Subst{t}{t'}{x} =
  c$, and $\Value{c}{M}[s] = \Assign{c}{M} =
  \Value{c}{M}[\Subst{s}{\Value{t'}{M}[s]}{x}]$.

\item If $t$ is a variable other than~$x$, say, $t \ident y$, then
  $\Subst{t}{t'}{x} = y$, and $\Value{y}{M}[s] =
  \Value{y}{M}[\Subst{s}{\Value{t'}{M}[s]}{x}]$ since
  $\varAssign{s}{\Subst{s}{\Value{t'}{M}[s]}{x}}{x}$.

\item If $t \ident x$, then $\Subst{t}{t'}{x} = t'$. But
  $\Value{x}{M}[\Subst{s}{\Value{t'}{M}[s]}{x}] = \Value{t'}{M}[s]$ by
  definition of~$\Subst{s}{\Value{t'}{M}[s]}{x}$.

\item If $t \ident \Atom{f}{t_1,\dots,t_n}$ then by the
  definition of substitution and the induction hypothesis:
\begin{align*}
  \Value{\Subst{t}{t'}{x}}{M}[s]
& = \Assign{f}{M}(\Value{\Subst{t_1}{t'}{x}}{M}[s], \dots,
    \Value{\Subst{t_n}{t'}{x}}{M}[s])\\
& = \Assign{f}{M}(\Value{t_1}{M}[\Subst{s}{\Value{t'}{M}[s]}{x}], \dots,
   \Value{t_n}{M}[\Subst{s}{\Value{t'}{M}[s]}{x}])\\
& = \Value{t}{M}[\Subst{s}{\Value{t'}{M}[s]}{x}].
\end{align*}
\end{enumerate}
\end{proof}

\begin{prop}[Substitution Lemma for Formulas] % THM-SYN003 (formulas)
\label{THM-SYN003:formulas}
Let $\Struct M$ be
a structure, $!A$ a formula, $t'$~a term, and $s$~a variable
assignment. Then $\Sat{M}{\Subst{!A}{t'}{x}}[s]$ iff
$\Sat{M}{!A}[\Subst{s}{\Value{t'}{M}[s]}{x}]$.
\end{prop}

\begin{proof}
By induction on~$!A$, parallel to the proof for terms.
\end{proof}

The point of the Substitution Lemma is the following. Suppose we have
a term $t$ or a formula~$!A$ and some term~$t'$, and we want to know
the value of $\Subst{t}{t'}{x}$ or whether or not $\Subst{!A}{t'}{x}$
is satisfied in a structure~$\Struct M$ relative to a variable
assignment~$s$. Then we can either perform the substitution first and
then consider the value or satisfaction relative to $\Struct{M}$
and~$s$, or we can first determine the value~$m = \Value{t'}{M}[s]$ of
$t'$ in $\Struct{M}$ relative to~$s$, change the variable assignment
to~$\Subst{s}{m}{x}$ and then consider the value of~$t$ in
$\Struct{M}$ and~$\Subst{s}{m}{x}$, or whether
$\Sat{M}{!A}[\Subst{s}{m}{x}]$. The Substitution Lemma guarantees
that the answer will be the same, whichever way we do it.
