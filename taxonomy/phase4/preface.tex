%% preface.tex
%% Preface for "A Lean Systematization of Mathematical Logic"

\chapter*{Preface}
\addcontentsline{toc}{chapter}{Preface}

\section*{What This Book Is}

This book presents the core of mathematical logic---from naive set
theory through G\"odel's incompleteness theorems---as a single,
self-contained development in which every concept is defined exactly
once, every theorem is traceable to its primitives, and the dependency
structure is explicit throughout.

The material is drawn from the Open Logic Project's \textit{Open
Logic: A Complete Text}, a collaboratively authored, open-source
textbook covering propositional and first-order logic, model theory,
computability, incompleteness, and set theory.  That text is modular
by design, offering instructors flexibility to assemble custom
courses.  The present volume takes the opposite approach: it
\emph{compresses} the source material into a single linear narrative
organized by a taxonomy of mathematical logic, eliminating redundancy
and making every dependency visible.

\section*{How It Is Organized}

The taxonomy identifies six \emph{domains}---irreducible subject
areas, each answering a single foundational question---plus a
\emph{metatheory} layer that records how the domains interact.

\begin{center}
\begin{tabular}{@{}clp{7.5cm}@{}}
\textbf{Ch.} & \textbf{Domain} & \textbf{Governing Question} \\[3pt]
1 & BST & What naive set-theoretic objects does the metalanguage use? \\
2 & SYN & How are well-formed expressions constructed? \\
3 & SEM & How is meaning assigned to expressions? \\
4 & DED & How are truths derived from assumptions? \\
5 & CMP & What is effectively computable? \\
6 & META & What are the properties of formal systems? \\
7 & SET & What is the formal mathematical universe? \\
8 & EXT & Where does the theory extend? \\
\end{tabular}
\end{center}

\noindent
The domains are not independent.  Their dependencies form the
following graph:

\begin{center}
\begin{tikzpicture}[
  node distance=2cm and 2.5cm,
  domain/.style={
    rectangle, rounded corners=4pt, draw, thick,
    minimum width=2.2cm, minimum height=0.9cm,
    font=\sffamily\small
  },
  meta/.style={
    rectangle, rounded corners=4pt, draw, thick, dashed,
    minimum width=2.2cm, minimum height=0.9cm,
    font=\sffamily\small
  },
  arr/.style={->, thick, >=stealth},
  darr/.style={->, thick, >=stealth, dashed}
]

% Level 0
\node[domain] (bst) {BST};

% Level 1
\node[domain, above right=1.5cm and 1cm of bst] (syn) {SYN};
\node[domain, above left=1.5cm and 1cm of bst] (cmp) {CMP};

% Level 2
\node[domain, above left=1.5cm and 0cm of syn] (sem) {SEM};
\node[domain, above right=1.5cm and 0cm of syn] (ded) {DED};

% Level 3
\node[meta, above right=1.5cm and -0.5cm of sem] (meta) {META};

% Level 4
\node[domain, above=1.5cm of meta] (set) {SET};

% Extensions
\node[domain, right=2cm of set, fill=gray!10] (ext) {EXT};

% Arrows (dependency = "uses")
\draw[arr] (bst) -- (syn);
\draw[arr] (bst) -- (cmp);
\draw[arr] (syn) -- (sem);
\draw[arr] (syn) -- (ded);
\draw[arr] (bst) to[bend left=20] (sem);
\draw[arr] (bst) to[bend right=20] (ded);
\draw[darr] (sem) -- (meta);
\draw[darr] (ded) -- (meta);
\draw[darr] (cmp) to[bend left=30] (meta);
\draw[arr] (meta) -- (set);
\draw[darr] (set) -- (ext);

% Labels
\node[left=0.3cm of bst, font=\footnotesize\itshape, text=gray] {Level 0};
\node[left=0.3cm of cmp, font=\footnotesize\itshape, text=gray] {Level 1};
\node[left=0.3cm of sem, font=\footnotesize\itshape, text=gray] {Level 2};
\node[right=0.3cm of meta, font=\footnotesize\itshape, text=gray]
  {Composition};
\node[left=1.7cm of set, font=\footnotesize\itshape, text=gray] {Level 3};

\end{tikzpicture}
\end{center}

\noindent
Solid arrows indicate definitional dependencies: SYN uses BST, SEM
uses SYN, and so on.  Dashed arrows indicate \emph{composition
patterns}---metatheorems that live at the intersection of two or more
domains.  For instance, the Completeness Theorem (Chapter~6) connects
SEM and DED but belongs to neither; it is a composition pattern.

\medskip

An important architectural feature is the two-level treatment of set
theory:

\begin{itemize}
\item \textbf{BST (Level~0)} provides the naive set-theoretic
  metalanguage---sets, functions, relations, cardinality---used to
  \emph{build} the formal apparatus of logic.

\item \textbf{SET (Level~1)} is Zermelo--Fraenkel set theory with
  Choice, treated as a formal first-order theory \emph{within} the
  apparatus.  Its axioms are sentences of $\Lang{L}_\in$, and its
  theorems are derived using the proof systems of Chapter~4.
\end{itemize}

\noindent
There is no circularity: Level-0 sets \emph{build} the system;
Level-1 set theory is a \emph{subject of} the system.

\section*{The Taxonomy in Numbers}

The taxonomy catalogs \textbf{205 formal items} across the six
domains: 74 primitives (concepts taken as given within their domain),
60 definitions (concepts defined from primitives), 9 axioms
(in formal set theory), and 62 theorems, lemmas, and propositions.
These are connected by \textbf{13 composition patterns}---metatheorems
such as Soundness (CP-001), Completeness (CP-002), Compactness
(CP-003), and the Incompleteness Theorems (CP-005, CP-006)---that
bridge domains.

Every formal item carries a unique identifier (e.g.,
\texttt{PRIM-SYN009} for the definition of a first-order language,
\texttt{CP-002} for the Completeness Theorem).  The full registry and
dependency specifications are maintained in the \texttt{taxonomy/}
directory of the project repository.

\section*{How to Read This Book}

The chapters are arranged in dependency order: each chapter uses only
material from earlier chapters.  A linear reading from Chapter~1
through Chapter~7 encounters no forward references.

Three reading paths are natural:

\begin{enumerate}
\item \textbf{Foundations first} (Chapters 1--4, then 6): Set
  theory, syntax, semantics, deduction, completeness.  This is the
  classical core of a first course in mathematical logic.

\item \textbf{Computability and incompleteness} (Chapters 1--2, 4--6):
  Syntax, deduction, computability, then the incompleteness theorems
  in Chapter~6.  Semantics (Chapter~3) can be deferred.

\item \textbf{Set theory} (Chapters 1, 3--4, 6--7): The path to
  formal set theory, ordinals, and cardinals, using the metatheory
  of Chapter~6 as a bridge.
\end{enumerate}

\noindent
Chapter~8 (Extensions) is a brief guide to topics beyond the scope of
this volume: modal logic, intuitionistic logic, many-valued logic,
second-order logic, and the lambda calculus.

\section*{Relation to the Source Material}

This text is compiled from the Open Logic Project's \textit{Open
Logic: A Complete Text}, released under a Creative Commons BY license.
The systematization involved four phases:

\begin{enumerate}
\item \textbf{Taxonomy}: Identification of the six domains, their
  primitives, and the composition patterns connecting them.
\item \textbf{Mapping}: Every section of the source text was mapped
  to one or more taxonomy items, identifying redundancies and gaps.
\item \textbf{Compression}: Redundant treatments were merged,
  redundant proofs were cut or condensed, and new bridging material
  was written to unify the narrative.
\item \textbf{Recomposition}: The compressed material was assembled
  into the present volume, with uniform notation, consistent
  cross-referencing, and a single dependency-ordered structure.
\end{enumerate}

\noindent
The taxonomy, mapping, compression plan, and all quality-control
audits are documented in the \texttt{taxonomy/} directory of the
project repository at \texttt{github.com/sqkim25/OpenLogic}.

\section*{Acknowledgments}

This volume would not exist without the Open Logic Project and its
contributors.  The original text, authored and maintained by
Richard Zach and collaborators, represents years of careful exposition.
We are grateful for their decision to release it as open-source
educational material.
