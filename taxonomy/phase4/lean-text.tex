% lean-text.tex
% A Lean Systematization of Mathematical Logic
%
% Compiled from the OpenLogic Project textbook, organized by the
% taxonomy of mathematical logic (6 domains + metatheory).
% Every concept traceable to primitives, no redundancy, full dependency graph.

\documentclass[openany]{memoir}

% lean-preamble.sty
% Preamble for the Lean Text of Mathematical Logic
% Macro definitions extracted and adapted from the OpenLogic Project
%
% Sources:
%   open-logic-config.sty  (mathematical notation macros)
%   sty/open-logic-formulas.sty  (formula metavariable active-char system)
%   sty/open-logic-book-envs.sty (theorem environments)
%   sty/open-logic.sty  (Goedel corners, special symbols, oltableau)
%

\makeatletter
% This file is self-contained: it declares all package dependencies
% and provides every macro needed by the lean text without relying on
% any OL infrastructure (no \olref, \ollabel, \settexttoken, \iftag, etc.)

\NeedsTeXFormat{LaTeX2e}

%% ===================================================================
%%% Package Dependencies
%% ===================================================================

\RequirePackage[english]{babel}
\RequirePackage{amsmath}
\RequirePackage{amssymb}
\RequirePackage{amsthm}
\RequirePackage{amsfonts}
\RequirePackage{mathpazo}         % Palatino math
\RequirePackage[scaled=0.95]{helvet}
\RequirePackage{courier}
\RequirePackage{xparse}           % \DeclareDocumentCommand
\RequirePackage[OMLmathsfit]{isomath}  % \mathsfit for object-language symbols
\RequirePackage{extarrows}        % \xLongrightarrow (parallel reduction)
\RequirePackage{nicefrac}
\RequirePackage{stmaryrd}         % \varolessthan and other symbols
\RequirePackage{etoolbox}         % \docsvlist (used by formula lookup)
\RequirePackage{graphicx}         % \reflectbox (used in \Gn)

% Proof systems
\RequirePackage{bussproofs}       % ND / SC proof trees
\renewenvironment{prooftree}
  {\begin{center}\bottomAlignProof}
  {\DisplayProof\end{center}}
% DeduceC: conclusion without inference line (from bussproofs-extra)
% Simplified: render as noLine + UnaryInfC
\newcommand{\DeduceC}[1]{\noLine\UnaryInfC{#1}}
\RequirePackage[tableaux]{prooftrees}  % tableau truth trees

% Theorem and box environments
\RequirePackage{mdframed}         % defish boxes
\RequirePackage{thmtools}         % theorem styling

% TikZ
\RequirePackage{tikz}
\usetikzlibrary{arrows,automata,positioning,lindenmayersystems}

% Cross-referencing (load last)
\RequirePackage{hyperref}
\RequirePackage{cleveref}

% Additional symbol font (boxright, fishhookright, diamondright)
\DeclareSymbolFont{symbolsC}{U}{ntxsyc}{m}{n}
\DeclareMathSymbol{\boxright}{\mathbin}{symbolsC}{128}
\DeclareMathSymbol{\diamondright}{\mathbin}{symbolsC}{132}
\DeclareMathSymbol{\fishhookright}{\mathbin}{symbolsC}{74}


%% ===================================================================
%%% DeclareDocumentMacro -- zero-argument form (not provided by xparse)
%% ===================================================================

% In the OL source this is defined as \def; we replicate it here so
% that all the macro definitions below compile without modification.
\newcommand*{\DeclareDocumentMacro}[2]{\def#1{#2}}


%% ===================================================================
%%% Helper: \applytofirst
%% ===================================================================

% Apply #1 to the first token of #2 only, so that e.g.
% \Struct{M_k} produces \mathfrak{M}_k rather than \mathfrak{M_k}.
\def\applytofirst#1#2{{\expandafter#1#2}}


%% ===================================================================
%%% Formula Metavariables  (from open-logic-formulas.sty)
%% ===================================================================

% Provide \mathexclaim  (the literal ! glyph in math mode)
\DeclareMathSymbol{\mathexclaim}{\mathord}{operators}{`!}%

% Make ! active in math mode for formula metavariables
{\catcode`!=\active
\gdef!{\formula}
\let\formula\relax % default: do nothing
}
\mathcode`\!="8000 % set ! active in math mode

% \ollatinformulas -- formula metavars are A, B, C, ...
\def\ollatinformulas{\let\formula\relax}

% \olgreekformulas -- formula metavars are \varphi, \psi, \chi, ...
\def\olgreekformulas
{\gdef\formula{\lookup{A/\varphi,B/\psi,C/\chi,D/\theta,E/\alpha,F/\beta,G/\gamma,H/\delta,K/\xi,L/\zeta,O/\omega,R/\rho,S/\sigma,T/\tau}}}

% \olalphagreekformulas -- formula metavars are \alpha, \beta, \gamma, ...
\def\olalphagreekformulas
{\gdef\formula{\lookup{A/\alpha,B/\beta,C/\gamma,D/\delta,E/\phi,F/\psi,G/\chi,H/\theta,K/\xi,L/\zeta,O/\omega,R/\rho,S/\sigma,T/\tau}}}

% args -- format a list of arguments separated by commas
\newcommand*{\args}[1]{%
  \let\@argsep\@argsepinit
  \@for\@arg:=#1\do{%
    \@argsep\@arg}%
}
\def\@argsepinit{\let\@argsep\argsep}
\newcommand{\argsep}{}

% lookup -- look up #2 in list #1 (comma-separated key/value pairs)
\def\@match#1/#2{\def\match@input{#1}\def\match@result{#2}}
\newcommand*{\lookup}[2]{%
  \def\lookup@input{#2}
  \def\lookup@result{#2}%
  \def\do##1{\@match##1\if\lookup@input\match@input
    \let\lookup@result\match@result\fi}%
  \docsvlist{#1}%
  \lookup@result}

% Default: use Greek formula metavariables
\olgreekformulas

% Prefer \varphi to \phi and \varepsilon to \epsilon
\let\oldphi\phi
\let\phi\varphi
\let\oldepsilon\epsilon
\let\epsilon\varepsilon


%% ===================================================================
%%% Goedel Corner Quotes  (from open-logic.sty)
%% ===================================================================

\newbox\gnBoxA
\newdimen\gnCornerHgt
\setbox\gnBoxA=\hbox{$\ulcorner$}
\global\gnCornerHgt=\ht\gnBoxA
\newdimen\gnArgHgt
\def\gn #1{%
\setbox\gnBoxA=\hbox{$#1$}%
\gnArgHgt=\ht\gnBoxA%
\ifnum     \gnArgHgt<\gnCornerHgt \gnArgHgt=0pt%
\else \advance \gnArgHgt by -\gnCornerHgt%
\fi \raise\gnArgHgt\hbox{$\ulcorner$} \box\gnBoxA %
\raise\gnArgHgt\hbox{$\urcorner$}}


%% ===================================================================
%%% Logical Symbols
%% ===================================================================

% --- Truth Values ---

\DeclareDocumentMacro \True {\ensuremath{\mathbb{T}}}
\DeclareDocumentMacro \False {\ensuremath{\mathbb{F}}}
\DeclareDocumentMacro \Indet {\ensuremath{\mathbb{I}}}
\DeclareDocumentMacro \Undef {\ensuremath{\mathbb{U}}}

% --- Propositional Constants and Connectives ---

\DeclareDocumentMacro \lfalse {\bot}
\DeclareDocumentMacro \ltrue {\top}

% \lnot  -- negation (already provided by LaTeX; default \lnot)
% \land  -- conjunction (already provided by LaTeX; default \land)
% \lor   -- disjunction (already provided by LaTeX; default \lor)

\DeclareDocumentMacro \lif {\mathbin{\rightarrow}}
\DeclareDocumentMacro \liff {\mathbin{\leftrightarrow}}
\DeclareDocumentMacro \cif {\boxright}           % counterfactual conditional
\DeclareDocumentMacro \strictif {\fishhookright}  % strict conditional


%% ===================================================================
%%% Quantifiers
%% ===================================================================

% \lexists[x][!A(x)]  -- existential; \lexists! for unique existence
\DeclareDocumentCommand \lexists { t{!} o o } {
    \exists
    \IfBooleanTF {#1}
        \mathexclaim    % unique
        \relax % not unique
    \IfNoValueTF {#2}
        \relax     % no arguments
        { #2 } % one argument: variable
    \IfNoValueTF {#3}
        \relax
        { \, #3 }      % two arguments: space and matrix
}

% \lforall[x][!A(x)]  -- universal
\DeclareDocumentCommand \lforall { o o } {
    \IfNoValueTF {#1}
        { \forall }    % no arguments
        { \forall #1 } % one argument: variable
    \IfNoValueTF {#2}
        \relax
        { \, #2 }      % two arguments: space and matrix
}

% \eq[x][y] -- identity; \eq/ for negated
\DeclareDocumentCommand \eq { t{/} o o } {
  \IfNoValueTF {#3}
    { \IfBooleanTF{#1}{ \neq }{ = } }
    { \IfBooleanTF{#1}{ #2 \neq #3}{#2 = #3} }
}


%% ===================================================================
%%% Proof and Derivation Symbols
%% ===================================================================

% Sequent symbols
\DeclareDocumentMacro \Sequent {\Rightarrow}
\DeclareDocumentMacro \nSequent {\mid}
\DeclareDocumentMacro \fCenter {\ensuremath{\,\Sequent\,}}

% Sequent calculus rule names
\DeclareDocumentCommand \LeftR { m } {\ensuremath{{#1}\mathrm{L}}}
\DeclareDocumentCommand \RightR { m } {\ensuremath{{#1}\mathrm{R}}}
\DeclareDocumentCommand \iR { m m o} {\ensuremath{{#1\IfNoValueTF{#3}{}{_{#3}}}{#2}}}

% Structural rules
\DeclareDocumentMacro \Weakening {\ensuremath{\mathrm{W}}}
\DeclareDocumentMacro \Contraction {\ensuremath{\mathrm{C}}}
\DeclareDocumentMacro \Exchange {\ensuremath{\mathrm{X}}}
\DeclareDocumentMacro \Cut {\ensuremath{\mathrm{Cut}}}

% Natural deduction rule names
\DeclareDocumentCommand \Intro { m } {\ensuremath{{#1}\mathrm{Intro}}}
\DeclareDocumentCommand \Elim { m } {\ensuremath{{#1}\mathrm{Elim}}}

% Absurdity rules
\DeclareDocumentMacro \FalseInt {\ensuremath{\lfalse_I}}
\DeclareDocumentMacro \FalseCl {\ensuremath{\lfalse_C}}

% Discharge notation
\DeclareDocumentCommand \Discharge { m m }{[#1]^{#2}}
\DeclareDocumentCommand \DischargeRule { m m }{
  \RightLabel{#1}
  \LeftLabel{\scriptsize $#2$}
}

% Proof terms (intuitionistic logic)
\DeclareDocumentCommand \typeof { m m } {#1^{#2}}
\DeclareDocumentCommand \andi { m m } {\tuple{#1, #2}}
\DeclareDocumentCommand \ande { m m } {\fn{p}_{#1}(#2)}
\DeclareDocumentCommand \ori { m m m } {\fn{in}_{#1}^{#2}(#3)}
\DeclareDocumentCommand \ore { m m m m m } {\fn{case}(#1, #2.#3, #4.#5)}
\DeclareDocumentCommand \falsee { m m } {\fn{contr}_{#1}(#2)}

% Axiomatic derivations
\DeclareDocumentMacro \MP {\textsc{mp}}
\DeclareDocumentMacro \QR {\textsc{qr}}
\DeclareDocumentMacro \Hyp {\textsc{Hyp}}

% Signed formulas
\DeclareDocumentCommand \sFmla { m m o }{
  \ensuremath{%
    \IfNoValueTF{#3}{}{#3\,}%
    \hbox to.8em{\ensuremath{#1}\hfil} #2}}

\DeclareDocumentCommand \pFmla { m m m }{
  \ensuremath{%
    \hskip 3em{\llap{$#3$}\,}%
    \hbox to1.3em{\ensuremath{#1}\hfil} #2}}

% Tableau rule names
\DeclareDocumentCommand \TRule { m m o }{%
  \ensuremath{{#2}{#1}\IfNoValueTF{#3}{}{\, #3}}}

\DeclareDocumentMacro \TAss {Assumption}


%% ===================================================================
%%% Substitution
%% ===================================================================

\DeclareDocumentCommand \subst { m m } {#1/#2}

\DeclareDocumentCommand \SSubst { m m } {
  #1[#2]}

\DeclareDocumentCommand \Subst { m m m } {
  #1[\subst{#2}{#3}]}

\DeclareDocumentCommand \pSubst { m m m } {
  #1[#2/#3]^{-}
}


%% ===================================================================
%%% Satisfaction and Truth
%% ===================================================================

% \Sat[/]{M}{!A}[s]  -- satisfaction in a structure
\DeclareDocumentCommand \Sat { t{/} m m o } {
  \IfBooleanTF{#1}{
    \IfNoValueTF {#4}
        { \Struct{#2} \nvDash #3 }
        { \Struct{#2}, #4 \nvDash #3}}{
    \IfNoValueTF {#4}
        { \Struct{#2} \vDash #3 }
        { \Struct{#2}, #4 \vDash #3 }}
}

% Propositional satisfaction
\DeclareDocumentCommand \pAssign { m } {\applytofirst{\mathfrak}{#1}}

\DeclareDocumentCommand \pValue { m d() o}{
  \overline{\pAssign{#1}}%
  \IfNoValueTF{#3}{}{_{#3}}%
  \IfNoValueTF {#2}{}{(#2)}
}

\DeclareDocumentCommand \pSat { t{/} m m o } {
  \pAssign{#2}
      \IfBooleanTF{#1}{\nvDash}{\vDash}%
      \IfNoValueTF{#4}{}{_{#4}}
  #3
}

% Truth functions
\DeclareDocumentCommand \tf { m o } {
      \widetilde{#1}%
      \IfNoValueTF{#2}{}{_{#2}}
}


%% ===================================================================
%%% Derivability
%% ===================================================================

% \Proves[L]  -- derivability; \Proves/ for negated
\DeclareDocumentCommand \Proves { t{/} o } {
  \IfBooleanTF {#1}{
    \IfNoValueTF {#2}
        { \nvdash }
        { \nvdash_{#2} }}{
    \IfNoValueTF {#2}
        { \vdash }
        { \vdash_{#2} }}
}

\DeclareDocumentCommand \Thms { m } {\mathrm{Thm}(#1)}
\DeclareDocumentMacro \PAx { \mathrm{Ax}_0 }

% \Entails[L]  -- semantic consequence; \Entails/ for negated
\DeclareDocumentCommand \Entails { t{/} o } {
  \IfBooleanTF {#1}{
    \IfNoValueTF {#2}
        { \nvDash }
        { \nvDash_{#2} }}{
    \IfNoValueTF {#2}
        { \vDash }
        { \vDash_{#2} }}
}


%% ===================================================================
%%% Model Theory
%% ===================================================================

\DeclareDocumentCommand \Domain { m }{\left| \Struct{#1} \right|}

\DeclareDocumentCommand \Assign { m m }{\mathord{#1^{\Struct{#2}}}}

\DeclareDocumentCommand \varAssign { m m m o } {
    \IfNoValueTF {#4}
        { #1 \sim_{#3} #2 }
        { #1 = #2[^{#4}/{#3}] }
}

\DeclareDocumentCommand \Value { m m o} {
    \IfNoValueTF {#3}
        { \mathrm{Val}^{\Struct{#2}}(#1) }
        { \mathrm{Val}^{\Struct{#2}}_{#3}(#1) }
}

\DeclareDocumentMacro \substruct {\subseteq}

\DeclareDocumentCommand \Theory { m } {\mathrm{Th}(\Struct{#1})}

\DeclareDocumentCommand \Mod { o d() m } {
  \IfNoValueTF {#2} {
    \IfNoValueTF {#1}{
      \mathrm{Mod}(#3) }{
      \mathrm{Mod}^{\Lang{#1}}(#3) }}{
    \IfNoValueTF {#1}{
      \mathrm{Mod}_{#2}(#3)}{
      \mathrm{Mod}_{#2}^{\Lang{#1}}(#3)}}
}

% Elementary equivalence
\DeclareDocumentCommand \elemequiv { t{/} o } {
  \IfBooleanTF {#1}{
    \IfNoValueTF {#2}
        { \not\equiv }
        { \not\equiv_{#2} }}{
    \IfNoValueTF {#2}
        { \equiv }
        { \equiv_{#2} }}
}

% Equivalence classes and representatives
\DeclareDocumentCommand \eqc { m o } {
  \IfNoValueTF {#2}
  {[#1]}
  {[#1]_{#2}}
}

\DeclareDocumentCommand \rep { m o } {
  \IfNoValueTF {#2}
  {\underline{#1}}
  {{\underline{#1}}_{#2}}
}

% Isomorphism
\DeclareDocumentCommand \iso { t{/} o } {
  \IfBooleanTF {#1}{
    \IfNoValueTF {#2}
        { \not\simeq }
        { \not\simeq_{#2} }}{
    \IfNoValueTF {#2}
        { \simeq }
        { \simeq_{#2} }}
}

% Syntactic identity
\DeclareDocumentMacro \ident {\equiv}

% Quantifier rank
\DeclareDocumentCommand \QuantRank { m } {\mathrm{qr}(#1)}

% Expansion of a structure
\DeclareDocumentCommand \Expan { m m } {(\Struct{#1}, #2)}

% Non-standard arithmetic operations
\DeclareDocumentMacro \nszero {\mathbf{z}}
\DeclareDocumentMacro \nssucc {*}
\DeclareDocumentMacro \nsplus {\oplus}
\DeclareDocumentMacro \nstimes {\otimes}
\DeclareDocumentMacro \nsless {\varolessthan}


%% ===================================================================
%%% Typesetting Commands for Logical Concepts
%% ===================================================================

% \Struct{M}  -- first-order structures (first token in Fraktur)
\DeclareDocumentCommand \Struct { m }{\applytofirst{\mathfrak}{#1}}

% \Lang{L}  -- languages (first token in calligraphic)
\DeclareDocumentCommand \Lang { m }{\applytofirst{\mathcal}{#1}}

% \Log{L}[subscript]  -- logics (boldface)
\DeclareDocumentCommand \Log { m o }{\ensuremath{\mathbf{#1}
\IfNoValueTF {#2}{}{_{#2}}}}

% Named logics
\DeclareDocumentMacro {\LogCL} {\Log{C}}
\DeclareDocumentMacro {\LogIL} {\Log{I}}
\DeclareDocumentMacro {\LogLuk} {\Log{\textbf{\L}}}
\DeclareDocumentMacro {\LogGod} {\Log{G}}
\DeclareDocumentMacro {\LogKs} {\Log{Ks}}
\DeclareDocumentMacro {\LogKw} {\Log{Kw}}
\DeclareDocumentMacro {\LogLP} {\Log{LP}}
\DeclareDocumentMacro {\LogRM} {\Log{RM}}
\DeclareDocumentMacro {\LogHal} {\Log{Hal}}

% \Obj{x}  -- object-language symbols (sans-serif italics)
\DeclareDocumentCommand \Obj { m }{\mathsfit{#1}}

% \Atom{P}{t1,t2}  -- atomic formula
\DeclareDocumentCommand \Atom { m m }{ \mathord{#1}(#2) }

% \Ax{A}  -- axiom name
\DeclareDocumentCommand \Ax { m } {\ensuremath{\mathrm{#1}}}

% \PIso{I}  -- partial isomorphisms
\DeclareDocumentCommand \PIso { m }{\mathcal{#1}}

% \fn{func}  -- typeset a function name
\DeclareDocumentCommand \fn { m } {\mathrm{#1}}

% \Th{T}  -- typeset a theory name
\DeclareDocumentCommand \Th { m } {\mathbf{#1}}


%% ===================================================================
%%% Sets of Expressions
%% ===================================================================

\DeclareDocumentMacro \Var { \mathrm{Var} }
\DeclareDocumentMacro \PVar { \mathrm{At}_0 }

\DeclareDocumentCommand \Trm { o } {
    \IfNoValueTF {#1}
        { \mathrm{Trm} }
        { \mathrm{Trm}({\Lang #1}) }
}

\DeclareDocumentCommand \Frm { o } {
    \IfNoValueTF {#1}
        { \mathrm{Frm} }
        { \mathrm{Frm}({\Lang #1}) }
}

\DeclareDocumentCommand \TrmSOL { o } {
    \IfNoValueTF {#1}
        { \mathrm{Trm}^2 }
        { \mathrm{Trm}^2({\Lang #1}) }
}

\DeclareDocumentCommand \FrmSOL { o } {
    \IfNoValueTF {#1}
        { \mathrm{Frm}^2 }
        { \mathrm{Frm}^2({\Lang #1}) }
}

\DeclareDocumentCommand \SubFrm { m } {
        \mathrm{SFrm}({#1})
}

\DeclareDocumentCommand \FV { m } {
        \mathrm{FV}({#1})
}

\DeclareDocumentCommand \Sent { o } {
    \IfNoValueTF {#1}
        { \mathrm{Sent} }
        { \mathrm{Sent}({\Lang{#1}}) }
}


%% ===================================================================
%%% Computability / Arithmetic
%% ===================================================================

\DeclareDocumentCommand \Proj { m m } {P^{#1}_{#2}}
\DeclareDocumentMacro \Zero {\fn{zero}}
\DeclareDocumentMacro \Succ {\fn{succ}}
\DeclareDocumentMacro \Add {\fn{add}}
\DeclareDocumentMacro \Mult {\fn{mult}}
\DeclareDocumentMacro \Exp {\fn{exp}}
\DeclareDocumentMacro \Pred {\fn{pred}}
\DeclareDocumentMacro \tsub {\mathbin{\dot-}}

\DeclareDocumentCommand \Char { m } {\chi_{#1}}

\DeclareDocumentMacro \defis {=}
\DeclareDocumentMacro \defiff {\Leftrightarrow}
\DeclareDocumentMacro \concat {\frown}

% Bounded/unbounded minimization and quantification
\DeclareDocumentCommand \umin { m m } {\mu #1 \; #2}
\DeclareDocumentCommand \bmin { m m } {(\fn{min} \; #1)\, #2}
\DeclareDocumentCommand \bexists { m m } {(\exists #1)\; #2}
\DeclareDocumentCommand \bforall { m m } {(\forall #1)\; #2}

% Partial computable functions
\DeclareDocumentCommand \cfind { m o } {%
    \IfNoValueTF {#2}
        { \varphi_{#1} }
        { \varphi_{#1}^{#2} }
}

% Defined / undefined
\DeclareDocumentMacro \fdefined {\downarrow}
\DeclareDocumentMacro \fundefined {\uparrow}

% Partial function arrow
\DeclareDocumentMacro \pto {\mathrel{\ooalign{\hfil$\mapstochar\mkern
      5mu$\hfil\cr$\to$}}}


%% ===================================================================
%%% Lambda Calculus
%% ===================================================================

% One-step reduction
\DeclareDocumentCommand \redone { o } {
  \IfNoValueTF {#1}
  {\xrightarrow{}}
  {\xrightarrow{#1}}
}

\DeclareDocumentMacro \aconvone {\redone[\alpha]}
\DeclareDocumentMacro \bredone {\redone[\beta]}
\DeclareDocumentMacro \eredone {\redone[\eta]}
\DeclareDocumentMacro \beredone {\redone[\beta\eta]}
\DeclareDocumentMacro \xredone {\redone[X]}

% Multi-step reduction (double arrow)
\DeclareDocumentCommand \xrightarrowdbl { o m } {
  \IfNoValueTF {#1}
  {\xrightarrow{#2} \mathrel{\mkern-14mu}\rightarrow}
  {\xrightarrow[#1]{#2} \mathrel{\mkern-14mu}\rightarrow}
}

\DeclareDocumentCommand \red { o } {
  \IfNoValueTF {#1}
  {\xrightarrowdbl{}}
  {\xrightarrowdbl{#1}}
}

\DeclareDocumentMacro \aconv {\red [\alpha]}
\DeclareDocumentMacro \bred {\red [\beta]}
\DeclareDocumentMacro \ered {\red [\eta]}
\DeclareDocumentMacro \bered {\red [\beta\eta]}
\DeclareDocumentMacro \xred {\red [X]}

% Equivalence with label
\DeclareDocumentCommand \equal { o } {
  \IfNoValueTF {#1}
  {\eq}
  {\stackrel{#1}{\eq}}
}

\DeclareDocumentMacro \aeq {\equal [\alpha]}
\DeclareDocumentMacro \eqs {\equiv}

% Parallel reduction
\DeclareDocumentCommand \redpar { o } {
  \IfNoValueTF {#1}
  {\xLongrightarrow{}}
  {\xLongrightarrow{#1}}
}

\DeclareDocumentMacro \bredpar {\redpar [\beta]}
\DeclareDocumentMacro \beredpar {\redpar [\beta\eta]}
\DeclareDocumentMacro \eqa {\equal{\alpha}}
\DeclareDocumentMacro \eqe {\equal{\eta}}
\DeclareDocumentMacro \ext {\ensuremath{\mathit{ext}}}

% Complete development
\DeclareDocumentCommand \cd { o m } {
  \IfNoValueTF {#1}
  {{ #2 }^*}
  {{ #2 }^{* {#1} }}
}

\DeclareDocumentCommand \bcd { m } {
  \cd[\beta]{#1}
}

\DeclareDocumentCommand \becd { m } {
  \cd[\beta\eta]{#1}
}

% Lambda abstract
\DeclareDocumentCommand \lambd { o o } {
    \IfNoValueTF {#1}
        { \lambda }    % no arguments
        { \lambda #1 } % one argument
    \IfNoValueTF {#2}
        \relax
        { .\, #2 }      % two arguments
}


%% ===================================================================
%%% Goedel Numbering and Arithmetic Predicates
%% ===================================================================

% \num{n}  -- numeral
\DeclareDocumentCommand \num { m } {\overline{#1}}

% \scode{s}  -- code for a symbol
\DeclareDocumentCommand \scode { m } {\fn{c}_{#1}}

% \Gn{!A}  -- Goedel number (sharp-sign notation)
\DeclareDocumentCommand \Gn { m } {{^{\reflectbox{\tiny\#}}}{#1}{^{\mbox{\tiny\#}}}}

% \gn{!A}  -- Goedel corner quotes (defined above in the corner-quotes section)

% Proof / provability predicates
\DeclareDocumentCommand \Prf { o } { \mathrm{Prf}\IfNoValueTF {#1} {} {_{#1}}}
\DeclareDocumentCommand \OPrf { o } { \mathsf{Prf}\IfNoValueTF {#1} {} {_{#1}}}

\DeclareDocumentCommand \Refut { o } { \mathrm{Ref}\IfNoValueTF {#1} {} {_{#1}}}
\DeclareDocumentCommand \ORefut { o } { \mathsf{Ref}\IfNoValueTF {#1} {} {_{#1}}}

\DeclareDocumentCommand \Prov { o } { \mathrm{Prov}\IfNoValueTF {#1} {} {_{#1}}}
\DeclareDocumentCommand \OProv { o } { \mathsf{Prov}\IfNoValueTF {#1} {} {_{#1}}}

\DeclareDocumentCommand \RProv { o } { \mathrm{RProv}\IfNoValueTF {#1} {} {_{#1}}}
\DeclareDocumentCommand \ORProv { o } { \mathsf{RProv}\IfNoValueTF {#1} {} {_{#1}}}

\DeclareDocumentCommand \OCon { o } { \mathsf{Con}\IfNoValueTF {#1} {} {_{#1}}}

% Parthood predicate
\DeclareDocumentCommand \Part { m m } {\Atom{\Obj P}{#1, #2}}

% Factorial
\DeclareDocumentCommand \fact {m} {#1\,\mathexclaim}


%% ===================================================================
%%% General Mathematics
%% ===================================================================

% --- Set-theoretic operators ---

\DeclareDocumentCommand \Setabs { m m }{\{ #1 : #2 \}}
\DeclareDocumentCommand \fregeext { m m }{\oldepsilon #1 \, #2 }
\DeclareDocumentCommand \fregenum { m m }{\# #1 \, #2 }
\DeclareDocumentCommand \Pow { m }{\wp(#1)}
\DeclareDocumentCommand \dom { m }{\fn{dom}(#1)}
\DeclareDocumentCommand \ran { m }{\fn{ran}(#1)}
\DeclareDocumentCommand \len { m }{\fn{len}(#1)}
\DeclareDocumentMacro \emptyseq {\Lambda}
\DeclareDocumentMacro \restrict {\upharpoonright}
\DeclareDocumentCommand \Complement { m } {\overline{#1}}
\DeclareDocumentCommand \card { m } {\left| #1 \right|}
\DeclareDocumentCommand \cardle { m m } {#1 \preceq #2}
\DeclareDocumentCommand \cardless { m m } {#1 \prec #2}
\DeclareDocumentCommand \cardeq { m m } {#1 \approx #2}

% Tuples
\DeclareDocumentMacro \openTuple {\langle}
\DeclareDocumentMacro \closeTuple {\rangle}
\DeclareDocumentCommand \tuple { m } {\openTuple #1 \closeTuple}

% Composition
\DeclareDocumentCommand \comp { m m }{#2 \circ #1}

% Cut rank / max rank
\DeclareDocumentCommand \cutrank { m }{\fn{cr}(#1)}
\DeclareDocumentCommand \maxrank { m }{\fn{mr}(#1)}

% --- Particular sets ---

\DeclareDocumentMacro \Nat {\mathbb{N}}
\DeclareDocumentMacro \Int {\mathbb{Z}}
\DeclareDocumentMacro \PosInt {\mathbb{Z}^+}
\DeclareDocumentMacro \Real {\mathbb{R}}
\DeclareDocumentMacro \Rat {\mathbb{Q}}
\DeclareDocumentMacro \Bin {\mathbb{B}}

% Identity relation
\DeclareDocumentCommand \Id { m } {\mathord{\mathrm{Id}_{#1}}}

% --- Topological notions ---

\DeclareDocumentCommand \Top { m }{\mathcal{#1}}
\DeclareDocumentCommand \Interior { m }{\mathrm{Int}(#1)}

% --- Turing Machine symbols ---

\DeclareDocumentMacro \TMendtape {\triangleright}
\DeclareDocumentMacro \TMblank {0}
\DeclareDocumentMacro \TMstroke {1}
\DeclareDocumentMacro \TMright {R}
\DeclareDocumentMacro \TMleft {L}
\DeclareDocumentMacro \TMstay {N}
\DeclareDocumentCommand \TMtrans { m m m } {\ensuremath{#1, #2, #3}}


%% ===================================================================
%%% Modal Logic
%% ===================================================================

\DeclareDocumentCommand \mModel { m }{\applytofirst{\mathfrak}{#1}}

\DeclareDocumentCommand \mSat { t{/} m m o } {%
  \IfBooleanTF{#1}{%
    \IfNoValueTF {#4}
        { \mModel{#2} \nVdash #3 }
        { \mModel{#2}, #4 \nVdash #3}}{%
    \IfNoValueTF {#4}
        { \mModel{#2} \Vdash #3 }
        { \mModel{#2}, #4 \Vdash #3 }}}

\DeclareDocumentCommand \mClass { m }{\mathcal{#1}}

\DeclareDocumentMacro \Nec {\textsc{nec}}
\DeclareDocumentMacro \RK {\textsc{rk}}
\DeclareDocumentMacro \Dual {\textsc{dual}}
\DeclareDocumentMacro \Taut {\textsc{taut}}
\DeclareDocumentMacro \PL {\textsc{pl}}

\DeclareDocumentCommand \Prop { m m } {
  {[\!\![} #2 {]\!\!]_{\mModel{#1}}}
}

\DeclareDocumentMacro \ST {\mathord{\mathrm{ST}}}

\DeclareDocumentCommand \mTrue { m }{\ensuremath{#1}}
\DeclareDocumentCommand \mFalse { m }{\ensuremath{\lnot #1}}

% Epistemic operators
\DeclareDocumentMacro {\Knows} {\mathord{\mathsf{K}}}
\DeclareDocumentMacro {\EKnows} {\mathord{\mathsf{E}}}
\DeclareDocumentMacro {\CKnows} {\mathord{\mathsf{C}}}

% Temporal operators
\DeclareDocumentMacro {\Ptemp} {\mathord{\mathsf{P}}}
\DeclareDocumentMacro {\Htemp} {\mathord{\mathsf{H}}}
\DeclareDocumentMacro {\Ftemp} {\mathord{\mathsf{F}}}
\DeclareDocumentMacro {\Gtemp} {\mathord{\mathsf{G}}}
\DeclareDocumentMacro {\Since} {\mathord{\mathsf{S}}}
\DeclareDocumentMacro {\Until} {\mathord{\mathsf{U}}}

% TikZ style for modal models
\tikzset{
  modal/.style={>=stealth',
    shorten >=1pt,
    shorten <=1pt,
    auto,
    node distance=1.5cm,
    label distance=2pt,
    semithick},
  every label/.style={phantom,align=left},
  world/.style = {circle,draw,minimum size=0.5cm,fill=gray!15},
  modal every node/.style={world},
  point/.style={circle,draw,inner sep=0.5mm,fill=black},
  phantom/.style={rectangle,inner sep=0pt,draw=none,fill=none},
  reflexive above/.style={->,loop,looseness=7,in=60,out=120},
  reflexive below/.style={->,loop,looseness=7,in=240,out=300},
  reflexive left/.style={->,loop,looseness=7,in=150,out=210},
  reflexive right/.style={->,loop,looseness=7,in=30,out=330}
}


%% ===================================================================
%%% Set Theory Symbols  (from Tim Button's Open Set Theory)
%% ===================================================================

\DeclareDocumentMacro \unitline {\text{L}}
\DeclareDocumentMacro \unitsquare {\text{S}}
\newcommand{\onesphere}{\mathbf{S}}
\newcommand\rotationsgroup{R}

\DeclareDocumentCommand \cardneq { m m } {#1 \not\approx #2}
\DeclareDocumentCommand \cardnless { m m } {#1 \npreceq #2}
\DeclareDocumentCommand \ordeq { m m } {#1 \cong #2}
\DeclareDocumentCommand \ordneq { m m } {#1 \ncong #2}
\DeclareDocumentCommand \funimage { m m } {#1[#2]}

\newcommand\closureofunder[2]{\mathrm{clo}_{#1}(#2)}
\newcommand\Closureofunder[2]{\mathrm{Clo}_{#1}(#2)}
\newcommand\equivrep[2]{[#1]_{#2}}
\newcommand\equivclass[2]{#1/_{\!{#2}}}
\newcommand\Intequiv{\sim}
\newcommand\Ratequiv{\backsim}
\newcommand\Realequiv{\Bumpeq}
\newcommand\funrestrictionto[2]{#1\mathord{\restriction}_{#2}}
\newcommand\isomorphic{\cong}
\newcommand\nisomorphic{\ncong}
\newcommand\precdot{\mathrel{\prec{\mkern -12mu \cdot}}}
\newcommand\disjointsum{\sqcup}
\newcommand\ordtype[1]{\mathrm{ord}(#1)}
\newcommand\ordsucc[1]{#1^{+}}
\newcommand\cardsucc[1]{#1^{\oplus}}
\newcommand\rlexless{\mathrel{\sphericalangle}}
\newcommand\canonord\lhd
\newcommand\ordplus{+}
\newcommand\ordtimes{\cdot}
\newcommand\ordexpo[2]{#1^{(#2)}}
\newcommand\cardplus{\oplus}
\newcommand\cardtimes{\otimes}
\newcommand\cardexpo[2]{#1^{#2}}
\newcommand\funfromto[2]{{}^{#1}{#2}}
\newcommand\setrank[1]{\mathrm{rank}(#1)}
\DeclareMathOperator*{\supstrict}{\mathrm{lsub}}
\DeclareMathOperator\lcm{\mathrm{lcm}}
\newcommand\trcl[1]{\mathrm{trcl}(#1)}

% Named set theories
\newcommand\ZF{\Th{ZF}}
\newcommand\SP{\Th{SP}}
\newcommand\ZFC{\Th{ZFC}}
\newcommand\Z{\Th{Z}}
\newcommand\ZFminus{\ZF^{-}}
\newcommand\Zminus{\Z^{-}}
\newcommand\Zr{\Th{Zr}}
\newcommand\LT{\Th{LT}}

\newcommand\cardfont[1]{\mathfrak{#1}}

% Hilbert curve (for set theory diagrams)
\pgfdeclarelindenmayersystem{Hilbert curve}{
	\rule{L -> +RF-LFL-FR+}
	\rule{R -> -LF+RFR+FL-}}

% Stage-theoretic principles (italic labels)
\newcommand\stageshier{\emph{Stages-are-key}}
\newcommand\stagesord{\emph{Stages-are-ordered}}
\newcommand\stagesacc{\emph{Stages-accumulate}}
\newcommand\stagessucc{\emph{Stages-keep-going}}
\newcommand\stagesinf{\emph{Stages-hit-infinity}}
\newcommand\limofsize{\emph{Limitation-of-size}}
\newcommand\stagesinex{\emph{Stages-are-inexhaustible}}
\newcommand\stagescofin{\emph{Stages-are-super-cofinal}}


%% ===================================================================
%%% Inductive Definitions Helper
%% ===================================================================

% \indcase{formula}{complex formula}{case text}
%   Starred version for atomic case; ! version for exercise
\DeclareDocumentCommand \indcase { s t{!} m m +m }{%
  \DeclareDocumentMacro \indfrm {#3}%
  \DeclareDocumentMacro \indfrmp {#3}%
  \DeclareDocumentMacro \indcomplex {#4}%
  \IfBooleanTF{#1}
     {$#3$ is atomic: }{$#3 \ident #4$: }
  \IfBooleanTF{#2}
     {exercise.}
     {#5}}


%% ===================================================================
%%% Theorem Environments  (from open-logic-book-envs.sty, adapted)
%% ===================================================================

% Define OLP colors (self-contained defaults: black / light grey)
\@ifundefined{OLPltcolor}{%
  \colorlet{OLPltcolor}{black!5}%
}{}
\@ifundefined{OLPdkcolor}{%
  \colorlet{OLPdkcolor}{black}%
}{}

% Theorem box width
\newlength{\thmwidth}
\setlength{\thmwidth}{16pt}
\addtolength{\thmwidth}{\textwidth}

% mdframed styles
\mdfdefinestyle{thmstyle}{backgroundcolor=OLPltcolor,
  innerleftmargin=8pt,
  innerrightmargin=8pt,
  userdefinedwidth=\thmwidth,
  everyline=true,
  needspace=1cm,
  splittopskip=2\topsep,
  hidealllines=true,
  beforesingleframe={\hskip-8pt},
  beforebreak={\hskip-8pt}
}

\mdfdefinestyle{defstyle}{linecolor=OLPltcolor,
  linewidth=2pt,
  innerleftmargin=6pt,
  innerrightmargin=6pt,
  userdefinedwidth=\thmwidth,
  everyline=true,
  needspace=1cm,
  beforesingleframe={\hskip-8pt},
  beforebreak={\hskip-8pt}
}

\mdfdefinestyle{defishstyle}{linecolor=OLPltcolor,
  linewidth=2pt,
  innerleftmargin=6pt,
  innertopmargin=6pt,
  innerrightmargin=6pt,
  userdefinedwidth=\thmwidth,
  everyline=true,
  needspace=1cm,
  beforesingleframe={\hskip-8pt},
  beforebreak={\hskip-8pt}
}

% Theorem-like environments (numbered within chapter)
\declaretheorem[
  style=plain,
  name=Theorem,
  mdframed={style=thmstyle},
  numberwithin=chapter]{thm}

\declaretheorem[
  style=definition,
  name=Example,
  sibling=thm]{ex}

\declaretheorem[
  style=plain,
  name=Lemma,
  mdframed={style=thmstyle},
  refname={Lemma,Lemmas},
  sibling=thm]{lem}

\declaretheorem[
  style=plain,
  name=Proposition,
  mdframed={style=thmstyle},
  sibling=thm]{prop}

\declaretheorem[
  style=plain,
  name=Corollary,
  mdframed={style=thmstyle},
  refname={Corollary,Corollaries},
  sibling=thm]{cor}

\declaretheorem[
  style=definition,
  name=Definition,
  mdframed={style=defstyle},
  sibling=thm]{defn}

\declaretheorem[
  style=definition,
  name=Problem,
  numberwithin=chapter]{prob}

\declaretheorem[
  style=definition,
  name=Axiom,
  mdframed={style=defstyle},
  unnumbered]{axiom}

\declaretheorem[
  style=remark,
  name=Remark]{rem}

\declaretheorem[
  style=remark,
  name=Note]{note}

\declaretheorem[
  style=remark,
  name=Case]{case}

\declaretheorem[
  style=remark,
  name=Convention,
  mdframed={style=defstyle},
  sibling=thm]{conv}

% cleveref names
\crefname{thm}{Theorem}{Theorems}
\crefname{ex}{Example}{Examples}
\crefname{defn}{Definition}{Definitions}
\crefname{lem}{Lemma}{Lemmas}
\crefname{prop}{Proposition}{Propositions}
\crefname{prob}{Problem}{Problems}
\crefname{rem}{Remark}{Remarks}
\crefname{figure}{Figure}{Figures}
\crefname{table}{Table}{Tables}


%% ===================================================================
%%% Special Environments
%% ===================================================================

% defish -- definition-like box for rules, etc.
\newenvironment{defish}{\begin{mdframed}[style=defishstyle]}{\end{mdframed}}

% derivation -- tabular environment for Hilbert-style proofs
%   Three columns: line number, formula, justification
\newenvironment{derivation}{%
  ~\begin{trivlist}\item\begin{tabular}[b]{@{}rll@{}}}
  {\end{tabular} \end{trivlist}}

% oltableau -- wrapper for prooftrees tableaux
\newenvironment{oltableau}{\center\tableau{}}
               {\endtableau\endcenter}

% probtag -- wrapper (tags are no-ops in the lean text)
\newenvironment{probtag}[1]{\begin{prob}}{\end{prob}}


%% ===================================================================
%%% End of lean-preamble.sty
%% ===================================================================

\makeatother

\endinput


% Page layout
\linespread{1.05}  % Optimized for Palatino

% Numbering
\setsecnumdepth{subsection}
\settocdepth{section}

% Chapter style
\chapterstyle{bianchi}

\begin{document}

%% ===================================================================
%%% Front Matter
%% ===================================================================

\frontmatter

\begin{titlingpage}
\begin{center}
\vspace*{3cm}
{\Huge\bfseries A Lean Systematization of\\[0.5em] Mathematical Logic}

\vspace{2cm}

{\Large Based on the OpenLogic Project}

\vspace{1cm}

{\large Organized by the Taxonomy of Mathematical Logic\\[0.5em]
6 Domains $+$ Metatheory $\cdot$ 205 Items $\cdot$ 13 Composition Patterns}

\vspace{3cm}

{\normalsize
Every concept traceable to primitives.\\
No redundancy. Full dependency graph.}

\vspace{2cm}

{\small Compiled from \textit{Open Logic: A Complete Text}\\
by the Open Logic Project\\[0.5em]
Systematized using the taxonomy at
\texttt{taxonomy/}}
\end{center}
\end{titlingpage}

\tableofcontents

%% ===================================================================
%%% Main Matter
%% ===================================================================

\mainmatter

% Chapter dependency order:
%   BST -> SYN -> SEM (parallel with DED) -> DED -> CMP -> META -> SET -> EXT
%
% BST provides the naive set-theoretic metalanguage (Level-0).
% SYN, SEM, DED define the formal logic.
% CMP adds computation theory.
% META collects cross-domain composition patterns (soundness, completeness, etc.).
% SET is ZFC as a formal first-order theory (Level-1).
% EXT points to extensions (modal, intuitionistic, etc.).

\chapter{Set-Theoretic Foundations} \label{ch:bst}

%% ===================================================================
%% BST.1: Sets and Membership
%% ===================================================================

\section{Sets and Membership} \label{BST.1}

A \emph{set} is a collection of objects, considered as a single
object. The objects making up the set are called \emph{elements} or
\emph{members} of the set. % PRIM-BST002
If $x$ is an element of a set~$A$, we
write $x \in A$; if not, we write $x \notin A$. The set which has no
elements is called the \emph{empty} set and
denoted~``$\emptyset$''. % PRIM-BST004

It does not matter how we \emph{specify} the set, or how we
\emph{order} its elements, or indeed how \emph{many times} we
count its elements. All that matters are what its elements
are. We codify this in the following principle.

\begin{defn}[Extensionality] % PRIM-BST001
\label{PRIM-BST001}
  If $A$ and $B$ are sets, then $A = B$ iff
  every element of~$A$ is also an element of~$B$, and vice
  versa.
\end{defn}

Extensionality licenses some notation. In general, when we have some
objects $a_{1}$, \dots, $a_{n}$, then $\{a_{1}, \dots, a_{n}\}$ is
\emph{the} set whose elements are $a_1, \ldots, a_n$. We emphasise
the word ``\emph{the}'', since extensionality tells us that there can
be only \emph{one} such set. Indeed, extensionality also licenses the
following:
  \[
    \{a, a, b\} = \{a, b\} = \{b,a\}.
  \]
This delivers on the point that, when we consider sets, we don't care
about the order of their elements, or how many times they are
specified.

Frequently we'll specify a set by some property that its elements
share. We'll use the following shorthand notation for that:
$\Setabs{x}{\phi(x)}$, where the $\phi(x)$ stands for the property
that~$x$ has to have in order to be counted among the elements of
the set.\footnote{Unrestricted set-builder notation leads to
paradoxes: for instance, the set $R = \Setabs{x}{x \notin x}$ cannot
exist (Russell's Paradox). Throughout this text, we work in naive set
theory and rely on set-builder notation only for unproblematic
instances of comprehension.}

Extensionality gives us a way for showing that sets are identical: to
show that $A = B$, show that whenever $x \in A$ then also $x \in B$,
and whenever $y \in B$ then also $y \in A$.


%%% Subsets and Power Sets (from sfr/set/sub — KEEP)

We will often want to compare sets. One obvious kind of comparison
is: \emph{everything in one set is in the other too}.

\begin{defn}[Subset] % PRIM-BST003
\label{PRIM-BST003}
If every element of a set $A$ is also an element of~$B$, then we
say that $A$ is a \emph{subset} of~$B$, and write $A \subseteq B$. If
$A$ is not a subset of~$B$ we write $A \not\subseteq B$.
If $A \subseteq B$ but $A \neq B$, we write $A \subsetneq B$ and say
that $A$ is a \emph{proper subset} of $B$.
\end{defn}

The notation $(\forall x \in A)\phi$ abbreviates $\forall x(x \in A
\lif \phi)$, and $(\exists x \in A)\phi$ abbreviates $\exists x(x \in
A \land \phi)$. Using this notation, $A \subseteq B$ iff $(\forall x
\in A)\, x \in B$.

\begin{prop} \label{PRIM-BST001:subset-char}
$A = B$ iff both $A \subseteq B$ and $B \subseteq A$.
\end{prop}

Now we consider a certain kind of set: the set of all subsets of a
given set.

\begin{defn}[Power Set] % PRIM-BST015
\label{PRIM-BST015}
The set consisting of all subsets of a set~$A$ is called the
\emph{power set of}~$A$, written $\Pow{A}$.
  \[
    \Pow{A} = \Setabs{B}{B \subseteq A}
  \]
\end{defn}

\begin{ex}
What are all the possible subsets of $\{ a, b, c \}$? They are:
$\emptyset$, $\{a \}$, $\{b\}$, $\{c\}$, $\{a, b\}$, $\{a, c\}$, $\{b,
c\}$, $\{a, b, c\}$. The set of all these subsets is
$\Pow{\{a,b,c\}}$:
\[
\Pow{\{ a, b, c \}} = \{\emptyset, \{a \}, \{b\}, \{c\}, \{a, b\},
\{b, c\}, \{a, c\}, \{a, b, c\}\}
\]
\end{ex}


%%% Unions and Intersections (from sfr/set/uni — KEEP)

We can define new sets from old by combining their elements, or by
taking only those elements they share.

\begin{defn}[Union] % PRIM-BST005
\label{PRIM-BST005}
The \emph{union} of two sets $A$ and $B$, written $A \cup B$, is the
set of all things which are elements of $A$, $B$, or both.
\[
A \cup B = \Setabs{x}{x \in A \lor x \in B}
\]
\end{defn}

\begin{defn}[Intersection] % PRIM-BST005
The \emph{intersection} of two sets $A$ and $B$, written $A \cap B$, is
the set of all things which are elements of both $A$ and~$B$.
\[
A \cap B = \Setabs{x}{x \in A \land x \in B}
\]
Two sets are called \emph{disjoint} if their intersection is
empty. This means they have no elements in common.
\end{defn}

We can also form the union or intersection of more than two
sets. An elegant way of dealing with this in general is the
following: suppose you collect all the sets you want to form the union
(or intersection) of into a single set. Then we can define the union
of all our original sets as the set of all objects which belong to at
least one element of the set, and the intersection as the set of
all objects which belong to every element of the set.

\begin{defn}[General Union]
If $A$ is a set of sets, then $\bigcup A$ is the set of elements of
elements of~$A$:
\begin{align*}
\bigcup A & = \Setabs{x}{x \text{ belongs to an element of } A}\\
& = \Setabs{x}{\text{there is a } B \in A
  \text{ so that } x \in B}
\end{align*}
\end{defn}

\begin{defn}[General Intersection]
If $A$ is a set of sets, then $\bigcap A$ is the set of objects which
all elements of~$A$ have in common:
\begin{align*}
\bigcap A & = \Setabs{x}{x \text{ belongs to every element of } A}\\
 & = \Setabs{x}{\text{for all } B \in A, x \in B}
\end{align*}
\end{defn}

\begin{ex}
Suppose $A = \{ \{ a, b \}, \{ a, d, e \}, \{ a, d \} \}$.
Then $\bigcup A = \{ a, b, d, e \}$ and $\bigcap A = \{ a \}$.
\end{ex}

We could also do the same for a sequence of sets $A_1$, $A_2$, \dots
\begin{align*}
\bigcup_i A_i & = \Setabs{x}{x \text{ belongs to one of the } A_i}\\
\bigcap_i A_i & = \Setabs{x}{x \text{ belongs to every } A_i}.
\end{align*}

When we have an \emph{index} of sets, i.e., some set $I$ such that we
are considering $A_i$ for each $i \in I$, we may also write:
\begin{align*}
	\bigcup_{i \in I} A_i & = \bigcup \Setabs{A_i }{i \in I}\\
	\bigcap_{i \in I} A_i & = \bigcap\Setabs{A_i}{i \in I}
\end{align*}

The \emph{set difference}~$A \setminus B$ is the set of all elements of
$A$ which are not also elements of~$B$, i.e.,
$A\setminus B = \Setabs{x}{x\in A \text{ and } x \notin B}$.


%%% Pairs, Tuples, Cartesian Products (from sfr/set/pai — ABSORB)

It follows from extensionality that sets have no order to their
elements. So if we want to represent order, we use \emph{ordered
pairs} $\tuple{x, y}$. In an unordered pair $\{x, y\}$, the order does
not matter: $\{x, y\} = \{y, x\}$. In an ordered pair, it does: if $x
\neq y$, then $\tuple{x, y} \neq \tuple{y, x}$.

We want to preserve the idea that ordered pairs are identical iff they
share the same first element and share the same second element, i.e.:
\[
  \tuple{a, b}= \tuple{c, d}\text{ iff both }a = c \text{ and }b=d.
\]
We can define ordered pairs in set theory using the Wiener--Kuratowski
definition.

\begin{defn}[Ordered pair] % PRIM-BST006
\label{PRIM-BST006}
	$\tuple{a, b} = \{\{a\}, \{a, b\}\}$.
\end{defn}

This reduces ordered pairs to sets: $\tuple{a,b}$ is simply the
set $\{\{a\},\{a,b\}\}$.

We can use ordered pairs to define ordered sequences of
more than two objects. \emph{Triples} $\tuple{x, y, z}$ are
$\tuple{\tuple{x, y},z}$,
\emph{quadruples} $\tuple{x,y,z,u}$ are
$\tuple{\tuple{\tuple{x,y},z},u}$, and so on. In general, we talk of
\emph{ordered $n$-tuples} $\tuple{x_1, \dots, x_n}$.

\begin{defn}[Cartesian product] % PRIM-BST007
\label{PRIM-BST007}
Given sets $A$ and $B$, their \emph{Cartesian product} $A \times B$ is
defined by
\[
  A \times B = \Setabs{\tuple{x, y}}{x \in A \text{ and } y \in B}.
\]
\end{defn}

If $A$ is a set, the product of $A$ with itself, $A \times A$, is also
written~$A^2$. It is the set of \emph{all} pairs $\tuple{x, y}$ with
$x, y \in A$. The set of all triples $\tuple{x, y, z}$ is $A^3$, and
so on. We can give a recursive definition:
\begin{align*}
  A^1 & = A\\
  A^{k+1} & = A^k \times A
\end{align*}

Finite sequences (words) over $A$ and the set $A^*$ are defined
formally in \S\ref{BST.4}.


%%% Important Sets (from sfr/set/imp — CONDENSE: N only here)

\begin{defn}[Natural Numbers] % PRIM-BST012
\label{PRIM-BST012}
The set of \emph{natural numbers} is denoted $\Nat = \{0, 1, 2, 3,
\ldots\}$.
\end{defn}

The natural numbers form the most basic infinite set and serve as the
foundation for counting, induction, and recursion throughout
mathematical logic.


%% ===================================================================
%% BST.2: Relations
%% ===================================================================

\section{Relations} \label{BST.2}

%%% Relations as Sets (from sfr/rel/set — KEEP)

Recall from \S\ref{BST.1} the notion of an ordered pair $\tuple{a,b}$ and
the Cartesian product $A \times B$. We can use these to give a
set-theoretic treatment of relations: a relation on a set is simply a
set of ordered pairs.

Consider a particular relation on a set: the $<$-relation
on the set~$\Nat$ of natural numbers. Consider the set of all pairs of
numbers $\tuple{n, m}$ where $n<m$, i.e.,
\[
  R=\Setabs{\tuple{n, m}}{n, m \in \Nat \text{ and } n<m}.
\]
There is a close connection between $n$ being less than $m$, and the
pair $\tuple{n, m}$ being a member of $R$, namely:
$n<m$ iff $\tuple{n, m} \in R$.
Indeed, without any loss of information, we can consider the set $R$
to \emph{be} the $<$-relation on $\Nat$. This justifies the following
definition:

\begin{defn}[Binary relation] % PRIM-BST008
\label{PRIM-BST008}
A \emph{binary relation} on a set $A$ is a subset of~$A^{2}$. If $R
\subseteq A^{2}$ is a binary relation on~$A$ and $x, y \in A$, we
sometimes write $Rxy$ (or $xRy$) for $\tuple{x, y} \in R$.
\end{defn}

\begin{ex}
The set $\Nat^{2}$ of pairs of natural numbers can be listed in a
2-dimensional matrix like this:
\[
  \begin{array}{ccccc}
  \mathbf{\tuple{ 0,0 }} & \tuple{ 0,1 } &
    \tuple{ 0,2 } & \tuple{ 0,3 } & \ldots\\
  \tuple{ 1,0 } & \mathbf{\tuple{ 1,1 }} &
    \tuple{ 1,2 } & \tuple{ 1,3 } & \ldots\\
  \tuple{ 2,0 } & \tuple{ 2,1 } &
    \mathbf{\tuple{ 2,2 }} & \tuple{ 2,3 } & \ldots\\
  \tuple{ 3,0 } & \tuple{ 3,1 } & \tuple{ 3,2 } &
    \mathbf{\tuple{ 3,3 }} & \ldots\\
  \vdots & \vdots & \vdots & \vdots & \mathbf{\ddots}
  \end{array}
\]
The diagonal pairs $\{\tuple{0,0 }, \tuple{ 1,1 }, \tuple{ 2,2 },
\dots\}$ form the \emph{identity relation on}~$\Nat$. We define
$\Id{A}=\Setabs{\tuple{ x,x }}{x \in A}$ for any set $A$. The pairs
above the diagonal form the \emph{less than} relation, those below the
diagonal form the \emph{greater than} relation.
\end{ex}

According to this definition, \emph{any} subset of $A^{2}$ is a
relation on~$A$. In particular, $\emptyset$ is a relation on any set
(the empty relation), and $A^{2}$~itself is a relation on~$A$ (the
universal relation).


%%% Special Properties of Relations (from sfr/rel/prp — KEEP)

Some kinds of relations are so common that they have been given special
names. We categorize relations according to special properties. Combinations
of these properties yield orders and equivalence relations.

\begin{defn}[Reflexivity]
A relation $R \subseteq A^2$ is \emph{reflexive} iff, for every $x \in
A$, $Rxx$.
\end{defn}

\begin{defn}[Transitivity]
A relation $R \subseteq A^2$ is \emph{transitive} iff, whenever $Rxy$
and $Ryz$, then also $Rxz$.
\end{defn}

\begin{defn}[Symmetry]
A relation~$R \subseteq A^2$ is \emph{symmetric} iff, whenever
$Rxy$, then also~$Ryx$.
\end{defn}

\begin{defn}[Anti-symmetry]
A relation~$R \subseteq A^2$ is \emph{anti-sym\-met\-ric} iff, whenever both
$Rxy$ and $Ryx$, then $x=y$ (or, in other words: if $x\neq y$ then
either $\lnot Rxy$ or $\lnot Ryx$).
\end{defn}

Note that being anti-symmetric and merely not being symmetric are
different conditions. A relation can be both symmetric and
anti-symmetric (e.g., the identity relation).

\begin{defn}[Connectivity]
A relation $R \subseteq A^2$ is \emph{connected} if for all $x,y\in
A$, if $x \neq y$, then either $Rxy$ or~$Ryx$.
\end{defn}

\begin{defn}[Irreflexivity]
A relation $R \subseteq A^2$ is called \emph{irreflexive} if, for all $x \in
A$, not $Rxx$.
\end{defn}

\begin{defn}[Asymmetry]
A relation $R \subseteq A^2$ is called \emph{asymmetric} if for no pair $x,y\in
A$ we have both $Rxy$ and~$Ryx$.
\end{defn}

Note that if $A \neq \emptyset$, then no irreflexive relation on~$A$
is reflexive and every asymmetric relation on~$A$ is also
anti-symmetric. However, there are $R \subseteq A^2$ that are not
reflexive and also not irreflexive, and there are anti-symmetric
relations that are not asymmetric.


%%% Equivalence Relations (from sfr/rel/eqv — KEEP)

The identity relation on a set is reflexive, symmetric, and
transitive. Relations~$R$ that have all three of these properties are very
common.

\begin{defn}[Equivalence relation] % DEF-BST004
\label{DEF-BST004}
A relation $R \subseteq A^2$ that is reflexive, symmetric, and
transitive is called an \emph{equivalence relation}. Elements $x$
and $y$ of~$A$ are said to be \emph{$R$-equivalent} if~$Rxy$.
\end{defn}

Equivalence relations give rise to the notion of an \emph{equivalence
class}. An equivalence relation ``chunks up'' the domain into
different partitions. Within each partition, all the objects are
related to one another; and no objects from different partitions
relate to one another.

\begin{defn}[Equivalence class]
Let $R \subseteq A^2$ be an equivalence relation. For each $x \in A$,
the \emph{equivalence class} of $x$ in~$A$ is the set $\equivrep{x}{R}
= \Setabs{y \in A}{Rxy}$. The \emph{quotient} of $A$ under~$R$ is
$\equivclass{A}{R} = \Setabs{\equivrep{x}{R}}{x \in A}$, i.e., the set
of these equivalence classes.
\end{defn}

The next result proves that the equivalence classes are indeed the
partitions of~$A$:

\begin{prop} \label{DEF-BST004:partition}
If $R \subseteq A^2$ is an equivalence relation, then $Rxy$ iff
$\equivrep{x}{R} = \equivrep{y}{R}$.
\end{prop}

\begin{proof}
For the left-to-right direction, suppose $Rxy$, and let $z \in
\equivrep{x}{R}$. By definition, then, $Rxz$. Since $R$ is an
equivalence relation, $Ryz$. (Spelling this out: as $Rxy$ and~$R$ is
symmetric we have $Ryx$, and as $Rxz$ and~$R$ is transitive we
have~$Ryz$.) So $z \in \equivrep{y}{R}$. Generalising,
$\equivrep{x}{R} \subseteq \equivrep{y}{R}$. But exactly similarly,
$\equivrep{y}{R} \subseteq \equivrep{x}{R}$. So $\equivrep{x}{R} =
\equivrep{y}{R}$, by extensionality.

For the right-to-left direction, suppose $\equivrep{x}{R} =
\equivrep{y}{R}$. Since $R$ is reflexive, $Ryy$, so $y \in
\equivrep{y}{R}$. Thus also $y \in \equivrep{x}{R}$ by the assumption
that $\equivrep{x}{R} = \equivrep{y}{R}$. So $Rxy$.
\end{proof}

\begin{ex}
A nice example of equivalence relations comes from modular arithmetic.
For any $a$, $b$, and $n \in \PosInt$, say that $a \equiv_n b$ iff
dividing $a$ by~$n$ gives the same remainder as dividing $b$ by~$n$.
(Somewhat more symbolically: $a \equiv_n b$ iff, for some $k \in
\Int$, $a - b = kn$.) Now, $\equiv_n$ is an equivalence relation, for
any~$n$. And there are exactly $n$ distinct equivalence classes
generated by~$\equiv_n$; that is, $\equivclass{\Nat}{\equiv_n}$ has
$n$ elements. These are: the set of numbers divisible by $n$
without remainder, i.e., $\equivrep{0}{\equiv_n}$; the set of numbers
divisible by $n$ with remainder~$1$, i.e., $\equivrep{1}{\equiv_n}$;
\ldots; and the set of numbers divisible by~$n$ with remainder~$n-1$,
i.e.,~$\equivrep{n-1}{\equiv_n}$.
\end{ex}


%%% Orders (from sfr/rel/ord — CONDENSE)

Many of our comparisons involve describing some objects as being
``less than'', ``equal to'', or ``greater than'' other objects, in a
certain respect. These involve \emph{order} relations. There are
different kinds: some require that any two objects be comparable,
others don't; some include identity (like~$\le$) and some exclude it
(like~$<$).

\begin{defn}[Preorder]
A relation which is both reflexive and transitive is called a
\emph{preorder.}
\end{defn}

\begin{defn}[Partial order] % DEF-BST005
\label{DEF-BST005}
A preorder which is also anti-symmetric is called a
\emph{partial order}.
\end{defn}

\begin{defn}[Linear order]
A partial order which is also connected is called a
\emph{total order} or \emph{linear order.}
\end{defn}

\begin{defn}[Strict order]
A \emph{strict order} is a relation which is irreflexive, asymmetric,
and transitive.
\end{defn}

\begin{defn}[Strict linear order]
A strict order which is also connected is called a
\emph{strict total order} or \emph{strict linear order.}
\end{defn}

\begin{ex}
An important partial order is the relation $\subseteq$ on a set of
sets. This is not in general a linear order, since if $a \neq b$ and
we consider $\Pow{\{a, b\}} = \{\emptyset, \{a\}, \{b\}, \{a,b\}\}$,
we see that $\{a\} \nsubseteq \{b\}$ and $\{a\} \neq \{b\}$ and $\{b\}
\nsubseteq \{a\}$.
\end{ex}

Any strict order $R$ on~$A$ can be turned into a partial order by
adding the diagonal $\Id{A}$, i.e., adding all the pairs~$\tuple{x,
x}$.  (This is called the \emph{reflexive closure} of~$R$.)

\begin{prop}\label{prop:stricttopartial}
If $R$ is a strict order on~$A$, then $R^+ = R \cup \Id{A}$ is a
partial order. Moreover, if $R$ is a strict linear order, then $R^+$ is
a linear order.
\end{prop}

\begin{proof}[Proof sketch]
Suppose $R$ is a strict order on~$A$ and let $R^+ = R \cup \Id{A}$.
\emph{Reflexivity}: $\tuple{x, x} \in \Id{A} \subseteq R^+$ for all
$x \in A$.
\emph{Anti-symmetry}: if $R^+xy$ and $R^+yx$ with $x \neq y$, then
$Rxy$ and $Ryx$, contradicting asymmetry.
\emph{Transitivity}: if $R^+xy$ and $R^+yz$, then either both pairs
are in $R$ (use transitivity of~$R$), or one is in $\Id{A}$ (use $x=y$
or $y=z$). The ``moreover'' clause follows because connectivity of~$R$
is inherited by~$R^+$.
\end{proof}


%%% Trees (from sfr/rel/tre — CONDENSE)

A particular kind of partial order that plays an important role in
logic is a \emph{tree}. Finite trees occur in syntax (formula
decomposition) and derivation systems, while infinite trees appear in
completeness proofs.

A \emph{minimal element} in a set~$A$ partially ordered
by~$\le$ is an element $x \in A$ such that for all $y \in A$ we have
that~$x \le y$. A set is \emph{well-ordered} by~$\le$ if every one of
its subsets has a minimal element.

\begin{defn}[Tree]
A \emph{tree} is a pair $T = \tuple{A, \le}$ such that $A$ is a set
and $\le$ is a partial order on~$A$ with a unique minimal element
$r \in A$ (called the \emph{root}) such that for all $x \in A$,
the set $\Setabs{y}{y \le x}$ is well-ordered by~$\le$.
\end{defn}

\begin{defn}[Successors]
Suppose $T = \tuple{A, \le}$ is a tree.
If $x,y \in A$, $x < y$, and there is no $z \in A$ such that
$x < z < y$, then we say that $y$ is a \emph{successor} (or
\emph{child}) of~$x$, and $x$ is the \emph{predecessor} (or
\emph{parent}) of~$y$.
\end{defn}

\begin{defn}[Branches]
Given a tree $T = \tuple{A, \le}$, a \emph{branch} of~$T$ is a
maximal chain in~$T$, i.e., a set $B \subseteq A$ such that
for any $x, y \in B$ either $x \le y$ or $y \le x$, and for any
$z \in A \setminus B$ there exists $u \in B$ such that neither
$z \le u$ nor $u \le z$.
We use $[T]$ to denote the set of all branches of $T$.
\end{defn}

A tree is \emph{infinite} if its underlying set is infinite, and
\emph{finitely branching} if every node has only finitely many
successors.

\begin{ex}
A classic example of a finitely branching tree is the
\emph{infinite binary tree} $\Bin^*$, ordered by the extension
relation $\sqsubseteq$ (e.g., $101 \sqsubseteq 101101$).
Every element~$s$ has exactly two successors, $s0$ and~$s1$, and its
root is the empty sequence $\emptyseq$.
\end{ex}


%%% Operations on Relations (from sfr/rel/ops — CONDENSE)

It is often useful to modify or combine relations. Here are some
fundamental operations.

\begin{defn}[Operations on relations]
Let $R$, $S$ be relations, and $A$ be any set.

The \emph{inverse} of $R$ is $R^{-1} = \Setabs{\tuple{y, x}}{\tuple{x,
    y} \in R}$.

The \emph{relative product} of $R$ and $S$ is $(R \mid S) =
\Setabs{\tuple{x, z}}{\exists y(Rxy \land Syz)}$.

The \emph{restriction} of $R$ to $A$ is $\funrestrictionto{R}{A}= R
\cap A^2$.

The \emph{application} of $R$ to $A$ is $\funimage{R}{A} = \Setabs{y}{
(\exists x \in A)\, Rxy}$.
\end{defn}

\begin{defn}[Transitive closure]
Let $R \subseteq A^2$ be a binary relation.

The \emph{transitive closure} of~$R$ is $R^+ = \bigcup_{0 < n \in
\Nat} R^n$, where we recursively define $R^1 = R$ and $R^{n+1} = R^n
\mid R$.

The \emph{reflexive transitive closure} of $R$ is $R^* = R^+ \cup
\Id{A}$.
\end{defn}


%% ===================================================================
%% BST.3: Functions
%% ===================================================================

\section{Functions} \label{BST.3}

%%% Basics (from sfr/fun/bas — KEEP)

A \emph{function} is a map which sends each element of a given set
to a specific element in some (other) given set. For instance, the
operation of adding~$1$ defines a function: each number~$n$ is mapped
to a unique number~$n+1$.

More generally, functions may take pairs, triples, etc., as inputs and
return some kind of output. In a mathematical, abstract sense, a function
is a \emph{black box}: what matters is only what output is paired with
what input, not the method for calculating the output.

\begin{defn}[Function] % PRIM-BST009
\label{PRIM-BST009}
A \emph{function} $f \colon A \to B$ is a mapping of each element
of~$A$ to an element of~$B$.

We call $A$ the \emph{domain} of~$f$ and $B$ the \emph{codomain}
of~$f$.  The elements of~$A$ are called inputs or \emph{arguments}
of~$f$, and the element of~$B$ that is paired with an argument~$x$
by~$f$ is called the \emph{value of~$f$} for argument~$x$,
written~$f(x)$.

The \emph{range} $\ran{f}$ of~$f$ is the subset of the codomain
consisting of the values of~$f$ for some argument; $\ran{f} =
\Setabs{f(x)}{x \in A}$.
\end{defn}

\begin{ex}
Let $f \colon \Nat \to \Nat$ be defined such that $f(x) = x+1$. This
is a function which takes in natural numbers and outputs natural numbers.
Given a natural number~$x$, $f$ will output its successor~$x+1$.
In this case, the codomain $\Nat$ is not the range of~$f$, since the
natural number~$0$ is not the successor of any natural number. The
range of~$f$ is the set of all positive integers, $\PosInt$.
\end{ex}

Two functions $f$ and $g$ are the same if they have the same domain,
codomain, and agree on all values: if $\forall x\, f(x) = g(x)$, then
$f = g$.


%%% Kinds of Functions (from sfr/fun/kin — KEEP)

We introduce a taxonomy for the most frequently encountered kinds of
functions.

\begin{defn}[Surjective function] % DEF-BST002
\label{DEF-BST002}
A function $f \colon A \rightarrow B$ is \emph{surjective} iff $B$
is also the range of~$f$, i.e., for every $y \in B$ there is at least
one $x \in A$ such that~$f(x) = y$, or in symbols:
\[
  (\forall y \in B)(\exists x \in A)\, f(x) = y.
\]
We call such a function a surjection from $A$ to $B$.
\end{defn}

\begin{defn}[Injective function] % DEF-BST001
\label{DEF-BST001}
A function $f \colon A \rightarrow B$ is \emph{injective} iff for
each $y \in B$ there is at most one $x \in A$ such that~$f(x) = y$. We
call such a function an injection from $A$ to~$B$.
\end{defn}

\begin{defn}[Bijection] % DEF-BST003
\label{DEF-BST003}
A function $f \colon A \to B$ is \emph{bijective} iff it is both
surjective and injective. We call such a function
a bijection from $A$ to~$B$ (or between $A$ and~$B$).
\end{defn}

\begin{ex}
The constant function $f\colon \Nat \to \Nat$ given by $f(x) = 1$ is
neither injective, nor surjective.
The identity function $f\colon \Nat \to \Nat$ given by $f(x) = x$ is
both injective and surjective.
The successor function $f \colon \Nat \to \Nat$ given by $f(x) = x+1$
is injective but not surjective.
The function $f \colon \Nat \to \Nat$ defined by:
\[
  f(x) =
  \begin{cases}
    \frac{x}{2} & \text{if $x$ is even} \\
    \frac{x+1}{2} & \text{if $x$ is odd.}
  \end{cases}
\]
is surjective, but not injective.
\end{ex}


%%% Inverses of Functions (from sfr/fun/inv — KEEP)

We now ask whether the mapping defined by a function can be
``reversed.'' This is made precise by the notion of an inverse.

\begin{defn}[Inverse]
A function $g \colon B \to A$ is an \emph{inverse} of a function $f
\colon A \to B$ if $f(g(y)) = y$ and $g(f(x)) = x$ for all $x \in A$
and $y \in B$.
\end{defn}

If $f$ has an inverse~$g$, we often write $f^{-1}$ instead of~$g$.

\begin{prop}\label{prop:inj-left-inv}
If $f\colon A \to B$ is injective, then there is a \emph{left
inverse}~$g\colon B \to A$ of~$f$ so that $g(f(x)) = x$ for all $x
\in A$.
\end{prop}

\begin{proof}[Proof sketch]
Since $f$ is injective, for each $y \in \ran{f}$ there is exactly one
$x$ with $f(x) = y$; define $g(y) = x$ in that case. For $y \notin
\ran{f}$, map $g(y)$ to any fixed $a \in A$. Then $g(f(x)) = x$ for
all $x \in A$ by construction.
\end{proof}

\begin{prop}\label{prop:surj-right-inv}
    If $f\colon A \to B$ is surjective, then there is a
    \emph{right inverse}~$h\colon B \to A$ of~$f$ so that $f(h(y)) =
    y$ for all~$y \in B$.
\end{prop}

The proof requires choosing, for each $y \in B$, some $x \in A$ with
$f(x) = y$. That such a choice is always possible is guaranteed by the
Axiom of Choice. In many specific cases (e.g., when $A = \Nat$, or when
$A$ is finite, or when $f$ is bijective), the Axiom of Choice is
not required.

\begin{prop}\label{prop:bijection-inverse}
If $f\colon A \to B$ is bijective, there is a
function~$f^{-1}\colon B \to A$ so that for all $x \in A$,
$f^{-1}(f(x)) = x$ and for all $y \in B$, $f(f^{-1}(y)) = y$.
\end{prop}

\begin{proof}[Proof sketch]
Since $f$ is injective, it has a left inverse~$g$. Since $f$ is
surjective, it has a right inverse~$h$. For any $y \in B$:
$g(y) = g(f(h(y))) = h(y)$, so $g = h$. This common function
$f^{-1} = g = h$ is both a left and right inverse of~$f$.
\end{proof}


%%% Composition of Functions (from sfr/fun/cmp — KEEP)

We can define a new function by composing two functions $f$ and~$g$,
i.e., by first applying $f$ and then~$g$. This is only possible if
the range of~$f$ is a subset of the domain of~$g$.

\begin{defn}[Composition]
Let $f\colon A \to B$ and $g\colon B \to C$ be functions. The
\emph{composition} of $f$ with~$g$ is $\comp{f}{g} \colon A \to C$,
where $(\comp{f}{g})(x) = g(f(x))$.
\end{defn}

\begin{ex}
Consider the functions $f(x) = x + 1$, and $g(x) = 2x$. Since
$(\comp{f}{g})(x) = g(f(x))$, for each input~$x$ you must first take
its successor, then multiply the result by two. So their composition
is given by $(\comp{f}{g})(x) = 2(x+1)$.
\end{ex}

Composition preserves injectivity and surjectivity: if $f \colon A \to
B$ and $g \colon B \to C$ are both injective, then $\comp{f}{g}$ is
injective; likewise for surjectivity.


%%% Functions as Relations (from sfr/fun/rel — CONDENSE)

A function $f \colon A \to B$ naturally defines a relation between $A$
and~$B$. In fact, we can \emph{identify} a function with this
relation, i.e., with a set of pairs.

\begin{defn}[Graph of a function]
Let $f\colon A \to B$ be a function.
The \emph{graph} of~$f$ is the relation $R_f \subseteq A \times B$
defined by
\[
R_f = \Setabs{\tuple{x,y}}{f(x) = y}.
\]
\end{defn}

Conversely, any relation $R \subseteq A \times B$ that is functional
(for each $x \in A$ there is exactly one $y \in B$ with $Rxy$) is the
graph of a function $f \colon A \to B$ defined by $f(x) = y$ iff
$Rxy$. Functions can thus be identified with certain relations, i.e.,
with certain sets of tuples.

%%% ===================================================================
%%% BST.4  Sequences and Numbers
%%% ===================================================================

\section{Sequences and Numbers}
\label{BST.4}

The ordered pair $\tuple{a,b}$ (PRIM-BST006, \S\ref{BST.1}) and
Cartesian product $A \times B$ (PRIM-BST007, \S\ref{BST.1}) allow us to form pairs
and grids of elements.  We now extend these constructions to
\emph{finite sequences} of arbitrary length and to \emph{infinite
sequences}, both of which are pervasive in logic and computability
theory.

\subsection{Tuples and Finite Sequences}

Ordered pairs generalise to longer sequences by iteration.  A
\emph{triple} $\tuple{x,y,z}$ is the pair $\tuple{\tuple{x,y},z}$; a
\emph{quadruple} $\tuple{x,y,z,u}$ is
$\tuple{\tuple{\tuple{x,y},z},u}$; and in general an \emph{ordered
$n$-tuple} $\tuple{x_1, \dots, x_n}$ is defined recursively by
\begin{align*}
  \tuple{x_1} &= x_1, \\
  \tuple{x_1, \dots, x_{k+1}} &= \tuple{\tuple{x_1, \dots, x_k},\, x_{k+1}}.
\end{align*}
The Cartesian product generalises correspondingly:
\begin{align*}
  A^1 &= A, \\
  A^{k+1} &= A^k \times A.
\end{align*}

\begin{defn}[Finite sequences / words] % PRIM-BST010
  \label{PRIM-BST010}
  Let $A$ be a set.  A \emph{word} (or \emph{finite sequence}) over~$A$ is
  any $n$-tuple of elements of~$A$, for some $n \geq 0$.  By convention
  elements of~$A$ are sequences of length~$1$, and $\emptyset$ is the
  unique sequence of length~$0$ (the \emph{empty word}).  The set of
  \emph{all} finite sequences over~$A$ is
  \[
    A^* = \{\emptyset\} \cup A \cup A^2 \cup A^3 \cup \dots
  \]
\end{defn}

\begin{ex}
If $A = \{a,b,c\}$, then the sequence ``$bac$'' is the triple
$\tuple{b,a,c} \in A^3$.  The set $A^*$ contains the empty word, all
single letters, all pairs, all triples, and so on.
\end{ex}

\subsection{Infinite Sequences}

\begin{defn}[Infinite sequences] % PRIM-BST011
  \label{PRIM-BST011}
  For any set $A$, the set $A^\omega$ consists of all infinite
  (one-way) sequences of elements of~$A$.  An element of $A^\omega$ is
  a sequence $a_1 a_2 a_3 \dots$ where each $a_i \in A$.
  Equivalently, $A^\omega$ is the set of all functions
  $f \colon \Nat \to A$ (or $f \colon \PosInt \to A$, depending on
  indexing convention).
\end{defn}

Finite sequences ($A^*$, PRIM-BST010) and infinite sequences
($A^\omega$, PRIM-BST011) will recur throughout the text: $A^*$
underlies the syntax of formal languages
(see PRIM-SYN001, CH-SYN), while $A^\omega$ appears in
diagonalisation arguments (\S\ref{BST.6}).

With the machinery of sequences and numbers in place, we now develop
the proof methods and recursive definitions that operate on these structures.

%%% ===================================================================
%%% BST.5  Induction and Recursion
%%% ===================================================================

\section{Induction and Recursion}
\label{BST.5}

Induction is a proof technique for establishing that \emph{every}
element of a suitably constructed set has a given property.  It applies
whenever the set is built up from basic elements by repeatedly applying
certain operations---that is, whenever the set is the \emph{closure}
of some base elements under those operations.

A set $S$ is \emph{closed} under a function $f$ iff $x \in S$ implies
$f(x) \in S$.  A property $P$ is \emph{preserved} under $f$ iff
$P(a)$ implies $P(f(a))$ for every $a$ in the domain of $f$.

\subsection{Mathematical Induction on $\Nat$}

The natural numbers $\Nat$ (PRIM-BST012, \S\ref{BST.1}) are closed
under the successor function $s(n) = n+1$, and every natural number is
obtained from $0$ by finitely many applications of $s$.  This yields
the principle of mathematical induction.

\begin{defn}[Mathematical induction on $\Nat$] % PRIM-BST013
  \label{PRIM-BST013}
  If $0$ has property $P$, and $P$ is preserved under the successor
  function, then every natural number has property~$P$.
\end{defn}

More formally: if $P(0)$ holds and $P(k) \Rightarrow P(k+1)$ for all
$k \in \Nat$, then $P(n)$ holds for all $n \in \Nat$.

\begin{rem}[Set-theoretic justification]
  Induction on $\Nat$ can be grounded in the set-theoretic notion of
  closure.  If $N, s, o$ form a Dedekind algebra (see below), then for
  any set $X$: if $o \in X$ and $s(n) \in X$ whenever $n \in N \cap X$,
  then $N \subseteq X$.  Applying this with
  $X = \{n \in N : P(n)\}$ recovers the familiar induction schema.
\end{rem}

\subsection{Structural Induction on Formulas}

The set of formulas of a formal language (see \S\ref{SYN.2} for the
formal definition) is built from atomic formulas
by the formula-building functions: negation ($\lnot$), conjunction
($\land$), disjunction ($\lor$), conditional ($\lif$), biconditional
($\liff$), and quantification ($\lforall[]$, $\lexists[]$).  The set
of formulas is closed under each of these operations, and every
formula is obtained from atomic formulas by finitely many applications
of them.

\begin{defn}[Structural induction on formulas] % PRIM-BST014
  \label{PRIM-BST014}
  If every atomic formula has property $P$, and $P$ is preserved under
  each formula-building function, then every formula has property~$P$.
\end{defn}

This suggests the following recipe for an inductive proof that every
formula has property~$P$:

\medskip\noindent
\textbf{Base case.}  Let $\varphi$ be an atomic formula.  [\ldots]
Therefore $\varphi$ has property~$P$.

\medskip\noindent
\textbf{Inductive step.}  Let $\varphi$ and $\psi$ be formulas, both
having property~$P$.

\begin{itemize}
  \item Case $\lnot$: [\ldots] Therefore $\lnot \varphi$ has property~$P$.
  \item Case $\land$: [\ldots] Therefore $\varphi \land \psi$ has property~$P$.
  \item Case $\lor$: [\ldots] Therefore $\varphi \lor \psi$ has property~$P$.
  \item Case $\lif$: [\ldots] Therefore $\varphi \lif \psi$ has property~$P$.
  \item Case $\liff$: [\ldots] Therefore $\varphi \liff \psi$ has property~$P$.
  \item Case $\lforall[]$: [\ldots] Therefore $\lforall[x]\varphi$ has property~$P$.
  \item Case $\lexists[]$: [\ldots] Therefore $\lexists[x]\varphi$ has property~$P$.
\end{itemize}

\medskip\noindent
Therefore every formula has property~$P$.

\subsection{Closure and Dedekind Algebras}

The closure machinery underlying induction can be made precise in
purely set-theoretic terms.

\begin{defn}[Closure] % DEF-BST006
  \label{DEF-BST006}
  For any function $f$ and any element $o$, a set $X$ is
  \emph{$f$-closed} iff $f(x) \in X$ for every $x \in X$.  The
  \emph{closure} of $o$ under $f$ is
  \[
    \closureofunder{f}{o}
    \;=\; \bigcap\Setabs{X}{o \in X \text{ and } X \text{ is $f$-closed}}.
  \]
\end{defn}

Intuitively, $\closureofunder{f}{o}$ is the \emph{smallest} $f$-closed
set containing~$o$.  One can verify that $o \in \closureofunder{f}{o}$,
that $\closureofunder{f}{o}$ is $f$-closed, and that it is contained in
every $f$-closed set that has $o$ as an element.

\begin{defn}[Dedekind algebra] % DEF-BST007
  \label{DEF-BST007}
  A \emph{Dedekind algebra} is a triple $(A, f, o)$ where $A$ is a set,
  $f \colon A \to A$ is a function, and $o \in A$, satisfying:
  \begin{enumerate}
    \item $o \notin \ran{f}$ \quad (zero is not a successor),
    \item $f$ is an injection \quad (distinct elements have distinct
      successors),
    \item $A = \closureofunder{f}{o}$ \quad ($A$ is the smallest
      $f$-closed set containing~$o$).
  \end{enumerate}
\end{defn}

Any Dedekind algebra can serve as a surrogate for the natural numbers:
conditions (1)--(3) are precisely the Dedekind--Peano characterisation
of $\Nat$, and arithmetic (addition, multiplication, exponentiation) can
be defined by recursion on a Dedekind algebra.  Moreover, any Dedekind
infinite set (see DEF-BST008, \S\ref{BST.6}) gives rise to a Dedekind
algebra.


%%% ===================================================================
%%% BST.6  Cardinality
%%% ===================================================================

\section{Cardinality}
\label{BST.6}

With functions (PRIM-BST009), injections (DEF-BST001), surjections
(DEF-BST002), and bijections (DEF-BST003) in hand (\S\ref{BST.3}),
we can make precise the idea of ``how many elements'' a set has, and
compare the sizes of infinite sets.

\subsection{Enumerations and Enumerable Sets}

Informally, an \emph{enumeration} of a set~$A$ is a (possibly
infinite) list of elements of~$A$ such that every element appears at
some finite position.  We make this precise as follows.

\begin{defn}[Enumeration] % PRIM-BST016 (part 1)
  An \emph{enumeration} of a non-empty set~$A$ is a surjective
  function $f \colon \Nat \to A$ (equivalently, $f \colon \PosInt \to A$).
\end{defn}

A surjection $f \colon \PosInt \to A$ gives the list $f(1), f(2), f(3), \dots$
in which every element of~$A$ appears.  The list may contain repetitions;
these can always be removed (by skipping any $f(k)$ already
listed), yielding a bijection between $A$ and either
$\Nat$ or an initial segment $\{0, \dots, n\}$
(equivalently, $\PosInt$ or $\{1, \dots, n\}$),
depending on whether $A$ is infinite or finite.

\begin{defn}[Enumerable set] % PRIM-BST016 (part 2)
  \label{PRIM-BST016}
  A set~$A$ is \emph{enumerable} (also called \emph{countable}) iff
  $A = \emptyset$ or there is an enumeration of~$A$.
  A set is \emph{non-enumerable} iff it is not enumerable.
\end{defn}

\begin{cor}[$\Nat$ is enumerable] % THM-BST002
  \label{THM-BST002}
  $\Nat$ is enumerable.
\end{cor}

\begin{proof}
  The identity function $\Id{\Nat}(n) = n$ is a bijection
  $\Nat \to \Nat$, hence an enumeration of~$\Nat$.
\end{proof}

\begin{ex}[Enumerating $\Int$ by hopping]
  The function $f \colon \Nat \to \Int$ defined by
  \[
    f(n) = (-1)^{n}\left\lceil \tfrac{n}{2} \right\rceil
  \]
  enumerates the integers by ``hopping'' between positive and negative
  values:
  \[
    \begin{array}{c c c c c c c c}
      f(0) & f(1) & f(2) & f(3) & f(4) & f(5) & f(6) & \dots \\[4pt]
      0 & -1 & 1 & -2 & 2 & -3 & 3 & \dots
    \end{array}
  \]
  Equivalently, $f$ can be written as:
  \[
    f(n) = \begin{cases}
      n/2 & \text{if $n$ is even,}\\
      -(n+1)/2 & \text{if $n$ is odd.}
    \end{cases}
  \]
\end{ex}

\subsection{Cantor's Zig-Zag Method}

We now show that the set of all pairs of natural numbers is enumerable.
Arrange the elements of $\Nat \times \Nat$ in an array:
\[
  \begin{array}{ c | c | c | c | c | c}
    & \mathbf 0 & \mathbf 1 & \mathbf 2 & \mathbf 3 & \dots \\
    \hline
    \mathbf 0 & \tuple{0,0} & \tuple{0,1} & \tuple{0,2} & \tuple{0,3} & \dots \\
    \hline
    \mathbf 1 & \tuple{1,0} & \tuple{1,1} & \tuple{1,2} & \tuple{1,3} & \dots \\
    \hline
    \mathbf 2 & \tuple{2,0} & \tuple{2,1} & \tuple{2,2} & \tuple{2,3} & \dots \\
    \hline
    \mathbf 3 & \tuple{3,0} & \tuple{3,1} & \tuple{3,2} & \tuple{3,3} & \dots \\
    \hline
    \vdots & \vdots & \vdots & \vdots & \vdots & \ddots
  \end{array}
\]
Every ordered pair appears exactly once.  To convert this
two-dimensional array into a one-dimensional list, traverse the
successive anti-diagonals (where $n+m$ is constant), reading each
anti-diagonal from bottom-left to top-right.  Numbering positions from
$0$, we obtain:
\[
  \begin{array}{ c | c | c | c | c | c | c}
    & \mathbf 0 & \mathbf 1 & \mathbf 2 & \mathbf 3 & \mathbf 4 &\dots \\
    \hline
    \mathbf 0 & 0  & 1& 3 & 6& 10 &\ldots \\
    \hline
    \mathbf 1 &2 & 4& 7 & 11 & \dots &\ldots \\
    \hline
    \mathbf 2 & 5 & 8 & 12 & \ldots & \dots&\ldots \\
    \hline
    \mathbf 3 & 9 & 13 & \ldots & \ldots & \dots & \ldots \\
    \hline
    \mathbf 4 & 14 & \ldots & \ldots & \ldots & \dots & \ldots \\
    \hline
    \vdots & \vdots & \vdots & \vdots & \vdots&\ldots & \ddots
  \end{array}
\]
This pattern is called \emph{Cantor's zig-zag method}.  It enumerates
$\Nat \times \Nat$ as:
\[
  \tuple{0,0},\; \tuple{0,1},\; \tuple{1,0},\; \tuple{0,2},\;
  \tuple{1,1},\; \tuple{2,0},\; \tuple{0,3},\; \tuple{1,2},\;
  \tuple{2,1},\; \tuple{3,0},\; \dots
\]

\begin{prop}[$\Nat \times \Nat$ is enumerable] % THM-BST003
  \label{THM-BST003}
  $\Nat \times \Nat$ is enumerable.
\end{prop}

\begin{proof}
Let $f \colon \Nat \to \Nat \times \Nat$ map each $k \in \Nat$ to the
pair $\tuple{n,m}$ occupying position $k$ in the zig-zag array.  Every
pair $\tuple{n,m}$ lies on the anti-diagonal where $n + m$ is constant,
and so appears at a definite finite position; hence $f$ is a surjection
(indeed a bijection).
\end{proof}

The same technique generalises by induction.  We can view
$\Nat^3 = (\Nat \times \Nat) \times \Nat$ and enumerate it by
zig-zagging the enumeration of $\Nat^2$ against $\Nat$.  Repeating:

\begin{prop}
  $\Nat^n$ is enumerable, for every $n \in \Nat$.
\end{prop}

\subsection{Pairing Functions}

It is useful to have an explicit formula for the zig-zag enumeration.
The \emph{inverse} of the enumeration is a bijection
$g \colon \Nat \times \Nat \to \Nat$ that assigns to each pair its
position in the zig-zag array.

\begin{defn}[Pairing function]
  An arithmetical \emph{pairing function} is an injective function
  $f \colon A \times B \to \Nat$.  The value $f(x,y)$ is called the
  \emph{code} for $\tuple{x,y}$.
\end{defn}

The zig-zag order yields the explicit pairing function
\[
  g(n,m) = \frac{(n+m+1)(n+m)}{2} + n.
\]
Indeed, the pair $\tuple{n,m}$ lies on the anti-diagonal where the sum
$n+m$ equals some value~$k$; the $k$th triangular number $k(k+1)/2$
gives the position of the first entry on that anti-diagonal, and $n$
counts the offset within it.  For instance,
$g(1,2) = \tfrac{4 \cdot 3}{2} + 1 = 7$, matching the array above.

\subsection{Non-Enumerable Sets}

Not every infinite set is enumerable.  The \emph{diagonal method},
introduced by Cantor, shows that certain natural sets have ``more''
elements than can be listed.

\begin{thm}[$\Bin^\omega$ is non-enumerable]
  \label{thm:nonenum-bin-omega}
  $\Bin^\omega$, the set of all infinite sequences of $0$'s and $1$'s,
  is non-enumerable.
\end{thm}

\begin{proof}
Suppose for contradiction that $\Bin^\omega$ is enumerable.  Then there
is a list $s_0, s_1, s_2, \dots$ of all elements of $\Bin^\omega$.
Write $s_i(j)$ for the $j$th entry of the $i$th sequence.  Arrange the
sequences into an array:
\[
  \begin{array}{c|c|c|c|c|c}
    & 0 & 1 & 2 & 3 & \dots \\\hline
    0 & \mathbf{s_{0}(0)} & s_{0}(1) & s_{0}(2) & s_0(3) & \dots \\\hline
    1 & s_{1}(0)& \mathbf{s_{1}(1)} & s_1(2) & s_1(3) & \dots \\\hline
    2 & s_{2}(0)& s_{2}(1) & \mathbf{s_2(2)} & s_2(3) & \dots \\\hline
    3 & s_{3}(0)& s_{3}(1) & s_3(2) & \mathbf{s_3(3)} & \dots \\\hline
    \vdots & \vdots & \vdots & \vdots & \vdots & \mathbf{\ddots}
  \end{array}
\]
Define a new sequence $d \in \Bin^\omega$ by flipping each diagonal
entry:
\[
  d(n) =
  \begin{cases}
    1 & \text{if } s_n(n) = 0, \\
    0 & \text{if } s_n(n) = 1.
  \end{cases}
\]
Then $d \in \Bin^\omega$, since it is an infinite sequence of $0$'s and
$1$'s.  But for every $n \in \Nat$, $d(n) \neq s_n(n)$, so $d \neq s_n$.
Hence $d$ does not appear on the list, contradicting the assumption that
the list enumerates all of $\Bin^\omega$.
\end{proof}

\begin{thm}[$\Pow{\Nat}$ is non-enumerable]
  \label{thm:nonenum-pownat}
  $\Pow{\Nat}$ is not enumerable.
\end{thm}

\begin{proof}[Proof sketch]
The argument follows the same diagonal pattern.  Given any list
$N_0, N_1, N_2, \dots$ of subsets of $\Nat$, define
$D = \Setabs{n \in \Nat}{n \notin N_n}$.  Then $D \subseteq \Nat$ but
$D \neq N_n$ for every~$n$ (since $n \in D$ iff $n \notin N_n$), so the
list omits~$D$.
\end{proof}

\subsection{Reduction}

Diagonalisation is not the only way to establish non-enumerability.
A general strategy is \emph{reduction}: to show that a set $B$ is
non-enumerable, exhibit a surjection $f \colon B \to A$ from $B$ onto a
known non-enumerable set~$A$.  Any enumeration of $B$ would then yield
an enumeration of $A$, a contradiction.

Equivalently, if $A$ is non-enumerable and there is an injection
$g \colon A \to B$, then $B$ is non-enumerable.

\subsection{Equinumerosity}

To compare sizes of arbitrary sets---not just to classify them as
``enumerable'' or ``non-enumerable''---we introduce a general notion
of size equivalence.

\begin{defn}[Equinumerosity] % DEF-BST009
  \label{DEF-BST009}
  $A$ is \emph{equinumerous} with $B$, written $\cardeq{A}{B}$, iff
  there is a bijection $f \colon A \to B$.
\end{defn}

Equinumerosity is an equivalence relation: reflexivity follows from the
identity function (DEF-BST003), symmetry from inverses of bijections,
and transitivity from composition of bijections.  Moreover, if
$\cardeq{A}{B}$, then $A$ is enumerable if and only if $B$ is.

\subsection{Dedekind Infinite Sets}

\begin{defn}[Dedekind infinite] % DEF-BST008
  \label{DEF-BST008}
  A set $A$ is \emph{Dedekind infinite} iff there is an injection from
  $A$ to a proper subset of $A$.  Equivalently, there exist some
  $o \in A$ and an injection $f \colon A \to A$ such that
  $o \notin \ran{f}$.
\end{defn}

Intuitively, a Dedekind infinite set can be put in bijection with a
proper subset of itself---the hallmark of infinity.  As noted in
\S\ref{BST.5}, any Dedekind infinite set gives rise to a Dedekind
algebra and hence supports induction.

\subsection{Sets of Different Sizes and Cantor's Theorem}

We now define a strict size ordering on sets and state the central
theorem of cardinality theory.

\begin{defn}[Size comparison by injection]
  $A$ is \emph{no larger than}~$B$, written $\cardle{A}{B}$, iff there
  is an injection $f \colon A \to B$.
\end{defn}

\begin{defn}[Strict size comparison]
  $A$ is \emph{smaller than}~$B$, written $\cardless{A}{B}$, iff
  $\cardle{A}{B}$ and $\cardneq{A}{B}$---i.e., there is an injection
  $f \colon A \to B$ but no bijection $g \colon A \to B$.
\end{defn}

The relation $\cardle{\cdot}{\cdot}$ is reflexive and transitive; the
relation $\cardless{\cdot}{\cdot}$ is irreflexive and transitive.

Using these definitions, a set $A$ is enumerable iff
$\cardle{A}{\Nat}$, and non-enumerable iff $\cardless{\Nat}{A}$.
The non-enumerability of $\Pow{\Nat}$
(Theorem~\ref{thm:nonenum-pownat}) can be restated as
$\cardless{\Nat}{\Pow{\Nat}}$.  Cantor proved that this
phenomenon is perfectly general:

\begin{thm}[Cantor's Theorem] % THM-BST001
  \label{THM-BST001}
  $\cardless{A}{\Pow{A}}$, for any set $A$.
\end{thm}

\begin{proof}
\emph{$\cardle{A}{\Pow{A}}$:} The function $f \colon A \to \Pow{A}$
defined by $f(x) = \{x\}$ is an injection, since $x \neq y$ implies
$\{x\} \neq \{y\}$ by extensionality (PRIM-BST001).

\emph{$\cardneq{A}{\Pow{A}}$:} Suppose for contradiction that there
is a bijection $g \colon A \to \Pow{A}$.  Define
\[
  D = \Setabs{x \in A}{x \notin g(x)}.
\]
Since $g(x) \subseteq A$ for every $x$, $D$ is a well-defined subset
of~$A$, so $D \in \Pow{A}$.  Because $g$ is a bijection, there exists
$y \in A$ with $g(y) = D$.  But then
\[
  y \in g(y) \;\text{ iff }\; y \in D \;\text{ iff }\; y \notin g(y),
\]
a contradiction.  Hence no bijection $A \to \Pow{A}$ exists, and
$\cardneq{A}{\Pow{A}}$.

Combining both parts: $\cardless{A}{\Pow{A}}$.
\end{proof}

The construction of the ``anti-diagonal'' set $D$ mirrors the diagonal
argument in Theorem~\ref{thm:nonenum-pownat}, and was the inspiration for
Russell's Paradox.

\subsection{Schr\"oder-Bernstein}

Cantor's Theorem shows that strict size differences exist among infinite
sets.  For the converse direction---showing two sets have the
\emph{same} size---the following deep result is essential.

\begin{thm}[Schr\"oder-Bernstein]
  \label{thm:schroder-bernstein}
  If $\cardle{A}{B}$ and $\cardle{B}{A}$, then $\cardeq{A}{B}$.
\end{thm}

In other words, if there is an injection $f \colon A \to B$ and an
injection $g \colon B \to A$, then there is a bijection
$h \colon A \to B$.  This justifies treating $\cardle{\cdot}{\cdot}$ as
a genuine size comparison: mutual ``no larger than'' implies
equinumerosity.

\begin{proof}[Proof sketch]
Given injections $f \colon A \to B$ and $g \colon B \to A$, the
composition $g \circ f$ is an injection from $A$ into $A$ whose range
$\funimage{g}{\funimage{f}{A}}$ satisfies
$\funimage{g}{\funimage{f}{A}} \subseteq \funimage{g}{B} \subseteq A$.
Thus $g \circ f$ is a bijection from $A$ onto
$\funimage{g}{\funimage{f}{A}}$, and we have the ``subset sandwich''
$\funimage{g}{\funimage{f}{A}} \subseteq \funimage{g}{B} \subseteq A$
with $\cardeq{\funimage{g}{\funimage{f}{A}}}{A}$.

The key step uses the generalised closure construction
(cf.\ DEF-BST006, \S\ref{BST.5}).  Define
$\Closureofunder{(g \circ f)}{A \setminus \funimage{g}{B}}$: the
smallest $(g \circ f)$-closed set containing
$A \setminus \funimage{g}{B}$.  One shows that the function
\[
  h(x) = \begin{cases}
    (g \circ f)(x) & \text{if $x$ belongs to the closure,} \\
    x & \text{otherwise}
  \end{cases}
\]
is a bijection from $A$ onto $\funimage{g}{B}$.  Composing $h$ with
$g^{-1} \colon \funimage{g}{B} \to B$ (which exists since $g$ is
injective) yields a bijection $A \to B$.
\end{proof}
   % CH-BST: Set-Theoretic Foundations (Level-0)
\chapter{Syntax} \label{ch:syn}

%% ===================================================================
%% SYN.1: Languages and Symbols
%% ===================================================================

\section{Languages and Symbols} \label{SYN.1}

Expressions of first-order logic are built up from a basic vocabulary
containing \emph{variables}, \emph{constant symbols},
\emph{predicate symbols}, and sometimes \emph{function symbols}.  From
them, together with logical connectives, quantifiers, and punctuation
symbols such as parentheses and commas, \emph{terms} and
\emph{formulas} are formed.

Predicate symbols are names for properties and relations, constant
symbols are names for individual objects, and function symbols are
names for mappings.  These, except for the identity
predicate~$\eq$, are the \emph{non-logical symbols} and together make
up a language.  Any first-order language~$\Lang L$ is determined by
its non-logical symbols.  In the most general case, $\Lang L$ contains
infinitely many symbols of each kind.

In the general case, we make use of the following symbols in
first-order logic: % PRIM-SYN001

\begin{enumerate}
\item \textbf{Logical symbols} % PRIM-SYN003, PRIM-SYN004
\begin{enumerate}
\item Logical connectives: $\lnot$ (negation), $\land$ (conjunction),
  $\lor$ (disjunction), $\lif$ (conditional), $\liff$
  (biconditional), $\lforall$ (universal quantifier), $\lexists$
  (existential quantifier).
\item The propositional constant for falsity~$\lfalse$.
\item The propositional constant for truth~$\ltrue$.
\item The two-place identity predicate~$\eq$. % PRIM-SYN018
\item A denumerable set of variables: $\Obj v_0$, $\Obj v_1$, $\Obj
  v_2$, \dots % PRIM-SYN002
\end{enumerate}
\item \textbf{Non-logical symbols}, making up the \emph{standard
  language} of first-order logic
\begin{enumerate}
\item A denumerable set of $n$-place predicate symbols for each $n>0$: $\Obj
  A^n_0$, $\Obj A^n_1$, $\Obj A^n_2$, \dots % PRIM-SYN007, PRIM-SYN008
\item A denumerable set of constant symbols: $\Obj c_0$, $\Obj c_1$, $\Obj
  c_2$, \dots % PRIM-SYN005
\item A denumerable set of $n$-place function symbols for each $n>0$:
  $\Obj f^n_0$, $\Obj f^n_1$, $\Obj f^n_2$, \dots % PRIM-SYN006
\end{enumerate}
\item Punctuation marks: (, ), and the comma.
\end{enumerate}

Most of our definitions and results will be formulated for the full
standard language of first-order logic.  However, depending on the
application, we may also restrict the language to only a few
predicate symbols, constant symbols, and function symbols.

\begin{defn}[Language] % PRIM-SYN009
\label{PRIM-SYN009}
A first-order \emph{language}~$\Lang L$ is determined by its
non-logical symbols: a set of predicate symbols, a set of constant
symbols, and a set of function symbols, each function and predicate
symbol having a fixed \emph{arity} $n \geq 1$ (the number of
arguments it takes).  Formally:
\[
  \Lang L = \langle \mathcal{C}, \mathcal{F}, \mathcal{R},
  \mathrm{ar} \rangle
\]
where $\mathcal{C}$ is a set of constant symbols, $\mathcal{F}$ is a
set of function symbols, $\mathcal{R}$ is a set of predicate symbols,
and $\mathrm{ar}$ assigns arities to the function and predicate
symbols.  The logical symbols (connectives, quantifiers, variables,
identity, and punctuation) are shared across all languages.
\end{defn}

\begin{ex}
The language~$\Lang L_A$ of arithmetic contains a single two-place
predicate symbol~$<$, a single constant symbol~$\Obj 0$, one one-place
function symbol~$\prime$, and two two-place function symbols~$+$
and~$\times$.  Officially: $\Lang L_A = \langle \{\Obj 0\}, \{
\prime, +, \times\}, \{<\}, \mathrm{ar} \rangle$ with
$\mathrm{ar}(\prime) = 1$ and $\mathrm{ar}(+) = \mathrm{ar}(\times) =
\mathrm{ar}(<) = 2$.
\end{ex}

\begin{rem}
We treat all the propositional operators and both quantifiers as
primitive symbols of the language.  One could instead choose a smaller
stock of primitive symbols and treat the other logical operators as
defined.  Truth-functionally complete sets of Boolean operators
include $\{ \lnot, \lor \}$, $\{ \lnot, \land \}$, and $\{ \lnot,
\lif\}$; any one of these, combined with either quantifier, yields an
expressively complete first-order language.  In this text, we keep all
connectives and both quantifiers as primitive for convenience.
\end{rem}


%% ===================================================================
%% SYN.2: Terms and Formulas
%% ===================================================================

\section{Terms and Formulas} \label{SYN.2}

Once a first-order language~$\Lang L$ is given, we can define
expressions built up from the basic vocabulary of~$\Lang L$.  These
include in particular \emph{terms} and \emph{formulas}.

\begin{defn}[Terms] % PRIM-SYN010, AX-SYN001
\label{PRIM-SYN010}
The set of \emph{terms}~$\Trm[L]$ of~$\Lang L$ is
defined inductively by:
\begin{enumerate}
\item Every variable is a term.
\item Every constant symbol of~$\Lang L$ is a term.
\item If $f$ is an $n$-place function symbol and $t_1$, \dots, $t_n$
  are terms, then $\Atom{f}{t_1, \ldots, t_n}$ is a term.
\item Nothing else is a term.
\end{enumerate}
A term containing no variables is a \emph{closed term}.
\end{defn}

\begin{defn}[Formulas] % PRIM-SYN011, PRIM-SYN012, AX-SYN002
\label{PRIM-SYN012}
The set of \emph{formulas}~$\Frm[L]$ of the language~$\Lang L$
is defined inductively as follows:
\begin{enumerate}
\item $\lfalse$ is an atomic formula.

\item $\ltrue$ is an atomic formula.

\item If $R$ is an $n$-place predicate symbol of~$\Lang L$ and $t_1$, \dots,
  $t_n$ are terms of~$\Lang L$, then $\Atom{R}{t_1,\ldots, t_n}$ is an
  atomic formula.

\item If $t_1$ and $t_2$ are terms of~$\Lang L$, then $\Atom{\eq}{t_1, t_2}$
  is an atomic formula.

\item If $!A$ is a formula, then $\lnot !A$ is
  a formula.

\item If $!A$ and $!B$ are formulas, then $(!A \land
  !B)$ is a formula.

\item If $!A$ and $!B$ are formulas, then $(!A \lor !B)$
  is a formula.

\item If $!A$ and $!B$ are formulas, then $(!A \lif !B)$
  is a formula.

\item If $!A$ and $!B$ are formulas, then $(!A \liff !B)$
  is a formula.

\item If $!A$ is a formula and $x$ is a variable,
  then $\lforall[x][!A]$ is a formula.

\item If $!A$ is a formula and $x$ is a variable,
  then $\lexists[x][!A]$ is a formula.

\item Nothing else is a formula.
\end{enumerate}
\end{defn}

By convention, we write $\eq$ between its arguments and leave out the
parentheses: $\eq[t_1][t_2]$ is an abbreviation for
$\Atom{\eq}{t_1,t_2}$.  Moreover, $\lnot \Atom{\eq}{t_1,t_2}$ is
abbreviated as $\eq/[t_1][t_2]$. When writing a formula $(!B \ast !C)$
constructed from $!B$, $!C$ using a two-place connective~$\ast$, we
will often leave out the outermost pair of parentheses and write
simply~$!B \ast !C$.

We write $!A \ident !B$ to express syntactic identity between strings of
symbols, i.e., $!A \ident !B$ iff $!A$ and $!B$ are strings of symbols
of the same length and which contain the same symbol in each place.

\begin{rem}[Propositional fragment] % AX-SYN003
\label{AX-SYN003}
Propositional (PL) formulas are the fragment of FOL formulas built
from propositional variables $\Obj p_0, \Obj p_1, \ldots$ using only
connectives---no quantifiers, function symbols, or predicate symbols
appear.  Formally, the set $\Frm[L_0]$ of PL formulas is defined by:
(i)~every propositional variable $\Obj p_i$ is an atomic formula;
(ii)~the formation rules for connectives apply as above (clauses 5--9);
(iii)~the quantifier clauses are omitted.
\end{rem}

As terms and formulas are built up from basic elements via inductive
definitions, we can use the following induction principles to prove
things about them.

\begin{lem}[Principle of induction on terms] % DEF-SYN005 (terms)
    \label{DEF-SYN005:terms}
    Let $\Lang L$ be a first-order language.
    If some property~$P$ is such that
    %
    \begin{enumerate}
        \item it holds for every variable~$v$,
        %
        \item it holds for every constant symbol~$a$ of~$\Lang L$, and
        %
        \item it holds for $f(t_1,\dotsc,t_n)$ whenever it holds for
        $t_1$,~\dots, $t_n$ and $f$~is an $n$-place
            function symbol of~$\Lang L$
    \end{enumerate}
    (assuming $t_1$,~\dots, $t_n$ are terms of~$\Lang{L}$),
    then $P$ holds for every term in~$\Trm[L]$.
\end{lem}

\begin{lem}[Principle of induction on formulas] % DEF-SYN005 (formulas)
    \label{DEF-SYN005}
    Let $\Lang L$ be a first-order language.
    If some property~$P$ holds for all the atomic formulas
    and is such that
    %
    \begin{enumerate}
        \item it holds for $\lnot !A$ whenever it
            holds for~$!A$;
        \item it holds for $(!A \land !B)$
            whenever it holds for $!A$ and~$!B$;
        \item it holds for $(!A \lor !B)$
            whenever it holds for $!A$ and~$!B$;
        \item it holds for $(!A \lif !B)$
            whenever it holds for $!A$ and~$!B$;
        \item it holds for $(!A \liff !B)$
            whenever it holds for $!A$ and~$!B$;
        \item it holds for $\lexists[x][!A]$
            whenever it holds for~$!A$;
        \item it holds for $\lforall[x][!A]$
            whenever it holds for~$!A$;
  \end{enumerate}
  (assuming $!A$ and $!B$ are formulas of~$\Lang{L}$),
  then $P$ holds for all formulas in~$\Frm[L]$.
\end{lem}

%%% Subformulas (from fol/syn/sbf --- KEEP)

It is often useful to talk about the formulas that ``make up'' a
given formula.  We call these its \emph{subformulas}.  Any
formula counts as a subformula of itself; a subformula of $!A$
other than $!A$ itself is a \emph{proper subformula}.

\begin{defn}[Immediate subformula] % PRIM-SYN017
\label{PRIM-SYN017}
If $!A$ is a formula, the \emph{immediate subformulas}
of $!A$ are defined as follows:
\begin{enumerate}
\item Atomic formulas have no immediate subformulas.

\item If $!A \ident \lnot !B$, the only immediate
    subformula of $!A$ is~$!B$.

\item If $!A \ident (!B \ast !C)$, the immediate subformulas of
  $!A$ are $!B$ and $!C$ ($\ast$ is any one of the two-place
  connectives).

\item If $!A \ident \lforall[x][!B]$, the only immediate
    subformula of $!A$ is~$!B$.

\item If $!A \ident \lexists[x][!B]$, the only immediate
    subformula of $!A$ is~$!B$.
\end{enumerate}
\end{defn}

\begin{defn}[Proper subformula]
\label{PRIM-SYN017:proper}
If $!A$ is a formula, the \emph{proper subformulas}
of $!A$ are defined recursively as follows:
\begin{enumerate}
\item Atomic formulas have no proper subformulas.

\item If $!A \ident \lnot !B$, the proper subformulas of
    $!A$ are~$!B$ together with all proper subformulas
    of~$!B$.

\item If $!A \ident (!B \ast !C)$, the proper subformulas of
  $!A$ are $!B$, $!C$, together with all proper subformulas
  of $!B$ and those of~$!C$.

\item If $!A \ident \lforall[x][!B]$, the proper
    subformulas of $!A$ are~$!B$ together with all proper
    subformulas of~$!B$.

\item If $!A \ident \lexists[x][!B]$, the proper
    subformulas of $!A$ are~$!B$ together with all proper
    subformulas of~$!B$.
\end{enumerate}
\end{defn}

\begin{defn}[Subformula]
\label{PRIM-SYN017:all}
The subformulas of $!A$ are $!A$ itself together with all its
proper subformulas.
\end{defn}

\begin{rem}
The subformula relation is transitive: if $!C$ is a subformula of
$!B$ and $!B$ is a subformula of $!A$, then $!C$ is a subformula
of~$!A$.
\end{rem}

%%% Subterms (from DOMAIN-SYN DEF-SYN008)

\begin{defn}[Subterm] % DEF-SYN008
\label{DEF-SYN008}
The \emph{subterms} of a term $t$ are defined recursively:
\begin{enumerate}
\item $t$ is a subterm of~$t$.
\item If $f(t_1, \ldots, t_n)$ is a subterm of $s$, then each $t_i$
  is a subterm of~$s$.
\end{enumerate}
\end{defn}

%%% Main operator (from fol/syn/mai --- CONDENSE)

\begin{defn}[Main operator] % DEF (OL-specific, useful vocabulary)
\label{def:main-op}
The \emph{main operator} of a formula~$!A$ is
defined as follows:
\begin{enumerate}
\item If $!A$ is atomic, $!A$ has no main operator.
\item If $!A \ident \lnot !B$, the main operator of $!A$
  is~$\lnot$.
\item If $!A \ident (!B \land !C)$, the main operator of
  $!A$ is~$\land$.
\item If $!A \ident (!B \lor !C)$, the main operator of
  $!A$ is~$\lor$.
\item If $!A \ident (!B \lif !C)$, the main operator of
  $!A$ is~$\lif$.
\item If $!A \ident (!B \liff !C)$, the main operator of
  $!A$ is~$\liff$.
\item If $!A \ident \lforall[x][!B]$, the main operator
  of $!A$ is~$\lforall$.
\item If $!A \ident \lexists[x][!B]$, the main operator of
  $!A$ is~$\lexists$.
\end{enumerate}
\end{defn}

%%% Formula complexity (FORMALIZE from DOMAIN-SYN DEF-SYN007)

\begin{defn}[Formula complexity] % DEF-SYN007
\label{DEF-SYN007}
The \emph{complexity} (or \emph{rank}) of a formula~$!A$, written
$\mathrm{cx}(!A)$, is a natural number defined by structural recursion:
\begin{enumerate}
\item If $!A$ is atomic, then $\mathrm{cx}(!A) = 0$.
\item $\mathrm{cx}(\lnot !B) = \mathrm{cx}(!B) + 1$.
\item $\mathrm{cx}((!B \ast !C)) = \max(\mathrm{cx}(!B),
  \mathrm{cx}(!C)) + 1$, where $\ast$ is any binary connective.
\item $\mathrm{cx}(\lforall[x][!B]) = \mathrm{cx}(!B) + 1$.
\item $\mathrm{cx}(\lexists[x][!B]) = \mathrm{cx}(!B) + 1$.
\end{enumerate}
For instance, $\mathrm{cx}(P(x)) = 0$, $\mathrm{cx}(\lnot P(x)) = 1$,
and $\mathrm{cx}(P(x) \lif \lforall[y][Q(y)]) = 2$.
\end{defn}

%%% Formation sequences (from fol/syn/fseq --- CONDENSE)

An alternative, bottom-up approach to the construction of formulas
uses \emph{formation sequences}, which make explicit the step-by-step
process by which a formula is built.

\begin{defn}[Formation sequences for terms] % DEF-SYN006 (terms)
\label{DEF-SYN006:terms}
A finite sequence of $\Lang L$-strings $\tuple{t_0,\dotsc,t_n}$ is a
\emph{formation sequence} for a term $t$ if $t \ident t_n$ and for all
$i \leq n$, either $t_i$ is a variable or a constant symbol, or $\Lang
L$ contains a $k$-ary function symbol~$f$ and there exist
$m_0,\dotsc,m_k < i$ such that $t_i \ident f(t_{m_0},\dotsc,t_{m_k})$.
\end{defn}

\begin{defn}[Formation sequences for formulas] % DEF-SYN006 (formulas)
\label{DEF-SYN006}
A finite sequence of $\Lang L$-strings $\tuple{!A_0,\dotsc,!A_n}$
is a \emph{formation sequence} for~$!A$ if $!A \ident !A_n$ and
for all $i \leq n$, either $!A_i$ is an atomic formula or there
exist $j,k < i$ and a variable~$x$ such that one of the following
holds:
\begin{enumerate}
    \item $!A_i \ident \lnot !A_j$.
    \item $!A_i \ident (!A_j \land !A_k)$.
    \item $!A_i \ident (!A_j \lor !A_k)$.
    \item $!A_i \ident (!A_j \lif !A_k)$.
    \item $!A_i \ident (!A_j \liff !A_k)$.
    \item $!A_i \ident \lforall[x][!A_j]$.
    \item $!A_i \ident \lexists[x][!A_j]$.
\end{enumerate}
\end{defn}

\begin{thm} % DEF-SYN006 equivalence
\label{DEF-SYN006:equiv}
$\Frm[L]$ is the set of all $\Lang L$-strings~$!A$ such that
there exists a formula formation sequence for~$!A$.
\end{thm}

\begin{proof}[Proof sketch]
One direction follows by induction on formulas: every formula
in~$\Frm[L]$ has a formation sequence (concatenate sequences for the
immediate subformulas, then append the formula itself).  The
converse is proved by strong induction on the length of the formation
sequence.
\end{proof}


%% ===================================================================
%% SYN.3: Variables and Scope
%% ===================================================================

\section{Variables and Scope} \label{SYN.3}

\begin{defn}[Free occurrences of a variable] % PRIM-SYN014
\label{PRIM-SYN014}
The \emph{free} occurrences of a variable in a formula are defined
inductively as follows:
\begin{enumerate}
\item If $!A$ is atomic, all variable occurrences in
  $!A$ are free.

\item If $!A \ident \lnot !B$, the free variable
  occurrences of $!A$ are exactly those of $!B$.

\item If $!A \ident (!B \ast !C)$, the free
  variable occurrences of $!A$ are those in $!B$
  together with those in~$!C$.

\item If $!A \ident \lforall[x][!B]$, the free variable
  occurrences in $!A$ are all of those in~$!B$ except for
  occurrences of~$x$.

\item If $!A \ident \lexists[x][!B]$, the free variable
  occurrences in $!A$ are all of those in~$!B$ except for
  occurrences of~$x$.
\end{enumerate}
\end{defn}

\begin{defn}[Free variables] % DEF-SYN003
\label{DEF-SYN003}
The set $\FV(!A)$ of \emph{free variables} of a formula~$!A$ is the
set of variables that have at least one free occurrence in~$!A$.
Explicitly:
\begin{enumerate}
\item If $!A \ident R(t_1, \ldots, t_n)$, then $\FV(!A) = \mathrm{Var}(t_1) \cup \cdots \cup \mathrm{Var}(t_n)$, where $\mathrm{Var}(t)$ is the set of variables occurring in term~$t$.
\item $\FV(\lnot !A) = \FV(!A)$.
\item $\FV((!A \ast !B)) = \FV(!A) \cup \FV(!B)$, where $\ast$ is any binary connective.
\item $\FV(\lforall[x][!A]) = \FV(!A) \setminus \{x\}$.
\item $\FV(\lexists[x][!A]) = \FV(!A) \setminus \{x\}$.
\end{enumerate}
\end{defn}

\begin{defn}[Bound variables] % PRIM-SYN015
\label{PRIM-SYN015}
An occurrence of a variable in a formula~$!A$ is \emph{bound} if
it is not free.
\end{defn}

\begin{defn}[Scope] % PRIM-SYN016
\label{PRIM-SYN016}
If $\lforall[x][!B]$ is an occurrence of a subformula
in a formula~$!A$, then the corresponding occurrence of~$!B$ in~$!A$
is called the \emph{scope} of the corresponding occurrence
of~$\lforall[x]$.  Similarly for $\lexists[x]$.

If $!B$ is the scope of a quantifier occurrence
$\lforall[x]$ or $\lexists[x]$ in~$!A$, then the free occurrences of
$x$ in~$!B$ are bound in $\lforall[x][!B]$ and
$\lexists[x][!B]$, respectively. We say that these
occurrences are \emph{bound by} the
mentioned quantifier occurrence.
\end{defn}

\begin{ex}
Consider the formula~$!A$:
\[
\lforall[\Obj v_0][\underbrace{(\Atom{\Obj A^1_0}{\Obj v_0} \lif
    \Atom{\Obj A^2_0}{\Obj v_0, \Obj v_1})}_{!B}] \lif \lexists[\Obj
  v_1][\underbrace{(\Atom{\Obj A^2_1}{\Obj v_0, \Obj v_1} \lor \lforall[\Obj v_0][\overbrace{\lnot \Atom{\Obj A^1_1}{\Obj v_0}}^{!D}])}_{!C}]
\]
$!B$ is the scope of the first $\lforall[\Obj v_0]$, $!C$ is the scope
of $\lexists[\Obj v_1]$, and $!D$ is the scope of the second
$\lforall[\Obj v_0]$.  The first $\lforall[\Obj v_0]$ binds the
occurrences of $\Obj v_0$ in~$!B$, $\lexists[\Obj v_1]$ binds the occurrence
of $\Obj v_1$ in $!C$, and the second $\lforall[\Obj v_0]$ binds the
occurrence of $\Obj v_0$ in~$!D$.  The first occurrence of $\Obj v_1$
and the third occurrence of $\Obj v_0$ are free in~$!A$. The last
occurrence of $\Obj v_0$ is free in $!D$, but bound in $!C$ and~$!A$.
\end{ex}

\begin{defn}[Sentence] % PRIM-SYN013
\label{PRIM-SYN013}
A formula~$!A$ is a \emph{sentence} iff it
contains no free occurrences of variables; equivalently, iff $\FV(!A)
= \emptyset$.
\end{defn}


%% ===================================================================
%% SYN.4: Substitution
%% ===================================================================

\section{Substitution} \label{SYN.4}

\begin{defn}[Substitution in a term] % DEF-SYN001 (terms)
\label{DEF-SYN001:terms}
We define $\Subst{s}{t}{x}$, the result of \emph{substituting} $t$
for every occurrence of~$x$ in $s$, recursively:
\begin{enumerate}
\item If $s \ident c$ for a constant symbol~$c$, then $\Subst{s}{t}{x}$ is just $s$.

\item If $s \ident y$ where $y$ is a variable and $y \not\ident x$, then $\Subst{s}{t}{x}$ is also just~$s$.

\item If $s \ident x$, then $\Subst{s}{t}{x}$ is~$t$.

\item If $s \ident \Atom{f}{t_1, \dots, t_n}$, then $\Subst{s}{t}{x}$ is
  $\Atom{f}{\Subst{t_1}{t}{x}, \dots, \Subst{t_n}{t}{x}}$.
\end{enumerate}
\end{defn}

\begin{defn}[Free for] % (capture-avoidance condition)
\label{def:free-for}
A term~$t$ is \emph{free for} $x$ in $!A$ if none of the free
occurrences of~$x$ in $!A$ occur in the scope of a quantifier that
binds a variable in~$t$.
\end{defn}

\begin{ex} ~
\begin{enumerate}
\item $\Obj v_8$ is free for $\Obj v_1$ in $\lexists[\Obj
  v_3]\Atom{\Obj A^2_4}{\Obj v_3,\Obj v_1}$.

\item $\Obj f^2_1(\Obj v_1, \Obj v_2)$ is \emph{not} free for $\Obj
  v_0$ in $\lforall[\Obj v_2]\Atom{\Obj A^2_4}{\Obj v_0,\Obj v_2}$.
\end{enumerate}
\end{ex}

\begin{defn}[Substitution in a formula] % DEF-SYN001
\label{DEF-SYN001}
If $!A$ is a formula, $x$~is a variable, and $t$~is a term
free for~$x$ in~$!A$, then $\Subst{!A}{t}{x}$ is the result of
substituting $t$ for all free occurrences of~$x$ in~$!A$.
\begin{enumerate}
\item If $!A \ident \lfalse$, then $\Subst{!A}{t}{x}$ is
    $\lfalse$.

\item If $!A \ident \ltrue$, then $\Subst{!A}{t}{x}$ is
    $\ltrue$.

\item If $!A \ident \Atom{P}{t_1,\dots,
    t_n}$, then $\Subst{!A}{t}{x}$ is $\Atom{P}{\Subst{t_1}{t}{x},
    \dots, \Subst{t_n}{t}{x}}$.

\item If $!A \ident \eq[t_1][t_2]$, then $\Subst{!A}{t}{x}$ is
  $\Subst{t_1}{t}{x} = \Subst{t_2}{t}{x}$.

\item If $!A \ident \lnot !B$, then $\Subst{!A}{t}{x}$ is
    $\lnot \Subst{!B}{t}{x}$.

\item If $!A \ident (!B \land
    !C)$, then $\Subst{!A}{t}{x}$ is $(\Subst{!B}{t}{x} \land
    \Subst{!C}{t}{x})$.

\item If $!A \ident (!B \lor
    !C)$, then $\Subst{!A}{t}{x}$ is $(\Subst{!B}{t}{x} \lor
    \Subst{!C}{t}{x})$.

\item If $!A \ident (!B \lif
    !C)$, then $\Subst{!A}{t}{x}$ is $(\Subst{!B}{t}{x} \lif
    \Subst{!C}{t}{x})$.

\item If $!A \ident (!B \liff
    !C)$, then $\Subst{!A}{t}{x}$ is $(\Subst{!B}{t}{x} \liff
    \Subst{!C}{t}{x})$.

\item If $!A \ident \lforall[y][!B]$, then $\Subst{!A}{t}{x}$
    is $\lforall[y][\Subst{!B}{t}{x}]$, provided $y$ is a variable
    other than $x$; otherwise $\Subst{!A}{t}{x}$
    is just $!A$.

\item If $!A \ident \lexists[y][!B]$, then $\Subst{!A}{t}{x}$
    is $\lexists[y][\Subst{!B}{t}{x}]$, provided $y$ is a variable
    other than $x$; otherwise $\Subst{!A}{t}{x}$
    is just $!A$.
\end{enumerate}
\end{defn}

Note that substitution may be vacuous: if $x$ does not occur in $!A$
at all, then $\Subst{!A}{t}{x}$ is just~$!A$.

The restriction that $t$ must be free for~$x$ in~$!A$ is necessary to
exclude cases where a free variable in~$t$ is ``captured'' by a
quantifier upon substitution.  For instance, if $!A \ident
\lexists[y][x < y]$ and $t \ident y$, then $\Subst{!A}{t}{x}$ would
be $\lexists[y][y < y]$---the free variable $y$ has been captured by
$\lexists[y]$, which changes the meaning.  We prevent this by
requiring that none of the free variables in~$t$ end up bound by a
quantifier in~$!A$.

We use the following convention: if $!A$ is a formula which may
contain the variable~$x$ free, we write~$!A(x)$; then $!A(t)$ is
short for $\Subst{!A}{t}{x}$.

%%% Simultaneous Substitution (NEW-CONTENT from DOMAIN-SYN DEF-SYN002)

\begin{defn}[Simultaneous substitution] % DEF-SYN002
\label{DEF-SYN002}
Let $!A$ be a formula, let $x_1, \ldots, x_n$ be distinct variables,
and let $t_1, \ldots, t_n$ be terms.  The \emph{simultaneous
substitution} $!A[t_1/x_1, \ldots, t_n/x_n]$ is the result of
replacing, in a single step, every free occurrence of each~$x_i$
in~$!A$ by the corresponding term~$t_i$.  The substitution is proper
provided each $t_i$ is free for $x_i$ in~$!A$.

Simultaneous substitution differs from iterated single substitution
when the substituted terms contain variables that are themselves being
replaced.  For example, $P(x,y)[y/x, x/y] = P(y, x)$ (the variables
are swapped), whereas the iterated substitution $P(x,y)[y/x][x/y] =
P(y,y)[x/y] = P(x,x)$ gives a different result.
\end{defn}

%%% Alphabetic Variant (NEW-CONTENT from DOMAIN-SYN DEF-SYN004)

\begin{defn}[Alphabetic variant] % DEF-SYN004
\label{DEF-SYN004}
Two formulas $!A$ and $!B$ are \emph{alphabetic variants} (written
$!A \equiv_\alpha !B$) if one can be obtained from the other by
consistently renaming bound variables, without introducing variable
capture.  That is, $!A$ and $!B$ differ only in the names of their
bound variables, with each renaming being a consistent bijection on
bound variable names that does not cause any free variable to become
bound.  For instance, $\lforall[x][P(x)]$ and $\lforall[y][P(y)]$
are alphabetic variants, since replacing the bound variable~$x$ by~$y$
throughout gives one formula from the other.
\end{defn}

\begin{rem}[Uniform substitution in PL]
In propositional logic, \emph{uniform substitution} replaces every
occurrence of a propositional variable~$\Obj p_i$ by a formula~$!B$
throughout~$!A$.  Formally, $\Subst{!A}{!B}{\Obj p_i}$ denotes the
result of replacing each occurrence of $\Obj p_i$ by~$!B$ in~$!A$.
This is a purely propositional operation: since propositional
variables have no internal structure, there is no notion of ``free
for'' and no risk of variable capture.  In the FOL setting, uniform
substitution is subsumed by the general substitution operation (see
DEF-SYN001, \S\ref{DEF-SYN001}).
\end{rem}

The syntactic operations defined so far---substitution, formation,
alphabetic variance---apply to arbitrary first-order languages.  We now
specialise to the \emph{language of arithmetic} and classify its
formulas by quantifier complexity.

%% ===================================================================
%% SYN.5: Arithmetic Hierarchy
%% ===================================================================

\section{Arithmetic Hierarchy} \label{SYN.5}

In the language of arithmetic~$\Lang L_A$, it is useful to single out
syntactic classes of formulas defined by the pattern of quantifiers
they contain.  These classes---the $\Delta_0$, $\Sigma_1$, and
$\Pi_1$ formulas---play a central role in the study of
incompleteness.

\begin{defn}[Bounded quantification] % DEF-SYN009
\label{DEF-SYN009}
A \emph{bounded existential formula} is one of the form
$\lexists[x][(x < t \land !A(x))]$ where $t$ is any term, which we
conventionally write as $\bexists{x < t}{!A(x)}$.
A \emph{bounded universal formula} is one of the form
$\lforall[x][(x < t \lif !A(x))]$ where $t$ is any term, which we
conventionally write as $\bforall{x < t}{!A(x)}$.
\end{defn}

\begin{defn}[$\Delta_0$, $\Sigma_1$, and $\Pi_1$ formulas] % DEF-SYN009, DEF-SYN010, DEF-SYN011
\label{DEF-SYN010}
\leavevmode
\begin{enumerate}
\item A formula~$!B$ is \emph{$\Delta_0$} if it is built up from atomic
formulas using only propositional connectives and bounded
quantification.

\item A formula~$!A$ is \emph{$\Sigma_1$} if $!A \ident \lexists[x][!B(x)]$
where $!B$ is $\Delta_0$.

\item A formula~$!A$ is \emph{$\Pi_1$} if $!A \ident \lforall[x][!B(x)]$
where $!B$ is $\Delta_0$.
\end{enumerate}
\end{defn}


%% ===================================================================
%% SYN.6: Theorems
%% ===================================================================

\section{Theorems} \label{SYN.6}

%%% Unique Readability (from fol/syn/unq --- KEEP)

The way we defined formulas guarantees that every formula has
a \emph{unique reading}, i.e., there is essentially only one way of
constructing it according to our formation rules for formulas and
only one way of ``interpreting'' it.  If this were not so, we would
have ambiguous formulas, i.e., formulas that have more than
one reading or interpretation---and that is clearly something we want
to avoid.  But more importantly, without this property, most of the
definitions and proofs we are going to give will not go through.

Perhaps the best way to see why unique readability matters is to
consider what would happen with bad formation rules.  For instance, if
we omitted parentheses and allowed: ``If $!A$ and $!B$ are formulas,
then so is $!A \lif !B$,'' then starting from an atomic formula $!D$
we could form $!D \lif !D \lif !D$ in two ways: taking $!A = !D$ and
$!B = !D \lif !D$, or taking $!A = !D \lif !D$ and $!B = !D$.  This
would make the main operator ambiguous (the first vs.\ the second
occurrence of~$\lif$), and recursive definitions on formulas would
be ill-defined.

\begin{lem}
The number of left and right parentheses in a formula~$!A$ are
equal.
\end{lem}

\begin{proof}
By induction on the construction of $!A$.  Let $l(!A)$ be the number
of left parentheses and $r(!A)$ the number of right parentheses
in~$!A$, and $l(t)$, $r(t)$ similarly for terms.

For atomic formulas: if $!A \ident \lfalse$ or $!A \ident \ltrue$,
then both counts are~$0$.  If $!A \ident \Atom{R}{t_1,\dots,t_n}$,
then $l(!A) = 1 + l(t_1) + \dots + l(t_n) = 1 + r(t_1) + \dots +
r(t_n) = r(!A)$, using the fact that $l(t) = r(t)$ for any term~$t$.
If $!A \ident \eq[t_1][t_2]$, then $l(!A) = l(t_1) + l(t_2) =
r(t_1) + r(t_2) = r(!A)$.

For the inductive cases: if $!A \ident \lnot !B$, then by hypothesis
$l(!B) = r(!B)$, so $l(!A) = l(!B) = r(!B) = r(!A)$.  If $!A \ident
(!B \ast !C)$, then $l(!A) = 1 + l(!B) + l(!C) = 1 + r(!B) + r(!C) =
r(!A)$.  If $!A \ident \lforall[x][!B]$ or $!A \ident
\lexists[x][!B]$, then $l(!A) = l(!B) = r(!B) = r(!A)$.
\end{proof}

\begin{prop}[Unique readability for atomic formulas] % THM-SYN002 (partial)
\label{THM-SYN002}
If $!A$ is an atomic formula, then it satisfies one, and only one
of the following conditions.
\begin{enumerate}
\item $!A \ident \lfalse$.
\item $!A \ident \ltrue$.
\item $!A \ident \Atom{R}{t_1,\dots,t_n}$ where $R$ is an $n$-place
  predicate symbol, $t_1$, \dots, $t_n$ are terms, and each of $R$,
  $t_1$, \dots, $t_n$ is uniquely determined.
\item $!A \ident \eq[t_1][t_2]$ where $t_1$ and $t_2$ are uniquely
  determined terms.
\end{enumerate}
\end{prop}

\begin{prop}[Unique Readability] % THM-SYN001
\label{THM-SYN001}
Every formula satisfies one, and only one of the following conditions.
\begin{enumerate}
\item $!A$ is atomic.

\item $!A$ is of the form $\lnot !B$.

\item $!A$ is of the form $(!B \land !C)$.

\item $!A$ is of the form $(!B \lor !C)$.

\item $!A$ is of the form $(!B \lif !C)$.

\item $!A$ is of the form $(!B \liff !C)$.

\item $!A$ is of the form $\lforall[x][!B]$.

\item $!A$ is of the form $\lexists[x][!B]$.
\end{enumerate}
Moreover, in each case $!B$, or $!B$ and $!C$, are uniquely
determined.  This means that, e.g., there are no different pairs $!B$,
$!C$ and $!B'$, $!C'$ so that $!A$ is both of the form
$(!B \lif !C)$ and $(!B' \lif !C')$.
\end{prop}

\begin{proof}[Proof sketch]
The proof proceeds in four steps:
\begin{enumerate}
\item \emph{Balanced parentheses}: Every formula has equal numbers of
  left and right parentheses (proved above).
\item \emph{Prefix-freeness}: No proper prefix of a formula is itself
  a formula.  (Proved by induction: every proper initial segment of a
  formula has strictly more left parentheses than right ones.)
\item \emph{Unique atomic parsing}: Atomic formulas are uniquely
  parsed (Proposition~\ref{THM-SYN002}).
\item \emph{Main result}: Suppose $!A \ident (!B \ast !C)$ and also
  $!A \ident (!B' \mathbin{\ast'} !C')$.  If $!B \ident !B'$, then
  clearly $\ast = {\ast'}$ and $!C \ident !C'$.  Otherwise, one of
  $!B$, $!B'$ is a proper prefix of the other, contradicting
  prefix-freeness.
\end{enumerate}
\end{proof}

\begin{rem}[PL unique readability]
Unique readability holds equally for propositional formulas
in~$\Frm[L_0]$, since PL formulas are a special case of FOL formulas
(with the quantifier clauses vacuously absent).
\end{rem}

%%% Structural Induction/Recursion Principles (FORMALIZE from THM-SYN004)

\begin{thm}[Structural induction and recursion principles] % THM-SYN004
\label{THM-SYN004}
\leavevmode
\begin{enumerate}
\item \textbf{Induction.}  If a property $P$ holds for all atomic
  formulas and is preserved by every formation rule (negation,
  binary connectives, quantification), then $P$ holds for all
  formulas.  Formally:
  \begin{multline*}
  [\forall \text{ atomic } !A\; P(!A)] \;\land\;
  [\forall !A\; (P(!A) \to P(\lnot !A))] \;\land \\
  [\forall !A\, \forall !B\; (P(!A) \land P(!B) \to P((!A \ast !B)))]
  \;\land\;
  [\forall !A\, \forall x\; (P(!A) \to P(\lforall[x][!A]))] \\
  \;\to\; \forall !A\; P(!A).
  \end{multline*}

\item \textbf{Recursion.}  Any definition by structural recursion on
  formulas---specifying the value for atomic formulas and how to
  combine values across each formation rule---determines a
  \emph{unique} function on~$\Frm[L]$.
\end{enumerate}
\end{thm}

\begin{proof}[Proof sketch]
Part~(1) follows from the fact that $\Frm[L]$ is inductively
defined: it is the smallest set of strings closed under the formation
rules.  If the set $S = \{!A \in \Frm[L] : P(!A)\}$ is also closed
under the formation rules, then $\Frm[L] \subseteq S$, so $P$ holds
for all formulas.

Part~(2) follows from unique readability (Proposition~\ref{THM-SYN001}):
since each formula has a unique outermost construction step, the
recursive clauses never conflict, and the function is well-defined.
\end{proof}
   % CH-SYN: Syntax
\chapter{Semantics} \label{ch:sem}

%% ===================================================================
%% SEM.1: Structures
%% Sources: fol/syn/str (KEEP), fol/syn/cov (CONDENSE)
%% ===================================================================

\section{Structures} \label{SEM.1}

First-order languages are, by themselves, \emph{uninterpreted}: the
constant symbols, function symbols, and predicate symbols have no specific
meaning attached to them.  Meanings are given by specifying
a \emph{structure}. It specifies the
\emph{domain}, i.e., the objects which the constant symbols pick out, the
function symbols operate on, and the quantifiers range over. In addition,
it specifies which constant symbols pick out which objects, how
a function symbol maps objects to objects, and which objects the
predicate symbols apply to.  Structures are the basis for
\emph{semantic} notions in logic, e.g., the notion of consequence,
validity, satisfiability. They are variously called ``structures,''
``interpretations,'' or ``models'' in the literature.

\begin{defn}[Structures] % PRIM-SEM001, PRIM-SEM002, PRIM-SEM003
\label{PRIM-SEM001}
A \emph{structure}~$\Struct M$ for a language
$\Lang{L}$ of first-order logic consists of the following elements:
\begin{enumerate}
\item \emph{Domain:} a non-empty set, $\Domain M$. \label{PRIM-SEM002}
\item \emph{Interpretation of constant symbols:} for each constant symbol~$c$ of
  $\Lang{L}$, an element $\Assign{c}{M} \in \Domain M$.
  \label{PRIM-SEM003}
\item \emph{Interpretation of predicate symbols:} for each $n$-place
  predicate symbol~$R$ of $\Lang{L}$ (other than $\eq$), an $n$-place
  relation $\Assign{R}{M} \subseteq \Domain{M}^n$.
\item \emph{Interpretation of function symbols:} for each $n$-place
  function symbol~$f$ of $\Lang{L}$, an $n$-place function $\Assign{f}{M}
  \colon \Domain{M}^n \to \Domain{M}$.
\end{enumerate}
\end{defn}

\begin{ex}
A structure~$\Struct M$ for the language of arithmetic consists of a
set, an element of $\Domain M$, $\Assign{\Obj 0}{M}$, as
interpretation of the constant symbol~$\Obj 0$, a one-place function
$\Assign{\Obj \prime}{M} \colon \Domain{M} \to \Domain M$, two
two-place functions $\Assign{\Obj +}{M}$ and $\Assign{\Obj
  \times}{M}$, both $\Domain M^2 \to \Domain M$, and a two-place
relation $\Assign{\Obj <}{M} \subseteq \Domain{M}^2$.

An obvious example of such a structure is the following:
\begin{enumerate}
\item $\Domain N = \Nat$
\item $\Assign{\Obj 0}{N} = 0$
\item $\Assign{\Obj \prime}{N}(n) = n + 1$ for all $n \in \Nat$
\item $\Assign{\Obj +}{N}(n, m) = n + m$ for all $n, m \in \Nat$
\item $\Assign{\Obj \times}{N}(n, m) = n\cdot m$ for all $n, m \in \Nat$
\item $\Assign{\Obj <}{N} = \Setabs{\tuple{n, m}}{n \in \Nat, m \in
  \Nat, n < m}$
\end{enumerate}
The structure~$\Struct N$ for $\Lang L_A$ so defined is called the
\emph{standard model of arithmetic}, because it interprets the
non-logical constants of~$\Lang L_A$ exactly how you would expect.
\end{ex}

The stipulations we make as to what counts as a structure impact
our logic. For example, the choice to prevent empty domains ensures
that $\lexists[x][(!A(x) \lor \lnot !A(x))]$ is
valid. Allowing empty domains or names that do not refer leads to
\emph{free logic}.\footnote{In free logic, existential generalization
requires an additional premise: $!A(a)$ and $\lexists[x][\eq[x][a]]$,
therefore $\lexists[x][!A(x)]$.}

%%% Value of Closed Terms (from fol/syn/cov — CONDENSE)

We can assign values to closed terms (terms containing no variables)
using only the structure, without needing a variable assignment.

\begin{defn}[Value of closed terms] % PRIM-SEM006 (closed-term case)
\label{PRIM-SEM006:closed}
If $t$ is a closed term of the language~$\Lang L$ and $\Struct M$ is a
structure for~$\Lang L$, the \emph{value}~$\Value{t}{M}$ is
defined as follows:
\begin{enumerate}
\item If $t$ is just the constant symbol~$c$, then $\Value{c}{M} = \Assign{c}{M}$.
\item If $t$ is of the form $\Atom{f}{t_1, \ldots, t_n}$, then
  \[
  \Value{t}{M} = \Assign{f}{M}(\Value{t_1}{M}, \ldots,
  \Value{t_n}{M}).
  \]
\end{enumerate}
\end{defn}

A structure is \emph{covered} if every element of the domain is the
value of some closed term.


%% ===================================================================
%% SEM.2: Satisfaction and Truth
%% Sources: fol/syn/sat (ABSORB:pl/syn/val), fol/syn/ass (CONDENSE)
%% ===================================================================

\section{Satisfaction and Truth} \label{SEM.2}

The basic notion that relates expressions such as terms and
formulas, on the one hand, and structures on the other, are
those of \emph{value} of a term and \emph{satisfaction} of
a formula.  Informally, the value of a term is an element of
a structure---if the term is just a constant, its value is the
object assigned to the constant by the structure, and if it is
built up using function symbols, the value is computed from the
values of constants and the functions assigned to the functions in
the term.  A formula is \emph{satisfied} in a structure if the
interpretation given to the predicates makes the formula true in
the domain of the structure. This notion of satisfaction is
specified inductively: the specification of the structure directly
states when atomic formulas are satisfied, and we define when a
complex formula is satisfied depending on the main connective or
quantifier and whether or not the immediate subformulas are
satisfied.

The case of the quantifiers is a bit tricky, as the
immediate subformula of a quantified formula has a free
variable, and structures don't specify the values of
variables.  In order to deal with this difficulty, we
introduce \emph{variable assignments} and define satisfaction not with
respect to a structure alone, but with respect to a structure
plus a variable assignment.

\begin{defn}[Variable Assignment] % PRIM-SEM004
\label{PRIM-SEM004}
A \emph{variable assignment}~$s$ for a structure~$\Struct{M}$ is a
function which maps each variable to an element of~$\Domain M$,
i.e., $s\colon \Var \to \Domain M$.
\end{defn}

A structure assigns a value to each constant symbol, and a
variable assignment to each variable.  But we want to use terms built
up from them to also name elements of the domain.  For this we
define the value of terms inductively. For constant symbols and
variables the value is just as the structure or the variable
assignment specifies it; for more complex terms it is computed
recursively using the functions the structure assigns to the
function symbols.

\begin{defn}[Value of Terms] % PRIM-SEM006
\label{PRIM-SEM006}
If $t$ is a term of the language~$\Lang L$, $\Struct M$ is a
structure for~$\Lang L$, and $s$ is a variable assignment
for~$\Struct M$, the \emph{value}~$\Value{t}{M}[s]$ is defined as
follows:
\begin{enumerate}
\item If $t$ is a constant symbol~$c$, then $\Value{c}{M}[s] = \Assign{c}{M}$.
\item If $t$ is a variable~$x$, then $\Value{x}{M}[s] = s(x)$.
\item If $t = \Atom{f}{t_1, \ldots, t_n}$, then
\[
\Value{t}{M}[s] = \Assign{f}{M}(\Value{t_1}{M}[s], \ldots,
\Value{t_n}{M}[s]).
\]
\end{enumerate}
\end{defn}

\begin{defn}[$x$-Variant] % PRIM-SEM005
\label{PRIM-SEM005}
If $s$ is a variable assignment for a structure~$\Struct M$, then any
variable assignment~$s'$ for~$\Struct M$ which differs from~$s$ at most
in what it assigns to~$x$ is called an \emph{$x$-variant} of~$s$.  If
$s'$ is an $x$-variant of~$s$ we write $\varAssign{s'}{s}{x}$.
\end{defn}

Note that an $x$-variant of an assignment~$s$ does not \emph{have} to
assign something different to~$x$.  In fact, every assignment counts
as an $x$-variant of itself.

\begin{defn}
  If $s$ is a variable assignment for a structure~$\Struct M$
  and $m \in \Domain{M}$, then the assignment~$\Subst{s}{m}{x}$ is the
  variable assignment defined by
  \[\Subst{s}{m}{x}(y) = \begin{cases}
    m & \text{if } y \ident x\\
    s(y) & \text{otherwise}.
  \end{cases}\]
\end{defn}

In other words, $\Subst{s}{m}{x}$ is the particular $x$-variant of~$s$
which assigns the domain element~$m$ to~$x$, and assigns the same
things to variables other than~$x$ that $s$ does.

\begin{defn}[Satisfaction] % PRIM-SEM007, DEF-SEM001
\label{PRIM-SEM007}
\label{DEF-SEM001}
Satisfaction of a formula~$!A$ in a structure~$\Struct M$
relative to a variable assignment~$s$, in symbols:
$\Sat{M}{!A}[s]$, is defined recursively as follows. (We write
$\Sat/{M}{!A}[s]$ to mean ``not $\Sat{M}{!A}[s]$.'')
\begin{enumerate}
\item $\Sat{M}{\Atom{R}{t_1, \dots, t_n}}[s]$
  iff $\langle \Value{t_1}{M}[s], \dots, \Value{t_n}{M}[s] \rangle \in
  \Assign{R}{M}$.

\item $\Sat{M}{\eq[t_1][t_2]}[s]$ iff
  $\Value{t_1}{M}[s] = \Value{t_2}{M}[s]$.

\item $\Sat{M}{\lnot !B}[s]$ iff
    $\Sat/{M}{!B}[s]$.

\item $\Sat{M}{(!B \land !C)}[s]$ iff $\Sat{M}{!B}[s]$
    and $\Sat{M}{!C}[s]$.

\item $\Sat{M}{(!B \lor !C)}[s]$ iff
    $\Sat{M}{!B}[s]$ or $\Sat{M}{!C}[s]$ (or both).

\item $\Sat{M}{(!B \lif !C)}[s]$ iff $\Sat/{M}{!B}[s]$
    or $\Sat{M}{!C}[s]$ (or both).

\item $\Sat{M}{(!B \liff !C)}[s]$ iff either both
    $\Sat{M}{!B}[s]$ and $\Sat{M}{!C}[s]$, or neither $\Sat{M}{!B}[s]$
    nor $\Sat{M}{!C}[s]$.

\item $\Sat{M}{\lforall[x][!B]}[s]$ iff for every
    element~$m \in \Domain M$, $\Sat{M}{!B}[\Subst{s}{m}{x}]$.

\item $\Sat{M}{\lexists[x][!B]}[s]$ iff for at least
  one element~$m \in \Domain M$, $\Sat{M}{!B}[\Subst{s}{m}{x}]$.
\end{enumerate}
\end{defn}

The variable assignments are important in the quantifier clauses. We
cannot define satisfaction of $\lforall[x][!B(x)]$ by ``for all $m \in
\Domain{M}$, $\Sat{M}{!B(m)}$,'' because if $m \in
\Domain M$, it is not a symbol of the language, and so $!B(m)$~is not
a formula (that is, $\Subst{!B}{m}{x}$ is undefined).  We also
cannot assume that we have constant symbols or terms available that name
every element of~$\Struct{M}$, since there is nothing in the
definition of structures that requires it.  Variable assignments
allow us to link variables directly with elements of the domain,
resolving this difficulty.

\begin{rem}[PL Specialization] % PRIM-SEM015
\label{PRIM-SEM015}
For propositional logic, a \emph{truth-value assignment} (or
\emph{valuation}) $v \colon \mathrm{PropVar} \to \{0,1\}$ replaces
the structure-plus-variable-assignment pair. Satisfaction reduces to:
$v \vDash p$ iff $v(p) = 1$; the connective clauses are the same as
above (without quantifier clauses). Propositional satisfaction is
thus the quantifier-free fragment of first-order satisfaction.
\end{rem}

%%% Satisfaction depends only on free variables (from fol/syn/ass — CONDENSE)

The value of a term~$t$, and whether or not a formula~$!A$ is
satisfied in a structure with respect to~$s$, only depend on the
assignments~$s$ makes to the variables in~$t$ and the free
variables of~$!A$.

The value of a term depends only on the values of its variables:
if $s_1$ and $s_2$ agree on all variables occurring in~$t$, then
$\Value{t}{M}[s_1] = \Value{t}{M}[s_2]$.

\begin{prop}
\label{prop:satindep}
If the free variables in $!A$ are among $x_1$, \dots,~$x_n$, and
$s_1(x_i) = s_2(x_i)$ for $i = 1$, \dots,~$n$, then $\Sat{M}{!A}[s_1]$
iff $\Sat{M}{!A}[s_2]$.
\end{prop}

\begin{proof}[Proof sketch]
By induction on the complexity of $!A$. The base case uses the
corresponding result for terms. For connectives, apply the induction
hypothesis to subformulas (whose free variables are among those
of~$!A$). For the quantifier case $\lexists[x][!B]$: if
$\Sat{M}{!B}[\Subst{s_1}{m}{x}]$ for some~$m$, then
$\Subst{s_1}{m}{x}$ and $\Subst{s_2}{m}{x}$ agree on the free
variables of~$!B$ (which are among $x_1, \ldots, x_n, x$), so the
induction hypothesis gives
$\Sat{M}{!B}[\Subst{s_2}{m}{x}]$. Similarly for $\lforall[x][!B]$.
\end{proof}

Sentences have no free variables, so any two variable assignments
assign the same things to all the (zero) free variables of any
sentence. Therefore:

\begin{cor}
\label{cor:sat-sentence}
If $!A$ is a sentence and $s$ a variable assignment, then
$\Sat{M}{!A}[s]$ iff $\Sat{M}{!A}[s']$ for every variable
assignment~$s'$.
\end{cor}

This justifies the following definition.

\begin{defn}[Truth in a Structure] % PRIM-SEM008
\label{PRIM-SEM008}
If $!A$ is a sentence, we say that a structure~$\Struct M$
\emph{satisfies}~$!A$, $\Sat{M}{!A}$, iff $\Sat{M}{!A}[s]$ for all
variable assignments~$s$.
\end{defn}

If $\Sat{M}{!A}$, we also simply say that \emph{$!A$ is true
in~$\Struct{M}$.}

\begin{defn}[Satisfaction for sets of sentences] % PRIM-SEM011 (alternative formulation)
\label{defn:sat-set}
If $\Gamma$ is a set of sentences, we say that
a structure~$\Struct M$ \emph{satisfies}~$\Gamma$,
$\Sat{M}{\Gamma}$, iff $\Sat{M}{!A}$ for all $!A \in \Gamma$.
\end{defn}

\begin{prop}
\label{prop:sentence-sat-true}
  Let $\Struct{M}$ be a structure, $!A$ be a sentence, and $s$ a
  variable assignment.  $\Sat{M}{!A}$ iff $\Sat{M}{!A}[s]$.
\end{prop}


%% ===================================================================
%% SEM.3: Validity and Consequence
%% Sources: fol/syn/sem (ABSORB:pl/syn/sem)
%% ===================================================================

\section{Validity and Consequence} \label{SEM.3}

Given the definition of structures for first-order languages, we can
define some basic semantic properties of and relationships between
sentences.  The simplest of these is the notion of \emph{validity} of
a sentence.  A sentence is valid if it is satisfied in every
structure.  Valid sentences are those that are satisfied regardless of
how the non-logical symbols in it are interpreted.  Valid sentences
are therefore also called \emph{logical truths}---they are true, i.e.,
satisfied, in any structure and hence their truth depends only on the
logical symbols occurring in them and their syntactic structure, but not
on the non-logical symbols or their interpretation.

\begin{defn}[Validity] % PRIM-SEM009
\label{PRIM-SEM009}
A sentence $!A$ is \emph{valid}, $\Entails !A$, iff $\Sat{M}{!A}$ for every
structure~$\Struct M$.
\end{defn}

\begin{defn}[Entailment] % PRIM-SEM010
\label{PRIM-SEM010}
A set of sentences~$\Gamma$ \emph{entails} a sentence~$!A$, $\Gamma
\Entails !A$, iff for every structure~$\Struct M$ with
$\Sat{M}{\Gamma}$, $\Sat{M}{!A}$.
\end{defn}

\begin{defn}[Satisfiability] % DEF-SEM002
\label{DEF-SEM002}
A set of sentences~$\Gamma$ is \emph{satisfiable} if $\Sat{M}{\Gamma}$
for some structure~$\Struct M$.  If $\Gamma$ is not satisfiable it is
called \emph{unsatisfiable}.
\end{defn}

\begin{defn}[Model] % PRIM-SEM011
\label{PRIM-SEM011}
Let $\Gamma$ be a set of sentences in a language~$\Lang L$.  We
say that a structure~$\Struct M$ \emph{is a model of}~$\Gamma$ if
$\Sat{M}{!A}$ for all $!A \in \Gamma$.
\end{defn}

\begin{defn}[Semantic Consistency] % DEF-SEM004
\label{DEF-SEM004}
A set of sentences~$\Gamma$ is \emph{semantically consistent} if it is
satisfiable, i.e., if it has at least one model. This is the semantic
counterpart of syntactic consistency (see \S\ref{DED.1}).
\end{defn}

We record some basic relationships:

\begin{itemize}
\item $\Gamma \Entails !A$ iff $\Gamma \cup \{\lnot !A\}$ is unsatisfiable.
\item (Monotonicity) If $\Gamma \subseteq \Gamma'$ and $\Gamma \Entails !A$,
  then $\Gamma' \Entails !A$.
\item (Semantic Deduction Theorem) $\Gamma \cup \{!A\} \Entails !B$ iff
  $\Gamma \Entails !A \lif !B$.
\end{itemize}

\begin{rem}[PL Specialization: Tautology] % DEF-SEM009
\label{DEF-SEM009}
A propositional formula is a \emph{tautology} if it is true under
every truth-value assignment. This specializes first-order validity to
propositional structures: a tautology is a formula valid under all
valuations $v \colon \mathrm{PropVar} \to \{0,1\}$.
\end{rem}

\begin{rem}[PL Specialization: PL Consequence] % DEF-SEM010
\label{DEF-SEM010}
PL entailment $\Gamma \vDash_{\mathrm{PL}} !A$ specializes first-order
entailment to propositional structures: every truth-value assignment
making all formulas in~$\Gamma$ true also makes~$!A$ true.
\end{rem}


%% ===================================================================
%% SEM.4: Models and Theories
%% Sources: fol/mat/exs (CONDENSE), fol/mat/int (DISTRIBUTE — DEF-SEM007)
%% ===================================================================

\section{Models and Theories} \label{SEM.4}

\begin{defn}[Semantic Completeness of a Theory] % DEF-SEM005
\label{DEF-SEM005}
A theory $T$ is \emph{semantically complete} iff for every
sentence~$!A$ in its language, either $T \Entails !A$ or
$T \Entails \lnot !A$.
\end{defn}

Note that semantic completeness of a theory is \emph{not} the same as
the Completeness Theorem (see \S\ref{META.2}), which states that
$\Gamma \Entails !A$ implies $\Gamma \Proves !A$.

\begin{defn}[Theory of a Structure] % DEF-SEM006
\label{DEF-SEM006}
Given a structure~$\Struct M$, the \emph{theory} of
$\Struct{M}$ is the set $\Theory{M}$ of sentences
that are true in $\Struct{M}$, i.e., $\Theory{M} =
\Setabs{!A}{\Sat{M}{!A}}$.
\end{defn}

We also use the term ``theory'' informally to refer to sets
of sentences having an intended interpretation, whether deductively
closed or not.

\begin{prop}
\label{prop:ThM-complete}
For any $\Struct{M}$, $\Theory{M}$ is semantically complete.
\end{prop}

\begin{proof}
For any sentence~$!A$ either $\Sat{M}{!A}$ or $\Sat{M}{\lnot !A}$,
so either $!A \in \Theory{M}$ or $\lnot !A \in \Theory{M}$.
\end{proof}

\begin{defn}[Definable Set] % DEF-SEM007
\label{DEF-SEM007}
A subset of $\Domain{M}^n$ is \emph{definable} (in $\Struct{M}$)
if there is a formula $!A(x_1, \ldots, x_n)$ with free variables
$x_1, \ldots, x_n$ such that the subset equals
$\Setabs{\langle a_1, \ldots, a_n \rangle \in
\Domain{M}^n}{\Sat{M}{!A}[\Subst{s}{a_1}{x_1}, \ldots,
\Subst{s}{a_n}{x_n}]}$.
\end{defn}

\begin{defn}[Elementary Equivalence] % DEF-SEM008
\label{DEF-SEM008}
Given two structures $\Struct{M}$ and $\Struct M'$ for the same
language~$\Lang{L}$, we say that $\Struct{M}$ is \emph{elementarily
  equivalent to} $\Struct M'$, written $\Struct{M} \equiv \Struct M'$,
if and only if for every sentence~$!A$ of~$\Lang{L}$,
$\Sat{M}{!A}$ iff $\Sat{M'}{!A}$.
\end{defn}

\begin{prop}
\label{prop:equiv}
  If $\Sat{N}{!A}$ for every $!A \in \Theory{M}$, then
  $\Struct{M} \equiv \Struct{N}$.
\end{prop}

\begin{proof}
Since $\Sat{N}{!A}$ for all $!A \in \Theory{M}$, $\Theory{M} \subseteq
\Theory{N}$. If $\Sat{N}{!A}$, then $\Sat/{N}{\lnot !A}$, so $\lnot !A
\notin \Theory{M}$. Since $\Theory{M}$ is complete, $!A \in
\Theory{M}$. So, $\Theory{N} \subseteq \Theory{M}$, and we have
$\Struct{M} \equiv \Struct{N}$.
\end{proof}

\begin{rem}
\label{rem:R}
  Consider $\Struct{R} = \langle\Real, <\rangle$, the structure
  whose domain is the set $\Real$ of the real numbers, in the language
  comprising only a 2-place predicate symbol interpreted as the $<$
  relation over the reals. Clearly $\Struct{R}$ is uncountable;
  however, since $\Theory{R}$ is obviously consistent, by the
  L\"owenheim--Skolem theorem (see \S\ref{META.4}) it has a countable model, say
  $\Struct{S}$, and by the above proposition, $\Struct{R}
  \equiv \Struct{S}$. Moreover, since $\Struct{R}$ and $\Struct{S}$
  are not isomorphic, this shows that the converse of
  the Isomorphism Lemma (Theorem~\ref{THM-SEM001}) fails in general:
  elementary equivalence does not imply isomorphism.
\end{rem}

\begin{defn}[Axiomatized Theory]
\label{defn:axiomatized-theory}
A set of sentences~$\Gamma$ is \emph{closed} iff, whenever
$\Gamma \Entails !A$ then $!A \in \Gamma$.  The \emph{closure} of a set
of sentences~$\Gamma$ is $\Setabs{!A}{\Gamma \Entails !A}$.
We say that~$\Gamma$ is \emph{axiomatized by} a set of
sentences~$\Delta$ if $\Gamma$ is the closure of~$\Delta$.
\end{defn}


%% ===================================================================
%% SEM.5: Arithmetic Models
%% Sources: inc/int/def (DISTRIBUTE — DEF-SEM017, DEF-SEM018),
%%          inc/tcp/itp (CONDENSE — DEF-SEM019 remark),
%%          mod/mar/stm, mod/mar/nst, mod/mar/mpa, mod/mar/cmp
%% ===================================================================

\section{Arithmetic Models} \label{SEM.5}

The structures studied in this section are models of the first-order
theories of arithmetic ($\Th{Q}$, $\Th{PA}$) introduced in
\S\ref{DED.6}.

\begin{defn}[Standard Model of Arithmetic] % DEF-SEM017
\label{DEF-SEM017}
The \emph{standard model of arithmetic} is the structure~$\Struct
N$ defined as follows:
\begin{enumerate}
\item $\Domain N = \Nat$
\item $\Assign{\Obj 0}{N} = 0$
\item $\Assign{\Obj \prime}{N}(n) = n + 1$ for all $n \in \Nat$
\item $\Assign{\Obj +}{N}(n, m) = n + m$ for all $n, m \in \Nat$
\item $\Assign{\Obj \times}{N}(n, m) = n\cdot m$ for all $n, m \in \Nat$
\item $\Assign{\Obj <}{N} = \Setabs{\tuple{n, m}}{n \in \Nat, m \in
  \Nat, n < m}$
\end{enumerate}
\end{defn}

\begin{defn}[True Arithmetic] % DEF-SEM018
\label{DEF-SEM018}
The theory of \emph{true arithmetic} is the set of sentences
satisfied in the standard model of arithmetic, i.e.,
\[
\Th{TA} = \Setabs{!A}{\Sat{N}{!A}}.
\]
\end{defn}

$\Th{TA}$ is a theory (closed under entailment), for whenever $\Th{TA} \Entails !A$, $!A$ is
satisfied in every structure which satisfies~$\Th{TA}$. Since
$\Sat{N}{\Th{TA}}$, we have that~$\Sat{N}{!A}$, and so $!A \in \Th{TA}$.

\begin{rem}[Interpretability] % DEF-SEM019
\label{DEF-SEM019}
Informally, an interpretation of a language $\Lang{L_1}$ in another
language $\Lang{L_2}$ involves defining the universe, relation symbols,
and function symbols of $\Lang{L_1}$ with formulas in $\Lang{L_2}$.
One can show: if a theory~$\Th{T}$ is consistent with the
interpretation of Robinson's $\Th{Q}$ (see \S\ref{DED.6}), then $\Th{T}$ is
undecidable, and no consistent extension of $\Th{T}$ is decidable. In
particular, there is no decidable or complete consistent axiomatizable extension of $\Th{ZFC}$.
\end{rem}


%% ===================================================================
%% SEM.6: Model-Theoretic Concepts
%% Sources: mod/bas/red, mod/bas/sub, mod/bas/iso, mod/bas/thm,
%%          mod/bas/dlo, mod/bas/ove
%% ===================================================================

\section{Model-Theoretic Concepts} \label{SEM.6}

%%% Reducts and Expansions (from mod/bas/red — CONDENSE)

\subsection*{Reducts and Expansions}

Often it is useful or necessary to compare languages which have
symbols in common, as well as structures for these languages.  The
most common case is when all the symbols in a language~$\Lang{L}$
are also part of a language~$\Lang{L'}$, i.e., $\Lang{L} \subseteq
\Lang{L'}$.

\begin{defn}[Reduct and Expansion]
\label{defn:reduct}
Suppose $\Lang L \subseteq \Lang L'$, $\Struct M$ is an
$\Lang L$-structure and $\Struct M'$ is an $\Lang L'$-structure.
$\Struct M$ is the \emph{reduct} of $\Struct M'$ to $\Lang L$, and
$\Struct M'$ is an \emph{expansion} of $\Struct M$ to $\Lang L'$ iff
\begin{enumerate}
\item $\Domain{M} = \Domain{M'}$;
\item For every constant symbol~$c \in \Lang L$, $\Assign{c}{M} =
  \Assign{c}{M'}$;
\item For every function symbol~$f \in \Lang L$, $\Assign{f}{M} =
  \Assign{f}{M'}$;
\item For every predicate symbol~$P \in \Lang L$, $\Assign{P}{M} =
  \Assign{P}{M'}$.
\end{enumerate}
\end{defn}

If $\Struct{M}$ is a reduct of $\Struct{M'}$, then for all
$\Lang{L}$-sentences~$!A$, $\Sat{M}{!A}$ iff $\Sat{M'}{!A}$.

When we have an $\Lang{L}$-structure $\Struct{M}$, and $\Lang{L'} =
\Lang{L} \cup \{P\}$ is the expansion of $\Lang{L}$ obtained by adding
a single $n$-place predicate symbol~$P$, and $R \subseteq \Domain{M}^n$
is an $n$-place relation, then we write $\Expan{M}{R}$ for the
expansion~$\Struct{M'}$ of~$\Struct{M}$ with $\Assign{P}{M'} = R$.

%%% Substructures (from mod/bas/sub — KEEP)

\subsection*{Substructures}

The domain of a structure~$\Struct{M}$ may be a subset of
another~$\Struct{M'}$.  But we should obviously only consider
$\Struct{M}$ a ``part'' of $\Struct{M'}$ if not only $\Domain{M}
\subseteq \Domain{M'}$, but $\Struct{M}$ and $\Struct{M'}$ ``agree''
in how they interpret the symbols of the language at least on the
shared part~$\Domain{M}$.

\begin{defn}[Substructure] % PRIM-SEM013
\label{PRIM-SEM013}
Given structures $\Struct M$ and $\Struct M'$ for the same
language~$\Lang L$, we say that $\Struct M$ is a \emph{substructure}
of $\Struct M'$, and $\Struct M'$ an \emph{extension} of $\Struct M$,
written $\Struct M \substruct \Struct M'$, iff
\begin{enumerate}
\item $\Domain{M} \subseteq \Domain{M'}$;
\item For each constant $c \in \Lang L$, $\Assign{c}{M} =
    \Assign{c}{M'}$;
\item For each $n$-place function symbol $f \in \Lang L$,
  $\Assign{f}{M}(a_1, \dots, a_n) = \Assign{f}{M'}(a_1, \dots, a_n)$
  for all $a_1$, \dots, $a_n \in \Domain{M}$;
\item For each $n$-place predicate symbol $R \in \Lang L$, $\langle
  a_1, \dots, a_n\rangle \in \Assign{R}{M}$ iff $\langle a_1, \dots,
  a_n\rangle \in \Assign{R}{M'}$ for all $a_1$, \dots, $a_n \in
  \Domain{M}$.
\end{enumerate}
\end{defn}

\begin{rem}
\label{rem:substructure}
If the language contains no constant or function symbols, then any
non-empty $N \subseteq \Domain{M}$ determines a substructure~$\Struct{N}$ of
$\Struct M$ with domain~$\Domain{N} = N$ by putting $\Assign{R}{N} =
\Assign{R}{M} \cap N^n$.
\end{rem}

%%% Homomorphisms (NEW-CONTENT from DOMAIN-SEMANTICS)

\subsection*{Homomorphisms}

\begin{defn}[Homomorphism] % PRIM-SEM014
\label{PRIM-SEM014}
A function $h \colon \Domain{M} \to \Domain{M'}$ between structures
$\Struct{M}$ and $\Struct{M'}$ for the same language~$\Lang{L}$ is a
\emph{homomorphism} if:
\begin{enumerate}
\item For every constant symbol $c$: $h(\Assign{c}{M}) = \Assign{c}{M'}$;
\item For every $n$-place function symbol $f$:
  $h(\Assign{f}{M}(a_1, \ldots, a_n)) = \Assign{f}{M'}(h(a_1), \ldots, h(a_n))$;
\item For every $n$-place predicate symbol $R$:
  $\langle a_1, \ldots, a_n \rangle \in \Assign{R}{M} \Rightarrow
  \langle h(a_1), \ldots, h(a_n) \rangle \in \Assign{R}{M'}$.
\end{enumerate}
A homomorphism is weaker than an isomorphism: it need not be bijective,
and it preserves relations in one direction only (it need not reflect them).
\end{defn}

%%% Embedding (NEW-CONTENT from DOMAIN-SEMANTICS)

\begin{defn}[Embedding] % DEF-SEM016
\label{DEF-SEM016}
An \emph{embedding} $h \colon \Domain{M} \hookrightarrow \Domain{M'}$
between structures $\Struct{M}$ and $\Struct{M'}$ for the same
language~$\Lang{L}$ is an injective homomorphism that also
\emph{reflects} relations: for every $n$-place predicate symbol $R$,
\[
\langle a_1, \ldots, a_n \rangle \in \Assign{R}{M} \quad\text{iff}\quad
\langle h(a_1), \ldots, h(a_n) \rangle \in \Assign{R}{M'}.
\]
Every isomorphism is a surjective embedding; every embedding is a
homomorphism that additionally reflects atomic formulas.
\end{defn}

%%% Isomorphism (from mod/bas/iso — KEEP)

\subsection*{Isomorphism}

First-order structures can be alike in one of two ways. One way in
which they can be alike is that they make the same sentences
true---we call such structures \emph{elementarily equivalent}
(Definition~\ref{DEF-SEM008}). But structures can be very different and still
make the same sentences true---for instance, one can be countable and
the other not.  So another, stricter, aspect in which structures can
be alike is if they are fundamentally the same, in the sense that they
only differ in the objects that make them up, but not in their
structural features.

\begin{defn}[Isomorphism] % PRIM-SEM012
\label{PRIM-SEM012}
Given two structures $\Struct{M}$ and
$\Struct M'$ for the same language~$\Lang L$, we say that
$\Struct{M}$ is \emph{isomorphic to}~$\Struct M'$, written $\Struct{M}
\simeq \Struct M'$, if and only if there is a function $h \colon
\Domain{M} \to \Domain{M'}$ such that:
\begin{enumerate}
\item $h$ is injective: if $h(x) = h(y)$ then $x = y$;
\item $h$ is surjective: for every $y \in \Domain{M'}$ there
  is $x \in \Domain{M}$ such that $h(x) = y$;
\item For every constant symbol $c$:
  $h(\Assign{c}{M}) = \Assign{c}{M'}$;
\item For every $n$-place predicate symbol~$P$:
  \[
  \tuple{a_1, \dots, a_n}\in \Assign{P}{M} \quad\text{iff}\quad
  \tuple{h(a_1), \dots, h(a_n)} \in \Assign{P}{M'};
  \]
\item For every $n$-place function symbol $f$:
  \[
  h(\Assign{f}{M}(a_1, \dots, a_n)) =
  \Assign{f}{M'}(h(a_1), \dots, h(a_n)).
  \]
\end{enumerate}
\end{defn}

\begin{thm}[Isomorphism Lemma] % THM-SEM001
\label{THM-SEM001}
If $\Struct{M} \iso \Struct M'$ then $\Struct{M} \equiv
\Struct{M'}$.
\end{thm}

\begin{proof}
Let $h$ be an isomorphism of $\Struct{M}$ onto $\Struct M'$. For any
assignment~$s$, $h \circ s$ is the composition of $h$ and $s$, i.e.,
the assignment in $\Struct{M'}$ such that $(h \circ s)(x) = h(s(x))$.
By induction on $t$ and $!A$ one proves the stronger claims:
\begin{enumerate}
  \item[a.] $h(\Value{t}{M}[s]) = \Value{t}{M'}[h\circ s]$.
  \item[b.] $\Sat{M}{!A}[s]$ iff $\Sat{M'}{!A}[h \circ s]$.
\end{enumerate}
Part (a) is proved by induction on the complexity of~$t$.
\begin{enumerate}
\item If $t \ident c$, then $h(\Value{c}{M}[s]) = h(\Assign{c}{M}) = \Assign{c}{M'} =
  \Value{c}{M'}[h \circ s]$.
\item If $t \ident x$, then $h(\Value{x}{M}[s]) =
  h(s(x)) = (h \circ s)(x) = \Value{x}{M'}[h \circ s]$.
\item If $t \ident f(t_1, \dots, t_n)$, then by induction hypothesis
  $h(\Value{t_i}{M}[s]) = \Value{t_i}{M'}[h\circ s]$ for each $i$, so
  \begin{align*}
    h(\Value{t}{M}[s])
    & = h(\Assign{f}{M}(\Value{t_1}{M}[s], \dots, \Value{t_n}{M}[s])) \\
    & = \Assign{f}{M'}(h(\Value{t_1}{M}[s]), \dots,
    h(\Value{t_n}{M}[s])) \\
    & = \Assign{f}{M'}(\Value{t_1}{M'}[h \circ s], \dots,
    \Value{t_n}{M'}[h \circ s]) \\
    & = \Value{t}{M'}[h\circ s].
  \end{align*}
\end{enumerate}
Part (b) is proved by induction on the complexity of~$!A$.
If $!A$ is a sentence, the assignments~$s$ and $h \circ s$ are
irrelevant, and we have $\Sat{M}{!A}$ iff $\Sat{M'}{!A}$.
\end{proof}

\begin{defn}
An \emph{automorphism} of a structure $\Struct{M}$ is an isomorphism
of $\Struct{M}$ onto itself.
\end{defn}

\begin{defn}[Elementary Substructure] % DEF-SEM011
\label{DEF-SEM011}
$\Struct{M}$ is an \emph{elementary substructure} of $\Struct{M'}$,
written $\Struct{M} \preccurlyeq \Struct{M'}$, if $\Struct{M}$ is a
substructure of $\Struct{M'}$ and for every formula
$!A(x_1, \ldots, x_n)$ and all $a_1, \ldots, a_n \in \Domain{M}$:
$\Sat{M}{!A}[a_1, \ldots, a_n]$ iff $\Sat{M'}{!A}[a_1, \ldots, a_n]$.
\end{defn}

%%% Diagram (NEW-CONTENT from DOMAIN-SEMANTICS)

\subsection*{Diagrams}

\begin{defn}[Diagram] % DEF-SEM012
\label{DEF-SEM012}
Let $\Struct{M}$ be a structure for $\Lang{L}$. Expand $\Lang{L}$ to
$\Lang{L}_M = \Lang{L} \cup \{c_a : a \in \Domain{M}\}$ by adding a
new constant symbol~$c_a$ for each element~$a$ of the domain. The
\emph{atomic diagram} of $\Struct{M}$ is the set
\[
\mathrm{Diag}(\Struct{M}) = \Setabs{!A}{!A \text{ is atomic or negated
    atomic in } \Lang{L}_M \text{ and } \Sat{M}{!A}}.
\]
The \emph{elementary diagram} of $\Struct{M}$ is the set
\[
\mathrm{ElDiag}(\Struct{M}) = \Setabs{!A}{!A \in
  \mathrm{Sent}(\Lang{L}_M) \text{ and } \Sat{M}{!A}}.
\]
Any model of $\mathrm{Diag}(\Struct{M})$ contains an isomorphic copy
of $\Struct{M}$ (via the map $a \mapsto c_a^{\Struct{M}}$), and any
model of $\mathrm{ElDiag}(\Struct{M})$ contains an elementary
extension of $\Struct{M}$.
\end{defn}

%%% Type (NEW-CONTENT from DOMAIN-SEMANTICS)

\begin{defn}[Complete Type] % DEF-SEM013
\label{DEF-SEM013}
Let $\Struct{M}$ be a structure for $\Lang{L}$ and $A \subseteq
\Domain{M}$. A \emph{complete $n$-type over $A$} is a maximal
consistent set~$p$ of formulas $!A(x_1, \ldots, x_n)$ with
parameters from $A$ that is finitely satisfiable in $\Struct{M}$.
The set of all complete $n$-types over $A$ is denoted
$S_n^{\Struct{M}}(A)$. Types classify the possible ``behaviors'' of
$n$-tuples in models extending~$\Struct{M}$.
\end{defn}

%%% Ultraproduct (NEW-CONTENT from DOMAIN-SEMANTICS)

\subsection*{Ultraproducts}

\begin{defn}[Ultraproduct] % DEF-SEM015
\label{DEF-SEM015}
Given a family of structures $\{\Struct{M}_i\}_{i \in I}$ for the
same language~$\Lang{L}$ and an ultrafilter $\mathcal{U}$ on $I$, the
\emph{ultraproduct} $\prod_{i \in I} \Struct{M}_i / \mathcal{U}$ is
the structure with domain
\[
\prod_{i \in I} \Domain{M_i} \big/ {\sim_{\mathcal{U}}},
\]
where $f \sim_{\mathcal{U}} g$ iff $\{i \in I : f(i) = g(i)\} \in
\mathcal{U}$. Constant, function, and predicate symbols are
interpreted componentwise modulo~$\mathcal{U}$. When all
$\Struct{M}_i = \Struct{M}$, the construction is called an
\emph{ultrapower} of $\Struct{M}$.
\end{defn}

Ultraproducts provide a purely model-theoretic proof of compactness
(see CP-003, \S\ref{META.3}): a set of sentences is satisfiable iff
every finite subset is, by taking an ultraproduct of the finite-subset
models. This avoids completeness entirely.

%%% Categoricity (from mod/bas/dlo — KEEP, with DEF-SEM014 added)

\subsection*{Categoricity and Dense Linear Orders}

\begin{defn}[Categoricity] % DEF-SEM014
\label{DEF-SEM014}
A theory~$T$ is \emph{$\kappa$-categorical} if all models of~$T$ of
cardinality~$\kappa$ are isomorphic. By the L\"owenheim--Skolem
theorem, no theory with infinite models is categorical in all
cardinalities.
\end{defn}

\begin{defn}[Dense Linear Ordering]
  A \emph{dense linear ordering without endpoints} is a structure
  $\Struct{M}$ for the language containing a single 2-place
  predicate symbol~$<$ satisfying the following sentences:
  \begin{enumerate}
  \item $\lforall[x][\lnot x < x]$ \hfill (irreflexivity)
  \item $\lforall[x][\lforall[y][\lforall[z][(x < y \lif (y < z \lif x
    <z ))]]]$ \hfill (transitivity)
  \item $\lforall[x][\lforall[y][(x< y \lor \eq[x][y] \lor y < x)]]$ \hfill (totality)
  \item $\lforall[x][\lexists[y][x < y]]$ \hfill (no greatest element)
  \item $\lforall[x][\lexists[y][y < x]]$ \hfill (no least element)
  \item $\lforall[x][\lforall[y][(x < y \lif \lexists[z][(x < z \land
        z < y)])]]$ \hfill (density)
 \end{enumerate}
\end{defn}

\begin{thm}[Cantor]
\label{thm:cantorQ}
  Any two countable dense linear orderings without
  endpoints are isomorphic.
\end{thm}

\begin{proof}
  Let $\Struct{M_1}$ and $\Struct{M_2}$ be countable dense linear
  orderings without endpoints, with ${<_1} = \Assign{<}{M_1}$ and ${<_2} =
  \Assign{<}{M_2}$, and let $\PIso{I}$ be the set of all partial
  isomorphisms between them. $\PIso{I}$ is not empty since at least
  $\emptyset \in \PIso{I}$. We show that $\PIso{I}$ satisfies the
  Back-and-Forth property.

  To show $\PIso{I}$ satisfies the Forth property, let $p \in
  \PIso{I}$ and let $p(a_i) = b_i$ for $i = 1$, \dots,~$n$, and
  without loss of generality suppose $a_1 <_1 a_2 <_1 \cdots <_1
  a_n$. Given $a \in \Domain{M_1}$, find $b \in \Domain{M_2}$ as
  follows:
  \begin{enumerate}
  \item if $a <_1 a_1$ let $b \in \Domain{M_2}$ be such that $b <_2
    b_1$;
  \item if $a_n <_1 a$ let $b \in \Domain{M_2}$ be such that $b_n <_2 b$;
 \item if $a_i <_1 a <_1 a_{i+1}$ for some $i$, then let $b \in
   \Domain{M_2}$ be such that $b_i <_2 b <_2 b_{i+1}$.
  \end{enumerate}
  It is always possible to find a $b$ with the desired property since
  $\Struct{M_2}$ is a dense linear ordering without endpoints. Define
  $q = p \cup \{ \langle a, b \rangle \}$ so that $q \in \PIso{I}$ is
  the desired extension of $p$. The Back property is similar. By
  the back-and-forth theorem (applied to countable structures),
  $\Struct{M_1} \iso \Struct{M_2}$.
\end{proof}

The theory of dense linear orders without endpoints is thus
$\aleph_0$-categorical.

\begin{rem}
  Let $\Struct{S}$ be any countable dense linear ordering without
  endpoints. Then by Cantor's theorem, $\Struct{S} \iso
  \Struct{Q}$, where $\Struct{Q} = (\Rat, <)$. Now consider the
  structure~$\Struct{R} =
  (\Real, <)$ from Remark~\ref{rem:R}. There is
  a countable structure~$\Struct{S}$ such that $\Struct{R}
  \equiv \Struct{S}$. But $\Struct{S}$ is a countable dense
  linear ordering without endpoints, and so it is isomorphic (and
  hence elementarily equivalent) to $\Struct{Q}$. By
  transitivity of elementary equivalence, $\Struct{R} \equiv
  \Struct{Q}$.
\end{rem}

%%% Standard and Non-Standard Models of Arithmetic
%%% (from mod/mar/stm, mod/mar/nst, mod/mar/mpa, mod/mar/cmp — CONDENSE)

\subsection*{Standard and Non-Standard Models of Arithmetic}

A structure for $\Lang{L_A}$ is \emph{standard} if it is
isomorphic to~$\Struct{N}$.

\begin{prop}
\label{prop:standard-domain}
If a structure~$\Struct{M}$ is standard,
then its domain is the set of values of the standard numerals, i.e.,
$\Domain{M} = \Setabs{\Value{\num{n}}{M}}{n \in \Nat}$.
\end{prop}

\begin{proof}[Proof sketch]
Since $\Struct{M}$ is standard, there is an isomorphism $g \colon \Nat
\to \Domain{M}$. Then $g(n) = g(\Value{\num{n}}{N}) =
\Value{\num{n}}{M}$, and $g$ is surjective.
\end{proof}

\begin{prop}
\label{prop:thq-standard}
If $\Sat{M}{\Th{Q}}$, and $\Domain{M} = \Setabs{\Value{\num{n}}{M}}{n
  \in \Nat}$, then $\Struct{M}$ is standard.
\end{prop}

\begin{proof}[Proof sketch]
The function $g(n) = \Value{\num{n}}{M}$ is surjective by hypothesis
and injective because $\Th{Q} \Proves \eq/[\num{n}][\num{m}]$ whenever
$n \neq m$, so $\Sat{M}{\eq/[\num{n}][\num{m}]}$. One verifies that
$g$ is an isomorphism by checking that $\Th{Q}$ proves the relevant
identities for $\Obj{0}$, $\prime$, $+$, $\times$, and $<$.
\end{proof}

A structure~$\Struct{M}$ for $\Lang{L_A}$ is \emph{non-standard}
if it is not isomorphic to~$\Struct{N}$. The elements $x \in
\Domain{M}$ which are equal to $\Value{\num{n}}{M}$ for some $n \in
\Nat$ are called \emph{standard numbers} (of $\Struct{M}$), and those
not, \emph{non-standard numbers}.

If a structure~$\Struct{M}$ for $\Lang{L_A}$ contains a
non-standard number, $\Struct{M}$ is non-standard.

\begin{prop}
$\Th{TA}$ has a countable non-standard model.
\end{prop}

\begin{proof}
Expand $\Lang{L_A}$ by a new constant symbol~$c$ and consider the set of
sentences
\[
\Gamma = \Th{TA} \cup \{\eq/[c][\num{0}], \eq/[c][\num{1}],
\eq/[c][\num{2}], \dots\}
\]
Any model~$\Struct{M^c}$ of~$\Gamma$ would contain an element~$x =
\Assign{c}{M}$ which is non-standard, since $x \neq
\Value{\num{n}}{M}$ for all $n \in \Nat$. Also, obviously,
$\Sat{M^c}{\Th{TA}}$, since $\Th{TA} \subseteq \Gamma$. If we turn
$\Struct{M^c}$ into a structure~$\Struct{M}$ for $\Lang{L_A}$
simply by forgetting about~$c$, its domain still contains the
non-standard~$x$, and also~$\Sat{M}{\Th{TA}}$.

We use the compactness theorem to show that~$\Gamma$ has a model. Consider any finite subset $\Gamma_0 \subseteq
\Gamma$. Suppose $k$ is
the largest number so that $\eq/[c][\num{k}] \in \Gamma_0$. Define
$\Struct{N_k}$ by expanding~$\Struct{N}$ to include the
interpretation~$\Assign{c}{N_k} = k+1$. Then $\Sat{N_k}{\Gamma_0}$,
since $c$ does not occur in~$\Th{TA}$, and $\Value{c}{N_k} = k+1 \neq n$
for $n \le k$. Thus every finite subset of~$\Gamma$ is
satisfiable, so by compactness, $\Gamma$ is satisfiable.
\end{proof}

%%% Block structure of non-standard models of PA

In a non-standard model~$\Struct{M}$ of $\Th{PA}$, the ordering
$\Assign{<}{M}$ is a linear strict order. The element
$\Assign{\Obj{0}}{M}$ is least, every element has a unique successor
and (except for $\Assign{\Obj{0}}{M}$) a unique predecessor, and all
standard elements are less than all non-standard elements.

Every non-standard element~$x$ is contained in a \emph{block}~$[x]$
consisting of all elements reachable from~$x$ by finitely many
applications of successor and predecessor. Each block has no least and
no greatest element. Distinct blocks are disjoint and respect the
ordering: if $x < y$ and $[x] \neq [y]$, then $u < v$ for all $u \in
[x]$ and $v \in [y]$.

\begin{prop}
\label{prop:blocks-dense}
The ordering of the non-standard blocks is dense: if $x < y$ and
$[x] \neq [y]$, then there is a block $[z]$ distinct from both that is
between them.
\end{prop}

\begin{proof}[Proof sketch]
$\Th{PA}$ proves that every element is divisible by~$2$ (possibly with
remainder). If $x$ is non-standard, the ``average'' $z$ of $x$
and~$y$ satisfies $x < z < y$ and $[z] \neq [x]$, $[z] \neq [y]$.
\end{proof}

The non-standard blocks are therefore ordered like the rationals: they
form a countable dense linear ordering without endpoints.  By
Cantor's theorem (Theorem~\ref{thm:cantorQ}), any two such orderings are isomorphic. It follows
that for any two countable non-standard models $\Struct{M}_1$ and
$\Struct{M_2}$ of true arithmetic, their reducts to the language
containing $<$ and $=$ only are isomorphic. However, they need not be
isomorphic in the full language of arithmetic, as there are
non-isomorphic ways to define addition and multiplication.

\begin{defn}[Computable Structure]
\label{defn:computable-structure}
A structure~$\Struct{M}$ for $\Lang{L_A}$ is \emph{computable} iff
  $\Domain{M} = \Nat$ and $\Assign{\prime}{M}$, $\Assign{+}{M}$,
  $\Assign{\times}{M}$ are computable functions and $\Assign{<}{M}$ is
  a decidable relation.
\end{defn}

\begin{thm}[Tennenbaum's Theorem]
\label{thm:tennenbaum}
$\Struct{N}$ is the only computable model of~$\Th{PA}$.
\end{thm}

%%% Overspill (from mod/bas/ove — CONDENSE)

\subsection*{Overspill}

A classical application of the compactness theorem is the overspill
principle.

\begin{thm}[Overspill]
\label{thm:overspill}
If a set $\Gamma$ of sentences has arbitrarily
large finite models, then it has an infinite model.
\end{thm}

\begin{proof}
Expand the language of $\Gamma$ by adding countably many new constants
$c_0$, $c_1$, \dots\ and consider $\Gamma \cup \{c_i \neq c_j :
i \neq j\}$. Since $\Gamma$ has arbitrarily large finite models,
every finite subset of this expanded set is satisfiable. By compactness, the whole set has a
model $\Struct M$ whose domain
must be infinite, since it satisfies all inequalities $c_i \neq c_j$.
\end{proof}

\begin{prop}
\label{prop:inf-not-fo}
There is no sentence $!A$ of any first-order language that is true in
a structure~$\Struct M$ if and only if the domain $\Domain{M}$ of
the structure is infinite.
\end{prop}

\begin{proof}
If there were such a $!A$, its negation $\lnot !A$ would be true in
all and only the finite structures, and it would therefore have
arbitrarily large finite models but lack an infinite model,
contradicting the overspill theorem.
\end{proof}


%% ===================================================================
%% SEM.7: Theorems
%% Sources: fol/syn/ext (KEEP)
%% ===================================================================

\section{Theorems} \label{SEM.7}

\subsection*{The Coincidence Lemma}

Extensionality, sometimes called relevance, can be expressed
informally as follows: the only factors that bear upon the
satisfaction of formula~$!A$ in a structure~$\Struct M$
relative to a variable assignment~$s$, are the size of the
domain and the assignments made by~$\Struct M$ and~$s$ to the
elements of the language that actually appear in~$!A$.

One immediate consequence is that where two
structures~$\Struct M$ and~$\Struct M'$ agree on all the elements
of the language appearing in a sentence~$!A$ and have the same
domain,~$\Struct M$ and~$\Struct M'$ must also agree on whether or not
$!A$ itself is true.

\begin{prop}[Extensionality / Coincidence Lemma] % THM-SEM002
\label{THM-SEM002}
  Let $!A$ be a formula, and $\Struct M_1$ and $\Struct M_2$ be
  structures with $\Domain{M_1} = \Domain{M_2}$, and $s$ a
  variable assignment on $\Domain{M_1} = \Domain{M_2}$.  If
  $\Assign{c}{M_1} = \Assign{c}{M_2}$, $\Assign{R}{M_1}=\Assign{R}{M_2}$,
  and $\Assign{f}{M_1} = \Assign{f}{M_2}$ for every constant symbol~$c$,
  relation symbol~$R$, and function symbol $f$ occurring in~$!A$, then
  $\Sat{M_1}{!A}[s]$ iff $\Sat{M_2}{!A}[s]$.
\end{prop}

\begin{proof}[Proof sketch]
  First prove (by induction on~$t$) that for every term,
  $\Value{t}{M_1}[s] = \Value{t}{M_2}[s]$.  Then prove the proposition
  by induction on~$!A$, making use of the claim just proved for the
  induction basis (where $!A$ is atomic).
\end{proof}

\begin{cor}[Extensionality for Sentences]
\label{cor:extensionality-sent}
  Let $!A$ be a sentence and $\Struct{M_1}$, $\Struct{M_2}$ as in
  the Coincidence Lemma. Then $\Sat{M_1}{!A}$ iff $\Sat{M_2}{!A}$.
\end{cor}

\begin{proof}
Follows from the Coincidence Lemma by Corollary~\ref{cor:sat-sentence}.
\end{proof}

\subsection*{The Substitution Lemma}

Moreover, the value of a term, and whether or not a structure
satisfies a formula, only depend on the values of its subterms.

\begin{prop}[Substitution Lemma for Terms] % THM-SYN003 (terms)
\label{THM-SYN003:terms}
Let $\Struct M$ be a structure, $t$ and $t'$ terms, and $s$ a
variable assignment. Then $\Value{\Subst{t}{t'}{x}}{M}[s] =
\Value{t}{M}[\Subst{s}{\Value{t'}{M}[s]}{x}]$.
\end{prop}

\begin{proof}
By induction on~$t$.
\begin{enumerate}
\item If $t$ is a constant, say, $t\ident c$, then $\Subst{t}{t'}{x} =
  c$, and $\Value{c}{M}[s] = \Assign{c}{M} =
  \Value{c}{M}[\Subst{s}{\Value{t'}{M}[s]}{x}]$.

\item If $t$ is a variable other than~$x$, say, $t \ident y$, then
  $\Subst{t}{t'}{x} = y$, and $\Value{y}{M}[s] =
  \Value{y}{M}[\Subst{s}{\Value{t'}{M}[s]}{x}]$ since
  $\varAssign{s}{\Subst{s}{\Value{t'}{M}[s]}{x}}{x}$.

\item If $t \ident x$, then $\Subst{t}{t'}{x} = t'$. But
  $\Value{x}{M}[\Subst{s}{\Value{t'}{M}[s]}{x}] = \Value{t'}{M}[s]$ by
  definition of~$\Subst{s}{\Value{t'}{M}[s]}{x}$.

\item If $t \ident \Atom{f}{t_1,\dots,t_n}$ then by the
  definition of substitution and the induction hypothesis:
\begin{align*}
  \Value{\Subst{t}{t'}{x}}{M}[s]
& = \Assign{f}{M}(\Value{\Subst{t_1}{t'}{x}}{M}[s], \dots,
    \Value{\Subst{t_n}{t'}{x}}{M}[s])\\
& = \Assign{f}{M}(\Value{t_1}{M}[\Subst{s}{\Value{t'}{M}[s]}{x}], \dots,
   \Value{t_n}{M}[\Subst{s}{\Value{t'}{M}[s]}{x}])\\
& = \Value{t}{M}[\Subst{s}{\Value{t'}{M}[s]}{x}].
\end{align*}
\end{enumerate}
\end{proof}

\begin{prop}[Substitution Lemma for Formulas] % THM-SYN003 (formulas)
\label{THM-SYN003:formulas}
Let $\Struct M$ be
a structure, $!A$ a formula, $t'$~a term, and $s$~a variable
assignment. Then $\Sat{M}{\Subst{!A}{t'}{x}}[s]$ iff
$\Sat{M}{!A}[\Subst{s}{\Value{t'}{M}[s]}{x}]$.
\end{prop}

\begin{proof}
By induction on~$!A$, parallel to the proof for terms.
\end{proof}

The point of the Substitution Lemma is the following. Suppose we have
a term $t$ or a formula~$!A$ and some term~$t'$, and we want to know
the value of $\Subst{t}{t'}{x}$ or whether or not $\Subst{!A}{t'}{x}$
is satisfied in a structure~$\Struct M$ relative to a variable
assignment~$s$. Then we can either perform the substitution first and
then consider the value or satisfaction relative to $\Struct{M}$
and~$s$, or we can first determine the value~$m = \Value{t'}{M}[s]$ of
$t'$ in $\Struct{M}$ relative to~$s$, change the variable assignment
to~$\Subst{s}{m}{x}$ and then consider the value of~$t$ in
$\Struct{M}$ and~$\Subst{s}{m}{x}$, or whether
$\Sat{M}{!A}[\Subst{s}{m}{x}]$. The Substitution Lemma guarantees
that the answer will be the same, whichever way we do it.
   % CH-SEM: Semantics
\chapter{Deduction} \label{ch:ded}

%% ===================================================================
%% DED.1: Generic Proof Theory
%% Sources: axd/rul (GENERALIZE), axd/ptn (GENERALIZE+ABSORB),
%%          axd/prv (ABSORB), axd/ppr (CONDENSE), axd/qpr (CONDENSE),
%%          ntd/rul (EXTRACT), seq/rul (EXTRACT), seq/srl (EXTRACT),
%%          com/mcs (CONDENSE)
%% ===================================================================

\section{Generic Proof Theory} \label{DED.1}

This section develops proof-theoretic concepts at the \emph{generic}
level, independent of any particular proof system.  Each concept
defined here---axiom schema, rule of inference, derivation,
provability, consistency, and so on---is instantiated differently by
the concrete proof systems presented in DED.2--DED.5.  The structural
properties established below (reflexivity, monotonicity, transitivity,
compactness) hold in each system, though the proofs differ in
detail. Canonical proofs are given here for the Hilbert-style
(axiomatic) case; remarks indicate how they adapt to other systems.

%%% -----------------------------------------------------------------
%%% DED.1.1  Axiom Schemas and Rules of Inference
%%% -----------------------------------------------------------------

\subsection{Axiom Schemas and Rules of Inference}

\begin{defn}[Axiom Schema] % PRIM-DED001
\label{PRIM-DED001}
An \emph{axiom schema} is a schematic expression involving metavariables
that stands for a (typically infinite) collection of formulas: every
formula obtained by uniformly replacing the metavariables with concrete
formulas is an \emph{instance} of the schema and counts as a logical
axiom.
\end{defn}

For instance, the schema $!A \lif (!B \lif !A)$ generates infinitely
many axioms---one for each choice of formulas~$!A$ and~$!B$.  Different
proof systems employ different collections of axiom schemas (or none at
all, as in the case of natural deduction).

\begin{defn}[Non-Logical Axiom] % PRIM-DED002
\label{PRIM-DED002}
A \emph{non-logical axiom} is a sentence adopted as a premise of
a theory but not derivable from the logical axioms alone.  A set of
non-logical axioms is sometimes called a set of \emph{proper axioms}
or simply the \emph{axioms of the theory}.  A theory is
\emph{axiomatized} by a set~$\Gamma_0$ of non-logical axioms when the
theory consists of all sentences that follow (semantically or
proof-theoretically) from~$\Gamma_0$.
\end{defn}

\begin{defn}[Rule of Inference] % PRIM-DED003
\label{PRIM-DED003}
A \emph{rule of inference} gives a sufficient condition for what counts
as a correct inference step in a derivation from a set of
assumptions~$\Gamma$.  More precisely, a rule specifies one or more
\emph{premises} and a \emph{conclusion}; a step is correct when each
premise either appears earlier in the derivation, is an axiom, or is
an element of~$\Gamma$.
\end{defn}

The simplest and most ubiquitous rule of inference is \emph{modus ponens}:

\begin{quote}
If $!B \lif !A$ and $!B$ both occur (as axioms, assumptions, or
earlier conclusions) in a derivation, then $!A$ is a correct inference
step.
\end{quote}

What counts as a correct derivation depends on which rules of inference
are admitted and on what is taken as an axiom.  Different proof systems
make different choices:
\begin{itemize}
\item \emph{Axiomatic (Hilbert-style) deduction} (see DED.2) uses
  many axiom schemas and few rules (typically just modus ponens and
  a quantifier rule).
\item \emph{Natural deduction} (see DED.3) uses no logical axioms,
  but has introduction and elimination rules for each connective and
  quantifier.
\item \emph{Sequent calculus} (see DED.4) operates on sequents and
  employs left/right introduction rules together with structural rules.
\item \emph{Tableaux} (see DED.5) work by attempted refutation,
  applying branch-extension rules to signed formulas.
\end{itemize}

%%% -----------------------------------------------------------------
%%% DED.1.2  Proof Systems and Derivations
%%% -----------------------------------------------------------------

\subsection{Proof Systems and Derivations}

\begin{defn}[Proof System] % PRIM-DED004
\label{PRIM-DED004}
A \emph{proof system} is a specification of:
\begin{enumerate}
\item A set of logical axioms (possibly empty, possibly given by schemas);
\item A set of rules of inference;
\item A definition of what constitutes a \emph{derivation}.
\end{enumerate}
A proof system determines a derivability relation $\Gamma \Proves !A$
on formulas.
\end{defn}

\begin{defn}[Derivation] % PRIM-DED005
\label{PRIM-DED005}
Let $\Gamma$ be a set of formulas of~$\Lang L$.  A \emph{derivation}
from~$\Gamma$ is a finite combinatorial object---a sequence of
formulas, a tree of formulas, or a tree of sequents, depending on the
proof system---in which every step is justified either as:
\begin{enumerate}
\item an element of~$\Gamma$ (an assumption); or
\item an instance of a logical axiom; or
\item the conclusion of a correct application of a rule of inference
  to earlier steps.
\end{enumerate}
A derivation \emph{derives} its final formula (its last element, its
root, or the succedent of its root sequent, depending on the system).
\end{defn}

\begin{rem}
The shape of derivations differs across proof systems.  In axiomatic
deduction, a derivation is a finite \emph{sequence} of formulas.  In
natural deduction and the sequent calculus, a derivation is a finite
\emph{tree}.  In tableaux, a derivation is a finitely branching tree
of signed formulas.  Despite these differences, the abstract notion
of derivability is uniform: $\Gamma \Proves !A$ means that there
exists a derivation of~$!A$ from~$\Gamma$ in the given system.
\end{rem}

%%% -----------------------------------------------------------------
%%% DED.1.3  Sequents and Structural Rules
%%% -----------------------------------------------------------------

\subsection{Sequents and Structural Rules}

\begin{defn}[Sequent] % PRIM-DED008
\label{PRIM-DED008}
A \emph{sequent} is an expression of the form
\[
\Gamma \Sequent \Delta
\]
where $\Gamma$ and $\Delta$ are finite (possibly empty) sequences of
sentences of the language~$\Lang L$.  $\Gamma$ is called the
\emph{antecedent} and $\Delta$ is called the \emph{succedent}.
\end{defn}

The intuitive reading of a sequent $\Gamma \Sequent \Delta$ is: if all
of the sentences in the antecedent hold, then at least one of the
sentences in the succedent holds.  That is, if $\Gamma = \tuple{!A_1,
\dots, !A_m}$ and $\Delta = \tuple{!B_1, \dots, !B_n}$, then $\Gamma
\Sequent \Delta$ holds iff
\[
(!A_1 \land \cdots \land !A_m) \lif (!B_1 \lor \cdots \lor !B_n)
\]
holds.  When $\Gamma$ is empty, $\Sequent \Delta$ asserts that
$!B_1 \lor \dots \lor !B_n$ holds.  When $\Delta$ is empty, $\Gamma
\Sequent$ asserts that $\lnot(!A_1 \land \dots \land !A_m)$.

\begin{defn}[Structural Rules] % PRIM-DED007
\label{PRIM-DED007}
The \emph{structural rules} of a proof system govern the manipulation
of the context (the set or sequence of assumptions or side formulas)
without introducing or eliminating any logical connective.  The
principal structural rules are:
\begin{enumerate}
\item \textbf{Weakening} ($\Weakening$): one may add a formula to the
  context without affecting derivability.
  \begin{defish}
  \Axiom$ \Gamma \fCenter \Delta $
  \RightLabel{\LeftR{\Weakening}}
  \UnaryInf$ !A, \Gamma \fCenter \Delta$
  \DisplayProof
  \hfill
  \Axiom$ \Gamma \fCenter \Delta$
  \RightLabel{\RightR{\Weakening}}
  \UnaryInf$ \Gamma \fCenter \Delta, !A$
  \DisplayProof
  \end{defish}

\item \textbf{Contraction} ($\Contraction$): two copies of the same
  formula may be collapsed into one.
  \begin{defish}
  \Axiom$ !A, !A, \Gamma \fCenter \Delta $
  \RightLabel{\LeftR{\Contraction}}
  \UnaryInf$ !A, \Gamma \fCenter \Delta$
  \DisplayProof
  \hfill
  \Axiom$ \Gamma \fCenter \Delta, !A, !A$
  \RightLabel{\RightR{\Contraction}}
  \UnaryInf$ \Gamma \fCenter \Delta, !A$
  \DisplayProof
  \end{defish}

\item \textbf{Exchange} ($\Exchange$): the order of formulas in the
  context may be permuted.
  \begin{defish}
  \Axiom$ \Gamma, !A, !B, \Pi \fCenter \Delta $
  \RightLabel{\LeftR{\Exchange}}
  \UnaryInf$ \Gamma, !B, !A, \Pi \fCenter \Delta$
  \DisplayProof
  \hfill
  \Axiom$ \Gamma \fCenter \Delta, !A, !B, \Lambda$
  \RightLabel{\RightR{\Exchange}}
  \UnaryInf$ \Gamma \fCenter \Delta, !B, !A, \Lambda$
  \DisplayProof
  \end{defish}

\item \textbf{Cut} ($\Cut$): if a formula can be derived on the right
  and consumed on the left, it may be eliminated.
  \begin{defish}
  \[
  \Axiom$ \Gamma \fCenter \Delta, !A$
  \Axiom$ !A, \Pi \fCenter \Lambda $
  \RightLabel{\Cut}
  \BinaryInf$ \Gamma, \Pi \fCenter \Delta, \Lambda$
  \DisplayProof
  \]
  \end{defish}
\end{enumerate}
\end{defn}

\begin{rem}
The structural rules are stated here in their sequent-calculus form,
but analogous phenomena occur in every proof system.  In axiomatic
deduction, weakening corresponds to monotonicity of derivability
(adding unused hypotheses), and cut corresponds to transitivity
(chaining derivations).  In natural deduction, weakening is built into
the assumption mechanism, contraction is implicit in re-use of
assumptions, and cut corresponds to substituting a derivation for an
assumption.  Substructural logics arise by restricting or removing
structural rules: e.g., linear logic drops weakening and contraction;
relevant logic restricts weakening.
\end{rem}

%%% -----------------------------------------------------------------
%%% DED.1.4  Assumption Discharge
%%% -----------------------------------------------------------------

\subsection{Assumption Discharge}

\begin{defn}[Assumption Discharge] % PRIM-DED009
\label{PRIM-DED009}
In natural deduction, an \emph{assumption} is any sentence occupying a
topmost (leaf) position in a derivation tree. Certain rules of
inference---notably $\Intro{\lif}$, $\Intro{\lnot}$, $\Elim{\lor}$,
and $\Elim{\lexists}$---permit one to \emph{discharge} assumptions:
the discharged assumption is cancelled and no longer counts among the
open (undischarged) assumptions of the derivation. The label notation
$\Discharge{!A}{n}$ indicates that the assumption~$!A$ bearing
label~$n$ is discharged by the corresponding inference step.
\end{defn}

Discharging is a permission, not a requirement: one may apply a
discharging rule even when the assumption to be discharged does not
actually occur in the derivation above.  The set of
\emph{undischarged} assumptions of a derivation is the set of
assumptions that have not been cancelled by any rule application.  A
derivation with no undischarged assumptions is a \emph{proof} (of a
theorem).

%%% -----------------------------------------------------------------
%%% DED.1.5  Provability and Theorems
%%% -----------------------------------------------------------------

\subsection{Provability and Theorems}

\begin{defn}[Provability] % PRIM-DED006
\label{PRIM-DED006}
A formula~$!A$ is \emph{derivable} from $\Gamma$, written
$\Gamma \Proves !A$, if there is a derivation from~$\Gamma$ ending
in~$!A$ (in whatever proof system is in force).
\end{defn}

\begin{defn}[Theorem] % PRIM-DED010
\label{PRIM-DED010}
A formula~$!A$ is a \emph{theorem} if there is a derivation
of~$!A$ from the empty set.  We write $\Proves !A$ if $!A$ is a
theorem and $\Proves/ !A$ if it is not.
\end{defn}

%%% -----------------------------------------------------------------
%%% DED.1.6  Structural Properties of Derivability
%%% -----------------------------------------------------------------

\subsection{Structural Properties of Derivability}

Just as we have defined semantic notions (validity, entailment,
satisfiability), we now establish corresponding \emph{proof-theoretic
properties}. These are not defined by appeal to satisfaction in
structures, but by appeal to the derivability or non-derivability of
formulas. It was an important discovery, the content of the
\emph{soundness} and \emph{completeness} theorems (see
CP-001, Soundness, \S\ref{META.1} and CP-002, Completeness, \S\ref{META.2}), that these notions coincide.

\begin{prop}[Reflexivity] % structural property
\label{DED-prop:reflexivity}
If $!A \in \Gamma$, then $\Gamma \Proves !A$.
\end{prop}

\begin{proof}
The formula~$!A$ by itself constitutes a (trivial) derivation of~$!A$
from~$\Gamma$: in axiomatic deduction, it is a one-element sequence
whose sole entry is justified as an element of~$\Gamma$; in natural
deduction, it is a single-node tree (an assumption); in the sequent
calculus, the initial sequent $!A \Sequent !A$ suffices.
\end{proof}

\begin{prop}[Monotonicity] % structural property
\label{DED-prop:monotonicity}
If $\Gamma \subseteq \Delta$ and $\Gamma \Proves !A$, then $\Delta
\Proves !A$.
\end{prop}

\begin{proof}
Any derivation of~$!A$ from~$\Gamma$ is also a derivation of~$!A$
from~$\Delta$, since every element of~$\Gamma$ used in the derivation
is also an element of~$\Delta$.
\end{proof}

\begin{prop}[Transitivity] % structural property
\label{DED-prop:transitivity}
If $\Gamma \Proves !A$ and $\{!A\} \cup \Delta \Proves
!B$, then $\Gamma \cup \Delta \Proves !B$.
\end{prop}

\begin{proof}
Suppose $\{!A\} \cup \Delta \Proves !B$.  Then there is
a derivation $!B_1, \dots, !B_l = !B$ from~$\{!A\} \cup
\Delta$. Some of the steps in that derivation will be correct
because of a rule which refers to a prior line~$!B_i = !A$. By
hypothesis, there is a derivation of~$!A$ from~$\Gamma$, i.e.,
a derivation~$!A_1, \dots, !A_k = !A$ where every $!A_i$ is an
axiom, an element of~$\Gamma$, or correct by a rule of
inference. Now consider the sequence
\[
!A_1, \dots, !A_k = !A, !B_1, \dots, !B_l = !B.
\]
This is a correct derivation of~$!B$ from $\Gamma \cup \Delta$
since every $!B_i = !A$ is now justified by the same rule which
justifies~$!A_k = !A$.
\end{proof}

\begin{rem}
The proof above is stated for axiomatic deduction, where derivations
are sequences and transitivity amounts to concatenation. In natural
deduction, transitivity is realized by substituting the derivation
of~$!A$ for the assumption~$!A$ in the derivation of~$!B$. In the
sequent calculus, transitivity is an instance of the cut rule.  Each
proof system instantiates these concepts differently; see DED.2--DED.5.
\end{rem}

Note that transitivity implies in particular: if $\Gamma \Proves !A$
and $!A \Proves !B$, then $\Gamma \Proves !B$.  It follows also that
if $!A_1, \dots, !A_n \Proves !B$ and $\Gamma \Proves !A_i$ for
each~$i$, then $\Gamma \Proves !B$.

\begin{prop}[Inconsistency Characterization] % structural property
\label{DED-prop:incons}
$\Gamma$ is inconsistent iff $\Gamma \Proves !A$ for every~$!A$.
\end{prop}

\begin{proof}
If $\Gamma$ is inconsistent, then $\Gamma \Proves \lfalse$.  From
$\lfalse$ any formula~$!A$ follows (by the logical axiom or rule
$\lfalse \lif !A$ in all standard proof systems).  Conversely, if
$\Gamma \Proves !A$ for every~$!A$, then in particular $\Gamma
\Proves \lfalse$, so $\Gamma$ is inconsistent.
\end{proof}

\begin{prop}[Compactness] % structural property
\label{DED-prop:proves-compact}
\begin{enumerate}
\item If $\Gamma \Proves !A$ then there is a finite subset $\Gamma_0
  \subseteq \Gamma$ such that $\Gamma_0 \Proves !A$.
\item If every finite subset of~$\Gamma$ is consistent, then $\Gamma$
  is consistent.
\end{enumerate}
\end{prop}

\begin{proof}
\begin{enumerate}
  \item If $\Gamma \Proves !A$, then there is a derivation---a finite
    object. Let $\Gamma_0$ be the set of elements of~$\Gamma$ that
    actually appear in the derivation.  Since the derivation is finite,
    $\Gamma_0$ is finite, and the derivation is equally a derivation
    from~$\Gamma_0$.  So $\Gamma_0 \Proves !A$.
  \item This is the contrapositive of~(1) for the special case $!A
    \ident \lfalse$.
\end{enumerate}
\end{proof}

%%% -----------------------------------------------------------------
%%% DED.1.7  Consistency and Derivability
%%% -----------------------------------------------------------------

\subsection{Consistency and Derivability}

We now establish a number of properties of the derivability relation.
They are independently interesting, and each plays a role in the proof
of the completeness theorem (see CP-002, Completeness, \S\ref{META.2}).

\begin{defn}[Syntactic Consistency] % DEF-DED001
\label{DEF-DED001}
A set $\Gamma$ of formulas is \emph{consistent} if and only if
$\Gamma \Proves/ \lfalse$; it is \emph{inconsistent} otherwise.
\end{defn}

\begin{prop} % provability-contr
\label{DED-prop:provability-contr}
If $\Gamma \Proves !A$ and $\Gamma \cup \{!A\}$ is inconsistent,
then $\Gamma$ is inconsistent.
\end{prop}

\begin{proof}
If $\Gamma \cup \{!A\}$ is inconsistent, then $\Gamma \cup \{!A\}
\Proves \lfalse$.  By reflexivity, $\Gamma \Proves !B$ for every
$!B \in \Gamma$.  Since also $\Gamma \Proves !A$ by hypothesis,
$\Gamma \Proves !B$ for every $!B \in \Gamma \cup \{!A\}$.  By
transitivity, $\Gamma \Proves \lfalse$, i.e., $\Gamma$ is
inconsistent.
\end{proof}

\begin{prop} % prov-incons
\label{DED-prop:prov-incons}
$\Gamma \Proves !A$ iff $\Gamma \cup \{\lnot !A\}$ is inconsistent.
\end{prop}

\begin{proof}
First suppose $\Gamma \Proves !A$.  Then $\Gamma \cup \{\lnot !A\}
\Proves !A$ by monotonicity, and $\Gamma \cup \{\lnot !A\} \Proves
\lnot !A$ by reflexivity.  Since from~$!A$ and~$\lnot !A$ together
we can derive~$\lfalse$ (via the axiom or rule $\lnot !A \lif (!A
\lif \lfalse)$ and modus ponens, in all standard proof systems),
$\Gamma \cup \{\lnot !A\} \Proves \lfalse$.

Now assume $\Gamma \cup \{\lnot !A\}$ is inconsistent.  By the
deduction theorem (THM-DED001, proved for each system in \S\ref{DED.7} below), $\Gamma \Proves \lnot
!A \lif \lfalse$.  Since $(\lnot !A \lif \lfalse) \lif \lnot\lnot !A$
and $\lnot\lnot !A \lif !A$ are derivable, we obtain $\Gamma \Proves
!A$ by modus ponens.
\end{proof}

\begin{prop} % explicit-inc
\label{DED-prop:explicit-inc}
If $\Gamma \Proves !A$ and $\lnot !A \in \Gamma$, then $\Gamma$ is
inconsistent.
\end{prop}

\begin{proof}
Since $\lnot !A \in \Gamma$, by reflexivity $\Gamma \Proves \lnot
!A$.  Together with $\Gamma \Proves !A$ and the derivability of
$\lnot !A \lif (!A \lif \lfalse)$, two applications of modus ponens
yield $\Gamma \Proves \lfalse$.
\end{proof}

\begin{prop} % provability-exhaustive
\label{DED-prop:provability-exhaustive}
If $\Gamma \cup \{!A\}$ and $\Gamma \cup \{\lnot !A\}$ are both
inconsistent, then $\Gamma$ is inconsistent.
\end{prop}

\begin{proof}
By \cref{DED-prop:prov-incons}, $\Gamma \cup \{\lnot !A\}$ inconsistent
implies $\Gamma \Proves !A$.  Since $\Gamma \cup \{!A\}$ is also
inconsistent, \cref{DED-prop:provability-contr} gives that $\Gamma$
is inconsistent.
\end{proof}

%%% -----------------------------------------------------------------
%%% DED.1.8  Derived and Admissible Rules
%%% -----------------------------------------------------------------

\subsection{Derived and Admissible Rules}

\begin{defn}[Derived Rule] % DEF-DED009
\label{DEF-DED009}
A rule $\frac{!A_1 \cdots !A_n}{!B}$ is a \emph{derived rule} of a
proof system if there exists a derivation of~$!B$ from assumptions
$!A_1, \ldots, !A_n$ using only the primitive rules. Derived rules
serve as shortcuts that do not extend the system's deductive power.
\end{defn}

\begin{defn}[Admissible Rule] % DEF-DED010
\label{DEF-DED010}
A rule $\frac{!A_1 \cdots !A_n}{!B}$ is \emph{admissible} if:
whenever $\Proves !A_1, \ldots, \Proves !A_n$ are all provable (as
theorems), then $\Proves !B$ is also provable. Unlike derived rules,
admissible rules need not yield a derivation using the premises
directly.
\end{defn}

\begin{rem}
Every derived rule is admissible, but not conversely.  For example,
the cut rule is admissible in the sequent calculus~$\Log{LK}$ (this is
the content of the cut-elimination theorem, see CP-010, Cut
Elimination, \S\ref{DED.4}), but in a system with cut it is a primitive
(hence trivially derived) rule.  The distinction matters for proof
search: derived rules can always be ``compiled away'' by inlining
their justifying derivation, while admissible rules may require a
global transformation of the proof.
\end{rem}

%%% -----------------------------------------------------------------
%%% DED.1.9  Deductive Closure and Conservative Extension
%%% -----------------------------------------------------------------

\subsection{Deductive Closure and Conservative Extension}

\begin{defn}[Deductive Closure] % DEF-DED003
\label{DEF-DED003}
The \emph{deductive closure} of a set of formulas~$\Gamma$ is the set
\[
\Thms{\Gamma} = \Setabs{!A}{\Gamma \Proves !A}.
\]
A set $\Gamma$ is \emph{deductively closed} if $\Gamma = \Thms{\Gamma}$.
\end{defn}

\begin{defn}[Conservative Extension] % DEF-DED004
\label{DEF-DED004}
A theory $T'$ in language $\Lang{L'} \supseteq \Lang{L}$ is a
\emph{conservative extension} of a theory $T$ in $\Lang{L}$ if for
every $\Lang{L}$-sentence $!A$: $T' \Proves !A$ implies $T \Proves
!A$.
\end{defn}

\begin{rem}
Conservative extensions are important in mathematical logic because
they guarantee that expanding a theory with new symbols and axioms does
not prove new theorems in the original language.  This notion appears
throughout the metatheory: definitional extensions are conservative,
and the method of Henkin constants used in the completeness proof (see
\S\ref{META.3}) produces a conservative extension of the original theory.
\end{rem}

%%% -----------------------------------------------------------------
%%% DED.1.10  Maximally Consistent Sets
%%% -----------------------------------------------------------------

\subsection{Maximally Consistent Sets}

\begin{defn}[Maximally Consistent Set] % DEF-DED002
\label{DEF-DED002}
A set~$\Gamma$ of sentences is \emph{maximally consistent} iff
\begin{enumerate}
\item $\Gamma$ is consistent, and
\item if $\Gamma \subsetneq \Gamma'$, then $\Gamma'$ is inconsistent.
\end{enumerate}
Equivalently, $\Gamma$ is maximally consistent iff $\Gamma$ is
consistent and for every sentence~$!A$: if $\Gamma \cup \{!A\}$ is
consistent, then $!A \in \Gamma$.
\end{defn}

Maximally consistent sets are central to the completeness proof.
Every consistent set of sentences is contained in a maximally
consistent set (by Lindenbaum's Lemma, see THM-DED005, \S\ref{DED.7}).
A maximally consistent set~$\Gamma$ contains, for each sentence~$!A$,
either $!A$ or~$\lnot !A$.  This property is what allows us to
construct a structure satisfying~$\Gamma$ in the completeness argument.

\begin{prop}
\label{DED-prop:mcs}
Suppose $\Gamma$ is maximally consistent. Then:
\begin{enumerate}
\item \label{DED-prop:mcs-prov-in} If $\Gamma \Proves !A$, then
  $!A \in \Gamma$.
\item \label{DED-prop:mcs-either-or} For any $!A$, either $!A \in
  \Gamma$ or $\lnot !A \in \Gamma$.
\item $(!A \land !B) \in \Gamma$ iff both $!A \in \Gamma$ and
  $!B \in \Gamma$.
\item $(!A \lor !B) \in \Gamma$ iff either $!A \in \Gamma$ or
  $!B \in \Gamma$.
\item $(!A \lif !B) \in \Gamma$ iff either $!A \notin \Gamma$ or
  $!B \in \Gamma$.
\end{enumerate}
\end{prop}

\begin{proof}
Let $\Gamma$ be maximally consistent throughout.
\begin{enumerate}
\item If $\Gamma \Proves !A$ and $!A \notin \Gamma$, then since
  $\Gamma$ is maximally consistent, $\Gamma \cup \{!A\}$ is
  inconsistent. By \cref{DED-prop:provability-contr}, $\Gamma$ is
  inconsistent, contradicting the hypothesis.  Hence $!A \in \Gamma$.

\item Suppose both $!A \notin \Gamma$ and $\lnot !A \notin \Gamma$.
  Then $\Gamma \cup \{!A\}$ and $\Gamma \cup \{\lnot !A\}$ are both
  inconsistent. By \cref{DED-prop:provability-exhaustive}, $\Gamma$ is
  inconsistent---a contradiction.

\item For the forward direction: if $(!A \land !B) \in \Gamma$ then
  $\Gamma \Proves !A \land !B$.  Since $!A \land !B \Proves !A$ and
  $!A \land !B \Proves !B$, we get $\Gamma \Proves !A$ and $\Gamma
  \Proves !B$ by transitivity. By~(1), $!A \in \Gamma$ and $!B \in
  \Gamma$.  For the reverse: if $!A, !B \in \Gamma$ then $\Gamma
  \Proves !A$ and $\Gamma \Proves !B$, so $\Gamma \Proves !A \land
  !B$, and by~(1), $(!A \land !B) \in \Gamma$.

\item Analogous, using the derivability properties of~$\lor$.

\item Analogous, using the derivability properties of~$\lif$.
\end{enumerate}
\end{proof}

%%% -----------------------------------------------------------------
%%% DED.1.11  Generic Connective and Quantifier Derivability
%%% -----------------------------------------------------------------

\subsection{Generic Connective and Quantifier Derivability}

The following propositions state basic derivability facts for the
propositional connectives and quantifiers that hold in all standard
proof systems.  They are used in the proof of the completeness theorem.
System-specific derivations establishing these facts are given in
DED.2--DED.5.

\begin{prop} % generic connective derivability
\label{DED-prop:provability-land}
\begin{enumerate}
\item Both $!A \land !B \Proves !A$ and $!A \land !B \Proves !B$.
\item $!A, !B \Proves !A \land !B$.
\end{enumerate}
\end{prop}

\begin{prop}
\label{DED-prop:provability-lor}
\begin{enumerate}
\item $!A \lor !B, \lnot !A, \lnot !B$ is inconsistent.
\item Both $!A \Proves !A \lor !B$ and $!B \Proves !A \lor !B$.
\end{enumerate}
\end{prop}

\begin{prop}
\label{DED-prop:provability-lif}
\begin{enumerate}
\item $!A, !A \lif !B \Proves !B$.
\item Both $\lnot !A \Proves !A \lif !B$ and $!B \Proves !A \lif !B$.
\end{enumerate}
\end{prop}

\begin{proof}[Proof sketch]
(1) In axiomatic deduction, modus ponens applied to $!A$ and the
axiom $!A \lif !B$ immediately yields $!B$.  In natural deduction,
$\lif$-elimination serves the same role.  In the sequent calculus,
Left-$\lif$ gives $!B$ from $!A \lif !B$ and $!A$.
(2) From $!B$ we derive $!A \lif !B$ by axiom~\eqref{ax:lif1} (or
$\lif$-introduction / Right-$\lif$); from $\lnot !A$ we derive
$!A \lif !B$ by axiom~\eqref{ax:lnot2} (or analogous rules).
Detailed derivations appear in DED.2 (axiomatic), DED.3 (natural
deduction), DED.4 (sequent calculus), and DED.5 (tableaux).
\end{proof}

\begin{thm}[Strong Generalization]
\label{DED-thm:strong-generalization}
If $c$ is a constant symbol not occurring in $\Gamma$ or $!A(x)$ and
$\Gamma \Proves !A(c)$, then $\Gamma \Proves \lforall[x][!A(x)]$.
\end{thm}

\begin{proof}
By the deduction theorem (THM-DED001), $\Gamma \Proves \ltrue \lif !A(c)$.  Since
$c$ does not occur in $\Gamma$ or~$\ltrue$, the quantifier rule (or
its equivalent in other systems) gives $\Gamma \Proves \ltrue \lif
\lforall[x][!A(x)]$.  By the deduction theorem again, $\Gamma \Proves
\lforall[x][!A(x)]$.
\end{proof}

\begin{prop}
\label{DED-prop:provability-quantifiers}
\begin{enumerate}
\item $!A(t) \Proves \lexists[x][!A(x)]$.
\item $\lforall[x][!A(x)] \Proves !A(t)$.
\end{enumerate}
\end{prop}

\begin{proof}
Both follow from the quantifier axioms (or quantifier rules, in
natural deduction and the sequent calculus) and the deduction theorem
(or the corresponding introduction/elimination rules).  For detailed
derivations, see DED.2--DED.5.
\end{proof}


%% ===================================================================
%% DED.2: Axiomatic (Hilbert) Systems
%% Sources: prf/axd (CONDENSE), axd/prp (KEEP), axd/qua (KEEP),
%%          axd/ded (KEEP+MERGE ddq), axd/ppr (CONDENSE),
%%          axd/qpr (CONDENSE), axd/sou (KEEP)
%% ===================================================================

\section{Axiomatic (Hilbert) Systems} \label{DED.2}

The axiomatic system instantiates the generic proof-theoretic framework
of \S\ref{DED.1} as follows.  A derivation (\ref{PRIM-DED005}) is a
finite \emph{sequence} of sentences; the system has many axiom schemas
(\ref{PRIM-DED001}) and only two rules of inference
(\ref{PRIM-DED003}): modus ponens and a quantifier rule.  Provability
(\ref{PRIM-DED006}) and consistency (\ref{DEF-DED001}) are defined
exactly as in \S\ref{DED.1}; all structural properties established
there (reflexivity, monotonicity, transitivity, compactness) hold with
derivations understood as sequences.

Axiomatic derivation systems were introduced by Gottlob Frege in 1879,
refined by Whitehead and Russell, and perfected by Hilbert and his
students in the 1920s.  Because derivations have a very simple
structure, it is relatively easy to prove things \emph{about} them
(e.g., the deduction theorem), though finding derivations in practice
is difficult.

%%% -----------------------------------------------------------------
%%% DED.2.1  Propositional Axioms and Modus Ponens
%%% -----------------------------------------------------------------

\subsection{Propositional Axioms and Modus Ponens}

\begin{defn}[Axiomatic System] % DEF-DED005
\label{DEF-DED005}
The \emph{axiomatic (Hilbert-style) deduction system} is defined by
the propositional axiom schemas of \ref{AX-DED003} below, the
quantifier axioms and quantifier rule of \S\ref{DED.2}.2, and the
rule of modus ponens (\ref{AX-DED001}).  A derivation from a set of
sentences~$\Gamma$ is a finite sequence $!B_1, \dots, !B_n$ in which
every~$!B_i$ is either
\begin{enumerate}
\item an element of~$\Gamma$, or
\item an instance of one of the axiom schemas, or
\item justified by a rule of inference applied to earlier items in the
  sequence.
\end{enumerate}
We write $\Gamma \Proves !A$ if there exists such a sequence ending
in~$!A$.
\end{defn}

\begin{defn}[Propositional Axioms] % AX-DED003 (propositional part)
\label{AX-DED003}
The set $\PAx$ of \emph{axioms} for the propositional connectives
comprises all formulas of the following forms:
\begin{align}
  & (!A \land !B) \lif !A \tag{A1}\label{ax:land1}\\
  & (!A \land !B) \lif !B \tag{A2}\label{ax:land2}\\
  & !A \lif (!B \lif (!A \land !B)) \tag{A3}\label{ax:land3}\\
  & !A \lif (!A \lor !B) \tag{A4}\label{ax:lor1}\\
  & !A \lif (!B \lor !A) \tag{A5}\label{ax:lor2}\\
  & (!A \lif !C) \lif ((!B \lif !C) \lif ((!A \lor !B) \lif !C))
    \tag{A6}\label{ax:lor3}\\
  & !A \lif (!B \lif !A) \tag{A7}\label{ax:lif1}\\
  & (!A \lif (!B \lif !C)) \lif ((!A \lif !B) \lif (!A \lif !C))
    \tag{A8}\label{ax:lif2}\\
  & (!A \lif !B) \lif ((!A \lif \lnot !B) \lif \lnot !A)
    \tag{A9}\label{ax:lnot1}\\
  & \lnot !A \lif (!A \lif !B) \tag{A10}\label{ax:lnot2}\\
  & \ltrue \tag{A11}\label{ax:ltrue}\\
  & \lfalse \lif !A \tag{A12}\label{ax:lfalse1}\\
  & (!A \lif \lfalse) \lif \lnot !A \tag{A13}\label{ax:lfalse2}\\
  & \lnot\lnot !A \lif !A \tag{A14}\label{ax:dne}
\end{align}
\end{defn}

\begin{defn}[Modus Ponens] % AX-DED001
\label{AX-DED001}
If $!B$ and $!B \lif !A$ already occur in a derivation, then $!A$ is
a correct inference step. We abbreviate this rule as~$\MP$.
\end{defn}

%%% -----------------------------------------------------------------
%%% DED.2.2  Quantifier Axioms and Rules
%%% -----------------------------------------------------------------

\subsection{Quantifier Axioms and Rules}

\begin{defn}[Quantifier Axioms] % AX-DED002 (part of AX-DED003)
\label{AX-DED002}
The \emph{axioms} governing quantifiers are all instances of:
\begin{align}
  & \lforall[x][!B] \lif !B(t), \tag{Q1}\label{ax:q1}\\
  & !B(t) \lif \lexists[x][!B], \tag{Q2}\label{ax:q2}
\end{align}
for any closed term~$t$.
\end{defn}

\begin{defn}[Quantifier Rule] % AX-DED002 (continued)
The quantifier rule $\QR$ has two forms:
\begin{enumerate}
\item If $!B \lif !A(a)$ already occurs in the derivation and $a$
  does not occur in~$\Gamma$ or~$!B$, then $!B \lif
  \lforall[x][!A(x)]$ is a correct inference step.
\item If $!A(a) \lif !B$ already occurs in the derivation and $a$
  does not occur in~$\Gamma$ or~$!B$, then $\lexists[x][!A(x)] \lif
  !B$ is a correct inference step.
\end{enumerate}
\end{defn}

%%% -----------------------------------------------------------------
%%% DED.2.3  The Deduction Theorem
%%% -----------------------------------------------------------------

\subsection{The Deduction Theorem (Axiomatic Proof)}

The deduction theorem (\ref{THM-DED001}) is the central metatheorem for
axiomatic systems.  Below we give its full proof, including the
quantifier-rule case.

\begin{prop}[Meta-Modus Ponens]
\label{DED2-prop:mp}
If $\Gamma \Proves !A$ and $\Gamma \Proves !A \lif !B$, then
$\Gamma \Proves !B$.
\end{prop}

\begin{proof}
We have that $\{!A, !A \lif !B\} \Proves !B$:
\begin{derivation}
  1. & $!A$ & Hyp.\\
  2. & $!A \lif !B$ & Hyp.\\
  3. & $!B$ & 1, 2, \MP
\end{derivation}
By transitivity (\ref{DED-prop:transitivity}), $\Gamma \Proves !B$.
\end{proof}

\begin{thm}[Deduction Theorem --- Axiomatic Proof] % THM-DED001, CP-009
\label{DED2-thm:deduction-thm}
$\Gamma \cup \{!A\} \Proves !B$ if and only if $\Gamma \Proves !A
\lif !B$.
\end{thm}

\begin{proof}
The ``if'' direction is immediate: if $\Gamma \Proves !A \lif !B$
then $\Gamma \cup \{!A\} \Proves !A \lif !B$ by monotonicity
(\ref{DED-prop:monotonicity}), $\Gamma \cup \{!A\} \Proves !A$ by
reflexivity (\ref{DED-prop:reflexivity}), and
\cref{DED2-prop:mp} gives $\Gamma \cup \{!A\} \Proves !B$.

For the ``only if'' direction, we proceed by induction on the length
of the derivation of~$!B$ from~$\Gamma \cup \{!A\}$.

\emph{Base case.}  A derivation of length~$1$ consists of $!B$ alone,
justified because $!B \in \Gamma \cup \{!A\}$ or $!B$ is an axiom.
If $!B \in \Gamma$ or $!B$ is an axiom, then $\Gamma \Proves !B$.
Since $\Gamma \Proves !B \lif (!A \lif !B)$ by axiom~\eqref{ax:lif1},
\cref{DED2-prop:mp} gives $\Gamma \Proves !A \lif !B$.  If $!B
\ident !A$, then $\Gamma \Proves !A \lif !A$ is derivable using
axioms \eqref{ax:lif1} and~\eqref{ax:lif2} and two applications
of~\MP.

\emph{Inductive step (modus ponens).}  Suppose the derivation of~$!B$
from $\Gamma \cup \{!A\}$ ends with a step justified by modus ponens
from earlier lines $!C \lif !B$ and~$!C$.  Then $\Gamma \cup \{!A\}
\Proves !C \lif !B$ and $\Gamma \cup \{!A\} \Proves !C$, and both
derivations are shorter. By induction hypothesis:
\begin{align*}
  & \Gamma \Proves !A \lif (!C \lif !B); \\
  & \Gamma \Proves !A \lif !C.
\end{align*}
By axiom~\eqref{ax:lif2},
\[
\Gamma \Proves (!A \lif (!C \lif !B)) \lif
((!A\lif !C)  \lif (!A \lif !B)),
\]
and two applications of \cref{DED2-prop:mp} give
$\Gamma \Proves !A \lif !B$.

\emph{Inductive step ($\QR$, universal case).}  Suppose $!B \ident !C
\lif \lforall[x][!D(x)]$ and a formula $!C \lif !D(a)$ appears
earlier in the derivation, where $a$ does not occur in~$!C$, $!A$,
or~$\Gamma$.  By induction hypothesis, $\Gamma \Proves !A \lif (!C
\lif !D(a))$.  From
\[
\Proves (!A \lif (!C \lif !D(a))) \lif ((!A \land !C) \lif !D(a))
\]
and modus ponens we get $\Gamma \Proves (!A \land !C) \lif !D(a)$.
Since the eigenvariable condition still holds, a step justified
by~$\QR$ gives $\Gamma \Proves (!A \land !C) \lif
\lforall[x][!D(x)]$. From
\[
\Proves ((!A \land !C) \lif \lforall[x][!D(x)]) \lif (!A \lif (!C
\lif \lforall[x][!D(x)])),
\]
modus ponens yields $\Gamma \Proves !A \lif (!C \lif
\lforall[x][!D(x)])$, i.e., $\Gamma \Proves !A \lif !B$.

The existential case ($!B \ident \lexists[x][!D(x)] \lif !C$) is
symmetric: one replaces $!C \lif !D(a)$ with $!D(a) \lif !C$
throughout and applies the second form of~$\QR$.
\end{proof}

Notice how axioms \eqref{ax:lif1} and~\eqref{ax:lif2} were chosen
precisely so that the Deduction Theorem would hold.

%%% -----------------------------------------------------------------
%%% DED.2.4  Derivability Properties
%%% -----------------------------------------------------------------

\subsection{Derivability Properties}

The following propositions provide the derivability facts stated
generically in \S\ref{DED.1}, now with explicit axiomatic proofs.

\begin{prop}\label{DED2-prop:provability-land}
\begin{enumerate}
\item Both $!A \land !B \Proves !A$ and $!A \land !B \Proves !B$.
\item $!A, !B \Proves !A \land !B$.
\end{enumerate}
\end{prop}

\begin{proof}
(1) From axioms \eqref{ax:land1} and~\eqref{ax:land2} by modus ponens.
(2) From axiom~\eqref{ax:land3} by two applications of modus ponens.
\end{proof}

\begin{prop}\label{DED2-prop:provability-lor}
\begin{enumerate}
\item $!A \lor !B, \lnot !A, \lnot !B$ is inconsistent.
\item Both $!A \Proves !A \lor !B$ and $!B \Proves !A \lor !B$.
\end{enumerate}
\end{prop}

\begin{proof}
(1) From axiom~\eqref{ax:lnot2} we derive $\{\lnot !A\} \Proves !A
\lif \lfalse$ and $\{\lnot !B\} \Proves !B \lif \lfalse$.  By
axiom~\eqref{ax:lor3} and the deduction theorem, $\{!A \lor !B, \lnot
!A, \lnot !B\} \Proves \lfalse$.
(2) From axioms \eqref{ax:lor1} and~\eqref{ax:lor2} by modus ponens.
\end{proof}

\begin{prop}\label{DED2-prop:provability-lif}
\begin{enumerate}
\item $!A, !A \lif !B \Proves !B$.
\item Both $\lnot !A \Proves !A \lif !B$ and $!B \Proves !A \lif !B$.
\end{enumerate}
\end{prop}

\begin{proof}
(1) Immediate from modus ponens.
(2) By axiom~\eqref{ax:lnot2} and axiom~\eqref{ax:lif1},
respectively, together with the deduction theorem.
\end{proof}

\begin{thm}[Strong Generalization]
\label{DED2-thm:strong-generalization}
If $c$ is a constant symbol not occurring in~$\Gamma$ or $!A(x)$ and
$\Gamma \Proves !A(c)$, then $\Gamma \Proves \lforall[x][!A(x)]$.
\end{thm}

\begin{proof}
By the deduction theorem, $\Gamma \Proves \ltrue \lif !A(c)$.  Since
$c$ does not occur in~$\Gamma$ or~$\ltrue$, the quantifier rule gives
$\Gamma \Proves \ltrue \lif \lforall[x][!A(x)]$. By the deduction
theorem again, $\Gamma \Proves \lforall[x][!A(x)]$.
\end{proof}

\begin{prop}\label{DED2-prop:provability-quantifiers}
\begin{enumerate}
\item $!A(t) \Proves \lexists[x][!A(x)]$.
\item $\lforall[x][!A(x)] \Proves !A(t)$.
\end{enumerate}
\end{prop}

\begin{proof}
(1) By axiom~\eqref{ax:q2} and modus ponens.
(2) By axiom~\eqref{ax:q1} and modus ponens.
\end{proof}

%%% -----------------------------------------------------------------
%%% DED.2.5  Soundness (Axiomatic)
%%% -----------------------------------------------------------------

\subsection{Soundness} \label{DED.2.sou}

\begin{prop}\label{DED2-prop:axioms-valid}
If $!A$ is an axiom (propositional, quantifier, or identity), then
$\Sat{M}{!A}[s]$ for each structure~$\Struct{M}$ and assignment~$s$.
\end{prop}

\begin{proof}
We verify that all the axioms are valid. For instance, here is the
case for axiom~\eqref{ax:q1}: suppose $t$ is free for~$x$ in~$!A$,
and assume $\Sat{M}{\lforall[x][!A]}[s]$. Then for each
$\varAssign{s'}{s}{x}$, also $\Sat{M}{!A}[s']$, and in particular
this holds when $s'(x) = \Value{t}{M}[s]$. By the substitution lemma
(see SYN.4), $\Sat{M}{\Subst{!A}{t}{x}}[s]$.  This shows that
$\Sat{M}{(\lforall[x][!A] \lif \Subst{!A}{t}{x})}[s]$. The remaining
propositional axioms are verified by truth-value analysis.

For the identity axioms: $\eq[t][t]$ is valid since $\Value{t}{M} =
\Value{t}{M}$. The axiom $\eq[t_1][t_2] \lif (!B(t_1) \lif !B(t_2))$
is valid because if $\Value{t_1}{M} = \Value{t_2}{M}$, then by the
substitution lemma $\Sat{M}{!B(t_1)}$ iff $\Sat{M}{!B(t_2)}$.
\end{proof}

\begin{thm}[Soundness] % CP-001(AX)
\label{DED2-thm:soundness}
If $\Gamma \Proves !A$ then $\Gamma \Entails !A$.
\end{thm}

\begin{proof}
By induction on the length of the derivation of~$!A$ from~$\Gamma$.

If there are no steps justified by inferences, then all formulas in
the derivation are either axiom instances or in~$\Gamma$. By
\cref{DED2-prop:axioms-valid}, all axioms are valid, so if $!A$ is an
axiom then $\Gamma \Entails !A$. If $!A \in \Gamma$, then trivially
$\Gamma \Entails !A$.

If the last step is justified by modus ponens, then there are formulas
$!B$ and $!B \lif !A$ in the derivation, and the induction hypothesis
applies to the parts of the derivation ending in those formulas (since
they contain at least one fewer inference step). So by induction
hypothesis, $\Gamma \Entails !B$ and $\Gamma \Entails !B \lif !A$.
Then $\Gamma \Entails !A$ by the semantic deduction theorem (see
THM-SEM, Semantic Deduction, \S SEM).

If the last step is justified by~$\QR$ and has the form $!C \lif
\lforall[x][!B(x)]$, then there is a preceding step $!C \lif !B(c)$
with $c$ not in~$\Gamma$, $!C$, or $\lforall[x][!B(x)]$. By
induction hypothesis, $\Gamma \Entails !C \lif !B(c)$. By the
semantic deduction theorem, $\Gamma \cup \{!C\} \Entails !B(c)$.
Consider a structure~$\Struct{M}$ with $\Sat{M}{\Gamma \cup \{!C\}}$.
We must show $\Sat{M}{\lforall[x][!B(x)]}$, i.e., for every variable
assignment~$s$, $\Sat{M}{!B(x)}[s]$. Since $\Gamma \cup \{!C\}$
consists of sentences, $\Sat{M}{!D}[s]$ for all $!D \in \Gamma \cup
\{!C\}$. Let $\Struct{M'}$ be like~$\Struct{M}$ except $\Assign{c}{M'}
= s(x)$. Since $c$ does not occur in~$\Gamma$ or~$!C$,
$\Sat{M'}{\Gamma \cup \{!C\}}$. Since $\Gamma \cup \{!C\} \Entails
!B(c)$, $\Sat{M'}{!B(c)}$. Since $!B(c)$ is a sentence,
$\Sat{M'}{!B(c)}[s]$. By the substitution lemma, $\Sat{M'}{!B(x)}[s]$.
Since $c$ does not occur in~$!B(x)$, $\Sat{M}{!B(x)}[s]$. Since $s$
was arbitrary, $\Sat{M}{\lforall[x][!B(x)]}$, and by the semantic
deduction theorem, $\Gamma \Entails !C \lif \lforall[x][!B(x)]$.

The case where the last step is $\lexists[x][!B(x)] \lif !C$ is
symmetric.

For the identity axioms, \cref{DED2-prop:axioms-valid} ensures they are
valid, so the base case of the induction applies.
\end{proof}

\begin{cor}[Weak Soundness]
\label{DED2-cor:weak-soundness}
If $\Proves !A$, then $!A$ is valid.
\end{cor}

\begin{cor}
\label{DED2-cor:consistency-soundness}
If $\Gamma$ is satisfiable, then it is consistent.
\end{cor}

\begin{proof}
We prove the contrapositive. If $\Gamma$ is inconsistent, then
$\Gamma \Proves \lfalse$. By \cref{DED2-thm:soundness}, any
structure~$\Struct{M}$ satisfying~$\Gamma$ must satisfy~$\lfalse$.
Since $\Sat/{M}{\lfalse}$ for every~$\Struct{M}$, no structure can
satisfy~$\Gamma$, so $\Gamma$ is unsatisfiable.
\end{proof}


%% ===================================================================
%% DED.3: Natural Deduction
%% Sources: prf/ntd (CONDENSE), ntd/rul (KEEP), ntd/prl (KEEP),
%%          ntd/qrl (KEEP), ntd/der (CONDENSE), ntd/sou (KEEP+ABSORB sid)
%% ===================================================================

\section{Natural Deduction} \label{DED.3}

Natural deduction instantiates the generic framework of \S\ref{DED.1}
differently from axiomatic deduction.  There are no logical axioms;
instead, each connective and quantifier has introduction and
elimination rules, corresponding to natural inference patterns (e.g.,
conditional proof, proof by cases, indirect proof).  A derivation
(\ref{PRIM-DED005}) is a finite \emph{tree} of sentences rather than
a sequence.  Provability (\ref{PRIM-DED006}) and consistency
(\ref{DEF-DED001}) are defined exactly as in \S\ref{DED.1}.

A distinguishing feature of natural deduction is the mechanism of
\emph{assumption discharge} (\ref{PRIM-DED009}): certain rules cancel
(\emph{discharge}) hypothetical assumptions, so that they no longer
count among the open assumptions of the derivation.  The undischarged
assumptions of a completed derivation are the set~$\Gamma$ from which
the conclusion is derived.

Natural deduction systems were developed by Gentzen and Ja\'skowski in
the 1930s, and later refined by Prawitz and Fitch.  In the philosophy
of logic, the rules of natural deduction have sometimes been taken to
define the meanings of the logical operators (``proof-theoretic
semantics'').

%%% -----------------------------------------------------------------
%%% DED.3.1  Rules and Derivations
%%% -----------------------------------------------------------------

\subsection{Rules and Derivations}

\begin{defn}[Assumption]
An \emph{assumption} is any sentence
in the topmost position of any branch of a derivation tree.
\end{defn}

Derivations in natural deduction are certain trees of sentences, where
the topmost sentences are assumptions, and if a sentence stands below
one, two, or three other sentences, it must follow correctly by a rule
of inference. The sentences at the top of the inference are the
\emph{premises} and the sentence below the \emph{conclusion}. The
rules come in pairs---an introduction and an elimination rule for each
operator---and some rules allow an assumption to be
\emph{discharged}. To indicate which assumption is discharged by which
inference, both receive a matching label: the assumption is written
``$\Discharge{!A}{n}$.''

It is customary to consider rules for all the operators $\land$,
$\lor$, $\lif$, $\lnot$, and $\lfalse$, even if some of those are
defined.

%%% -----------------------------------------------------------------
%%% DED.3.2  Propositional Rules
%%% -----------------------------------------------------------------

\subsection{Propositional Rules} % DEF-DED006
\label{DEF-DED006}

\subsubsection*{Rules for $\land$}

\begin{defish}
\AxiomC{$!A$}
\AxiomC{$!B$}
\RightLabel{\Intro{\land}}
\BinaryInfC{$!A \land !B$}
\DisplayProof
\hfill
\begin{tabular}{r}
\AxiomC{$!A \land !B$}
\RightLabel{\Elim{\land}}
\UnaryInfC{$!A$}
\DisplayProof
\\[3ex]
\AxiomC{$!A \land !B$}
\RightLabel{\Elim{\land}}
\UnaryInfC{$!B$}
\DisplayProof
\end{tabular}
\end{defish}

\subsubsection*{Rules for $\lor$}

\begin{defish}
\begin{tabular}{r}
\AxiomC{$!A$}
\RightLabel{\Intro{\lor}}
\UnaryInfC{$!A \lor !B$}
\DisplayProof
\\[3ex]
\AxiomC{$!B$}
\RightLabel{\Intro{\lor}}
\UnaryInfC{$!A \lor !B$}
\DisplayProof
\end{tabular}
\hfill
\AxiomC{$!A \lor !B$}
\AxiomC{$\Discharge{!A}{n}$}
\DeduceC{$!C$}
\AxiomC{$\Discharge{!B}{n}$}
\DeduceC{$!C$}
\DischargeRule{\Elim{\lor}}{n}
\TrinaryInfC{$!C$}
\DisplayProof
\end{defish}

\subsubsection*{Rules for $\lif$}

\begin{defish}
\AxiomC{$\Discharge{!A}{n}$}
\DeduceC{$!B$}
\DischargeRule{\Intro{\lif}}{n}
\UnaryInfC{$!A \lif !B$}
\DisplayProof
\hfill
\AxiomC{$!A \lif !B$}
\AxiomC{$!A$}
\RightLabel{\Elim{\lif}}
\BinaryInfC{$!B$}
\DisplayProof
\end{defish}

\subsubsection*{Rules for $\lnot$}

\begin{defish}
\AxiomC{$\Discharge{!A}{n}$}
\noLine
\DeduceC{$\lfalse$}
\DischargeRule{\Intro{\lnot}}{n}
\UnaryInfC{$\lnot !A$}
\DisplayProof
\hfill
\AxiomC{$\lnot !A$}
\AxiomC{$!A$}
\RightLabel{\Elim{\lnot}}
\BinaryInfC{$\lfalse$}
\DisplayProof
\end{defish}

\subsubsection*{Rules for $\lfalse$}

\begin{defish}
\AxiomC{$\lfalse$}
\RightLabel{\FalseInt}
\UnaryInfC{$!A$}
\DisplayProof
\hfill
\AxiomC{$\Discharge{\lnot !A}{n}$}
\DeduceC{$\lfalse$}
\DischargeRule{\FalseCl}{n}
\UnaryInfC{$!A$}
\DisplayProof
\end{defish}

Note that $\Intro{\lnot}$ and $\FalseCl$ are very similar: the
difference is that $\Intro{\lnot}$ derives a negated sentence~$\lnot
!A$ while $\FalseCl$ derives a positive sentence~$!A$.

Whenever a rule indicates that some assumption may be discharged, this
is a permission, not a requirement.  In the $\Intro{\lif}$ rule, for
instance, we may discharge any number of assumptions of the
form~$!A$, including zero.

%%% -----------------------------------------------------------------
%%% DED.3.3  Quantifier Rules
%%% -----------------------------------------------------------------

\subsection{Quantifier Rules}

\subsubsection*{Rules for $\lforall$}

\begin{defish}
\AxiomC{$!A(a)$}
\RightLabel{\Intro{\lforall}}
\UnaryInfC{$\lforall[x][\Atom{!A}{x}]$}
\DisplayProof
\hfill
\AxiomC{$\lforall[x][\Atom{!A}{x}]$}
\RightLabel{\Elim{\lforall}}
\UnaryInfC{$!A(t)$}
\DisplayProof
\end{defish}

In $\Elim{\lforall}$, $t$ is a closed term. In $\Intro{\lforall}$,
$a$ is a constant symbol which does not occur in the
conclusion~$\lforall[x][!A(x)]$ or in any undischarged assumption. We
call $a$ the \emph{eigenvariable} of the $\Intro{\lforall}$ inference.

\subsubsection*{Rules for $\lexists$}

\begin{defish}
\AxiomC{$\Atom{!A}{t}$}
\RightLabel{\Intro{\lexists}}
\UnaryInfC{$\lexists[x][\Atom{!A}{x}]$}
\DisplayProof
\hfill
\AxiomC{$\lexists[x][\Atom{!A}{x}]$}
\AxiomC{[$\Atom{!A}{a}$]$^n$}
\DeduceC{$!C$}
\DischargeRule{\Elim{\lexists}}{n}
\BinaryInfC{$!C$}
\DisplayProof
\end{defish}

Again, $t$ is a closed term, and $a$ is a constant symbol which does
not occur in $\lexists[x][!A(x)]$, in~$!C$, or in any undischarged
assumption (other than~$!A(a)$). We call $a$ the \emph{eigenvariable}
of the $\Elim{\lexists}$ inference.

The condition that an eigenvariable not occur in the premises or in
any undischarged assumption is the \emph{eigenvariable condition}.  It
is necessary to ensure soundness.

\subsubsection*{Rules for Identity}

\begin{defish}
\AxiomC{}
\RightLabel{\Intro{\eq}}
\UnaryInfC{$\eq[t][t]$}
\DisplayProof
\hfill
\begin{tabular}{r}
\AxiomC{$\eq[t_1][t_2]$}
\AxiomC{$!A(t_1)$}
\RightLabel{\Elim{\eq}}
\BinaryInfC{$!A(t_2)$}
\DisplayProof
\\[3ex]
\AxiomC{$\eq[t_1][t_2]$}
\AxiomC{$!A(t_2)$}
\RightLabel{\Elim{\eq}}
\BinaryInfC{$!A(t_1)$}
\DisplayProof
\end{tabular}
\end{defish}

In the above rules, $t$, $t_1$, and $t_2$ are closed terms. The
$\Intro{\eq}$ rule derives $\eq[t][t]$ from no assumptions.

%%% -----------------------------------------------------------------
%%% DED.3.4  Derivations
%%% -----------------------------------------------------------------

\subsection{Derivations}

\begin{defn}[Derivation] % ND instantiation of PRIM-DED005
\label{DED3-defn:derivation}
A \emph{derivation} of a sentence~$!A$ from assumptions~$\Gamma$ is a
finite tree of sentences satisfying:
\begin{enumerate}
\item The topmost sentences are either in~$\Gamma$ or are discharged
  by an inference in the tree.
\item The bottommost sentence is~$!A$.
\item Every sentence except~$!A$ is a premise of a correct application
  of an inference rule whose conclusion stands directly below it.
\end{enumerate}
We call $!A$ the \emph{conclusion} and $\Gamma$ the
\emph{undischarged assumptions}. If such a derivation exists, we write
$\Gamma \Proves !A$. If every assumption is discharged, we
write~$\Proves !A$.
\end{defn}

%%% -----------------------------------------------------------------
%%% DED.3.5  Soundness (Natural Deduction)
%%% -----------------------------------------------------------------

\subsection{Soundness} \label{DED.3.sou}

\begin{thm}[Soundness] % CP-001(ND)
\label{DED3-thm:soundness}
If $!A$ is derivable from the undischarged assumptions~$\Gamma$, then
$\Gamma \Entails !A$.
\end{thm}

\begin{proof}
Let $\delta$ be a derivation of~$!A$. We proceed by induction on the
number of inferences in~$\delta$.

\emph{Base case.} If there are no inferences, $\delta$ consists of a
single sentence~$!A$ that is an undischarged assumption. Any
structure satisfying~$\Gamma$ satisfies~$!A$ since $!A \in \Gamma$.

\emph{Inductive step.} Suppose $\delta$ contains~$n$ inferences.  We
assume the induction hypothesis for each sub-derivation (which has
fewer than~$n$ inferences) and distinguish cases by the last
inference.

\begin{enumerate}
\item \emph{$\Intro{\lnot}$:} The derivation ends in
\begin{prooftree}
  \AxiomC{$\Gamma, \Discharge{!A}{n}$}
  \RightLabel{$\delta_1$}
  \DeduceC{$\lfalse$}
  \DischargeRule{\Intro{\lnot}}{n}
  \UnaryInfC{$\lnot !A$}
\end{prooftree}
By induction hypothesis, $\Gamma \cup \{!A\} \Entails \lfalse$.
Suppose $\Sat{M}{\Gamma}$ but $\Sat/{M}{\lnot !A}$, i.e.,
$\Sat{M}{!A}$. Then $\Sat{M}{\Gamma \cup \{!A\}}$, so
$\Sat{M}{\lfalse}$, a contradiction. Hence $\Sat{M}{\lnot !A}$.

\item \emph{$\Elim{\land}$:} By induction hypothesis $\Gamma \Entails
!A \land !B$; by definition of satisfaction, $\Sat{M}{!A}$ (or
$\Sat{M}{!B}$, for the other variant).

\item \emph{$\Intro{\lor}$:} By induction hypothesis $\Gamma \Entails
!A$; since $\Sat{M}{!A}$ implies $\Sat{M}{!A \lor !B}$.

\item \emph{$\Intro{\lif}$:} By induction hypothesis $\Gamma \cup
\{!A\} \Entails !B$. Suppose $\Sat{M}{\Gamma}$ but
$\Sat/{M}{!A \lif !B}$. Then $\Sat{M}{!A}$ and $\Sat/{M}{!B}$, so
$\Sat{M}{\Gamma \cup \{!A\}}$, giving $\Sat{M}{!B}$, contradiction.

\item \emph{$\Elim{\lif}$:} By induction hypothesis $\Gamma_1
\Entails !A \lif !B$ and $\Gamma_2 \Entails !A$. If
$\Sat{M}{\Gamma_1 \cup \Gamma_2}$, then $\Sat{M}{!A \lif !B}$ and
$\Sat{M}{!A}$, so $\Sat{M}{!B}$.

\item \emph{$\Elim{\lnot}$:} By induction hypothesis $\Gamma_1
\Entails \lnot !A$ and $\Gamma_2 \Entails !A$. If
$\Sat{M}{\Gamma_1 \cup \Gamma_2}$ then both $\Sat{M}{\lnot !A}$ and
$\Sat{M}{!A}$, giving $\Sat{M}{\lfalse}$.

\item \emph{$\FalseInt$:} By induction hypothesis $\Gamma \Entails
\lfalse$. Since $\Sat/{M}{\lfalse}$ for every~$\Struct{M}$, no
structure satisfies~$\Gamma$, so $\Gamma \Entails !A$ vacuously.

\item \emph{$\FalseCl$:} By induction hypothesis $\Gamma \cup \{\lnot
!A\} \Entails \lfalse$. Suppose $\Sat{M}{\Gamma}$ but
$\Sat/{M}{!A}$, i.e., $\Sat{M}{\lnot !A}$. Then $\Sat{M}{\Gamma \cup
\{\lnot !A\}}$, so $\Sat{M}{\lfalse}$, contradiction.

\item \emph{$\Intro{\land}$:} By induction hypothesis $\Gamma_1
\Entails !A$ and $\Gamma_2 \Entails !B$. If $\Sat{M}{\Gamma_1 \cup
\Gamma_2}$, then $\Sat{M}{!A}$ and $\Sat{M}{!B}$, so $\Sat{M}{!A
\land !B}$.

\item \emph{$\Elim{\lor}$:} By induction hypothesis $\Gamma_1
\Entails !A \lor !B$, $\Gamma_2 \cup \{!A\} \Entails !C$, and
$\Gamma_3 \cup \{!B\} \Entails !C$. If $\Sat{M}{\Gamma_1 \cup
\Gamma_2 \cup \Gamma_3}$, then $\Sat{M}{!A \lor !B}$, so either
$\Sat{M}{!A}$ or $\Sat{M}{!B}$; in either case $\Sat{M}{!C}$.

\item \emph{$\Intro{\lforall}$:} By induction hypothesis $\Gamma
\Entails !A(a)$ where $a$ does not occur in~$\Gamma$ or
$\lforall[x][!A(x)]$. Suppose $\Sat{M}{\Gamma}$. We must show
$\Sat{M}{\lforall[x][!A(x)]}$, i.e., for every variable
assignment~$s$, $\Sat{M}{!A(x)}[s]$. Let $\Struct{M'}$ be
like~$\Struct{M}$ except $\Assign{a}{M'} = s(x)$. Since $a$ does not
occur in~$\Gamma$, $\Sat{M'}{\Gamma}$, so $\Sat{M'}{!A(a)}$, whence
$\Sat{M'}{!A(a)}[s]$. By the substitution lemma,
$\Sat{M'}{!A(x)}[s]$. Since $a$ does not occur in~$!A(x)$,
$\Sat{M}{!A(x)}[s]$.

\item \emph{$\Intro{\lexists}$:} By induction hypothesis $\Gamma
\Entails !A(t)$. By the substitution lemma, $\Sat{M}{!A(t)}$ implies
$\Sat{M}{\lexists[x][!A(x)]}$.

\item \emph{$\Elim{\lforall}$:} By induction hypothesis $\Gamma
\Entails \lforall[x][!A(x)]$. If $\Sat{M}{\lforall[x][!A(x)]}$, by
the substitution lemma $\Sat{M}{!A(t)}$.

\item \emph{$\Elim{\lexists}$:} By induction hypothesis $\Gamma_1
\Entails \lexists[x][!A(x)]$ and $\Gamma_2 \cup \{!A(a)\} \Entails
!C$, where $a$ does not occur in $\lexists[x][!A(x)]$, $!C$, or the
undischarged assumptions. The argument is analogous to the
$\Intro{\lforall}$ case, constructing a suitable~$\Struct{M'}$.

\item \emph{$\Intro{\eq}$:} $\eq[t][t]$ is valid since
$\Value{t}{M} = \Value{t}{M}$ for every~$\Struct{M}$.

\item \emph{$\Elim{\eq}$:} By induction hypothesis $\Gamma_1
\Entails \eq[t_1][t_2]$ and $\Gamma_2 \Entails !A(t_1)$. If
$\Sat{M}{\Gamma_1 \cup \Gamma_2}$, then $\Value{t_1}{M} =
\Value{t_2}{M}$. By the substitution lemma, $\Sat{M}{!A(t_1)}$
iff $\Sat{M}{!A(t_2)}$, so $\Sat{M}{!A(t_2)}$.
\end{enumerate}
\end{proof}

\begin{cor}[Weak Soundness]
\label{DED3-cor:weak-soundness}
If $\Proves !A$, then $!A$ is valid.
\end{cor}

\begin{cor}
\label{DED3-cor:consistency-soundness}
If $\Gamma$ is satisfiable, then it is consistent.
\end{cor}

\begin{proof}
Contrapositive: if $\Gamma$ is inconsistent, then $\Gamma \Proves
\lfalse$. By \cref{DED3-thm:soundness}, any structure satisfying
$\Gamma$ must satisfy~$\lfalse$. Since $\Sat/{M}{\lfalse}$ for
every~$\Struct{M}$, $\Gamma$ is unsatisfiable.
\end{proof}

\begin{rem}
The structural properties of \S\ref{DED.1} (reflexivity, monotonicity,
transitivity, compactness) hold for natural deduction. Transitivity
uses implication-introduction and elimination rather than sequence
concatenation.
\end{rem}


%% ===================================================================
%% DED.4: Sequent Calculus
%% Sources: seq/rul (KEEP), seq/prl (KEEP), seq/srl (KEEP),
%%          seq/qrl (KEEP), seq/der (CONDENSE), seq/sou (KEEP+ABSORB sid),
%%          seq/ide (CONDENSE)
%% ===================================================================

\section{Sequent Calculus} \label{DED.4}

The sequent calculus (Gentzen's $\Log{LK}$) instantiates the generic
framework of \S\ref{DED.1} with derivations that are trees of
\emph{sequents} (\ref{PRIM-DED008}) rather than trees of single
formulas.  The system has no axiom schemas in the Hilbert sense;
instead, it employs initial sequents, logical rules (one left and one
right rule per connective), and the structural rules of
\S\ref{DED.1}.3 (\ref{PRIM-DED007}).  Provability
(\ref{PRIM-DED006}) and consistency (\ref{DEF-DED001}) are defined as
before: $\Gamma \Proves !A$ iff for some finite $\Gamma_0 \subseteq
\Gamma$, the sequent $\Gamma_0 \Sequent !A$ has a derivation.

%%% -----------------------------------------------------------------
%%% DED.4.1  Sequents and Initial Sequents
%%% -----------------------------------------------------------------

\subsection{Sequents and Initial Sequents} % DEF-DED007
\label{DEF-DED007}

For the following, let $\Gamma, \Delta, \Pi, \Lambda$ represent finite
sequences of sentences.

\begin{defn}[Sequent]
A \emph{sequent} is an expression $\Gamma \Sequent \Delta$ where
$\Gamma$ (the \emph{antecedent}) and $\Delta$ (the \emph{succedent})
are finite (possibly empty) sequences of sentences.
\end{defn}

The intuitive reading of $\Gamma \Sequent \Delta$ is: if all
sentences in the antecedent hold, then at least one of the sentences
in the succedent holds.  That is, if $\Gamma = \tuple{!A_1, \dots,
!A_m}$ and $\Delta = \tuple{!B_1, \dots, !B_n}$, then $\Gamma
\Sequent \Delta$ corresponds to
\[
(!A_1 \land \cdots \land !A_m) \lif (!B_1 \lor \cdots \lor !B_n).
\]
When $\Gamma$ is empty, $\Sequent \Delta$ asserts $!B_1 \lor \dots
\lor !B_n$. When $\Delta$ is empty, $\Gamma \Sequent$ asserts
$\lnot(!A_1 \land \dots \land !A_m)$.

If $\Gamma$ is a sequence, we write $\Gamma, !A$ for the result of
appending $!A$ to the right end of~$\Gamma$ (and $!A, \Gamma$ for
appending to the left). $\Gamma, \Delta$ denotes concatenation.

\begin{defn}[Initial Sequent]
An \emph{initial sequent} is a sequent of one of the following forms:
\begin{enumerate}
\item $!A \Sequent !A$ for any sentence~$!A$;
\item $\quad \Sequent \ltrue$;
\item $\lfalse \Sequent \quad$.
\end{enumerate}
\end{defn}

Derivations in the sequent calculus are trees of sequents, where the
topmost sequents are initial sequents, and the logical rules are named
for the main operator of the sentence they introduce. Each comes in
two forms: a left rule (introducing the operator in the antecedent)
and a right rule (introducing it in the succedent).

%%% -----------------------------------------------------------------
%%% DED.4.2  Propositional Rules
%%% -----------------------------------------------------------------

\subsection{Propositional Rules}

\subsubsection*{Rules for $\lnot$}

\begin{defish}
\Axiom$ \Gamma \fCenter \Delta, !A $
\RightLabel{\LeftR{\lnot}}
\UnaryInf$ \lnot !A, \Gamma \fCenter \Delta$
\DisplayProof
\hfill
\Axiom$!A, \Gamma \fCenter \Delta$
\RightLabel{\RightR{\lnot}}
\UnaryInf$ \Gamma \fCenter \Delta, \lnot !A $
\DisplayProof
\end{defish}

\subsubsection*{Rules for $\land$}

\begin{defish}\noindent
\begin{tabular}{l}
\Axiom$ !A, \Gamma \fCenter \Delta$
\RightLabel{\LeftR{\land}}
\UnaryInf$ !A \land !B, \Gamma \fCenter \Delta$
\DisplayProof
\\[3ex]
\Axiom$!B, \Gamma \fCenter \Delta$
\RightLabel{\LeftR{\land}}
\UnaryInf$!A \land !B, \Gamma \fCenter \Delta$
\DisplayProof
\end{tabular}
\hfill
\Axiom$\Gamma \fCenter \Delta, !A$
\Axiom$ \Gamma \fCenter \Delta, !B$
\RightLabel{\RightR{\land}}
\BinaryInf$ \Gamma \fCenter \Delta, !A \land !B $
\DisplayProof
\end{defish}

\subsubsection*{Rules for $\lor$}

\begin{defish}
\Axiom$!A, \Gamma \fCenter \Delta$
\Axiom$!B, \Gamma \fCenter \Delta$
\RightLabel{\LeftR{\lor}}
\BinaryInf$!A \lor !B, \Gamma \fCenter \Delta$
\DisplayProof
\hfill
\begin{tabular}{r}
\Axiom$\Gamma \fCenter \Delta, !A$
\RightLabel{\RightR{\lor}}
\UnaryInf$ \Gamma \fCenter \Delta, !A \lor !B$
\DisplayProof
\\[3ex]
\Axiom$ \Gamma \fCenter \Delta, !B$
\RightLabel{\RightR{\lor}}
\UnaryInf$ \Gamma \fCenter \Delta, !A \lor !B$
\DisplayProof
\end{tabular}
\end{defish}

\subsubsection*{Rules for $\lif$}

\begin{defish}
\Axiom$ \Gamma \fCenter \Delta, !A$
\Axiom$ !B, \Pi \fCenter \Lambda$
\RightLabel{\LeftR{\lif}}
\BinaryInf$ !A \lif !B, \Gamma, \Pi \fCenter \Delta, \Lambda$
\DisplayProof
\hfill
\Axiom$ !A, \Gamma \fCenter \Delta, !B$
\RightLabel{\RightR{\lif}}
\UnaryInf$ \Gamma \fCenter \Delta, !A \lif !B $
\DisplayProof
\end{defish}

%%% -----------------------------------------------------------------
%%% DED.4.3  Structural Rules
%%% -----------------------------------------------------------------

\subsection{Structural Rules}

The structural rules (\ref{PRIM-DED007}) of $\Log{LK}$ allow
rearranging sentences in the antecedent and succedent. Because the
logical rules require that the principal sentence stand at a specific
position, exchange is needed to move it there; weakening and
contraction handle unused or duplicated formulas.

\subsubsection*{Weakening}

\begin{defish}
\Axiom$ \Gamma \fCenter \Delta $
\RightLabel{\LeftR{\Weakening}}
\UnaryInf$ !A, \Gamma \fCenter \Delta$
\DisplayProof
\hfill
\Axiom$ \Gamma \fCenter \Delta$
\RightLabel{\RightR{\Weakening}}
\UnaryInf$ \Gamma \fCenter \Delta, !A$
\DisplayProof
\end{defish}

\subsubsection*{Contraction}

\begin{defish}
\Axiom$ !A, !A, \Gamma \fCenter \Delta $
\RightLabel{\LeftR{\Contraction}}
\UnaryInf$ !A, \Gamma \fCenter \Delta$
\DisplayProof
\hfill
\Axiom$ \Gamma \fCenter \Delta, !A, !A$
\RightLabel{\RightR{\Contraction}}
\UnaryInf$ \Gamma \fCenter \Delta, !A$
\DisplayProof
\end{defish}

\subsubsection*{Exchange}

\begin{defish}
\Axiom$ \Gamma, !A, !B, \Pi \fCenter \Delta $
\RightLabel{\LeftR{\Exchange}}
\UnaryInf$ \Gamma, !B, !A, \Pi \fCenter \Delta$
\DisplayProof
\hfill
\Axiom$ \Gamma \fCenter \Delta, !A, !B, \Lambda$
\RightLabel{\RightR{\Exchange}}
\UnaryInf$ \Gamma \fCenter \Delta, !B, !A, \Lambda$
\DisplayProof
\end{defish}

A series of weakening, contraction, and exchange inferences is often
indicated by double inference lines.

\subsubsection*{Cut}

\begin{defish}
\[
\Axiom$ \Gamma \fCenter \Delta, !A$
\Axiom$ !A, \Pi \fCenter \Lambda $
\RightLabel{\Cut}
\BinaryInf$ \Gamma, \Pi \fCenter \Delta, \Lambda$
\DisplayProof
\]
\end{defish}

The cut rule is not strictly necessary, but makes it considerably
easier to reuse and combine derivations.

\begin{rem}[Cut Elimination] % CP-010
\label{CP-010}
\emph{Cut Elimination (Gentzen's Hauptsatz):} The $\Cut$ rule is
\emph{admissible} in~$\Log{LK}$---every $\Log{LK}$-derivation using
$\Cut$ can be transformed into one without~$\Cut$. The proof is a
detailed structural induction on the complexity of the cut formula and
the height of the derivation; it is beyond our scope here. Cut
elimination has profound consequences: it implies the subformula
property (every formula in a cut-free derivation is a subformula of
the end-sequent), which in turn yields consistency proofs, decidability
results, and interpolation theorems.
\end{rem}

%%% -----------------------------------------------------------------
%%% DED.4.4  Quantifier Rules
%%% -----------------------------------------------------------------

\subsection{Quantifier Rules}

\subsubsection*{Rules for $\lforall$}

\begin{defish}
\Axiom$ !A(t), \Gamma \fCenter \Delta$
\RightLabel{\LeftR{\lforall}}
\UnaryInf$ \lforall[x][!A(x)],\Gamma \fCenter \Delta$
\DisplayProof
\hfill
\Axiom$ \Gamma \fCenter \Delta, !A(a) $
\RightLabel{\RightR{\lforall}}
\UnaryInf$ \Gamma \fCenter \Delta, \lforall[x][!A(x)]$
\DisplayProof
\end{defish}

In $\LeftR{\lforall}$, $t$ is a closed term. In $\RightR{\lforall}$,
$a$ is a constant symbol which must not occur in the lower sequent.
We call $a$ the \emph{eigenvariable} of the $\RightR{\lforall}$
inference.

\subsubsection*{Rules for $\lexists$}

\begin{defish}
\Axiom$ !A(a), \Gamma \fCenter \Delta $
\RightLabel{\LeftR{\lexists}}
\UnaryInf$ \lexists[x][!A(x)], \Gamma \fCenter \Delta$
\DisplayProof
\hfill
\Axiom$ \Gamma \fCenter \Delta, !A(t) $
\RightLabel{\RightR{\lexists}}
\UnaryInf$ \Gamma \fCenter \Delta, \lexists[x][!A(x)]$
\DisplayProof
\end{defish}

Again, $t$ is a closed term, and $a$ is a constant symbol not
occurring in the lower sequent. The \emph{eigenvariable condition}
requires that the eigenvariable~$a$ not occur anywhere in the lower
sequent of the $\RightR{\lforall}$ or $\LeftR{\lexists}$ inference.

\subsubsection*{Identity Rules}

\begin{defn}[Initial Sequents for $\eq$]
If $t$ is a closed term, then ${} \Sequent \eq[t][t]$ is an initial
sequent.
\end{defn}

The rules for~$\eq$ are ($t_1$ and $t_2$ are closed terms):

\begin{defish}
\Axiom$ \eq[t_1][t_2], \Gamma \fCenter \Delta, !A(t_1) $
\RightLabel{$\eq$}
\UnaryInf$\eq[t_1][t_2], \Gamma \fCenter \Delta, !A(t_2)$
\DisplayProof
\hfill
\Axiom$\eq[t_1][t_2], \Gamma \fCenter \Delta, !A(t_2) $
\RightLabel{$\eq$}
\UnaryInf$\eq[t_1][t_2], \Gamma  \fCenter \Delta, !A(t_1)$
\DisplayProof
\end{defish}

%%% -----------------------------------------------------------------
%%% DED.4.5  Derivations
%%% -----------------------------------------------------------------

\subsection{Derivations}

\begin{defn}[$\Log{LK}$-Derivation]
\label{DED4-defn:derivation}
An \emph{$\Log{LK}$-derivation} of a sequent~$S$ is a finite tree of
sequents satisfying:
\begin{enumerate}
\item The topmost sequents are initial sequents.
\item The bottommost sequent is~$S$.
\item Every sequent except $S$ is a premise of a correct application
  of an inference rule whose conclusion stands directly below it.
\end{enumerate}
We say $S$ is the \emph{end-sequent} and that $S$ is
\emph{$\Log{LK}$-derivable}.
\end{defn}

%%% -----------------------------------------------------------------
%%% DED.4.6  Soundness (Sequent Calculus)
%%% -----------------------------------------------------------------

\subsection{Soundness} \label{DED.4.sou}

\begin{defn}[Valid Sequent]
\label{DED4-defn:valid-sequent}
A structure~$\Struct{M}$ \emph{satisfies} a sequent $\Gamma \Sequent
\Delta$ iff either $\Sat/{M}{!A}$ for some $!A \in \Gamma$ or
$\Sat{M}{!A}$ for some $!A \in \Delta$.  A sequent is \emph{valid}
iff every structure satisfies it.
\end{defn}

\begin{thm}[Soundness] % CP-001(SC)
\label{DED4-thm:sequent-soundness}
If $\Log{LK}$ derives $\Theta \Sequent \Xi$, then $\Theta \Sequent
\Xi$ is valid.
\end{thm}

\begin{proof}
Let $\pi$ be a derivation of $\Theta \Sequent \Xi$. We proceed by
induction on the number of inferences~$n$ in~$\pi$.

If $n = 0$, then $\pi$ is an initial sequent. Every initial sequent
$!A \Sequent !A$ is valid, since for every~$\Struct{M}$, either
$\Sat/{M}{!A}$ or $\Sat{M}{!A}$.  The sequents $\Sequent \ltrue$ and
$\lfalse \Sequent$ are also valid.  Identity initial sequents
$\Sequent \eq[t][t]$ are valid since $\Value{t}{M} = \Value{t}{M}$.

If $n > 0$, we distinguish cases by the last inference.  By induction
hypothesis the premise(s) are valid.

\begin{enumerate}
\item \emph{Weakening:} If the premise $\Gamma \Sequent \Delta$ is
valid, then so is $!A, \Gamma \Sequent \Delta$ (and $\Gamma \Sequent
\Delta, !A$), since any witness to the validity of $\Gamma \Sequent
\Delta$ also witnesses the validity of the weakened sequent.

\item \emph{$\LeftR{\lnot}$:} The premise is $\Gamma \Sequent \Delta,
!A$ and the conclusion is $\lnot !A, \Gamma \Sequent \Delta$. Given
$\Struct{M}$, if $\Struct{M}$ falsifies some $!C \in \Gamma$ or
satisfies some $!C \in \Delta$, the conclusion is satisfied. Otherwise,
the validity of the premise forces $\Sat{M}{!A}$, whence
$\Sat/{M}{\lnot !A}$, and $\lnot !A \in \Theta$ is falsified.

\item \emph{$\RightR{\lnot}$:} Symmetric to the $\LeftR{\lnot}$ case.

\item \emph{$\LeftR{\land}$:} The premise is $!A, \Gamma \Sequent
\Delta$ and the conclusion is $!A \land !B, \Gamma \Sequent \Delta$.
If $\Sat/{M}{!A}$, then $\Sat/{M}{!A \land !B}$, and the conclusion
is satisfied. Otherwise $\Sat{M}{!A}$, and the validity of the
premise gives the result. The case with $!B$ is analogous.

\item \emph{$\RightR{\lor}$:} The premise is $\Gamma \Sequent \Delta,
!A$ and the conclusion is $\Gamma \Sequent \Delta, !A \lor !B$. If
$\Sat{M}{!A}$ then $\Sat{M}{!A \lor !B}$. Otherwise the validity of
the premise gives a witness in $\Gamma$ or~$\Delta$.

\item \emph{$\RightR{\lif}$:} The premise is $!A, \Gamma \Sequent
\Delta, !B$ and the conclusion is $\Gamma \Sequent \Delta, !A \lif
!B$. If $\Sat/{M}{!A}$ or $\Sat{M}{!B}$, then $\Sat{M}{!A \lif !B}$.
Otherwise the validity of the premise gives a witness.

\item \emph{$\RightR{\land}$:} (Two premises.) If $\Struct{M}$ does
not satisfy $\Gamma \Sequent \Delta$, then the validity of the first
premise forces $\Sat{M}{!A}$ and the validity of the second forces
$\Sat{M}{!B}$, whence $\Sat{M}{!A \land !B}$.

\item \emph{$\LeftR{\lor}$:} (Two premises.) If $\Sat{M}{!A \lor !B}$
then either $\Sat{M}{!A}$ or $\Sat{M}{!B}$. In the former case the
validity of the left premise gives the result; in the latter, the
right premise.

\item \emph{$\LeftR{\lif}$:} (Two premises.) Suppose $\Struct{M}$
does not satisfy $\Gamma, \Pi \Sequent \Delta, \Lambda$. Then
$\Struct{M}$ satisfies neither $\Gamma \Sequent \Delta$ nor $\Pi
\Sequent \Lambda$. The validity of $\Gamma \Sequent \Delta, !A$
forces $\Sat{M}{!A}$, and the validity of $!B, \Pi \Sequent \Lambda$
forces $\Sat/{M}{!B}$. Hence $\Sat/{M}{!A \lif !B}$.

\item \emph{$\Cut$:} (Two premises.) Either $\Sat/{M}{!A}$ or
$\Sat{M}{!A}$. In the first case, $\Struct{M}$ must satisfy $\Gamma
\Sequent \Delta$ by the validity of the left premise. In the second,
$\Struct{M}$ must satisfy $\Pi \Sequent \Lambda$ by the validity of
the right premise.

\item \emph{$\LeftR{\lforall}$:} The premise is $!A(t), \Gamma
\Sequent \Delta$ and the conclusion is $\lforall[x][!A(x)], \Gamma
\Sequent \Delta$. If $\Sat{M}{\lforall[x][!A(x)]}$, then by the
substitution lemma $\Sat{M}{!A(t)}$. The validity of the premise then
gives the result. If $\Sat/{M}{\lforall[x][!A(x)]}$, the conclusion
is satisfied since $\lforall[x][!A(x)]$ is in the antecedent.

\item \emph{$\RightR{\lforall}$:} The premise is $\Gamma \Sequent
\Delta, !A(a)$ where the eigenvariable condition holds. Suppose
$\Sat{M}{\Gamma}$ and $\Sat/{M}{!C}$ for all $!C \in \Delta$. We
must show $\Sat{M}{\lforall[x][!A(x)]}$, i.e., for all~$s$,
$\Sat{M}{!A(x)}[s]$. Let $\Struct{M'}$ be like~$\Struct{M}$ except
$\Assign{a}{M'} = s(x)$. Since $a$ does not occur in $\Gamma$ or
$\Delta$, the same truth values hold in~$\Struct{M'}$. The validity
of the premise then gives $\Sat{M'}{!A(a)}$. By the substitution
lemma, $\Sat{M'}{!A(x)}[s]$, and since $a$ does not occur
in~$!A(x)$, $\Sat{M}{!A(x)}[s]$.

\item \emph{$\LeftR{\lexists}$:} Symmetric to the $\RightR{\lforall}$
case.

\item \emph{$\RightR{\lexists}$:} Symmetric to the $\LeftR{\lforall}$
case.

\item \emph{Identity rule~$\eq$:} The premise is $\eq[t_1][t_2],
\Gamma \Sequent \Delta, !A(t_1)$ and the conclusion is
$\eq[t_1][t_2], \Gamma \Sequent \Delta, !A(t_2)$. By induction
hypothesis the premise is valid. If $\Sat{M}{\eq[t_1][t_2]}$ and
$\Sat{M}{!A(t_1)}$, then $\Value{t_1}{M} = \Value{t_2}{M}$, and by
the substitution lemma $\Sat{M}{!A(t_2)}$.
\end{enumerate}
\end{proof}

\begin{cor}[Weak Soundness]
\label{DED4-cor:weak-soundness}
If $\Proves !A$ then $!A$ is valid.
\end{cor}

\begin{cor}
\label{DED4-cor:entailment-soundness}
If $\Gamma \Proves !A$ then $\Gamma \Entails !A$.
\end{cor}

\begin{proof}
If $\Gamma \Proves !A$ then for some finite $\Gamma_0 \subseteq
\Gamma$, there is a derivation of $\Gamma_0 \Sequent !A$. By
\cref{DED4-thm:sequent-soundness}, every structure~$\Struct{M}$
either falsifies some $!B \in \Gamma_0$ or satisfies~$!A$. Hence, if
$\Sat{M}{\Gamma}$ then $\Sat{M}{!A}$.
\end{proof}

\begin{cor}
\label{DED4-cor:consistency-soundness}
If $\Gamma$ is satisfiable, then it is consistent.
\end{cor}

\begin{proof}
Contrapositive: if $\Gamma$ is inconsistent, then there is a finite
$\Gamma_0 \subseteq \Gamma$ and a derivation of $\Gamma_0 \Sequent$.
By \cref{DED4-thm:sequent-soundness}, $\Gamma_0 \Sequent$ is valid,
i.e., for every $\Struct{M}$ there is $!C \in \Gamma_0$ with
$\Sat/{M}{!C}$. Hence no structure satisfies~$\Gamma$.
\end{proof}

\begin{rem}
In the sequent calculus, derivability is expressed via $\Gamma_0
\Sequent !A$. Transitivity corresponds to the $\Cut$ rule.
\end{rem}


%% ===================================================================
%% DED.5: Tableaux
%% Sources: tab/rul (KEEP), tab/prl (KEEP), tab/qrl (KEEP),
%%          tab/der (CONDENSE), tab/sou (KEEP+ABSORB sid),
%%          tab/ide (CONDENSE)
%% ===================================================================

\section{Tableaux} \label{DED.5}

Tableaux instantiate the generic framework of \S\ref{DED.1} by
working \emph{refutationally}: to show $\Gamma \Proves !A$, one
attempts to build a systematic survey of all ways the assumptions
in~$\Gamma$ could be true and~$!A$ false, and demonstrates that every
such possibility leads to a contradiction. Derivations
(\ref{PRIM-DED005}) are finitely branching trees of \emph{signed
formulas} rather than trees of plain formulas or sequents. Provability
(\ref{PRIM-DED006}) and consistency (\ref{DEF-DED001}) are defined as
in \S\ref{DED.1}: $\Gamma \Proves !A$ iff $\{\sFmla{\True}{!B_1},
\dots, \sFmla{\True}{!B_n}, \sFmla{\False}{!A}\}$ has a closed
tableau for some $!B_1, \dots, !B_n \in \Gamma$.

%%% -----------------------------------------------------------------
%%% DED.5.1  Signed Formulas and Tableau Rules
%%% -----------------------------------------------------------------

\subsection{Signed Formulas and Tableau Rules} % DEF-DED008
\label{DEF-DED008}

\begin{defn}[Signed Formula]
A \emph{signed formula} is a pair consisting of a truth value sign and
a sentence:
\[
\sFmla{\True}{!A} \quad\text{or}\quad \sFmla{\False}{!A}.
\]
\end{defn}

Intuitively, $\sFmla{\True}{!A}$ means ``$!A$ might be true'' and
$\sFmla{\False}{!A}$ means ``$!A$ might be false'' in some structure.

Each signed formula in a tableau is either an \emph{assumption} (at
the top) or obtained from a signed formula above it by a rule of
inference. There are two rules per main operator---one for sign~$\True$
and one for sign~$\False$---and some rules branch the tree.

A branch is \emph{closed} when it contains both $\sFmla{\True}{!A}$
and $\sFmla{\False}{!A}$. A \emph{closed tableau} is one where every
branch is closed. A closed tableau \emph{for $!A$} is a closed
tableau with root~$\sFmla{\False}{!A}$. If such a closed tableau
exists, all possibilities for~$!A$ being false have been ruled out.

%%% -----------------------------------------------------------------
%%% DED.5.2  Propositional Rules
%%% -----------------------------------------------------------------

\subsection{Propositional Rules}

\subsubsection*{Rules for $\lnot$}

\begin{defish}
\AxiomC{\sFmla{\True}{\lnot !A}}
\RightLabel{\TRule{\True}{\lnot}}
\UnaryInfC{\sFmla{\False}{!A}}
\DisplayProof
\hfill
\AxiomC{\sFmla{\False}{\lnot !A}}
\RightLabel{\TRule{\False}{\lnot}}
\UnaryInfC{\sFmla{\True}{!A}}
\DisplayProof
\end{defish}

\subsubsection*{Rules for $\land$}

\begin{defish}\noindent
\AxiomC{\sFmla{\True}{!A \land !B}}
\RightLabel{\TRule{\True}{\land}}
\UnaryInfC{\sFmla{\True}{!A}}
\noLine
\UnaryInfC{\sFmla{\True}{!B}}
\DisplayProof
\hfill
\AxiomC{\sFmla{\False}{!A \land !B}}
\RightLabel{\TRule{\False}{\land}}
\UnaryInfC{$\sFmla{\False}{!A} \quad \mid \quad \sFmla{\False}{!B}$}
\DisplayProof
\end{defish}

\subsubsection*{Rules for $\lor$}

\begin{defish}
\AxiomC{\sFmla{\True}{!A \lor !B}}
\RightLabel{\TRule{\True}{\lor}}
\UnaryInfC{$\sFmla{\True}{!A} \quad \mid \quad \sFmla{\True}{!B}$}
\DisplayProof
\hfill
\AxiomC{\sFmla{\False}{!A \lor !B}}
\RightLabel{\TRule{\False}{\lor}}
\UnaryInfC{\sFmla{\False}{!A}}
\noLine
\UnaryInfC{\sFmla{\False}{!B}}
\DisplayProof
\end{defish}

\subsubsection*{Rules for $\lif$}

\begin{defish}
\AxiomC{\sFmla{\True}{!A \lif !B}}
\RightLabel{\TRule{\True}{\lif}}
\UnaryInfC{$\sFmla{\False}{!A} \quad \mid \quad \sFmla{\True}{!B}$}
\DisplayProof
\hfill
\AxiomC{\sFmla{\False}{!A \lif !B}}
\RightLabel{\TRule{\False}{\lif}}
\UnaryInfC{\sFmla{\True}{!A}}
\noLine
\UnaryInfC{\sFmla{\False}{!B}}
\DisplayProof
\end{defish}

\subsubsection*{The Cut Rule}

\begin{defish}
\AxiomC{}
\RightLabel{\Cut}
\UnaryInfC{$\sFmla{\True}{!A} \quad \mid \quad \sFmla{\False}{!A}$}
\DisplayProof
\end{defish}

The $\Cut$ rule splits every branch in two. It is not necessary (any
set of signed formulas with a closed tableau has one not using $\Cut$),
but it allows convenient combination of tableaux.

%%% -----------------------------------------------------------------
%%% DED.5.3  Quantifier Rules
%%% -----------------------------------------------------------------

\subsection{Quantifier Rules}

\subsubsection*{Rules for $\lforall$}

\begin{defish}
\AxiomC{\sFmla{\True}{\lforall[x][!A(x)]}}
\RightLabel{\TRule{\True}{\forall}}
\UnaryInfC{\sFmla{\True}{!A(t)}}
\DisplayProof
\hfill
\AxiomC{\sFmla{\False}{\lforall[x][!A(x)]}}
\RightLabel{\TRule{\False}{\lforall}}
\UnaryInfC{\sFmla{\False}{!A(a)}}
\DisplayProof
\end{defish}

In $\TRule{\True}{\lforall}$, $t$ is a closed term. In
$\TRule{\False}{\lforall}$, $a$ is a constant symbol not occurring
anywhere in the branch above. We call $a$ the \emph{eigenvariable}.

\subsubsection*{Rules for $\lexists$}

\begin{defish}
\AxiomC{\sFmla{\True}{\lexists[x][!A(x)]}}
\RightLabel{\TRule{\True}{\lexists}}
\UnaryInfC{\sFmla{\True}{!A(a)}}
\DisplayProof
\hfill
\AxiomC{\sFmla{\False}{\lexists[x][!A(x)]}}
\RightLabel{\TRule{\False}{\lexists}}
\UnaryInfC{\sFmla{\False}{!A(t)}}
\DisplayProof
\end{defish}

Again, $t$ is a closed term, and $a$ is a constant symbol not
occurring in the branch above. The \emph{eigenvariable condition}
requires that $a$ not occur in the branch above the
$\TRule{\False}{\lforall}$ or $\TRule{\True}{\lexists}$ inference.

\subsubsection*{Identity Rules}

\begin{defish}
\AxiomC{}
\RightLabel{$\eq$}
\UnaryInfC{\sFmla{\True}{\eq[t][t]}}
\DisplayProof
\hfill
\AxiomC{\sFmla{\True}{\eq[t_1][t_2]}}
\noLine
\UnaryInfC{\sFmla{\True}{!A(t_1)}}
\RightLabel{$\TRule{\True}{\eq}$}
\UnaryInfC{\sFmla{\True}{!A(t_2)}}
\DisplayProof
\hfill
\AxiomC{\sFmla{\True}{\eq[t_1][t_2]}}
\noLine
\UnaryInfC{\sFmla{\False}{!A(t_1)}}
\RightLabel{$\TRule{\False}{\eq}$}
\UnaryInfC{\sFmla{\False}{!A(t_2)}}
\DisplayProof
\end{defish}

In contrast to the other rules, $\TRule{\True}{\eq}$ and
$\TRule{\False}{\eq}$ require two signed formulas already on the
branch: both $\sFmla{\True}{\eq[t_1][t_2]}$ and $\sFmla{S}{!A(t_1)}$.

%%% -----------------------------------------------------------------
%%% DED.5.4  Tableaux
%%% -----------------------------------------------------------------

\subsection{Tableaux}

\begin{defn}[Tableau]
\label{DED5-defn:tableau}
A \emph{tableau} for assumptions $\sFmla{S_1}{!A_1}$, \dots,
$\sFmla{S_n}{!A_n}$ (where each $S_i$ is $\True$ or~$\False$) is a
finite tree of signed formulas satisfying:
\begin{enumerate}
\item The $n$ topmost signed formulas are $\sFmla{S_i}{!A_i}$, one
  below the other.
\item Every signed formula not among the assumptions results from a
  correct application of an inference rule to a signed formula in the
  branch above it.
\end{enumerate}
A branch is \emph{closed} iff it contains both $\sFmla{\True}{!A}$
and~$\sFmla{\False}{!A}$, and \emph{open} otherwise. A tableau with
every branch closed is a \emph{closed tableau}; otherwise it is
\emph{open}.
\end{defn}

%%% -----------------------------------------------------------------
%%% DED.5.5  Soundness (Tableaux)
%%% -----------------------------------------------------------------

\subsection{Soundness} \label{DED.5.sou}

\begin{defn}[Satisfaction of Signed Formulas]
\label{DED5-defn:satisfies-signed}
A structure~$\Struct{M}$ \emph{satisfies} $\sFmla{\True}{!A}$ iff
$\Sat{M}{!A}$, and satisfies $\sFmla{\False}{!A}$ iff $\Sat/{M}{!A}$.
$\Struct{M}$ satisfies a set~$\Gamma$ of signed formulas iff it
satisfies every member.  $\Gamma$ is \emph{satisfiable} if some
structure satisfies it, and \emph{unsatisfiable} otherwise.
\end{defn}

\begin{thm}[Soundness] % CP-001(Tab)
\label{DED5-thm:tableau-soundness}
If $\Gamma$ has a closed tableau, $\Gamma$ is unsatisfiable.
\end{thm}

\begin{proof}
Call a branch \emph{satisfiable} iff the set of signed formulas on it
is satisfiable, and call a tableau \emph{satisfiable} if it has at
least one satisfiable branch.

We show: extending a satisfiable tableau by one rule of inference
always results in a satisfiable tableau. This proves the theorem: a
closed tableau results from applying rules to the tableau consisting
only of the assumptions from~$\Gamma$. If $\Gamma$ were satisfiable,
the initial tableau would be satisfiable, and hence every extension
would be satisfiable. But a closed tableau is clearly not satisfiable:
every branch contains both $\sFmla{\True}{!A}$ and
$\sFmla{\False}{!A}$.

Suppose we have a satisfiable tableau with a satisfiable branch to
which a rule is applied. Let $\Gamma$ be the set of signed formulas on
that branch, and let $\sFmla{S}{!A} \in \Gamma$ be the signed formula
to which the rule is applied. If the rule does not split, we show that
the extended branch is satisfiable. If the rule splits, we show at
least one resulting branch is satisfiable.

\emph{Non-splitting rules:}
\begin{enumerate}
\item \emph{$\TRule{\True}{\lnot}$ applied to
$\sFmla{\True}{\lnot !B}$:} The extended branch contains $\Gamma \cup
\{\sFmla{\False}{!B}\}$. If $\Sat{M}{\Gamma}$, then
$\Sat{M}{\lnot !B}$, so $\Sat/{M}{!B}$, i.e., $\Struct{M}$ satisfies
$\sFmla{\False}{!B}$.

\item \emph{$\TRule{\False}{\lnot}$ applied to
$\sFmla{\False}{\lnot !B}$:} If $\Sat/{M}{\lnot !B}$, then
$\Sat{M}{!B}$, so $\Struct{M}$ satisfies $\sFmla{\True}{!B}$.

\item \emph{$\TRule{\True}{\land}$ applied to
$\sFmla{\True}{!B \land !C}$:} If $\Sat{M}{!B \land !C}$, then
$\Sat{M}{!B}$ and $\Sat{M}{!C}$, so $\Struct{M}$ satisfies both new
signed formulas.

\item \emph{$\TRule{\False}{\lor}$ applied to
$\sFmla{\False}{!B \lor !C}$:} If $\Sat/{M}{!B \lor !C}$, then
$\Sat/{M}{!B}$ and $\Sat/{M}{!C}$.

\item \emph{$\TRule{\False}{\lif}$ applied to
$\sFmla{\False}{!B \lif !C}$:} If $\Sat/{M}{!B \lif !C}$, then
$\Sat{M}{!B}$ and $\Sat/{M}{!C}$.

\item \emph{$\TRule{\True}{\lforall}$ applied to
$\sFmla{\True}{\lforall[x][!A(x)]}$:} This adds
$\sFmla{\True}{!A(t)}$. If $\Sat{M}{\lforall[x][!A(x)]}$, by the
substitution lemma $\Sat{M}{!A(t)}$.

\item \emph{$\TRule{\False}{\lforall}$ applied to
$\sFmla{\False}{\lforall[x][!A(x)]}$:} This adds
$\sFmla{\False}{!A(a)}$ where $a$ does not occur in~$\Gamma$. Since
$\Sat/{M}{\lforall[x][!A(x)]}$, for some variable assignment~$s$,
$\Sat/{M}{!A(x)}[s]$. Let $\Struct{M'}$ be like~$\Struct{M}$ except
$\Assign{a}{M'} = s(x)$. Since $a$ does not occur in~$\Gamma$,
$\Struct{M'}$ still satisfies~$\Gamma$. By the substitution lemma,
$\Sat/{M'}{!A(a)}$, so $\Struct{M'}$ satisfies
$\sFmla{\False}{!A(a)}$.

\item \emph{$\TRule{\True}{\lexists}$ applied to
$\sFmla{\True}{\lexists[x][!A(x)]}$:} Symmetric to the
$\TRule{\False}{\lforall}$ case, constructing $\Struct{M'}$ with
$\Assign{a}{M'} = s(x)$ for a suitable~$s$.

\item \emph{$\TRule{\False}{\lexists}$ applied to
$\sFmla{\False}{\lexists[x][!A(x)]}$:} Symmetric to the
$\TRule{\True}{\lforall}$ case.

\item \emph{Identity rule~$\eq$} (adding $\sFmla{\True}{\eq[t][t]}$):
Trivially $\Sat{M}{\eq[t][t]}$.

\item \emph{$\TRule{\True}{\eq}$} (adding $\sFmla{\True}{!A(t_2)}$
from $\sFmla{\True}{\eq[t_1][t_2]}$ and $\sFmla{\True}{!A(t_1)}$):
Since $\Value{t_1}{M} = \Value{t_2}{M}$, the substitution lemma gives
$\Sat{M}{!A(t_2)}$. The $\TRule{\False}{\eq}$ case is similar.
\end{enumerate}

\emph{Splitting rules:}
\begin{enumerate}
\item \emph{$\TRule{\False}{\land}$ applied to
$\sFmla{\False}{!B \land !C}$:} Splits into $\sFmla{\False}{!B}$ and
$\sFmla{\False}{!C}$. If $\Sat/{M}{!B \land !C}$, then
$\Sat/{M}{!B}$ or $\Sat/{M}{!C}$; accordingly $\Struct{M}$ satisfies
the left or right branch.

\item \emph{$\TRule{\True}{\lor}$ applied to
$\sFmla{\True}{!B \lor !C}$:} Splits into $\sFmla{\True}{!B}$ and
$\sFmla{\True}{!C}$. Since $\Sat{M}{!B}$ or $\Sat{M}{!C}$, at
least one branch is satisfiable.

\item \emph{$\TRule{\True}{\lif}$ applied to
$\sFmla{\True}{!B \lif !C}$:} Splits into $\sFmla{\False}{!B}$ and
$\sFmla{\True}{!C}$. Since either $\Sat/{M}{!B}$ or $\Sat{M}{!C}$,
at least one branch is satisfiable.

\item \emph{$\Cut$:} Splits into $\sFmla{\True}{!B}$ and
$\sFmla{\False}{!B}$. Since either $\Sat{M}{!B}$ or $\Sat/{M}{!B}$,
at least one branch is satisfiable.
\end{enumerate}
\end{proof}

\begin{cor}[Weak Soundness]
\label{DED5-cor:weak-soundness}
If $\Proves !A$ then $!A$ is valid.
\end{cor}

\begin{cor}
\label{DED5-cor:entailment-soundness}
If $\Gamma \Proves !A$ then $\Gamma \Entails !A$.
\end{cor}

\begin{proof}
If $\Gamma \Proves !A$ then for some $!B_1, \dots, !B_n \in \Gamma$,
$\{\sFmla{\False}{!A}, \sFmla{\True}{!B_1}, \dots,
\sFmla{\True}{!B_n}\}$ has a closed tableau. By
\cref{DED5-thm:tableau-soundness}, every structure~$\Struct{M}$ either
falsifies some~$!B_i$ or satisfies~$!A$. Hence if
$\Sat{M}{\Gamma}$ then $\Sat{M}{!A}$.
\end{proof}

\begin{cor}
\label{DED5-cor:consistency-soundness}
If $\Gamma$ is satisfiable, then it is consistent.
\end{cor}

\begin{proof}
Contrapositive: if $\Gamma$ is inconsistent, then there are $!B_1,
\dots, !B_n \in \Gamma$ with a closed tableau for
$\{\sFmla{\True}{!B_1}, \dots, \sFmla{\True}{!B_n}\}$. By
\cref{DED5-thm:tableau-soundness}, no structure satisfies all~$!B_i$,
so $\Gamma$ is unsatisfiable.
\end{proof}

\begin{rem}
Tableau consistency---the absence of a closed tableau from $\True$-signed
assumptions---is a reformulation of the generic consistency notion of
\S\ref{DED.1}. Transitivity uses the $\Cut$ rule.
\end{rem}


%% ===================================================================
%% DED.6: Theories and Arithmetic
%% Sources: inc/int/def (DISTRIBUTE: Q, PA), inc/tcp/oqn (DEF only),
%%          inc/inp/prc (DEF-DED014)
%% ===================================================================

\section{Theories and Arithmetic} \label{DED.6}

This section introduces the formal theories of arithmetic that serve
as the principal objects of study in the incompleteness theorems (see
CH-BST, Boundedness). We define Robinson's~$\Th{Q}$, Peano
Arithmetic~$\Th{PA}$, the notion of $\omega$-consistency, and the
derivability conditions that any sufficiently strong theory must
satisfy for the incompleteness theorems to apply.

%%% -----------------------------------------------------------------
%%% DED.6.1  Robinson Arithmetic Q
%%% -----------------------------------------------------------------

\subsection{Robinson Arithmetic $\Th{Q}$}

The natural language in which to express facts of arithmetic is~$\Lang
L_A$.  $\Lang L_A$ contains a single two-place predicate symbol~$<$, a
single constant symbol~$\Obj 0$, one one-place function symbol~$\prime$,
and two two-place function symbols~$+$ and~$\times$ (see
PRIM-SYN009, Language, \S\ref{SYN.1} for the general notion of a
first-order language).

\begin{defn}[Robinson Arithmetic $\Th{Q}$] % DEF-DED011
\label{DEF-DED011}
The theory $\Th{Q}$ axiomatized by the following sentences is known
as ``Robinson's $\Th{Q}$'' and is a very simple theory of arithmetic.
\begin{align*}
& \lforall[x][\lforall[y][(\eq[x'][y'] \lif \eq[x][y])]]
  \tag{$!Q_1$}\\
& \lforall[x][\eq/[\Obj 0][x']] \tag{$!Q_2$}\\
& \lforall[x][(\eq[x][\Obj 0] \lor \lexists[y][\eq[x][y']])]
  \tag{$!Q_3$}\\
& \lforall[x][\eq[(x + \Obj 0)][x]] \tag{$!Q_4$}\\
& \lforall[x][\lforall[y][\eq[(x + y')][(x + y)']]] \tag{$!Q_5$}\\
& \lforall[x][\eq[(x \times \Obj 0)][\Obj 0]] \tag{$!Q_6$}\\
& \lforall[x][\lforall[y][\eq[(x \times y')][((x \times y) + x)]]]
  \tag{$!Q_7$}\\
& \lforall[x][\lforall[y][(x < y \liff
  \lexists[z][\eq[(z' + x)][y]])]] \tag{$!Q_8$}
\end{align*}
The sentences $\{!Q_1, \dots, !Q_8\}$ are the axioms of~$\Th{Q}$, so
\[
\Th{Q} = \Setabs{!A}{\{!Q_1, \dots, !Q_8\} \Entails !A}.
\]
\end{defn}

The axioms~$!Q_1$ and~$!Q_2$ express that the successor function is
injective with $0$ not in its range. Axiom~$!Q_3$ says every number is
either~$0$ or a successor.  Axioms~$!Q_4$--$!Q_7$ give the recursive
definitions of addition and multiplication.  Axiom~$!Q_8$ defines the
ordering in terms of addition.

$\Th{Q}$ is weak: it cannot even prove the commutativity of addition.
Its importance lies in the fact that $\Th{Q}$ is strong enough to
represent all computable functions and all decidable relations (see
\S\ref{CMP.4}), which is the key hypothesis of the incompleteness theorems.
Since any theory that extends~$\Th{Q}$ inherits this representability,
the incompleteness theorems apply to a wide class of theories.

%%% -----------------------------------------------------------------
%%% DED.6.2  Peano Arithmetic PA
%%% -----------------------------------------------------------------

\subsection{Peano Arithmetic $\Th{PA}$}

\begin{defn}[Peano Arithmetic $\Th{PA}$] % DEF-DED012
\label{DEF-DED012}
Suppose $!A(x)$ is a formula in $\Lang L_A$ with free variables~$x$
and $y_1$, \dots, $y_n$.  Then any sentence of the form
\[
\lforall[y_1][\dots\lforall[y_n][((!A(\Obj 0) \land \lforall[x][(!A(x)
\lif !A(x'))]) \lif \lforall[x][!A(x)])]]
\]
is an instance of the \emph{induction schema}.

\emph{Peano Arithmetic}~$\Th{PA}$ is the theory axiomatized by the
axioms of $\Th{Q}$ together with all instances of the induction schema.
\end{defn}

Every instance of the induction schema is true in the standard model
of arithmetic~$\Struct{N}$ (see DEF-SEM012, Standard Model, \S\ref{SEM.5}).
If $!A(x)$ defines a subset~$X_{!A}$ of~$\Nat$ in~$\Struct{N}$, then the
induction schema asserts that if $0 \in X_{!A}$ and $X_{!A}$ is closed
under the successor function, then $X_{!A} = \Nat$.

The induction schema is genuinely a \emph{schema}: it generates
infinitely many axioms, and $\Th{PA}$ is not finitely axiomatizable.
However, since one can effectively determine whether a string of
symbols is an instance of an induction axiom, the set of axioms
for~$\Th{PA}$ is decidable, and $\Th{PA}$ is an axiomatizable theory
in the sense of PRIM-DED002.

$\Th{PA}$ is a much more robust theory than~$\Th{Q}$: one can prove
the commutativity and associativity of addition and multiplication, and
in fact most finitary number-theoretic and combinatorial arguments can
be carried out in~$\Th{PA}$.

%%% -----------------------------------------------------------------
%%% DED.6.3  Omega-Consistency
%%% -----------------------------------------------------------------

\subsection{$\omega$-Consistency}

\begin{defn}[$\omega$-Consistency] % DEF-DED013
\label{DEF-DED013}
A theory $\Th{T}$ is \emph{$\omega$-consistent} if the following
holds: if $\lexists[x][!A(x)]$ is any sentence and $\Th{T}$ proves
$\lnot !A(\num 0)$, $\lnot !A(\num 1)$, $\lnot !A(\num 2)$, \dots,
then $\Th{T}$ does not prove $\lexists[x][!A(x)]$.
\end{defn}

$\omega$-consistency is strictly stronger than ordinary consistency:
every $\omega$-consistent theory is consistent, but the converse fails.
G\"odel's original 1931 proof of the first incompleteness theorem
assumed $\omega$-consistency.  Rosser subsequently strengthened the
result by replacing $\omega$-consistency with ordinary consistency (see
Rosser's Theorem, \S\ref{META.5}).

%%% -----------------------------------------------------------------
%%% DED.6.4  Derivability Conditions
%%% -----------------------------------------------------------------

\subsection{Derivability Conditions}

\begin{defn}[Derivability Conditions] % DEF-DED014 (authoritative: \S META.6)
Let $\Th{T}$ be an axiomatizable theory extending~$\Th{Q}$, and let
$\OProv[\Th{T}](y)$ be a formula representing the derivability
predicate for~$\Th{T}$ (i.e., $\OProv[\Th{T}](y) \defis
\lexists[x][\OPrf[\Th{T}](x,y)]$, where $\OPrf[\Th{T}](x,y)$
represents the proof relation).  The \emph{derivability conditions} (also
called Hilbert--Bernays--L\"ob conditions) are:
\begin{enumerate}
\item[\textbf{P1.}] If $\Th{T} \Proves !A$, then $\Th{T} \Proves
  \OProv[\Th{T}](\gn{!A})$.
\item[\textbf{P2.}] For all formulas $!A$ and $!B$,
  \[
  \Th{T} \Proves \OProv[\Th{T}](\gn{!A \lif !B}) \lif
  (\OProv[\Th{T}](\gn{!A}) \lif \OProv[\Th{T}](\gn{!B})).
  \]
\item[\textbf{P3.}] For every formula~$!A$,
  \[
  \Th{T} \Proves \OProv[\Th{T}](\gn{!A})
  \lif \OProv[\Th{T}](\gn{\OProv[\Th{T}](\gn{!A})}).
  \]
\end{enumerate}
\end{defn}

Condition P1 says that $\Th{T}$ is aware of its own theorems: if it
proves~$!A$, it proves that it proves~$!A$.  Condition P2 says that
the provability predicate distributes over the conditional, so that
$\Th{T}$ can reason internally about modus ponens.  Condition P3 is a
form of positive introspection: $\Th{T}$ can verify its own provability.

All three conditions hold for $\Th{PA}$ (and, more generally, for any
axiomatizable extension of~$\Th{Q}$ with a suitably chosen provability
predicate).  Conditions P1 and P2 are relatively easy to verify; P3
requires substantial formalization of proof theory inside~$\Th{T}$
itself.

The derivability conditions are the essential hypotheses for L\"ob's
Theorem (see THM-DED007, \S\ref{DED.7}) and the Second Incompleteness
Theorem (see CP-006, \S\ref{META.6}).


%% ===================================================================
%% DED.7: Theorems
%% Sources: axd/ded (THM-DED001 cross-ref), com/lin (THM-DED005),
%%          inc/inp/fix (THM-DED006), inc/inp/lob (THM-DED007)
%% ===================================================================

\section{Theorems} \label{DED.7}

This section collects the principal theorems of deduction theory.

%%% -----------------------------------------------------------------
%%% DED.7.1  The Deduction Theorem
%%% -----------------------------------------------------------------

\subsection{The Deduction Theorem}

\begin{thm}[Deduction Theorem] % THM-DED001
\label{THM-DED001}
$\Gamma \cup \{!A\} \Proves !B$ if and only if $\Gamma \Proves !A
\lif !B$.
\end{thm}

The deduction theorem is a fundamental metatheorem relating assumption
introduction to the conditional connective.  Its proof depends on the
proof system:
\begin{itemize}
\item In axiomatic deduction, the proof is by induction on the length
  of the derivation and uses the logical axiom $!A \lif (!B \lif !A)$
  and the distribution axiom $(!A \lif (!B \lif !C)) \lif ((!A \lif
  !B) \lif (!A \lif !C))$.  See DED.2 for the full proof.
\item In natural deduction, the deduction theorem is immediate from
  the $\Intro{\lif}$ rule, which allows one to discharge an
  assumption~$!A$ and conclude $!A \lif !B$.  See DED.3.
\item In the sequent calculus, the deduction theorem corresponds to
  the right conditional rule $\RightR{\lif}$, which moves a formula
  from the antecedent to the succedent.  See DED.4.
\end{itemize}

%%% -----------------------------------------------------------------
%%% DED.7.2  Lindenbaum's Lemma
%%% -----------------------------------------------------------------

\subsection{Lindenbaum's Lemma}

\begin{lem}[Lindenbaum's Lemma] % THM-DED005 (authoritative: \S META.2)
Every consistent set~$\Gamma$ in a language~$\Lang{L}$ can be
extended to a complete and consistent set~$\Gamma^*$.
\end{lem}

\begin{proof}
Let $\Gamma$ be consistent.  Let $!A_0, !A_1, \dots$ be an
enumeration of all the sentences of~$\Lang L$.  Define $\Gamma_0 =
\Gamma$, and
\[
\Gamma_{n+1} =
\begin{cases}
\Gamma_n \cup \{ !A_n \} & \text{if $\Gamma_n \cup \{!A_n\}$ is
  consistent;} \\
\Gamma_n \cup \{ \lnot !A_n \} & \text{otherwise.}
\end{cases}
\]
Let $\Gamma^* = \bigcup_{n \geq 0} \Gamma_n$.

Each $\Gamma_n$ is consistent: $\Gamma_0$ is consistent by definition.
If $\Gamma_{n+1} = \Gamma_n \cup \{!A_n\}$, this is because the latter
is consistent.  If it is not, $\Gamma_{n+1} = \Gamma_n \cup \{\lnot
!A_n\}$. We verify that $\Gamma_n \cup \{\lnot !A_n\}$ is consistent.
If it were not, then \emph{both} $\Gamma_n \cup \{!A_n\}$ and $\Gamma_n
\cup \{\lnot !A_n\}$ would be inconsistent. By
\cref{DED-prop:provability-exhaustive}, $\Gamma_n$ would be
inconsistent, contradicting the induction hypothesis.

For every~$n$ and every $i < n$, $\Gamma_i \subseteq \Gamma_n$. This
follows by a simple induction on~$n$.

From this it follows that $\Gamma^*$ is consistent: let $\Gamma'
\subseteq \Gamma^*$ be finite. Each $!B \in \Gamma'$ is also
in~$\Gamma_i$ for some~$i$.  Let $n$ be the largest of these. Since
$\Gamma_i \subseteq \Gamma_n$ if $i \le n$, every $!B \in \Gamma'$ is
also $\in \Gamma_n$, i.e., $\Gamma' \subseteq \Gamma_n$, and
$\Gamma_n$~is consistent.  So every finite subset $\Gamma' \subseteq
\Gamma^*$ is consistent.  By compactness
(\cref{DED-prop:proves-compact}), $\Gamma^*$ is consistent.

Every sentence of $\Frm[L]$ appears on the list used to
define~$\Gamma^*$.  If $!A_n \notin \Gamma^*$, then that is because
$\Gamma_n \cup \{!A_n\}$ was inconsistent.  But then $\lnot !A_n
\in \Gamma^*$, so $\Gamma^*$ is complete.
\end{proof}

%%% -----------------------------------------------------------------
%%% DED.7.3  The Fixed-Point Lemma
%%% -----------------------------------------------------------------

\subsection{The Fixed-Point Lemma}

The fixed-point lemma (also called the diagonal lemma or
self-referential lemma) is the engine behind the incompleteness
theorems.  It guarantees that any expressible property can be asserted
of its own G\"odel number.

\begin{lem}[Fixed-Point Lemma] % THM-DED006 (authoritative: \S META.5)
Let $!B(x)$ be any formula with one free variable~$x$. Then there is a
sentence~$!A$ such that $\Th{Q} \Proves !A \liff !B(\gn{!A})$.
\end{lem}

\begin{proof}
Given $!B(x)$, let $!E(x)$ be the formula
$\lexists[y][(!D_{\fn{diag}}(x,y) \land !B(y))]$, where
$!D_{\fn{diag}}(x,y)$ is a formula representing the primitive
recursive diagonalization function~$\fn{diag}$ in~$\Th{Q}$. The
function $\fn{diag}(n)$ computes, given the G\"odel number~$n$ of a
formula~$!E(x)$, the G\"odel number of its diagonalization
$!E(\gn{!E(x)})$.

Let $!A$ be the diagonalization of~$!E(x)$, i.e., $!A$ is
$!E(\gn{!E(x)})$.

Since $!D_{\fn{diag}}$ represents $\fn{diag}$, and
$\fn{diag}(\Gn{!E(x)}) = \Gn{!A}$, $\Th{Q}$ can derive:
\begin{align}
  & !D_{\fn{diag}}(\gn{!E(x)}, \gn{!A}) \label{repdiag1} \\
  & \lforall[y][(!D_{\fn{diag}}(\gn{!E(x)},y) \lif
  \eq[y][\gn{!A}])]. \label{repdiag2}
\end{align}
We show that $\Th{Q} \Proves !A \liff !B(\gn{!A})$, arguing informally
using logic and facts derivable in~$\Th{Q}$.

\emph{Forward direction.} Suppose~$!A$, i.e., $!E(\gn{!E(x)})$, which
by definition of~$!E(x)$ is
\[
\lexists[y][(!D_{\fn{diag}}(\gn{!E(x)},y) \land !B(y))].
\]
Consider such a~$y$. Since $!D_{\fn{diag}}(\gn{!E(x)},y)$, by
\eqref{repdiag2}, $y = \gn{!A}$. So from $!B(y)$ we
have~$!B(\gn{!A})$.

\emph{Reverse direction.} Suppose $!B(\gn{!A})$. By
\eqref{repdiag1}, we have
\[
!D_{\fn{diag}}(\gn{!E(x)}, \gn{!A}) \land !B(\gn{!A}).
\]
It follows that
\[
\lexists[y][(!D_{\fn{diag}}(\gn{!E(x)},y) \land !B(y))],
\]
which is $!E(\gn{!E(x)})$, i.e.,~$!A$.
\end{proof}

%%% -----------------------------------------------------------------
%%% DED.7.4  L\"ob's Theorem
%%% -----------------------------------------------------------------

\subsection{L\"ob's Theorem}

\begin{thm}[L\"ob's Theorem] % THM-DED007 (authoritative: \S META.6)
Let $\Th{T}$ be an axiomatizable theory extending $\Th{Q}$, and
suppose $\OProv[\Th{T}](y)$ is a formula satisfying the derivability
conditions P1--P3 (see DEF-DED014, \S\ref{DED.6}). If $\Th{T}$ derives
$\OProv[\Th{T}](\gn{!A}) \lif !A$, then in fact $\Th{T}$ derives $!A$.
\end{thm}

Equivalently: if $\Th{T} \Proves/ !A$, then $\Th{T} \Proves/
\OProv[\Th{T}](\gn{!A}) \lif !A$.  The schema
$\OProv[\Th{T}](\gn{!A}) \lif !A$ is called the \emph{reflection
principle}; L\"ob's theorem says that a consistent theory can only
derive those instances of the reflection principle where~$!A$ is
already a theorem.

\begin{proof}
Suppose $!A$ is a sentence such that $\Th{T}$ derives
$\OProv[\Th{T}](\gn{!A}) \lif !A$.  Let $!B(y)$ be the
formula~$\OProv[\Th{T}](y) \lif !A$, and use the fixed-point lemma
(THM-DED006) to find a sentence~$!D$ such that $\Th{T}$
derives $!D \liff !B(\gn{!D})$. Then each of the following is
derivable in $\Th{T}$:
\begin{align}
  & !D \liff (\OProv[\Th{T}](\gn{!D}) \lif !A) \label{L-1}\\
  & \qquad \text{$!D$ is a fixed point of~$!B(y)$}\notag \\
  & !D \lif (\OProv[\Th{T}](\gn{!D}) \lif !A) \label{L-2}\\
  & \qquad\text{from \eqref{L-1}}\notag\\
  & \OProv[\Th{T}](\gn{!D \lif (\OProv[\Th{T}](\gn{!D}) \lif !A)})
    \label{L-3}\\
  & \qquad \text{from \eqref{L-2} by condition P1}\notag \\
  & \OProv[\Th{T}](\gn{!D}) \lif
    \OProv[\Th{T}](\gn{\OProv[\Th{T}](\gn{!D}) \lif !A})
    \label{L-4}\\
  &\qquad \text{from \eqref{L-3} using condition P2}\notag \\
  & \OProv[\Th{T}](\gn{!D}) \lif
    (\OProv[\Th{T}](\gn{\OProv[\Th{T}](\gn{!D})}) \lif
    \OProv[\Th{T}](\gn{!A})) \label{L-5}\\
  &\qquad \text{from \eqref{L-4} using P2 again} \notag\\
  & \OProv[\Th{T}](\gn{!D}) \lif
    \OProv[\Th{T}](\gn{\OProv[\Th{T}](\gn{!D})}) \label{L-6}\\
  & \qquad\text{by derivability condition P3} \notag\\
  & \OProv[\Th{T}](\gn{!D}) \lif \OProv[\Th{T}](\gn{!A}) \label{L-7}\\
  &\qquad\text{from \eqref{L-5} and \eqref{L-6}}\notag\\
  & \OProv[\Th{T}](\gn{!A}) \lif !A \label{L-8}\\
  &\qquad\text{by assumption of the theorem} \notag\\
  & \OProv[\Th{T}](\gn{!D}) \lif !A \label{L-9}\\
  &\qquad\text{from \eqref{L-7} and \eqref{L-8}}\notag\\
  & (\OProv[\Th{T}](\gn{!D}) \lif !A) \lif !D \label{L-10}\\
  & \qquad \text{from \eqref{L-1}}\notag \\
  & !D \label{L-11}\\
  & \qquad\text{from \eqref{L-9} and \eqref{L-10}}\notag \\
  & \OProv[\Th{T}](\gn{!D}) \label{L-12}\\
  & \qquad\text{from \eqref{L-11} by condition P1}\notag \\
  & !A \qquad\qquad\text{from \eqref{L-8} and \eqref{L-12}}\notag
\end{align}
\end{proof}

\begin{rem}
With L\"ob's theorem in hand, there is a short proof of the second
incompleteness theorem (see CP-006, Second Incompleteness Theorem,
\S\ref{META.6}).  Take $!A \defis \lfalse$.  If $\Th{T} \Proves
\OProv[\Th{T}](\gn{\lfalse}) \lif \lfalse$, then by L\"ob's theorem
$\Th{T} \Proves \lfalse$.  Contrapositively, if $\Th{T}$ is
consistent, then $\Th{T} \Proves/ \OProv[\Th{T}](\gn{\lfalse}) \lif
\lfalse$, i.e., $\Th{T} \Proves/ \OCon[\Th{T}]$.

L\"ob's theorem also settles the status of the fixed point~$!H$ of
$\OProv[\Th{T}](x)$, i.e., a sentence~$!H$ such that $\Th{T} \Proves
\OProv[\Th{T}](\gn{!H}) \liff !H$.  Since in particular $\Th{T}
\Proves \OProv[\Th{T}](\gn{!H}) \lif !H$, L\"ob's theorem gives
$\Th{T} \Proves !H$.
\end{rem}
   % CH-DED: Deduction
\chapter{Computation} \label{ch:cmp}

%% ===================================================================
%% CMP.1: Recursive Functions
%% ===================================================================

\section{Recursive Functions} \label{CMP.1}

We now develop the theory of computable functions on the natural
numbers, beginning with the primitive recursive functions and extending
to the partial recursive functions via unbounded search.  The primitive
recursive functions form a robust class closed under composition and
primitive recursion, but they do not exhaust the computable functions.
To capture computability in full, we must allow partial functions and
the $\mu$-operator.

\subsection{Partial and total functions} \label{CMP.1.1}

We work throughout with functions whose domain and codomain are
subsets of $\Nat$.  A function may fail to be defined on some inputs;
we make this precise.

\begin{defn}[Partial function] % DEF-CMP001
\label{DEF-CMP001}
A \emph{partial function} $f\colon \Nat^k \pto \Nat$ is a function
from a subset of $\Nat^k$ to $\Nat$.  We write $f(\vec{x}) \fdefined$
to mean that $f$ is defined at $\vec{x}$ (i.e., $\vec{x}$ is in the
domain of $f$), and $f(\vec{x}) \fundefined$ to mean that $f$ is not
defined at $\vec{x}$.  We write $f(\vec{x}) \simeq g(\vec{x})$ to
mean that either both $f(\vec{x})$ and $g(\vec{x})$ are undefined, or
both are defined and equal.
\end{defn}

\begin{defn}[Total function] % DEF-CMP002
\label{DEF-CMP002}
A partial function $f\colon \Nat^k \pto \Nat$ is \emph{total} if
$f(\vec{x}) \fdefined$ for every $\vec{x} \in \Nat^k$, i.e., if
its domain is all of $\Nat^k$.
\end{defn}

We adopt the convention that if $h$ and $g_0, \dots, g_k$ are all
partial functions, then $h(g_0(\vec{x}), \dots, g_k(\vec{x}))$ is
defined if and only if each $g_i$ is defined at $\vec{x}$, and $h$ is
defined at $g_0(\vec{x}), \dots, g_k(\vec{x})$.

\subsection{Composition and primitive recursion} \label{CMP.1.2}

\begin{defn}[Composition] % PRIM-CMP001a
\label{DEF-CMP-COMP}
Suppose $f$ is a $k$-place function, and $g_0$, \dots, $g_{k-1}$ are
$k$ functions which are all $n$-place.  The function defined by
\emph{composition from $f$ and $g_0$, \dots,~$g_{k-1}$} is the
$n$-place function~$h$ defined by
\[
h(x_0, \dots, x_{n-1}) =
f(g_0(x_0, \dots, x_{n-1}), \dots, g_{k-1}(x_0, \dots, x_{n-1})).
\]
\end{defn}

Together with the projection functions $\Proj{n}{i}(x_0, \dots,
x_{n-1}) = x_i$, composition provides sufficient flexibility to
rearrange, duplicate, or discard arguments.  For instance, a
three-place function $g(x_0, y, z) = \Succ(z)$ can be defined as
$g(x_0, y, z) = \Succ(\Proj{3}{2}(x_0, y, z))$.

\begin{defn}[Primitive recursion] % PRIM-CMP002
\label{PRIM-CMP002}
Suppose $f$ is a $k$-place function ($k \geq 1$) and $g$ is a
$(k+2)$-place function.  The function defined by \emph{primitive
  recursion from $f$ and $g$} is the $(k+1)$-place function~$h$
defined by the equations
\begin{align*}
  h(x_0, \dots, x_{k-1}, 0) & = f(x_0, \dots, x_{k-1}) \\
  h(x_0, \dots, x_{k-1}, y+1) & = g(x_0, \dots, x_{k-1}, y, h(x_0,
  \dots, x_{k-1}, y))
\end{align*}
\end{defn}

\begin{defn}[Primitive recursive function] % PRIM-CMP002
\label{PRIM-CMP002-SET}
The set of \emph{primitive recursive functions} is the set of
functions from $\Nat^n$ to $\Nat$, defined inductively by the
following clauses:
\begin{enumerate}
\item $\Zero$ is primitive recursive.
\item $\Succ$ is primitive recursive.
\item Each projection function $\Proj{n}{i}$ is primitive recursive.
\item If $f$ is a $k$-place primitive recursive function and $g_0$,
  \dots,~$g_{k-1}$ are $n$-place primitive recursive functions, then
  the composition of $f$ with $g_0$, \dots,~$g_{k-1}$ is primitive
  recursive.
\item If $f$ is a $k$-place primitive recursive function and $g$ is a
  $k+2$-place primitive recursive function, then the function defined
  by primitive recursion from $f$ and $g$ is primitive recursive.
\end{enumerate}
Equivalently, the set of primitive recursive functions is the smallest
set containing $\Zero$, $\Succ$, and the projection
functions~$\Proj{n}{j}$, and which is closed under composition and
primitive recursion.
\end{defn}


\subsection{Primitive recursive relations} \label{CMP.1.3}

\begin{defn}[Characteristic function] % DEF-CMP003
\label{DEF-CMP003}
The \emph{characteristic function} of a relation $R(\vec{x})$ is the
function
\[
\Char{R}(\vec{x}) = \begin{cases}
  1 & \text{if $R(\vec{x})$} \\
  0 & \text{otherwise.}
\end{cases}
\]
A relation $R(\vec{x})$ is said to be \emph{primitive recursive} if
its characteristic function $\Char{R}$ is primitive recursive.
\end{defn}

For example, the relation $\fn{IsZero}(x)$, which holds if and only
if $x = 0$, corresponds to the function $\Char{\fn{IsZero}}$ defined
by primitive recursion: $\Char{\fn{IsZero}}(0) = 1$ and
$\Char{\fn{IsZero}}(x+1) = 0$.  The equality relation $x = y$ is
primitive recursive, defined by $\fn{IsZero}(\left|x - y\right|)$,
and the ordering $x \leq y$ is primitive recursive, defined by
$\fn{IsZero}(x \tsub y)$.

\begin{prop} \label{PROP-CMP-BOOL}
The set of primitive recursive relations is closed under Boolean
operations, that is, if $P(\vec{x})$ and $Q(\vec{x})$ are primitive
recursive, so are
\begin{enumerate}
\item $\lnot P(\vec{x})$
\item $P(\vec{x}) \land Q(\vec{x})$
\item $P(\vec{x}) \lor Q(\vec{x})$
\item $P(\vec{x}) \lif Q(\vec{x})$
\end{enumerate}
\end{prop}

\begin{proof}
Suppose $P(\vec{x})$ and $Q(\vec{x})$ are primitive recursive, i.e.,
their characteristic functions $\Char{P}$ and $\Char{Q}$ are.  We
have to show that the characteristic functions of $\lnot P(\vec{x})$,
etc., are also primitive recursive.
\[
\Char{\lnot P}(\vec{x}) = \begin{cases}
  0 & \text{if $\Char{P}(\vec{x}) = 1$}\\
  1 & \text{otherwise}
\end{cases}
\]
We can define $\Char{\lnot P}(\vec{x})$ as $1 \tsub \Char{P}(\vec{x})$.
\[
\Char{P \land Q}(\vec{x}) = \begin{cases}
  1 & \text{if $\Char{P}(\vec{x}) = \Char{Q}(\vec{x}) = 1$}\\
  0 & \text{otherwise}
\end{cases}
\]
We can define $\Char{P \land Q}(\vec{x})$ as $\Char{P}(\vec{x})
\cdot \Char{Q}(\vec{x})$ or as $\fn{min}(\Char{P}(\vec{x}),
\Char{Q}(\vec{x}))$.  Similarly,
\begin{align*}
  \Char{P \lor Q}(\vec{x}) & = \fn{max}(\Char{P}(\vec{x}),
  \Char{Q}(\vec{x})) \text{ and}\\
  \Char{P \lif Q}(\vec{x}) & = \fn{max}(1 \tsub \Char{P}(\vec{x}),
  \Char{Q}(\vec{x})).
\end{align*}
\end{proof}

\begin{prop} \label{PROP-CMP-BQUANT}
The set of primitive recursive relations is closed under bounded
quantification, i.e., if $R(\vec{x}, z)$ is a primitive recursive
relation, then so are the relations
\begin{align*}
  & \bforall{z < y}{R(\vec{x}, z)} \text{ and}\\
  & \bexists{z < y}{R(\vec{x}, z)}.
\end{align*}
$\bforall{z < y}{R(\vec{x}, z)}$ holds of $\vec{x}$ and $y$ if and
only if $R(\vec{x}, z)$ holds for every~$z$ less than~$y$, and
similarly for $\bexists{z < y}{R(\vec{x}, z)}$.
\end{prop}

\begin{proof}
By convention, we take $\bforall{z < 0}{R(\vec{x}, z)}$ to be true
(for the trivial reason that there are no $z$ less than~$0$) and
$\bexists{z < 0}{R(\vec{x}, z)}$ to be false.  A bounded universal
quantifier functions like an iterated minimum: if
$P(\vec{x}, y) \defiff \bforall{z < y}{R(\vec{x}, z)}$ then
$\Char{P}(\vec{x}, y)$ can be defined by
\begin{align*}
  \Char{P}(\vec{x}, 0) & = 1\\
  \Char{P}(\vec{x}, y+1) & = \fn{min}(\Char{P}(\vec{x}, y),
  \Char{R}(\vec{x}, y)).
\end{align*}
Bounded existential quantification can similarly be defined using
$\fn{max}$.  Alternatively, it can be defined from bounded universal
quantification, using the equivalence $\bexists{z < y}{R(\vec{x}, z)}
\liff \lnot \bforall{z < y}{\lnot R(\vec{x}, z)}$.  Note that a
bounded quantifier of the form $\bexists{x \leq y}{\dots x \dots}$ is
equivalent to $\bexists{x < y+1}{\dots x \dots}$.
\end{proof}

\begin{prop}[Definition by cases] \label{PROP-CMP-CASES}
If $g_0(\vec{x})$, \dots,~$g_m(\vec{x})$ are primitive recursive
functions, and $R_0(\vec{x})$, \dots, $R_{m-1}(\vec{x})$ are
primitive recursive relations, then the function $f$ defined by
\[
f(\vec{x}) = \begin{cases}
    g_0(\vec{x}) & \text{if $R_0(\vec{x})$} \\
    g_1(\vec{x}) & \text{if $R_1(\vec{x})$ and not $R_0(\vec{x})$} \\
    \vdots & \\
    g_{m-1}(\vec{x}) & \text{if $R_{m-1}(\vec{x})$ and none of the
      previous hold} \\
    g_m(\vec{x}) & \text{otherwise}
\end{cases}
\]
is also primitive recursive.
\end{prop}

\begin{proof}
The conditional function $\fn{cond}(x,y,z)$, defined by
$\fn{cond}(0,y,z) = y$ and $\fn{cond}(x+1,y,z) = z$, is primitive
recursive.  When $m = 1$, the function $f$ is just
$f(\vec{x}) = \fn{cond}(\Char{\lnot R_0}(\vec{x}), g_0(\vec{x}),
g_1(\vec{x}))$.  For $m$ greater than $1$, one composes definitions of
this form.
\end{proof}


\subsection{Bounded minimization} \label{CMP.1.4}

\begin{prop} \label{PROP-CMP-BMIN}
If $R(\vec{x}, z)$ is primitive recursive, so is the function
$m_R(\vec{x}, y)$ which returns the least~$z$ less than~$y$ such that
$R(\vec{x}, z)$ holds, if there is one, and $y$ otherwise.  We write
this function as
\[
\bmin{z < y}{R(\vec{x}, z)}.
\]
\end{prop}

\begin{proof}
Since there is no $z < 0$, we have $m_R(\vec{x}, 0) = 0$.  For the
successor case, there are three possibilities: (1) there is a $z < y$
such that $R(\vec{x}, z)$, so $m_R(\vec{x}, y+1) = m_R(\vec{x}, y)$;
(2) there is no such $z < y$ but $R(\vec{x}, y)$ holds, so
$m_R(\vec{x}, y+1) = y$; (3) there is no $z < y+1$ such that
$R(\vec{x}, z)$, so $m_R(\vec{x}, y+1) = y+1$.  Thus:
\begin{align*}
m_R(\vec{x}, 0) & = 0\\
m_R(\vec{x}, y+1) & = \begin{cases}
  m_R(\vec{x}, y) & \text{if $m_R(\vec{x}, y) \neq y$}\\
  y & \text{if $m_R(\vec{x}, y) = y$ and $R(\vec{x}, y)$}\\
  y+1 & \text{otherwise.}
\end{cases}
\end{align*}
This is a definition by primitive recursion combined with definition by
cases from primitive recursive relations, so $m_R$ is primitive
recursive.
\end{proof}

Bounded minimization finds the least witness below a given bound.  In
contrast, the unbounded search operator (introduced next) searches
without any bound and may therefore fail to terminate, taking us out of
the realm of primitive recursive functions.


\subsection{Computability of primitive recursive functions} \label{CMP.1.5}

\begin{prop} \label{PROP-CMP-PRCOMP}
Every primitive recursive function is computable.
\end{prop}

\begin{proof}[Proof sketch]
The basic functions $\Zero$, $\Succ$, and $\Proj{n}{i}$ are
computable.  Composition preserves computability: to compute
$h(\vec{x}) = f(g_0(\vec{x}), \dots, g_{k-1}(\vec{x}))$, first
compute each $g_i(\vec{x})$ and then apply $f$.  Primitive recursion
preserves computability: to compute $h(\vec{x}, y)$, successively
compute $h(\vec{x}, 0)$, $h(\vec{x}, 1)$, \dots, until reaching
$h(\vec{x}, y)$.  Since each step is computable, so is the result.
\end{proof}


\subsection{Partial recursive functions and unbounded search}
\label{CMP.1.6}

We now extend the primitive recursive functions by allowing partial
functions and adding the unbounded search operator.

\begin{defn}[Unbounded search / $\mu$-recursion] % PRIM-CMP003
\label{PRIM-CMP003}
If $f(x, \vec{z})$ is any partial function on the natural numbers,
define $\umin{x}{f(x, \vec{z})}$ to be
\begin{quote}
the least $x$ such that $f(0, \vec{z}), f(1, \vec{z}), \dots,
f(x, \vec{z})$ are all defined, and $f(x, \vec{z}) = 0$, if such an
$x$ exists,
\end{quote}
with the understanding that $\umin{x}{f(x, \vec{z})}$ is undefined
otherwise.

If $R(x, \vec{z})$ is any relation, $\umin{x}{R(x, \vec{z})}$ is
defined to be $\umin{x}{(1 \tsub \Char{R}(x, \vec{z}))}$, i.e., the
least $x$ such that $R(x, \vec{z})$ holds.
\end{defn}

Computationally, the procedure for computing $\umin{x}{f(x, \vec{z})}$
amounts to computing $f(0, \vec{z}), f(1, \vec{z}), f(2, \vec{z}),
\dots$ until a value of~$0$ is returned.  If any intermediate
computation does not halt, neither does the computation of
$\umin{x}{f(x, \vec{z})}$.

\begin{defn}[Partial recursive function] % PRIM-CMP001
\label{PRIM-CMP001}
The set of \emph{partial recursive functions} is the smallest set of
partial functions from the natural numbers to the natural numbers (of
various arities) containing zero, successor, and projections, and
closed under composition, primitive recursion, and unbounded search.
\end{defn}

\begin{defn}[Recursive function] % DEF-CMP002
\label{DEF-CMP002-REC}
The set of \emph{recursive functions} (also called \emph{total
  recursive functions}) is the set of partial recursive functions that
are total.
\end{defn}


%% ===================================================================
%% CMP.2: Turing Machines
%% ===================================================================

\section{Turing Machines} \label{CMP.2}

We now introduce Turing machines, an independent model of computation
that approaches computability from a concrete, mechanical perspective
rather than the function-algebraic perspective of the recursive
functions.

\subsection{Definition of Turing machines} \label{CMP.2.1}

\begin{defn}[Turing machine] % PRIM-CMP004
\label{PRIM-CMP004}
A \emph{Turing machine} $M$ is a tuple $\langle Q, \Sigma, q_0,
\delta\rangle$ consisting of
\begin{enumerate}
\item a finite set of \emph{states}~$Q$,
\item a finite \emph{alphabet} $\Sigma$ which includes $\TMendtape$
  and $\TMblank$,
\item an \emph{initial state}~$q_0 \in Q$,
\item a finite \emph{instruction set}~$\delta\colon Q \times \Sigma
  \pto Q \times \Sigma \times \{\TMleft, \TMright, \TMstay\}$.
\end{enumerate}
The partial function~$\delta$ is also called the \emph{transition
  function} of~$M$.
\end{defn}

We assume the tape is infinite in one direction only.  The
symbol~$\TMendtape$ serves as a marker for the left end of the tape,
making it possible for programs to detect when they are at the
leftmost square.


\subsection{Configurations and computations} \label{CMP.2.2}

\begin{defn}[Configuration] % DEF-CMP-CONFIG
\label{DEF-CMP-CONFIG}
A \emph{configuration} of Turing machine $M = \tuple{Q, \Sigma, q_0,
  \delta}$ is a triple $\tuple{C, m, q}$ where
\begin{enumerate}
\item $C \in \Sigma^*$ is a finite sequence of symbols from $\Sigma$,
\item $m \in \Nat$ is a number $< \len{C}$, and
\item $q \in Q$.
\end{enumerate}
Intuitively, the sequence~$C$ is the content of the tape (from the
leftmost square to the last non-blank or previously visited square),
$m$~is the position of the read/write head, and $q$ is the current
state of the machine.
\end{defn}

\begin{defn}[Initial configuration] % DEF-CMP-INITCONFIG
\label{DEF-CMP-INITCONFIG}
The \emph{initial configuration} of $M$ for input $I \in \Sigma^*$ is
\[
\tuple{\TMendtape \concat I, 1, q_0}.
\]
\end{defn}

\begin{defn}[Yields in one step] % DEF-CMP-YIELD
\label{DEF-CMP-YIELD}
We say that a configuration $\tuple{C, m, q}$ \emph{yields the
  configuration $\tuple{C', m', q'}$ in one step} (according to~$M$)
iff
\begin{enumerate}
\item the $m$-th symbol of $C$ is $\sigma$,
\item the instruction set of $M$ specifies $\delta(q, \sigma) =
  \tuple{q', \sigma', D}$,
\item the $m$-th symbol of $C'$ is $\sigma'$, and
\item\begin{enumerate}
  \item $D = L$ and $m' = m - 1$ if $m > 0$, otherwise $m' = 0$, or
  \item $D = R$ and $m' = m + 1$, or
  \item $D = N$ and $m' = m$,
\end{enumerate}
\item if $m' = \len{C}$, then $\len{C'} = \len{C} + 1$ and the
  $m'$-th symbol of $C'$ is~$\TMblank$; otherwise
  $\len{C'} = \len{C}$,
\item for all $i$ such that $i < \len{C}$ and $i \neq m$,
  $C'(i) = C(i)$.
\end{enumerate}
\end{defn}

\begin{defn}[Run, halting, output] % DEF-CMP-RUN
\label{DEF-CMP-RUN}
A \emph{run of $M$ on input~$I$} is a sequence $C_i$ of
configurations of $M$, where $C_0$ is the initial configuration of $M$
for input~$I$, and each $C_i$ yields $C_{i+1}$ in one step.

We say that $M$ \emph{halts on input $I$ after $k$ steps} if $C_k =
\tuple{C, m, q}$, the $m$th symbol of~$C$ is~$\sigma$, and
$\delta(q, \sigma)$ is undefined.  In that case, the \emph{output}
of~$M$ for input~$I$ is~$O$, where $O$ is a string of symbols not
ending in~$\TMblank$ such that $C = \TMendtape \concat O \concat
\TMblank^j$ for some~$j \in \Nat$.
\end{defn}


\subsection{Unary representation and computation of functions}
\label{CMP.2.3}

We represent natural numbers on the tape using unary notation: if
$n \in \Nat$, let $\TMstroke^n$ be the empty sequence if $n = 0$, and
otherwise the sequence consisting of exactly $n$~$\TMstroke$'s.

\begin{defn}[Turing computation of a total function] % DEF-CMP-TMCOMP
\label{DEF-CMP-TMCOMP}
A Turing machine~$M$ \emph{computes} the function
$f\colon \Nat^k \to \Nat$ iff $M$~halts on input
\[
\TMstroke^{n_1} \TMblank \TMstroke^{n_2} \TMblank \cdots \TMblank
\TMstroke^{n_k}
\]
with output $\TMstroke^{f(n_1, \dots, n_k)}$.
\end{defn}

\begin{defn}[Turing computation of a partial function] % DEF-CMP-TMPCOMP
\label{DEF-CMP-TMPCOMP}
A Turing machine~$M$ computes the partial function
$f\colon \Nat^k \pto \Nat$ iff
\begin{enumerate}
  \item $M$ halts on input
    $\TMstroke^{n_1} \concat \TMblank \concat \cdots
    \concat \TMblank \concat \TMstroke^{n_k}$ with output
    $\TMstroke^{m}$ if $f(n_1, \dots, n_k) = m$.
  \item $M$ does not halt at all, or halts with an output that is not
    a single block of~$\TMstroke$'s, if $f(n_1, \dots, n_k)$ is
    undefined.
\end{enumerate}
\end{defn}


\subsection{Disciplined machines} \label{CMP.2.4}

\begin{defn}[Disciplined Turing machine] % DEF-CMP-DISC
\label{DEF-CMP-DISC}
A Turing machine~$M$ is \emph{disciplined} iff
\begin{enumerate}
    \item it has a designated single halting state~$h$,
    \item it halts, if it halts at all, while scanning square~$1$,
    \item it never erases the $\TMendtape$ symbol on square~$0$, and
    \item it never attempts to move left from square~$0$.
\end{enumerate}
\end{defn}

\begin{prop} \label{PROP-CMP-DISC}
For every Turing machine~$M$, there is a disciplined Turing
machine~$M'$ which halts with output~$O$ if $M$~halts with output~$O$,
and does not halt if $M$~does not halt.  In particular, any function
$f\colon \Nat^n \to \Nat$ computable by a Turing machine is also
computable by a disciplined Turing machine.
\end{prop}

\begin{proof}[Proof sketch]
If $M$ halts in a state other than a designated halting state, add a
new state~$h$ and transition to it.  If $M$ halts with the head not on
square~$1$, add instructions to move the head left until the tape-end
marker is found, then move one square right, then halt.  The other
conditions (not erasing $\TMendtape$, not moving left from square~$0$)
can be enforced similarly by adding a bounded number of extra states.
\end{proof}


\subsection{Combining Turing machines} \label{CMP.2.5}

Given Turing machines $M = \tuple{Q, \Sigma, q_0, \delta}$ and
$M' = \tuple{Q', \Sigma', q_0', \delta'}$, the \emph{sequential
  composition} $M \frown M'$ is constructed as follows: renumber the
states of~$M'$ so that $Q \cap Q' = \emptyset$; the states of
$M \frown M'$ are $Q \cup Q'$; the alphabet is $\Sigma \cup \Sigma'$;
the start state is~$q_0$; and the transition function is
\[
\delta''(q, \sigma) = \begin{cases}
  \delta(q, \sigma) & \text{if $q \in Q$
    and $\delta(q,\sigma)$ is defined}\\
  \tuple{q_0', \sigma, \TMstay} & \text{if $q \in Q$
    and $\delta(q,\sigma)$ is undefined}\\
  \delta'(q, \sigma) & \text{if $q \in Q'$.}
\end{cases}
\]
The idea is that when $M$ would halt (i.e., $\delta$ is undefined), the
combined machine instead enters the start state of~$M'$ and continues.

\begin{prop} \label{PROP-CMP-COMBINE}
If $M$ and $M'$ are disciplined and compute the functions
$f\colon \Nat^k \to \Nat$ and $f'\colon \Nat \to \Nat$, respectively,
then $M \frown M'$ is disciplined and computes~$\comp{f}{f'}$.
\end{prop}

\begin{proof}[Proof sketch]
Since $M$ is disciplined, when it halts with output
$f(n_1, \dots, n_k) = m$, the head is scanning square~$1$.  Entering
the start state of~$M'$ at that point, $M'$ then computes $f'(m)$ and
halts on square~$1$.  The other conditions of
\cref{DEF-CMP-DISC} are preserved by the construction.
\end{proof}


\subsection{The Church--Turing thesis} \label{CMP.2.6}

\begin{defn}[Church--Turing thesis] % PRIM-CMP005
\label{PRIM-CMP005}
The \emph{Church--Turing Thesis} states that anything computable via
an effective procedure is Turing computable.
\end{defn}

The Church--Turing thesis is supported by the fact that every proposed
precise model of effective computability---Turing machines, the
$\lambda$-calculus, partial recursive functions, register machines,
Post production systems, Markov algorithms, and others---turns out to
compute exactly the same class of functions.  The thesis is invoked in
two ways: first, as justification that an informal procedure described
in ``pseudo-code'' could in principle be implemented by a Turing
machine; second, and more importantly, to conclude that functions
which provably cannot be computed by any Turing machine cannot be
computed by \emph{any} effective procedure whatsoever.

\begin{rem}[Equivalence of computation models] % THM-CMP001
\label{THM-CMP001}
Turing machines, partial recursive functions, the $\lambda$-calculus,
unlimited register machines, and all other standard models of
computation define exactly the same class of computable (partial)
functions.  This convergence of independent formalizations constitutes
the primary evidence for the Church--Turing thesis.
\end{rem}


%% ===================================================================
%% CMP.3: Decidability
%% ===================================================================

\section{Decidability} \label{CMP.3}

We now lift the notion of computability from functions to sets and
relations.  A set is \emph{decidable} (computable) if its
characteristic function is computable; it is \emph{semi-decidable}
(computably enumerable) if there is a computable procedure that
eventually confirms membership for elements of the set, but may fail
to terminate for non-members.

\subsection{Computable sets} \label{CMP.3.1}

\begin{defn}[Computable / decidable set] % PRIM-CMP006
\label{PRIM-CMP006}
Let $S$ be a set of natural numbers.  Then $S$ is \emph{computable}
(equivalently, \emph{decidable}) iff its characteristic
function~$\Char{S}$ is computable, i.e., the function
\[
\Char{S}(x) = \begin{cases}
  1 & \text{if $x \in S$} \\
  0 & \text{otherwise}
\end{cases}
\]
is computable.  Similarly, a relation $R(x_0, \dots, x_{k-1})$ is
computable iff its characteristic function is computable.
\end{defn}

Note the distinction: the computation of a partial function returns
the output of the function for input values at which the function is
defined; the computation for a decidable set always halts and returns
either~$1$ or~$0$, indicating membership.


\subsection{Computably enumerable sets} \label{CMP.3.2}

\begin{defn}[Computably enumerable set] % PRIM-CMP007
\label{PRIM-CMP007}
A set $S$ is \emph{computably enumerable} (abbreviated \emph{c.e.};
also called \emph{recursively enumerable} or \emph{r.e.}) if it is
empty or the range of a computable function.
\end{defn}

If $S$ is the range of the computable function~$f$, then
$S = \{f(0), f(1), f(2), \dots\}$, and $f$ can be seen as
``enumerating'' the elements of~$S$.  The enumeration need not be in
increasing order, and repetitions are allowed.

Any computable set is computably enumerable.  For if $S$ is
computable and non-empty, let $a$ be any element of~$S$ and define
\[
f(x) = \begin{cases}
  x & \text{if $\Char{S}(x) = 1$} \\
  a & \text{otherwise.}
\end{cases}
\]
Then $f$ is computable and $S$ is the range of~$f$.


\subsection{Equivalent characterizations of c.e.\ sets} \label{CMP.3.3}

\begin{thm} % DEF-CMP010
\label{DEF-CMP010}
Let $S$ be a set of natural numbers.  Then the following are
equivalent:
\begin{enumerate}
\item\label{ce:ce} $S$ is computably enumerable.
\item\label{ce:ran-pc} $S$ is the range of a \emph{partial}
  computable function.
\item\label{ce:ran-prim} $S$ is empty or the range of a primitive
  recursive function.
\item\label{ce:domain} $S$ is the \emph{domain} of a partial
  computable function.
\end{enumerate}
\end{thm}

\begin{proof}
Since every primitive recursive function is computable and every
computable function is partial computable,
\ref{ce:ran-prim}~$\Rightarrow$~\ref{ce:ce} and
\ref{ce:ce}~$\Rightarrow$~\ref{ce:ran-pc}.  (If $S$ is empty, it is
the range of the partial computable function that is nowhere defined.)
It suffices to show that \ref{ce:ran-pc}~$\Rightarrow$~\ref{ce:ran-prim}
and that \ref{ce:ce}~$\Leftrightarrow$~\ref{ce:domain}.

\medskip\noindent\textbf{\ref{ce:ran-pc}~$\Rightarrow$~\ref{ce:ran-prim}:}
Suppose $S$ is the range of the partial computable function
$\cfind{e}$.  If $S$ is empty, we are done.  Otherwise, let $a$ be any
element of~$S$.  By Kleene's normal form theorem (see DEF-CMP-NF,
\S CMP.4),
\[
\cfind{e}(x) = U(\umin{s}{T(e, x, s)}).
\]
In particular, $\cfind{e}(x) \fdefined$ and equals $y$ if and only if
there is an~$s$ such that $T(e, x, s)$ and $U(s) = y$.  Define
$f(z)$ by
\[
f(z) = \begin{cases}
  U((z)_1) & \text{if $T(e, (z)_0, (z)_1)$} \\
  a        & \text{otherwise.}
\end{cases}
\]
Then $f$ is primitive recursive, because $T$ and $U$ are.  We show
$S$ is the range of~$f$.  In the forward direction, if $y \in S$,
then $y$ is in the range of $\cfind{e}$, so for some $x$ and~$s$,
$T(e, x, s)$ holds and $U(s) = y$; but then $y = f(\tuple{x, s})$.
Conversely, if $y$ is in the range of~$f$, then either $y = a \in S$,
or for some~$z$, $T(e, (z)_0, (z)_1)$ and $U((z)_1) = y$; in the
latter case $\cfind{e}((z)_0) \fdefined = y$, so $y \in S$.

\medskip\noindent\textbf{\ref{ce:ce}~$\Rightarrow$~\ref{ce:domain}:}
Suppose $S$ is the range of a computable function~$f$, i.e.,
$S = \Setabs{y}{\text{for some } x, \, f(x) = y}$.
Let
\[
g(y) = \umin{x}{(f(x) = y)}.
\]
Then $g$ is a partial computable function, and $g(y)$ is defined if and
only if for some~$x$, $f(x) = y$.  So the domain of~$g$ is the range
of~$f$, which is~$S$.

\medskip\noindent\textbf{\ref{ce:domain}~$\Rightarrow$~\ref{ce:ce}:}
Suppose $S$ is the domain of the partial computable
function~$\cfind{e}$, i.e.,
$S = \Setabs{x}{\cfind{e}(x) \fdefined}$.
If $S$ is empty, we are done; otherwise, let $a$ be any element
of~$S$.  Define $f$ by
\[
f(z) = \begin{cases}
  (z)_0 & \text{if $T(e, (z)_0, (z)_1)$} \\
  a     & \text{otherwise.}
\end{cases}
\]
Then a number $x$ is in the range of~$f$ if and only if
$\cfind{e}(x) \fdefined$, i.e., if and only if $x \in S$.
\end{proof}

Clause~\ref{ce:domain} provides a convenient way of enumerating the
c.e.\ sets: for each~$e$, let $W_e$ denote the domain of $\cfind{e}$,
i.e., % DEF-CMP004
\[
W_e = \Setabs{x}{\cfind{e}(x) \fdefined}.
\]
Then if $A$ is any computably enumerable set, $A = W_e$ for some~$e$.

\begin{thm}[$\exists$-characterization of c.e.\ sets] % DEF-CMP010b
\label{DEF-CMP010b}
A set $S$ is computably enumerable if and only if there is a
computable relation $R(x, y)$ such that
\[
S = \Setabs{x}{\lexists[y][R(x, y)]}.
\]
\end{thm}

\begin{proof}
In the forward direction, suppose $S$ is computably enumerable.  Then
for some~$e$, $S = W_e$.  For this value of~$e$ we can write
\[
S = \Setabs{x}{\lexists[y][T(e, x, y)]}.
\]
In the reverse direction, suppose
$S = \Setabs{x}{\lexists[y][R(x, y)]}$.  Define $f$ by
\[
f(x) \simeq \umin{y}{R(x, y)}.
\]
Then $f$ is partial computable, and $S$ is the domain of~$f$.
\end{proof}


\subsection{Closure properties of c.e.\ sets} \label{CMP.3.4}

\begin{thm} \label{THM-CMP-CECLOSURE}
Suppose $A$ and $B$ are computably enumerable.  Then so are
$A \cap B$ and $A \cup B$.
\end{thm}

\begin{proof}[Proof sketch]
Suppose $A$ is the domain of $\cfind{d}$ and $B$ is the domain of
$\cfind{e}$.  Then $A \cap B$ is the domain of the partial function
$\cfind{d}(x) + \cfind{e}(x)$ (both must halt for the sum to be
defined).  For $A \cup B$, define
$p(x) = \umin{y}{(T(d, x, y) \lor T(e, x, y))}$; then $A \cup B$ is
the domain of~$p$, since $p$ halts whenever either $\cfind{d}$ or
$\cfind{e}$ halts.
\end{proof}


\subsection{C.e.\ sets are not closed under complement} \label{CMP.3.5}

\begin{thm} \label{THM-CMP-CECOMP}
Let $A$ be any set of natural numbers.  Then $A$ is computable if and
only if both $A$ and $\Complement{A}$ are computably enumerable.
\end{thm}

\begin{proof}
The forward direction is straightforward: if $A$ is computable, then
$\Complement{A}$ is also computable (since $\Char{\Complement{A}} =
1 \tsub \Char{A}$), and so both are c.e.

In the reverse direction, suppose $A$ and $\Complement{A}$ are both
computably enumerable.  Let $A$ be the domain of~$\cfind{d}$, and let
$\Complement{A}$ be the domain of~$\cfind{e}$.  Define $h$ by
\[
h(x) = \umin{s}{(T(d, x, s) \lor T(e, x, s))}.
\]
On input~$x$, $h$ searches for either a halting computation
of~$\cfind{d}$ or a halting computation of~$\cfind{e}$.  Since every
$x$ is in either $A$ or $\Complement{A}$, one of these searches must
succeed, so $h$ is total computable.  Now for every~$x$: $x \in A$ if
and only if $T(d, x, h(x))$, i.e., if $\cfind{d}$ is the one that
halts.  Since $T(d, x, h(x))$ is a computable relation, $A$ is
computable.
\end{proof}


\subsection{Non-computable sets} \label{CMP.3.6}

\begin{thm} \label{THM-CMP-K0}
Let $K_0 = \Setabs{\tuple{e, x}}{\cfind{e}(x) \fdefined}$.  Then
$K_0$ is computably enumerable but not computable.
\end{thm}

\begin{proof}
To see that $K_0$ is computably enumerable, note that it is the
domain of the function~$f$ defined by
\[
f(z) = \umin{y}{(\len{z} = 2 \land T((z)_0, (z)_1, y))}.
\]
For, if $\cfind{e}(x)$ is defined, $f(\tuple{e, x})$ finds a halting
computation sequence; if $\cfind{e}(x)$ is undefined, so is
$f(\tuple{e, x})$; and if $z$ doesn't code a pair, then $f(z)$ is
also undefined.

The fact that $K_0$ is not computable is the undecidability of the
halting problem: if $K_0$ were decidable, one could decide for any
program~$e$ and input~$x$ whether $\cfind{e}(x)$ halts, contradicting
the halting problem (see CMP.4).
\end{proof}

The set $K_0$ is the \emph{halting set}: $\tuple{e, x} \in K_0$ iff
$\cfind{e}$ is defined on input~$x$.

\begin{thm} \label{THM-CMP-K}
The \emph{self-halting set} $K = \Setabs{e}{\cfind{e}(e) \fdefined}$
is computably enumerable but not decidable.
\end{thm}

\begin{proof}
Suppose $K$ is decidable, i.e., its characteristic function $\Char{K}$
is computable.  Define
\[
d(e) = \begin{cases}
  1 & \text{if $\Char{K}(e) = 0$}\\
  \fundefined & \text{otherwise.}
\end{cases}
\]
Let $k$ be the index of~$d$, i.e., $d \simeq \cfind{k}$.  Then
$d(k) \simeq \cfind{k}(k)$.  But by definition, $d(k) \fdefined$ iff
$\Char{K}(k) = 0$ iff $k \notin K$ iff
$\cfind{k}(k) \fundefined$---a contradiction.

$K$ is c.e.\ because it is the domain of $f(x) = \umin{y}{T(x, x, y)}$.
\end{proof}

\begin{cor} \label{COR-CMP-KBAR}
$\Complement{K_0}$ is not computably enumerable.
\end{cor}

\begin{proof}
We know that $K_0$ is computably enumerable but not computable.  If
$\Complement{K_0}$ were computably enumerable, then $K_0$ would be
computable by \cref{THM-CMP-CECOMP}, contradicting
\cref{THM-CMP-K0}.
\end{proof}


%% ===================================================================
%% CMP.4: Diagonalization and Halting
%% ===================================================================

\section{Diagonalization and Halting} \label{CMP.4}

Diagonalization is one of the most powerful techniques in the theory of
computation.  It was first used by Cantor to show that the set of real
numbers is uncountable, and it was adapted by G\"odel and Turing to
establish fundamental limits on what can be computed.  In this section
we use diagonalization to prove that the halting problem is unsolvable,
and that there is no universal computable function for the total
computable functions.

\subsection{No universal computable function} \label{CMP.4.1}

Although there is a partial computable function that is universal for
the partial computable functions (see \cref{PRIM-CMP012} below), there
is no total computable function that is universal for the total
computable functions.

\begin{thm} % PRIM-CMP010
\label{PRIM-CMP010}
There is no universal computable function.  In other words, any
function $\fn{Un}'(k, x)$ which is such that if $f(x)$ is a total
computable function, then there is a natural number~$k$ such that
$f(x) = \fn{Un}'(k,x)$ for every~$x$, is not computable.
\end{thm}

\begin{proof}
The proof is a simple diagonalization: if $\fn{Un}'(k,x)$ were total
and computable, then
\[
d(x) = \fn{Un}'(x, x) + 1
\]
would also be total and computable.  However, by definition, $d(k)$ is
not equal to $\fn{Un}'(k,k)$.  Hence, for every $k$, the values of
$d(x)$ and~$\fn{Un}'(k, x)$ differ for at least one~$x$, namely $x = k$.
\end{proof}

The normal form theorem (see \cref{DEF-CMP-NF}) shows that we can get
around this diagonalization argument, but only at the expense of
allowing the universal function to be partial.  That is, $\fn{Un}$ is
universal for the total computable functions, it just isn't total.  The
diagonalization argument does not work in the partial case.

\begin{rem}[Diagonalization for primitive recursive functions]
\label{REM-CMP-DIAGNPR}
The same technique shows that the primitive recursive functions do not
exhaust the computable functions.  One can effectively enumerate all
unary primitive recursive functions $f_0, f_1, f_2, \dots$ (by
assigning codes to their definitions; see PRIM-CMP011, \S CMP.5).  The
function $h(x) = f_x(x) + 1$ is then computable but not primitive
recursive, since it differs from each $f_i$ at argument~$i$.
\end{rem}


\subsection{The halting problem} \label{CMP.4.2}

Assume we have fixed an enumeration of Turing machine descriptions
$M_1, M_2, M_3, \dots$  (see \cref{PRIM-CMP011}).  Each Turing machine
thus receives an \emph{index}: its place in the enumeration.  We know
that there must be non-Turing-computable functions---the set of Turing
machine descriptions is enumerable, but the set of all functions from
$\Nat$ to $\Nat$ is not.  But we can find specific examples of
non-computable functions.

\begin{defn}[Halting function] % PRIM-CMP008
\label{PRIM-CMP008}
The \emph{halting function}~$h$ is defined as
\[
h(e,n) =
\begin{cases}
  0 & \text{if machine~$M_e$ does not halt for input $n$} \\
  1 & \text{if machine~$M_e$ halts for input $n$}
\end{cases}
\]
\end{defn}

\begin{defn}[Halting problem] % PRIM-CMP008
\label{PRIM-CMP008-PROB}
The \emph{Halting Problem} is the problem of determining (for any $e$,
$n$) whether the Turing machine~$M_e$ halts for an input of~$n$
strokes.
\end{defn}

We show that $h$ is not Turing-computable by showing that a related
function~$s$ is not Turing-computable.  This proof relies on the fact
that anything computable by a Turing machine can be computed by a
disciplined Turing machine (\cref{DEF-CMP-DISC}), and the fact that
two Turing machines can be combined into a single machine
(\cref{PROP-CMP-COMBINE}).

\begin{defn}
The function~$s$ is defined as
\[
s(e) =
\begin{cases}
  0 & \text{if machine~$M_e$ does not halt for input $e$} \\
  1 & \text{if machine~$M_e$ halts for input $e$}
\end{cases}
\]
\end{defn}

\begin{lem} \label{LEM-CMP-SNOTCOMP}
The function~$s$ is not Turing computable.
\end{lem}

\begin{proof}
We suppose, for contradiction, that the function~$s$ is Turing
computable.  Then there would be a Turing machine~$S$ that
computes~$s$. We may assume, without loss of generality, that when $S$
halts, it does so while scanning the first square (i.e., that it is
disciplined).  This machine can be ``hooked up'' to another
machine~$J$, which halts if it is started on input~$0$ (i.e., if it
reads $\TMblank$ in the initial state while scanning the square to the
right of the end-of-tape symbol), and otherwise wanders off to the
right, never halting. $S \concat J$, the machine created by hooking
$S$ to~$J$, is a Turing machine, so it is $M_e$ for some~$e$ (i.e., it
appears somewhere in the enumeration). Start $M_e$ on an input of~$e$
$\TMstroke$s. There are two possibilities: either $M_e$ halts or it
does not halt.
\begin{enumerate}
\item Suppose $M_e$ halts for an input of $e$ $\TMstroke$s. Then $s(e)
  = 1$. So $S$, when started on~$e$, halts with a single $\TMstroke$
  as output on the tape.  Then $J$ starts with a $\TMstroke$ on the
  tape. In that case $J$ does not halt. But $M_e$ is the machine $S
  \concat J$, so it should do exactly what $S$ followed by $J$ would
  do (i.e., in this case, wander off to the right and never halt).  So
  $M_e$ cannot halt for an input of $e$ $\TMstroke$'s.

\item Now suppose $M_e$ does not halt for an input of $e$
  $\TMstroke$s.  Then $s(e) = 0$, and $S$, when started on input~$e$,
  halts with a blank tape.  $J$,~when started on a blank tape,
  immediately halts.  Again, $M_e$ does what $S$ followed by~$J$ would
  do, so $M_e$ must halt for an input of $e$ $\TMstroke$'s.
\end{enumerate}
In each case we arrive at a contradiction with our assumption. This
shows there cannot be a Turing machine~$S$: $s$~is not Turing
computable.
\end{proof}

\begin{thm}[Unsolvability of the Halting Problem] % THM-CMP002
\label{THM-CMP002}
The halting problem is unsolvable, i.e., the function~$h$ is not Turing
computable.
\end{thm}

\begin{proof}
Suppose $h$ were Turing computable, say, by a Turing machine~$H$. We
could use $H$ to build a Turing machine that computes~$s$: First, make
a copy of the input (separated by a~$\TMblank$ symbol). Then move back
to the beginning, and run~$H$.  We can clearly make a machine that
does the former, and if $H$ existed, we would be able to ``hook it up''
to such a copier machine to get a new machine which would determine if
$M_e$ halts on input~$e$, i.e., computes~$s$. But we've already shown
that no such machine can exist. Hence, $h$~is also not Turing
computable.
\end{proof}

\begin{rem}[Halting problem for partial recursive functions]
\label{REM-CMP-HALTPRF}
The same result holds in the setting of partial recursive functions.
Let $\fn{Un}(e,x)$ denote the universal partial computable function.
Define
\[
h(e, x) =
\begin{cases}
1 & \text{if $\fn{Un}(e, x)$ is defined} \\
0 & \text{otherwise.}
\end{cases}
\]
Then $h$ is not computable.  One proof goes via the no-universal-function
result (\cref{PRIM-CMP010}): if $h$ were computable, one could define a
total computable $\fn{Un}'(e,x)$ agreeing with $\fn{Un}$ wherever the latter
is defined (by returning $0$ when $h(e,x)=0$), contradicting
\cref{PRIM-CMP010}.  An alternative, more direct proof proceeds by
diagonalization: define $g(x)$ to equal $0$ when $h(x,x)=0$ and be
undefined otherwise. Then $g$ is partial computable, so $g \simeq
\cfind{e}$ for some~$e$, and asking whether $g(e)$ is defined leads to
a contradiction.
\end{rem}


%% ===================================================================
%% CMP.5: Coding and Universality
%% ===================================================================

\section{Coding and Universality} \label{CMP.5}

We now develop the machinery of G\"odel numbering, which allows us to
treat syntactic objects---terms, formulas, derivations, and
programs---as natural numbers.  This opens the door to the normal form
theorem, the universal Turing machine, the $s$-$m$-$n$ theorem,
arithmetization of proof predicates, and the representability theorem.

\subsection{Sequence coding} \label{CMP.5.1}

The set of primitive recursive functions is remarkably robust, and we
can extend its power further with a coding of finite sequences of
natural numbers as single natural numbers.  We identify the
sequence $\langle a_0, a_1, \dots, a_k \rangle$ with the number
\[
p_0^{a_0+1} \cdot p_1^{a_1+1} \cdot p_2^{a_2+1} \cdot \dots \cdot
p_k^{a_k+1},
\]
where $p_i$ is the $i$th prime.  Adding one to the exponents
ensures that, e.g., $\langle 2, 7, 3\rangle$ and $\langle 2, 7, 3,
0, 0 \rangle$ have distinct codes.  The Fundamental Theorem of
Arithmetic guarantees that this mapping is injective.

The operations of determining the length $\len{s}$ of a sequence~$s$,
extracting its $i$th element $(s)_i$, appending an element
$\fn{append}(s,a)$, and concatenating two sequences $s \concat t$, are
all primitive recursive.  We can also bound the code of a sequence of
length~$k$ with elements at most~$x$ by
$\fn{sequenceBound}(x,k) = p_{k-1}^{k \cdot (x+1)}$.


\subsection{Kleene's normal form theorem} \label{CMP.5.2}

\begin{thm}[Kleene's Normal Form Theorem] % THM-CMP004
\label{THM-CMP004}
\label{DEF-CMP-NF}
There is a primitive recursive relation $T(e, x, s)$ and a primitive
recursive function $U(s)$, with the following property: if $f$ is any
partial recursive function, then for some~$e$,
\[
f(x) \simeq U(\umin{s}{T(e, x, s)})
\]
for every $x$.
\end{thm}

Every partial recursive function has an \emph{index}~$e$---intuitively,
a number coding its program or definition.  If $f(x) \fdefined$, the
computation can be recorded and coded by some number~$s$, and the fact
that $s$ codes the computation of~$f$ on input~$x$ can be checked
primitive recursively.  Consequently, $T(e,x,s)$ (``the function with
index~$e$ has a computation for input~$x$ coded by~$s$'') is primitive
recursive, and the output can be extracted from~$s$ by the primitive
recursive function~$U$.

\begin{defn}[Index / program] % DEF-CMP005
\label{DEF-CMP005}
The normal form theorem shows that only a single unbounded search is
required for the definition of any partial recursive function.  We use
the numbers~$e$ as ``names'' of partial recursive functions, and write
$\cfind{e}$ for the function~$f$ defined by the equation
$\cfind{e}(x) \simeq U(\umin{s}{T(e, x, s)})$.
Note that any partial recursive function can have more than one
index---in fact, every partial recursive function has infinitely many
indices.
\end{defn}


\subsection{Enumerating Turing machines} \label{CMP.5.3}

Every Turing machine can be described by a finite sequence of positive
integers encoding its states, alphabet, start state, and instructions.
By considering only \emph{standard} machines (where states and symbols
are positive integers), we can canonically encode any Turing machine as
an element of $(\PosInt)^*$.  Since $(\PosInt)^*$ is enumerable, so is
the set of standard Turing machine descriptions.

\begin{defn}[Index of a Turing machine] % DEF-CMP005
\label{DEF-CMP005-TM}
If $M$ is the $e$th Turing machine (in our fixed enumeration), we say
that $e$ is an \emph{index} of~$M$.  We write $M_e$ for the $e$th
Turing machine.
\end{defn}

A machine may have more than one index; for example, two descriptions
of~$M$ that list the instructions in different orders will have
different indices.  Given the enumeration, we can effectively compute
the description of~$M$ from its index and vice versa.

\begin{thm} \label{THM-CMP-UNCOMPEXIST}
There are functions from $\Nat$ to~$\Nat$ which are not Turing
computable.
\end{thm}

\begin{proof}
The set of descriptions of standard Turing machines is a subset of
$(\PosInt)^*$, so it is enumerable.  The set of all Turing computable
functions is therefore also enumerable.  But the set of all functions
from $\Nat$ to $\Nat$ is not enumerable.  So there must be functions
that are not Turing computable.
\end{proof}


\subsection{The universal Turing machine} \label{CMP.5.4}

\begin{thm}[Universal Turing machine] % PRIM-CMP012
\label{PRIM-CMP012}
There is a \emph{universal Turing machine}~$U$ which, when started on
input $\tuple{e,n}$:
\begin{enumerate}
  \item halts iff $M_e$ halts on input~$n$, and
  \item if $M_e$ halts with output $m$, so does~$U$.
\end{enumerate}
$U$ thus computes the function $f\colon \Nat \times \Nat \pto \Nat$
given by $f(e,n) = m$ if $M_e$ started on input~$n$ halts with
output~$m$, and undefined otherwise.
\end{thm}

\begin{proof}
We describe how $U$ works and invoke the Church--Turing thesis.  When
$U$ starts, its tape contains a block of $e$~$\TMstroke$'s followed by
a block of $n$~$\TMstroke$'s.  It first ``decodes'' the index~$e$,
producing the description of~$M_e$ (a list of instruction
$5$-tuples).  Then $U$ sets up the initial configuration: it records
the start state of~$M_e$ and the initial head position, and converts
the input into coded symbols on a simulated ``tape.''

$U$ now simulates $M_e$ step by step:
\begin{enumerate}
  \item Find the current head position~$k$.
  \item Read the coded symbol at position~$k$ on the simulated tape.
  \item Find the instruction matching the current state and symbol.
  \item Write the new symbol at position~$k$.
  \item Update the stored state to the new state.
  \item Adjust the stored head position (increment or decrement
    according to the direction).
  \item Repeat.
\end{enumerate}
If $M_e$ never halts, then $U$ never halts either.  If $M_e$ halts
(i.e., no instruction matches the current state/symbol pair), then $U$
decodes the simulated tape contents back into a unary output and halts.
\end{proof}


\subsection{The $s$-$m$-$n$ theorem} \label{CMP.5.5}

\begin{thm}[$s$-$m$-$n$ theorem] % THM-CMP004
\label{THM-CMP004-SMN}
  For each pair of natural numbers $n$ and~$m$, there is a primitive
  recursive function~$s^m_n$ such that for every sequence
  $e$, $a_0$, \dots, $a_{m-1}$, $y_0$, \dots, $y_{n-1}$, we have
  \[
  \cfind{s^m_n(e, a_0, \dots, a_{m-1})}[n](y_0, \dots, y_{n-1}) \simeq
  \cfind{e}[m+n](a_0, \dots, a_{m-1}, y_0, \dots, y_{n-1}).
\]
\end{thm}

It is helpful to think of $s^m_n$ as acting on \emph{programs}.  The
function $s^m_n$ takes a program~$e$ for an $(m+n)$-ary function, as
well as fixed inputs $a_0, \dots, a_{m-1}$, and returns a program
$s^m_n(e, a_0, \dots, a_{m-1})$ for the $n$-ary function of the
remaining arguments.  In Turing machine terms,
$s^m_n(e, a_0, \dots, a_{m-1})$ is the machine that, on input
$y_0, \dots, y_{n-1}$, prepends $a_0, \dots, a_{m-1}$ to the input
string and runs~$e$.  Each $s^m_n$ is a primitive recursive function that
finds a code for the appropriate machine.


\subsection{Representing Turing machines in first-order logic}
\label{CMP.5.6}

To connect computation to logic, we show how to represent the behavior
of a Turing machine~$M$ on input~$w$ by a sentence of first-order
logic.

\begin{defn}[Language $\Lang{L}_M$] % DEF-CMP009
\label{DEF-CMP009}
Given a Turing machine $M = \tuple{Q, \Sigma, q_0, \delta}$, the
language~$\Lang{L}_M$ consists of:
\begin{enumerate}
\item A two-place predicate symbol $\Obj Q_q(x, y)$ for every state~$q \in
  Q$.  Intuitively, $\Obj Q_q(\num{m}, \num{n})$ expresses ``after $n$
  steps, $M$ is in state~$q$ scanning the $m$th square.''
\item A two-place predicate symbol $\Obj S_\sigma(x, y)$ for every
  symbol~$\sigma\in \Sigma$.  Intuitively, $\Obj S_\sigma(\num{m},
  \num{n})$ expresses ``after $n$ steps, the $m$th square contains
  symbol~$\sigma$.''
\item A constant symbol $\Obj 0$, a one-place function symbol~$\prime$,
  and a two-place predicate symbol~$<$.
\end{enumerate}
\end{defn}

The sentence $!T(M, w)$ consists of axioms for $\Obj{0}$, $\prime$,
and $<$ (ensuring $\lforall[x][x < x']$ and transitivity), axioms
describing the input configuration, and axioms describing the
transition from one configuration to the next.  For each instruction
$\delta(q_i, \sigma) = \tuple{q_j, \sigma', D}$ there is a universally
quantified sentence stating that if $M$ is in state~$q_i$ scanning a
square containing~$\sigma$, then after one more step, the state is
$q_j$, the symbol on that square is~$\sigma'$, the head has moved
according to~$D$, and all other squares are unchanged.

The sentence $!E(M, w)$ asserts that $M$ eventually reaches a halting
configuration:
$\lexists[x][\lexists[y][(\bigvee_{\tuple{q,\sigma} \in X}
(\Obj Q_q(x,y) \land \Obj S_\sigma(x,y)))]]$, where $X$ is the set of
state/symbol pairs on which $\delta$ is undefined.


\subsection{Verification lemmas} \label{CMP.5.7}

Let $!C(M, w, n)$ be the sentence describing the configuration of~$M$
run on~$w$ after $n$~steps: it specifies the state, head position,
and contents of every tape square.

\begin{lem} \label{LEM-CMP-HALTCFG}
If $M$ run on input~$w$ is in a halting configuration after $n$ steps,
then $!C(M, w, n) \Entails !E(M, w)$.
\end{lem}

\begin{proof}[Proof sketch]
If $M$ halts after $n$ steps in state~$q$ scanning square~$m$
containing~$\sigma$ with $\delta(q,\sigma)$ undefined, then $!C(M,w,n)$
includes conjuncts $\Obj Q_q(\num{m},\num{n})$ and
$\Obj S_\sigma(\num{m},\num{n})$.  Since $\tuple{q,\sigma} \in X$,
these imply $!E(M,w)$ by existential generalization.
\end{proof}

\begin{lem} \label{LEM-CMP-CONFIGENT}
For each $n$, if $M$ has not halted after $n$ steps, $!T(M, w)
\Entails !C(M, w, n)$.
\end{lem}

\begin{proof}[Proof sketch]
By induction on~$n$.  The base case ($n=0$) holds because the
conjuncts of $!C(M,w,0)$ are conjuncts of $!T(M,w)$.  For the
inductive step, suppose $!T(M,w) \Entails !C(M,w,n)$ and $M$ has not
halted.  The transition axiom corresponding to the instruction
executed at step~$n$, together with $!C(M,w,n)$, entails all conjuncts
of $!C(M,w,n+1)$.  Unchanged squares follow from the frame axiom
$!A(x,y)$; if the head visits a new square, the axiom
$\lforall[x][x < x']$ and transitivity establish the required
properties.
\end{proof}

\begin{lem} \label{LEM-CMP-VALIDHALT}
If $M$ halts on input~$w$, then $!T(M, w) \lif !E(M, w)$ is valid.
\end{lem}

\begin{proof}[Proof sketch]
If $M$ halts after $k$ steps, then by \cref{LEM-CMP-CONFIGENT},
$!T(M,w) \Entails !C(M,w,k)$, and by \cref{LEM-CMP-HALTCFG},
$!C(M,w,k) \Entails !E(M,w)$.
\end{proof}

\begin{lem} \label{LEM-CMP-HALTVALID}
If $\Entails !T(M, w) \lif !E(M, w)$, then $M$ halts on input~$w$.
\end{lem}

\begin{proof}[Proof sketch]
Construct a structure~$\Struct{M}$ with domain~$\Nat$ that interprets
$\Obj 0$ as~$0$, $\prime$ as successor, $<$ as less-than, and $\Obj
Q_q$, $\Obj S_\sigma$ according to the actual run of~$M$ on~$w$.
Then $\Sat{M}{!T(M,w)}$ by construction.  If $\Entails !T(M,w) \lif
!E(M,w)$, then $\Sat{M}{!E(M,w)}$, which means there exist $m, n \in
\Nat$ such that $M$ is in a halting configuration after $n$~steps.
\end{proof}


\subsection{Arithmetization of syntax} \label{CMP.5.8}

We now show that syntactic objects of first-order logic can be coded as
natural numbers, and that the relevant properties are primitive
recursive.

\begin{defn}[Symbol code and G\"odel number] % PRIM-CMP011
\label{PRIM-CMP011}
If $s$ is a symbol of the language~$\Lang{L}$, its \emph{symbol
  code}~$\scode{s}$ is defined as follows: logical symbols receive
codes $\tuple{0, i}$ for various~$i$; the $i$th variable~$\Obj v_i$
receives $\scode{\Obj v_i} = \tuple{1, i}$; the $i$th constant
symbol~$\Obj c_i$ receives $\tuple{2, i}$; the $i$th $n$-ary
function symbol~$\Obj f_i^n$ receives $\tuple{3, n, i}$; and the
$i$th $n$-ary predicate symbol~$\Obj P_i^n$ receives $\tuple{4, n,
  i}$.  If $s_0, \dots, s_{n-1}$ is a sequence of symbols, its
\emph{G\"odel number} is $\tuple{\scode{s_0}, \dots,
  \scode{s_{n-1}}}$.
\end{defn}

\begin{rem}
The relations $\fn{Fn}(x,n)$ (``$x$ codes an $n$-ary function
symbol'') and $\fn{Pred}(x,n)$ (``$x$ codes an $n$-ary predicate
symbol'') are primitive recursive.
\end{rem}

The following properties are all primitive recursive (each is
established by bounded search over formation sequences or codes):

\begin{prop} \label{PROP-CMP-TERMPRIM}
The relation $\fn{Term}(x)$, which holds iff $x$ is the G\"odel number
of a term, is primitive recursive.  So is $\fn{num}(n) = \Gn{\num{n}}$.
\end{prop}

\begin{proof}[Proof sketch]
A number $x$ is the G\"odel number of a term iff there is a formation
sequence $s_0, \dots, s_{k-1}$ of terms ending in the expression coded
by~$x$.  Each $s_i$ is either a variable, a constant, or is built from
earlier terms by a function symbol.  The code of the formation sequence
is bounded by $p_{k-1}^{k(x+1)}$ where $k = \len{x}$, so the check
involves only bounded quantification.  The function $\fn{num}(n)$ is
defined by primitive recursion: $\fn{num}(0) = \Gn{\Obj 0}$ and
$\fn{num}(n+1) = \Gn{\prime(} \concat \fn{num}(n) \concat \Gn{)}$.
\end{proof}

\begin{prop} \label{PROP-CMP-FRMPRIM}
The relations $\fn{Frm}(x)$ (``$x$ is the G\"odel number of a
formula'') and $\fn{Sent}(x)$ (``$x$ is the G\"odel number of a
sentence'') are primitive recursive.
\end{prop}

\begin{prop} \label{PROP-CMP-SUBSTPRIM}
There is a primitive recursive function $\fn{Subst}(x, y, z)$ such that
$\fn{Subst}(\Gn{!A}, \Gn{t}, \Gn{u}) = \Gn{\Subst{!A}{t}{u}}$.
\end{prop}


\subsection{The proof predicate} \label{CMP.5.9}

We now arithmetize derivations.  Since derivations are structured
syntactic objects (trees of sequents or sequences of formulas), they
can be coded as numbers.  The details depend on the proof system
chosen---sequent calculus ($\Log{LK}$), natural deduction, or axiomatic
derivations---but the result is the same in each case.

\begin{defn}[G\"odel number of a derivation] % DEF-CMP012
\label{DEF-CMP012}
A derivation~$\pi$ receives a G\"odel number $\Gn{\pi}$ by recursively
coding its structure.  For sequent calculus: an initial sequent $\Gamma
\Sequent \Delta$ is coded as $\tuple{0, \Gn{\Gamma \Sequent \Delta}}$;
a one-premise inference is coded as $\tuple{1, \Gn{\pi_1}, \Gn{\Gamma
    \Sequent \Delta}, k}$ where $k$ identifies the rule; and similarly
for two-premise inferences.  Analogous codings exist for natural
deduction and axiomatic derivations.
\end{defn}

\begin{prop} \label{PROP-CMP-CORRECT}
The property $\fn{Correct}(p)$, which holds iff the last inference in
the derivation with G\"odel number~$p$ is a correct application of a
rule, is primitive recursive.
\end{prop}

\begin{proof}[Proof sketch]
For each rule~$R$, the relation $\fn{FollowsBy}_R(p)$ checks that the
end-sequent of~$p$ follows from the premises by a correct application
of~$R$.  This involves verifying the structure of the G\"odel numbers
using bounded quantification and the primitive recursive functions for
sequence manipulation, $\fn{Frm}$, $\fn{Subst}$, etc.  The property
$\fn{Correct}(p)$ is the disjunction of all $\fn{FollowsBy}_R(p)$
together with the case that $p$ codes an initial sequent.
\end{proof}

\begin{prop} \label{PROP-CMP-DERIV}
The relation $\fn{Deriv}(p)$, which holds if $p$ is the G\"odel number
of a correct derivation, is primitive recursive.
\end{prop}

\begin{prop}[Proof predicate] % DEF-CMP012
\label{PROP-CMP-PRF}
Suppose $\Gamma$ is a primitive recursive set of sentences.  Then the
relation $\Prf[\Gamma](x, y)$ expressing ``$x$ is the code of a
derivation of~$!A$ from~$\Gamma$ and $y$ is the G\"odel number
of~$!A$'' is primitive recursive.
\end{prop}

\begin{rem}[Variant proof systems]
The above results hold for sequent calculus, natural deduction, and
axiomatic proof systems.  The internal details of the coding differ
(trees vs.\ sequences, treatment of discharge labels, etc.), but in
every case $\fn{Deriv}$ and $\Prf[\Gamma]$ are primitive recursive.
The key ingredients are the same: primitive recursive checking of each
inference step, and bounded search over sub-derivations.
\end{rem}


\subsection{Representability in $\Th{Q}$} \label{CMP.5.10}

\begin{defn}[Representable function] % DEF-CMP009
\label{DEF-CMP009-REP}
A function $f(x_0, \dots, x_k)$ is \emph{representable in $\Th{Q}$}
if there is a formula $!A_f(x_0, \dots, x_k, y)$ such that whenever
$f(n_0, \dots, n_k) = m$, then:
\begin{enumerate}
\item $\Th{Q} \Proves !A_f(\num{n_0}, \dots, \num{n_k}, \num{m})$, and
\item $\Th{Q} \Proves \lforall[y][(!A_f(\num{n_0}, \dots, \num{n_k},
  y) \lif y = \num{m})]$.
\end{enumerate}
\end{defn}

\begin{defn}[Representable relation]
\label{DEF-CMP-REPREL}
A relation $R(x_0,\dots,x_k)$ is \emph{representable in $\Th{Q}$} if
there is a formula $!A_R(x_0,\dots,x_k)$ such that whenever
$R(n_0,\dots,n_k)$ is true, $\Th{Q} \Proves !A_R(\num{n_0}, \dots,
\num{n_k})$, and whenever $R(n_0,\dots,n_k)$ is false, $\Th{Q} \Proves
\lnot !A_R(\num{n_0}, \dots, \num{n_k})$.
\end{defn}

The representability theorem establishes a deep connection between
computability and provability.

\begin{thm}[Representability theorem] % DEF-CMP009
\label{THM-CMP-REPCOMP}
A function is representable in $\Th{Q}$ if and only if it is computable.
A relation is representable in $\Th{Q}$ if and only if it is computable.
\end{thm}

\begin{proof}[Proof sketch]
\textbf{Representable $\Rightarrow$ computable:} If $f$ is represented
by $!A_f$, we compute $f(n_0, \dots, n_k)$ by searching through all
derivations from $\Th{Q}$ until we find one proving
$!A_f(\num{n_0}, \dots, \num{n_k}, \num{m})$ for some~$m$.  Since the
proof predicate $\Prf[\Th{Q}]$ is primitive recursive, the search can
be formalized as a regular minimization.

\textbf{Computable $\Rightarrow$ representable:} We show that the
basic functions ($\Zero$, $\Succ$, $\Proj{n}{i}$, $\Add$, $\Mult$,
$\Char{=}$) are representable, and that representable functions are
closed under composition and regular minimization.  Primitive recursion
is handled via the beta function, as follows.

The basic functions are represented by their natural formulas: $\Zero$
by $\eq[y][\Obj{0}]$, $\Succ$ by $\eq[y][x']$, $\Proj{n}{i}$ by
$\eq[y][x_i]$, $\Add$ by $\eq[y][(x_0 + x_1)]$, $\Mult$ by $\eq[y][(x_0
\times x_1)]$, and $\Char{=}$ by
$(\eq[x_0][x_1] \land \eq[y][\num{1}]) \lor (\eq/[x_0][x_1] \land
\eq[y][\num{0}])$.  That these work requires showing, for instance,
that $\Th{Q} \Proves \eq[(\num{n} + \num{m})][\num{n+m}]$ (by
induction on~$m$ using axioms $!Q_4$ and $!Q_5$) and that $\Th{Q}$
proves distinct numerals unequal (by induction using axioms $!Q_1$ and
$!Q_2$).

Closure under composition: if $!A_f$ represents~$f$ and $!A_{g_i}$
represents~$g_i$, then
$\lexists[y_0][\dots\lexists[y_{k-1}][(!A_{g_0}(\vec{x}, y_0) \land
    \dots \land !A_{g_{k-1}}(\vec{x}, y_{k-1}) \land !A_f(y_0, \dots,
    y_{k-1}, z))]]$
represents $h(\vec{x}) = f(g_0(\vec{x}), \dots, g_{k-1}(\vec{x}))$.

Closure under regular minimization: if $!A_g$ represents~$g$, then
$!A_g(y, z, \Obj{0}) \land \lforall[w][(w < y \lif \lnot !A_g(w, z,
  \Obj{0}))]$ represents $f(z) = \umin{x}{[g(x,z) = 0]}$.  The proof
uses lemmas showing that $\Th{Q}$ proves $\lforall[x][\lnot x <
  \Obj{0}]$, that $\Th{Q}$ proves $x < \num{n+1} \lif (\eq[x][\Obj{0}]
\lor \dots \lor \eq[x][\num{n}])$, and that $\Th{Q}$ proves
trichotomy for numerals.
\end{proof}


\subsection{The beta function} \label{CMP.5.11}

\begin{lem}[Beta function lemma] % DEF-CMP013
\label{DEF-CMP013}
There is a function $\beta(d,i)$ such that for every sequence $a_0,
\dots, a_n$ there is a number~$d$ such that for every $i \le n$,
$\beta(d,i) = a_i$.  Moreover, $\beta$ can be defined from the basic
functions using just composition and regular minimization.
\end{lem}

The function $\beta$ provides a way of decoding finite sequences
without using primitive recursion.  It is defined using the Chinese
Remainder Theorem (Sunzi's Theorem): given $a_0, \dots, a_n$, let $j =
\max(n, a_0 + 1, \dots, a_n + 1)$ and $m = \text{lcm}(1, \dots, j)$.
Then $x_i = 1 + (i+1) \cdot m$ are pairwise relatively prime and each
exceeds~$a_i$.  By Sunzi's Theorem, there exists~$d_0$ with $d_0
\equiv a_i \pmod{x_i}$ for each~$i$.  Setting $d = J(d_0, m)$ (where
$J$ is the pairing function) and $\beta(d, i) = \fn{rem}(1 + (i+1)
\cdot L(d), K(d))$ gives the required decoding.

Using $\beta$, primitive recursion can be simulated by regular
minimization: if $h(\vec{x}, 0) = f(\vec{x})$ and $h(\vec{x}, y+1) =
g(\vec{x}, y, h(\vec{x}, y))$, then $h(\vec{x}, y) = \beta(\hat{h}(\vec{x},
y), y)$ where $\hat{h}(\vec{x}, y) = \umin{d}{(\beta(d,0) = f(\vec{x})
\land \bforall{i < y}{\beta(d, i+1) = g(\vec{x}, i, \beta(d, i))})}$.


\subsection{Productive sets} \label{CMP.5.12}

\begin{defn}[Productive set] % DEF-CMP007
\label{DEF-CMP007}
A set $A \subseteq \Nat$ is \emph{productive} if there exists a
computable function~$f$ such that for every c.e.\ set $W_e \subseteq
A$, we have $f(e) \in A \setminus W_e$.  Such a function~$f$ is called
a \emph{productive function} for~$A$.
\end{defn}

Productive sets witness a strong form of non-computability: not only is
a productive set not c.e., but given any c.e.\ approximation $W_e
\subseteq A$, we can computably find a specific element that $W_e$
misses.

\begin{prop} \label{PROP-CMP-KBARPROD}
The complement of $K = \Setabs{e}{\cfind{e}(e) \fdefined}$ is
productive, with the identity function as a productive function.
\end{prop}

\begin{proof}
Suppose $W_e \subseteq \Complement{K}$.  We must show $e \in
\Complement{K} \setminus W_e$.  If $e \in W_e$, then since $W_e
\subseteq \Complement{K}$, we have $e \in \Complement{K}$, i.e.,
$\cfind{e}(e) \fundefined$.  But $e \in W_e$ means $\cfind{e}(e)
\fdefined$, a contradiction.  So $e \notin W_e$.  If $e \in K$, then
$\cfind{e}(e) \fdefined$, so $e \in W_e$ (since $W_e$ is the domain
of~$\cfind{e}$), contradicting what we just showed.  So $e \in
\Complement{K}$, and thus $e \in \Complement{K} \setminus W_e$.
\end{proof}


%% ===================================================================
%% CMP.6: Computability Theory
%% ===================================================================

\section{Computability Theory} \label{CMP.6}

Having established the basic framework of computability---recursive
functions, Turing machines, decidability, coding, and
universality---we now develop the structural theory.  The key tools are
many-one reducibility, which provides a method for comparing the
difficulty of decision problems, and the notions of complete c.e.\
sets, axiomatizable theories, and computable inseparability.

\subsection{Many-one reducibility} \label{CMP.6.1}

\begin{defn}[Many-one reduction] % PRIM-CMP009
\label{PRIM-CMP009}
Let $A$ and $B$ be sets of natural numbers.  A computable
function~$f\colon \Nat \to \Nat$ is a \emph{many-one reduction} of $A$
to~$B$ iff, for every natural number~$x$,
\[
x \in A \quad \text{if and only if} \quad f(x) \in B.
\]
If such a reduction $f$ exists, we say that $A$ is \emph{many-one
  reducible} to~$B$, written $A \leq_m B$.  If $A \leq_m B$ and $B
\leq_m A$, then $A$ and $B$ are \emph{many-one equivalent}, written
$A \equiv_m B$.
\end{defn}

As an example, the function $f(x) = \tuple{x,x}$ is a many-one
reduction of $K = \Setabs{x}{\cfind{x}(x) \fdefined}$ to $K_0 =
\Setabs{\tuple{e,x}}{\cfind{e}(x) \fdefined}$, since $x \in K$ iff
$\cfind{x}(x) \fdefined$ iff $\tuple{x,x} \in K_0$.

If $f$ happens to be injective, $A$ is said to be \emph{one-one
  reducible} to~$B$.

\begin{prop}[Transitivity] \label{PROP-CMP-TRANSRED}
If $A \leq_m B$ and $B \leq_m C$, then $A \leq_m C$.
\end{prop}

\begin{proof}
Composing a reduction of $A$ to~$B$ with a reduction of $B$ to~$C$
yields a reduction of $A$ to~$C$: if $f$ reduces $A$ to~$B$ and $g$
reduces $B$ to~$C$, then $\comp{f}{g}$ reduces $A$ to~$C$, since for
every~$x$, $x \in A$ iff $f(x) \in B$ iff $g(f(x)) \in C$.
\end{proof}

\begin{prop}[Preservation under reduction] \label{PROP-CMP-REDUCE}
Let $A$ and $B$ be any sets, and suppose $A \leq_m B$.
\begin{enumerate}
\item If $B$ is computably enumerable, so is~$A$.
\item If $B$ is computable, so is~$A$.
\end{enumerate}
\end{prop}

\begin{proof}
Let $f$ be a many-one reduction from $A$ to~$B$.

For (1): if $B$ is the domain of a partial computable function~$g$,
then $A$ is the domain of~$\comp{f}{g}$, since $x \in A$ iff $f(x)
\in B$ iff $g(f(x)) \fdefined$.

For (2): note that $f$ is also a reduction of $\Complement{A}$ to
$\Complement{B}$.  If $B$ is computable, then both $B$ and
$\Complement{B}$ are c.e., so by (1), both $A$ and $\Complement{A}$
are c.e., whence $A$ is computable by \cref{THM-CMP-CECOMP}.
\end{proof}


\subsection{Complete c.e.\ sets} \label{CMP.6.2}

\begin{defn}[Complete c.e.\ set] % DEF-CMP008
\label{DEF-CMP008}
A set $A$ is a \emph{complete computably enumerable set} (under
many-one reducibility) if
\begin{enumerate}
\item $A$ is computably enumerable, and
\item for any other computably enumerable set $B$, $B \leq_m A$.
\end{enumerate}
\end{defn}

In other words, complete c.e.\ sets are the ``hardest'' c.e.\ sets
possible: they allow one to answer questions about \emph{any} c.e.\
set.

\begin{thm} \label{THM-CMP-KKCOMPLETE}
$K$, $K_0$, and $K_1$ are all complete computably enumerable sets.
\end{thm}

\begin{proof}
To see that $K_0$ is complete, let $B$ be any computably
enumerable set.  Then for some index $e$,
$B = W_e = \Setabs{x}{\cfind{e}(x) \fdefined}$.  Let $f$ be the
function $f(x) = \tuple{e, x}$.  Then for every natural number $x$,
$x \in B$ if and only if $f(x) \in K_0$.  In other words, $f$ reduces
$B$ to~$K_0$.

$K$ can be reduced to $K_0$ in the same way (via $x \mapsto
\tuple{x,x}$), so $K$ is also complete by transitivity.

To see that $K_1$ is complete, one shows that $K_0$ reduces to
$K_1$ (via an $s$-$m$-$n$ argument), and then completeness follows by
transitivity.
\end{proof}


\subsection{Totality is undecidable} \label{CMP.6.3}

\begin{prop} \label{PROP-CMP-TOT}
The set $\fn{Tot} = \Setabs{x}{\text{for every } y,\; \cfind{x}(y)
  \fdefined}$ is not computable.
\end{prop}

\begin{proof}[Proof sketch]
We reduce $K$ to $\fn{Tot}$.  Define $h(x,y) \simeq 0$ if $x \in K$,
and $h(x,y) \fundefined$ otherwise (the function $h$ does not depend
on~$y$: it simply simulates $\cfind{x}(x)$ and outputs $0$ if it
halts).  By the $s$-$m$-$n$ theorem, there is a primitive recursive
$k(x)$ such that $\cfind{k(x)}(y) = h(x,y)$.  Then $\cfind{k(x)}$ is
total iff $x \in K$, so $k$ reduces $K$ to $\fn{Tot}$.
\end{proof}


\subsection{Axiomatizable theories} \label{CMP.6.4}

\begin{defn}[Axiomatizable theory] % DEF-CMP011
\label{DEF-CMP011}
A theory~$\Gamma$ is \emph{axiomatizable} if it is axiomatized by a
decidable set of axioms, i.e., $\Gamma = \Setabs{!A}{\Gamma_0 \Entails
  !A}$ for some decidable set~$\Gamma_0$.
\end{defn}

Any theory with a finite set of axioms is axiomatizable (since finite
sets are decidable).  Schematically axiomatized theories like Peano
arithmetic $\Th{PA}$ are also axiomatizable, since one can effectively
test whether a given sentence is an instance of the induction schema.

\begin{lem} \label{LEM-CMP-AXTCE}
If $\Th{T}$ is axiomatizable, then $\Th{T}$ is computably enumerable.
\end{lem}

\begin{proof}[Proof sketch]
If $A$ is a decidable set of axioms for $\Th{T}$, then $!A \in \Th{T}$
iff there is a finite list of axioms $!B_1, \dots, !B_k$ in $A$ and a
derivation of $(!B_1 \land \dots \land !B_k) \lif !A$ in first-order
logic.  Since the set of all derivations is enumerable and we can
check membership in~$A$ decidably, $\Th{T}$ is c.e.
\end{proof}


\subsection{Computable inseparability} \label{CMP.6.5}

\begin{defn}[Computable inseparability] % DEF-CMP014
\label{DEF-CMP014}
Two disjoint sets $A$ and $B$ are \emph{computably inseparable} if
there is no computable set $C$ such that $A \subseteq C$ and $B
\subseteq \Complement{C}$.
\end{defn}

\begin{lem} \label{LEM-CMP-QQBARINSEP}
$\Th{Q}$ and $\Th{\bar{Q}} = \Setabs{!A}{\Th{Q} \Proves \lnot !A}$
are computably inseparable.
\end{lem}

\begin{proof}[Proof sketch]
Suppose $C$ is a computable set with $\Th{Q} \subseteq C$ and
$\Th{\bar{Q}} \subseteq \Complement{C}$.  Define $R(x,y)$ to hold iff
$x$ codes a formula~$!D(u)$ and $!D(\num{y}) \in C$.  Since $C$ is
computable, $R$ is computable.  We show $R$ is a universal computable
relation, contradicting the fact (established by diagonalization) that
no such relation exists.  If $S(y)$ is any computable relation,
represented by $!D_S(u)$ in $\Th{Q}$, then $S(n) \Rightarrow \Th{Q}
\Proves !D_S(\num{n}) \Rightarrow !D_S(\num{n}) \in C \Rightarrow
R(\Gn{!D_S(u)}, n)$, and $\lnot S(n) \Rightarrow \Th{Q} \Proves
\lnot !D_S(\num{n}) \Rightarrow !D_S(\num{n}) \in \Th{\bar{Q}}
\subseteq \Complement{C} \Rightarrow \lnot R(\Gn{!D_S(u)}, n)$.
\end{proof}


%% ===================================================================
%% CMP.7: Theorems
%% ===================================================================

\section{Theorems} \label{CMP.7}

We conclude the chapter with the central theorems of computability
theory: Rice's theorem, the recursion (fixed-point) theorem, the
unsolvability of the decision problem for first-order logic, and a
collection of undecidability and incompleteness results for
arithmetical theories.

\subsection{Rice's theorem} \label{CMP.7.1}

\begin{thm}[Rice's Theorem] % THM-CMP003
\label{THM-CMP003}
  Let $C$ be any set of partial computable functions, and let $A =
  \Setabs{n}{\cfind{n} \in C}$.  If $A$ is computable, then either $C$
  is empty or $C$ is the set of all partial computable functions.
\end{thm}

Rice's theorem says that no nontrivial \emph{index set} is decidable.
It is important to understand what the theorem does and does not say.
There are certainly computable questions about programs as syntactic
objects (``does this program have more than 150 symbols?'').  Rice's
theorem says that no nontrivial question about a program's
\emph{behavior}---what function it computes---is decidable.  This
includes questions like: does it halt on input~$0$?  Does it ever
halt?  Does it ever output an even number?

\begin{proof}
Suppose $C$ is neither empty nor the set of all partial
computable functions, and let $A$ be the set of indices of functions
in~$C$.  We show that if $A$ were computable, we could solve the
halting problem; so $A$~is not computable.

Without loss of generality, assume that the nowhere-defined function
$f$ is not in $C$ (otherwise, switch $C$ and its complement).  Let $g$
be any function in~$C$.  Define
\[
h(x,y) \simeq
\begin{cases}
\text{undefined} & \text{if $\cfind{x}(x) \fundefined$} \\
g(y) & \text{otherwise.}
\end{cases}
\]
More precisely, $h(x,y) \simeq \Proj{2}{0}(g(y), \fn{Un}(x,x))$,
which is defined and equals $g(y)$ exactly when both $\fn{Un}(x,x)$
and $g(y)$ are defined.

For a fixed~$x$: if $\cfind{x}(x)$ is undefined, then $h(x,y)$ is
undefined for every~$y$, so $h_x$ acts like~$f$; if $\cfind{x}(x)$ is
defined, then $h(x,y) \simeq g(y)$, so $h_x$ acts like~$g$.

Since $h$ is partial computable, it equals $\cfind{e}$ for some~$e$.
By the $s$-$m$-$n$ theorem, there is a primitive recursive~$s$ such
that $\cfind{s(e,x)}(y) = h_x(y)$.  Now for each $x$: if
$\cfind{x}(x) \fdefined$, then $\cfind{s(e,x)}$ computes~$g \in C$,
so $s(e,x) \in A$; if $\cfind{x}(x) \fundefined$, then
$\cfind{s(e,x)}$ computes~$f \notin C$, so $s(e,x) \notin A$.
Hence $x \in K$ iff $s(e,x) \in A$.  If $A$ were computable, so would
$K$---contradiction.
\end{proof}

\begin{cor} \label{COR-CMP-RICE}
The following sets are undecidable:
\begin{enumerate}
\item $\Setabs{x}{\text{$17$ is in the range of $\cfind{x}$}}$,
\item $\Setabs{x}{\text{$\cfind{x}$ is constant}}$,
\item $\Setabs{x}{\text{$\cfind{x}$ is total}}$,
\item $\Setabs{x}{\text{whenever $y < y'$, $\cfind{x}(y) \fdefined$,
    and if $\cfind{x}(y') \fdefined$, then $\cfind{x}(y) <
    \cfind{x}(y')$}}$.
\end{enumerate}
\end{cor}

\begin{proof}
These are all nontrivial index sets.
\end{proof}


\subsection{The recursion theorem} \label{CMP.7.2}

The recursion theorem (also known as the fixed-point theorem) says
that any computable transformation of programs has a fixed point.
Think of it this way: given any partial computable $g(x,y)$, one can
find a program~$e$ that computes $g_e(y) = g(e,y)$---a program whose
behavior depends on its own code.  This is the theoretical basis for
self-referential constructions such as quines (programs that print
their own source code).

\begin{lem} \label{LEM-CMP-FIXEDEQUIV}
The following statements are equivalent:
\begin{enumerate}
\item For every partial computable function $g(x,y)$, there is an
  index~$e$ such that for every~$y$,
  $\cfind{e}(y) \simeq g(e,y)$.
\item For every computable function~$f(x)$, there is an index~$e$ such
  that for every~$y$,
  $\cfind{e}(y) \simeq \cfind{f(e)}(y)$.
\end{enumerate}
\end{lem}

\begin{proof}
$(1) \Rightarrow (2)$: Given $f$, define $g(x,y) \simeq
\fn{Un}(f(x),y)$ and apply (1).

$(2) \Rightarrow (1)$: Given $g$, use the $s$-$m$-$n$ theorem to get
$f$ such that $\cfind{f(x)}(y) \simeq g(x,y)$ and apply (2).
\end{proof}

\begin{thm}[Recursion Theorem / Fixed-Point Theorem] % THM-CMP005
\label{THM-CMP005}
For every partial computable function $g(x,y)$, there is an index~$e$
such that for every~$y$,
\[
\cfind{e}(y) \simeq g(e,y).
\]
\end{thm}

\begin{proof}
Let $\fn{diag}(x)$ be a computable function such that
$\cfind{\fn{diag}(x)}(y) \simeq \cfind{x}(x,y)$.
Such a function exists by the $s$-$m$-$n$ theorem: define $s(x,y)
\simeq \fn{Un}^2(x,x,y)$ and let $\fn{diag}$ satisfy
$\cfind{\fn{diag}(x)}(y) \simeq s(x,y)$.

Now define $l(x,y) \simeq g(\fn{diag}(x), y)$ and let $\gn{l}$ be an
index for~$l$.  Set $e = \fn{diag}(\gn{l})$.  Then:
\begin{align*}
\cfind{e}(y) &\simeq \cfind{\fn{diag}(\gn{l})}(y)
  \simeq \cfind{\gn{l}}(\gn{l}, y)
  \simeq l(\gn{l}, y) \\
  &\simeq g(\fn{diag}(\gn{l}), y)
  \simeq g(e, y). \qedhere
\end{align*}
\end{proof}


\subsection{Unsolvability of the decision problem} \label{CMP.7.3}

\begin{thm} \label{THM-CMP-DECPROB}
The decision problem is unsolvable: there is no Turing machine~$D$
which, when started on a tape containing a sentence~$!B$ of
first-order logic as input, eventually halts and outputs~$1$ iff $!B$
is valid and $0$ otherwise.
\end{thm}

\begin{proof}
Suppose the decision problem were solvable by a Turing machine~$D$.
We construct a Turing machine~$E$ that, given input~$e$ and~$w$,
computes the sentence $!T(M_e, w) \lif !E(M_e, w)$ (see
\cref{DEF-CMP009}).  The machine $E \concat D$ then computes
$!T(M_e,w) \lif !E(M_e,w)$ and runs~$D$ on the result.  By
\cref{LEM-CMP-VALIDHALT} and \cref{LEM-CMP-HALTVALID}, $!T(M_e,w)
\lif !E(M_e,w)$ is valid iff $M_e$ halts on input~$w$.  So $E \concat
D$ solves the halting problem, contradicting \cref{THM-CMP002}.
\end{proof}

\begin{cor} \label{COR-CMP-UNDECSAT}
It is undecidable whether an arbitrary sentence of first-order logic is
satisfiable.
\end{cor}

\begin{proof}
If satisfiability were decidable by~$S$, we could decide validity:
given~$!B$, run $S$ on $\lnot !B$; then $!B$ is valid iff $\lnot !B$
is unsatisfiable.
\end{proof}

\begin{thm} \label{THM-CMP-VALIDCE}
Validity of first-order sentences is semi-decidable: there is a Turing
machine~$E$ which halts with output~$1$ iff $!B$ is valid, but does
not halt otherwise.
\end{thm}

\begin{proof}
All possible derivations can be generated one after another by an
effective algorithm.  The machine~$E$ generates derivations and halts
with output~$1$ when it finds one showing $\Proves !B$.  By
soundness, if $E$ halts then $\Entails !B$.  By completeness, if
$\Entails !B$ then such a derivation exists and will eventually be
found.
\end{proof}


\subsection{Undecidability of arithmetic theories} \label{CMP.7.4}

We now connect computability to the incompleteness phenomena.  The
results below show that essentially any theory strong enough to
represent computable functions is undecidable, and that axiomatizable
extensions cannot be complete.

\begin{thm} \label{THM-CMP-CONSDECRELS}
If $\Gamma$ is a consistent theory that represents every decidable
relation, then $\Gamma$ is not decidable.
\end{thm}

\begin{proof}[Proof sketch]
Suppose $\Gamma$ were decidable.  Enumerate all formulas with one free
variable as $!A_0(x), !A_1(x), \dots$ and define $D =
\Setabs{n}{\Gamma \Proves \lnot !A_n(\num{n})}$.  Since $\Gamma$ is
decidable, $D$ is decidable.  Since $\Gamma$ represents all decidable
relations, $D$ is represented by some $!A_d(x)$.  Then $d \in D$ leads
to $\Gamma \Proves !A_d(\num{d})$ (since $!A_d$ represents~$D$) and
$\Gamma \Proves \lnot !A_d(\num{d})$ (by definition of~$D$), making
$\Gamma$ inconsistent.
\end{proof}

\begin{thm} \label{THM-CMP-AXTCOMPDEC}
If $\Gamma$ is axiomatizable and complete, then $\Gamma$ is decidable.
\end{thm}

\begin{proof}[Proof sketch]
If $\Gamma$ is inconsistent, it is trivially decidable.  Otherwise,
simultaneously search for a derivation of~$!A$ and a derivation
of~$\lnot !A$ from the axioms of~$\Gamma$.  Since $\Gamma$ is
complete, one search must succeed; since $\Gamma$ is consistent, the
other cannot.
\end{proof}

\begin{cor}[Weak first incompleteness] \label{COR-CMP-WEAKINC}
If $\Gamma$ is consistent, axiomatizable, and represents every
decidable property, then $\Gamma$ is not complete.
\end{cor}

\begin{proof}
If $\Gamma$ were complete, it would be decidable (since it is
axiomatizable), contradicting \cref{THM-CMP-CONSDECRELS}.
\end{proof}

\begin{thm} \label{THM-CMP-QCE}
$\Th{Q}$ is c.e.\ but not decidable.  In fact, $\Th{Q}$ is a complete
c.e.\ set.
\end{thm}

\begin{proof}[Proof sketch]
$\Th{Q}$ is c.e.\ since $\Th{Q} = \Setabs{y}{\lexists[x][\Prf[\Th{Q}](x,y)]}$
and $\Prf[\Th{Q}]$ is primitive recursive.  To show c.e.-completeness,
we reduce $K$ to $\Th{Q}$: since Kleene's predicate $T(e,x,s)$ is
represented in $\Th{Q}$ by some~$!A_T$, we have $x \in K$ iff $\Th{Q}
\Proves \lexists[s][!A_T(\num{x}, \num{x}, s)]$.  The function
mapping $x$ to (a code for) $\lexists[s][!A_T(\num{x}, \num{x}, s)]$
is a reduction of $K$ to~$\Th{Q}$.
\end{proof}

\begin{thm} \label{THM-CMP-CONSQEXT}
Let $\Th{T}$ be any consistent theory that includes $\Th{Q}$.  Then
$\Th{T}$ is not decidable.
\end{thm}

\begin{proof}[Proof sketch]
If $\Th{T}$ were consistent and decidable, we could define a universal
computable relation $R(x,y)$: ``$x$ codes a formula $!D(u)$ and
$\Th{T}$ proves $!D(\num{y})$.''  Since $\Th{T}$ extends $\Th{Q}$,
every computable relation $S(y)$ represented by $!D_S(u)$ satisfies
$S(n)$ iff $R(\Gn{!D_S(u)}, n)$.  But no universal computable
relation exists (by diagonalization), so $\Th{T}$ is not decidable.
\end{proof}

\begin{thm} \label{THM-CMP-CONSWITHQ}
Let $\Th{T}$ be any theory in the language of arithmetic that is
consistent with $\Th{Q}$ (i.e., $\Th{T} \cup \Th{Q}$ is
consistent).  Then $\Th{T}$ is undecidable.
\end{thm}

\begin{proof}[Proof sketch]
Suppose $\Th{T}$ is decidable and consistent with~$\Th{Q}$.  Let $!E
= !Q_1 \land \dots \land !Q_8$ be the conjunction of the axioms
of~$\Th{Q}$, and define $C = \Setabs{!A}{\Th{T} \Proves !E \lif !A}$.
Then $C$ is computable.  If $!A \in \Th{Q}$, then $\Proves !E \lif
!A$, so $!A \in C$.  If $!A \in \Th{\bar{Q}}$, then $\Proves !E \lif
\lnot !A$; if also $\Th{T} \Proves !E \lif !A$, then $\Th{T} \Proves
\lnot !E$, contradicting consistency of $\Th{T} \cup \Th{Q}$.  So $!A
\notin C$.  Thus $C$ computably separates $\Th{Q}$ and $\Th{\bar{Q}}$,
contradicting \cref{LEM-CMP-QQBARINSEP}.
\end{proof}

\begin{thm} \label{THM-CMP-INTERP}
Suppose $\Th{T}$ is a theory in a language in which one can interpret
the language of arithmetic, in such a way that $\Th{T}$ is consistent
with the interpretation of $\Th{Q}$.  Then $\Th{T}$ is undecidable.
If $\Th{T}$ proves the interpretation of the axioms of $\Th{Q}$, then
no consistent extension of $\Th{T}$ is decidable.
\end{thm}

\begin{proof}[Proof sketch]
The proof is a small modification of the proof of
\cref{THM-CMP-CONSWITHQ}: a counterexample would yield a computable
separation of $\Th{Q}$ and $\Th{\bar{Q}}$ via the interpretation.
\end{proof}

\begin{cor} \label{COR-CMP-ZFC}
There is no decidable extension of $\Th{ZFC}$ (Zermelo--Fraenkel set
theory with the axiom of choice).  In particular, there is no
complete, consistent, axiomatizable extension of $\Th{ZFC}$.
\end{cor}

\begin{cor} \label{COR-CMP-FOLBIN}
First-order logic for any language with a binary relation symbol is
undecidable.
\end{cor}

\begin{rem}[Decidability boundary]
\label{REM-CMP-DECBOUND}
The undecidability of first-order logic extends to any language with
two unary function symbols (since these can simulate a binary
relation).  On the other hand, first-order logic for a language with
only unary relation symbols and at most one unary function symbol is
decidable.  Similarly, Presburger arithmetic---the set of sentences in
the language $\Obj{0}$, $\prime$, $+$ true in $\Struct{N}$---is
decidable (though computationally very expensive), while the set of
true sentences in the language $\Obj{0}$, $\prime$, $\times$ is
already undecidable.
\end{rem}
   % CH-CMP: Computation
\chapter{Metatheory} \label{ch:meta}

%% ===================================================================
%% META.1: Soundness (CP-001)
%% Source: written from scratch (unified statement referencing DED.2--DED.5)
%% ===================================================================

\section{Soundness} \label{META.1}

The Soundness Theorem establishes that the derivability relation~$\Proves$
is truth-preserving: nothing that can be derived from a set of
sentences~$\Gamma$ goes beyond what is semantically entailed by~$\Gamma$.

\begin{thm}[Soundness Theorem] % CP-001
\label{CP-001}
If $\Gamma \Proves !A$, then $\Gamma \Entails !A$.
\end{thm}

This result is proved independently for each of the four proof systems
treated in this text:
\begin{itemize}
\item \textbf{Axiomatic (Hilbert-style) deduction} --- see \S\ref{DED.2}.
\item \textbf{Natural deduction} --- see \S\ref{DED.3}.
\item \textbf{Sequent calculus} --- see \S\ref{DED.4}.
\item \textbf{Tableaux} --- see \S\ref{DED.5}.
\end{itemize}

In each case, the proof proceeds by induction on the length (or tree
structure) of the derivation.  The base cases verify that every logical
axiom (in the axiomatic system) or initial sequent / leaf assumption is
logically valid or follows from the assumptions.  The inductive step shows
that each rule of inference---modus ponens, introduction/elimination
rules, structural rules, or tableau expansion rules, depending on the
system---preserves truth: if the premises of the rule are true in every
structure satisfying~$\Gamma$, so is the conclusion.  Since every step in
the derivation preserves truth, the final conclusion is entailed
by~$\Gamma$.

An important corollary is that every satisfiable set is consistent:

\begin{cor} % CP-001
\label{cor:satisfiable-consistent}
If $\Gamma$ is satisfiable, then $\Gamma$ is consistent.
Equivalently, if $\Gamma$ is inconsistent, then $\Gamma$ is
unsatisfiable.
\end{cor}

\begin{proof}
Suppose $\Gamma$ is satisfiable and, toward a contradiction, suppose
$\Gamma$ is inconsistent, i.e., $\Gamma \Proves \lfalse$.  By Soundness
(\ref{CP-001}), $\Gamma \Entails \lfalse$.  But $\lfalse$ is false in
every structure, so no structure satisfies~$\Gamma$---contradicting the
assumption that $\Gamma$ is satisfiable.
\end{proof}


%% ===================================================================
%% META.2: Completeness (CP-002)
%% Sources: fol/com/ccs, fol/com/mcs, fol/com/hen, fol/com/lin,
%%          fol/com/mod, fol/com/ide, fol/com/cth
%% ===================================================================

\section{Completeness} \label{META.2}

The Completeness Theorem, due to G\"odel (1930) with an influential
alternative proof by Henkin (1949), establishes the converse of
soundness: every sentence that is semantically entailed by a set of
sentences is derivable from it.  Equivalently, every consistent set of
sentences is satisfiable.  We follow the Henkin proof strategy, which
constructs a model out of the syntax itself.

%%% -----------------------------------------------------------------
%%% META.2.1  Complete Consistent Sets
%%% -----------------------------------------------------------------

\subsection{Complete Consistent Sets of Sentences}

\begin{defn}[Complete set] % DEF-SEM005 (authoritative: \S SEM.3)
A set~$\Gamma$ of sentences is \emph{complete} iff for any
sentence~$!A$, either $!A \in \Gamma$ or $\lnot !A \in \Gamma$.
\end{defn}

Complete consistent sets are central to the completeness proof: we will
show that every consistent set of sentences~$\Gamma$ can be extended to
a complete consistent set~$\Gamma^*$, from which a satisfying structure
is constructed.

In what follows, we will often tacitly use the properties of
reflexivity, monotonicity, and transitivity of $\Proves$ (see
\S\ref{DED.1}).

\begin{prop} \label{prop:ccs}
Suppose $\Gamma$ is complete and consistent. Then:
\begin{enumerate}
\item \label{prop:ccs-prov-in} If $\Gamma \Proves !A$, then $!A \in
  \Gamma$.

\item \label{prop:ccs-and} $!A \land !B \in \Gamma$
  iff both $!A \in \Gamma$ and $!B \in \Gamma$.

\item \label{prop:ccs-or} $!A \lor !B \in \Gamma$ iff
  either $!A \in \Gamma$ or $!B \in \Gamma$.

\item \label{prop:ccs-if} $!A \lif !B \in \Gamma$ iff
  either $!A \notin \Gamma$ or $!B \in \Gamma$.
\end{enumerate}
\end{prop}

\begin{proof}
Let us suppose for all of the following that $\Gamma$ is complete and
consistent.
\begin{enumerate}
\item If $\Gamma \Proves !A$, then $!A \in \Gamma$.

Suppose that $\Gamma \Proves !A$.  Suppose to the contrary that $!A
\notin \Gamma$.  Since $\Gamma$ is complete, $\lnot !A \in \Gamma$.
By the properties of derivability (see DEF-DED003, \S\ref{DED.1}),
$\Gamma$ is inconsistent.  This contradicts the assumption that
$\Gamma$ is consistent.  Hence, it cannot be the case that $!A \notin
\Gamma$, so $!A \in \Gamma$.

\item $!A \land !B \in \Gamma$ iff both $!A \in \Gamma$ and $!B \in \Gamma$:

For the forward direction, suppose $!A \land !B \in \Gamma$.  Then
by the provability properties of~$\land$ (see \S\ref{DED.1}), item~(1),
$\Gamma \Proves !A$ and $\Gamma \Proves !B$.  By
\ref{prop:ccs-prov-in}, $!A \in \Gamma$ and $!B \in \Gamma$, as
required.

For the reverse direction, let $!A \in \Gamma$ and $!B \in
\Gamma$.  By the provability properties of~$\land$, item~(2),
$\Gamma \Proves !A \land !B$.  By \ref{prop:ccs-prov-in}, $!A \land
!B \in \Gamma$.

\item First we show that if $!A \lor !B \in \Gamma$, then either $!A \in
\Gamma$ or $!B \in \Gamma$.  Suppose $!A \lor !B \in \Gamma$ but $!A
\notin \Gamma$ and $!B \notin \Gamma$.  Since $\Gamma$ is
complete, $\lnot !A \in \Gamma$ and $\lnot !B \in \Gamma$.  By
the provability properties of~$\lor$ (see \S\ref{DED.1}),
item (1), $\Gamma$ is inconsistent, a contradiction.  Hence, either $!A
\in \Gamma$ or $!B \in \Gamma$.

For the reverse direction, suppose that $!A \in \Gamma$ or $!B \in
\Gamma$.  By the provability properties of~$\lor$, item (2),
$\Gamma \Proves !A \lor !B$.  By \ref{prop:ccs-prov-in}, $!A \lor
!B \in \Gamma$, as required.

\item For the forward direction, suppose $!A \lif !B \in \Gamma$, and suppose
to the contrary that $!A \in \Gamma$ and $!B \notin \Gamma$.  On these
assumptions, $!A \lif !B \in \Gamma$ and $!A \in \Gamma$.  By
the provability properties of~$\lif$ (see \S\ref{DED.1}), item~(1),
$\Gamma \Proves !B$.  But then by \ref{prop:ccs-prov-in}, $!B \in
\Gamma$, contradicting the assumption that $!B \notin \Gamma$.

For the reverse direction, first consider the case where $!A \notin
\Gamma$.  Since $\Gamma$ is complete, $\lnot !A \in \Gamma$.  By
the provability properties of~$\lif$, item~(2),
$\Gamma \Proves !A \lif !B$.  Again by \ref{prop:ccs-prov-in}, we get
that $!A \lif !B \in \Gamma$, as required.

Now consider the case where $!B \in \Gamma$.  By
the provability properties of~$\lif$,
item (2) again, $\Gamma \Proves !A \lif !B$.  By
\ref{prop:ccs-prov-in}, $!A \lif !B \in \Gamma$.
\end{enumerate}
\end{proof}

%%% -----------------------------------------------------------------
%%% META.2.2  Maximally Consistent Sets
%%% -----------------------------------------------------------------

\subsection{Maximally Consistent Sets of Sentences}

\begin{defn}[Maximally consistent set] % DEF-DED002 (authoritative: \S DED.1)
A set~$\Gamma$ of sentences is \emph{maximally consistent} iff
\begin{enumerate}
\item $\Gamma$ is consistent, and
\item if $\Gamma \subsetneq \Gamma'$, then $\Gamma'$ is inconsistent.
\end{enumerate}
\end{defn}

Every maximally consistent set is complete and consistent; it therefore
has all the properties established in Proposition~\ref{prop:ccs}.

%%% -----------------------------------------------------------------
%%% META.2.3  Henkin Expansion
%%% -----------------------------------------------------------------

\subsection{Henkin Expansion}

In order to guarantee that the model we construct from a complete
consistent set~$\Gamma$ makes all the quantified formulas in~$\Gamma$
true, we use a trick due to Leon Henkin: expand the language by
infinitely many constants and add, for each formula with one free
variable $!A(x)$, a formula of the form $\lexists[x][!A(x)] \lif !A(c)$,
where $c$ is one of the new constants.

\begin{prop} \label{prop:lang-exp}
If $\Gamma$ is consistent in $\Lang L$ and $\Lang L'$ is obtained from
$\Lang L$ by adding a denumerable set of new constants $\Obj d_0$,
$\Obj d_1$, \dots, then $\Gamma$ is consistent in~$\Lang L'$.
\end{prop}

\begin{defn}[Saturated set] \label{defn:saturated-set}
A set $\Gamma$ of formulas of a language $\Lang {L}$ is
\emph{saturated} iff for each formula~$!A(x) \in \Frm[L]$ with one
free variable~$x$ there is a constant~$c \in \Lang{L}$ such
that $\lexists[x][!A(x)] \lif !A(c) \in \Gamma$.
\end{defn}

The following definition will be used in the proof of the next lemma.

\begin{defn} \label{defn:henkin-exp}
Let $\Lang L'$ be as in Proposition~\ref{prop:lang-exp}.  Fix an enumeration
$!A_0(x_0)$, $!A_1(x_1)$, \dots of all formulas~$!A_i(x_i)$
of~$\Lang L'$ in which one variable ($x_i$) occurs free.  We define
the sentences~$!D_n$ by induction on~$n$.

Let $c_0$ be the first constant among the $\Obj d_i$ we added
to~$\Lang{L}$ which does not occur in~$!A_0(x_0)$.  Assuming that
$!D_0$, \dots,~$!D_{n-1}$ have already been defined, let $c_n$ be the
first among the new constants~$\Obj d_i$ that occurs neither in
$!D_0$, \dots,~$!D_{n-1}$ nor in~$!A_n(x_n)$.

Now let $!D_{n}$ be the formula
$\lexists[x_{n}][!A_{n}(x_{n})] \lif !A_{n}(c_{n})$.
\end{defn}

\begin{lem}[Henkin's Lemma] % THM-META-HEN
\label{lem:henkin}
Every consistent set~$\Gamma$ can be extended to a saturated
consistent set~$\Gamma'$.
\end{lem}

\begin{proof}
Given a consistent set of sentences~$\Gamma$ in a language~$\Lang{L}$,
expand the language by adding a denumerable set of new
constants to form~$\Lang{L'}$.  By Proposition~\ref{prop:lang-exp}, $\Gamma$
is still consistent in the richer language.  Further, let $!D_i$ be as
in Definition~\ref{defn:henkin-exp}.  Let
\begin{align*}
\Gamma_0 & = \Gamma \\
\Gamma_{n+1} & = \Gamma_n \cup \{!D_n \}
\end{align*}
i.e., $\Gamma_{n+1} = \Gamma \cup \{ !D_0, \dots, !D_n \}$, and let
$\Gamma' = \bigcup_{n} \Gamma_n$.  $\Gamma'$ is clearly saturated.

If $\Gamma'$ were inconsistent, then for some $n$, $\Gamma_n$ would be
inconsistent.  So to show that $\Gamma'$ is
consistent it suffices to show, by induction on~$n$, that each
set~$\Gamma_n$ is consistent.

The induction basis is simply the claim that $\Gamma_0 = \Gamma$ is
consistent, which is the hypothesis of the lemma.  For the induction
step, suppose that $\Gamma_{n}$ is consistent but $\Gamma_{n+1} =
\Gamma_n \cup \{!D_n\}$ is inconsistent.  Recall that $!D_n$~is
$\lexists[x_{n}][!A_{n}(x_n)] \lif !A_{n}(c_{n})$,
where $!A_n(x_n)$ is a formula of $\Lang{L'}$ with only the
variable~$x_n$ free. By the way we have chosen the~$c_n$ (see
Definition~\ref{defn:henkin-exp}), $c_n$ does not occur in~$!A_n(x_n)$ nor
in~$\Gamma_n$.

If $\Gamma_n \cup \{!D_n\}$ is inconsistent, then $\Gamma_n
\Proves \lnot !D_n$, and hence both of the following hold:
\[
\Gamma_n \Proves \lexists[x_n][!A_n(x_n)]
\qquad
\Gamma_n \Proves \lnot !A_n(c_n)
\]
Since $c_n$ does not occur in
$\Gamma_n$ or in~$!A_n(x_n)$,
the strong generalization theorem applies (see \S\ref{DED.1}).
From $\Gamma_n \Proves \lnot !A_n(c_n)$,
we obtain
$\Gamma_n \Proves \lforall[x_n][\lnot !A_n(x_n)]$.
Thus we have that both
$\Gamma_n \Proves \lexists[x_n][!A_n(x_n)]$ and
$\Gamma_n \Proves \lforall[x_n][\lnot !A_n(x_n)]$,
so $\Gamma_n$ itself is inconsistent.
Contradiction: $\Gamma_n$ was supposed to be consistent.  Hence
$\Gamma_n \cup \{ !D_n\}$ is consistent.
\end{proof}

We now show that complete, consistent sets which are saturated
have the property that they contain an existentially quantified sentence
iff they contain at least one instance, and they contain a universally
quantified sentence iff they contain all instances.

\begin{prop}\label{prop:saturated-instances}
Suppose $\Gamma$ is complete, consistent, and saturated.
\begin{enumerate}
\item $\lexists[x][!A(x)] \in \Gamma$ iff $!A(t) \in \Gamma$
  for at least one closed term~$t$.
\item $\lforall[x][!A(x)] \in \Gamma$ iff $!A(t) \in \Gamma$
  for all closed terms~$t$.
\end{enumerate}
\end{prop}

\begin{proof}
\begin{enumerate}
\item First suppose that $\lexists[x][!A(x)]
      \in \Gamma$.  Because $\Gamma$ is saturated,
      $(\lexists[x][!A(x)] \lif !A(c)) \in \Gamma$ for some
      constant~$c$. By the provability properties of~$\lif$ (see
      Proposition~\ref{prop:ccs}\ref{prop:ccs-if}),
      and Proposition~\ref{prop:ccs}\ref{prop:ccs-prov-in}, $!A(c)
      \in \Gamma$.

    For the other direction, saturation is not necessary: Suppose
    $!A(t) \in \Gamma$.  Then $\Gamma \Proves \lexists[x][!A(x)]$ by
    the provability properties of quantifiers (see \S\ref{DED.1}), item~(1). By
    Proposition~\ref{prop:ccs}\ref{prop:ccs-prov-in},
    $\lexists[x][!A(x)] \in \Gamma$.

\item Suppose that $!A(t) \in \Gamma$ for
      all closed terms~$t$.  By way of contradiction, assume
      $\lforall[x][!A(x)] \notin \Gamma$.  Since $\Gamma$ is complete,
      $\lnot\lforall[x][!A(x)] \in \Gamma$.  By saturation,
      $(\lexists[x][\lnot !A(x)] \lif \lnot !A(c)) \in
        \Gamma$ for some constant~$c$.  By assumption, since $c$
      is a closed term, $!A(c) \in \Gamma$.  But this would make
      $\Gamma$ inconsistent, since
      $\lnot \lforall[x][!A(x)]$,
      $\lexists[x][\lnot !A(x)] \lif \lnot !A(c)$, $!A(c)$
      is inconsistent.

      For the reverse direction, we do not need saturation: Suppose
      $\lforall[x][!A(x)] \in \Gamma$.  Then $\Gamma \Proves !A(t)$
      by the provability properties of quantifiers (see \S\ref{DED.1}),
      item~(2). We get $!A(t) \in \Gamma$ by
      Proposition~\ref{prop:ccs}.
\end{enumerate}
\end{proof}

%%% -----------------------------------------------------------------
%%% META.2.4  Lindenbaum's Lemma
%%% -----------------------------------------------------------------

\subsection{Lindenbaum's Lemma}

\begin{lem}[Lindenbaum's Lemma] % THM-DED005
\label{THM-DED005}
Every consistent set~$\Gamma$ in a language~$\Lang{L}$ can be
extended to a complete and consistent set~$\Gamma^*$.
\end{lem}

\begin{proof}
Let $\Gamma$ be consistent.  Let $!A_0$, $!A_1$,
\dots{} be an enumeration of all the sentences of~$\Lang L$.
Define $\Gamma_0 = \Gamma$, and
\[
\Gamma_{n+1} =
\begin{cases}
\Gamma_n \cup \{ !A_n \} & \textrm{if $\Gamma_n \cup \{!A_n\}$ is
  consistent;} \\
\Gamma_n \cup \{ \lnot !A_n \} & \textrm{otherwise.}
\end{cases}
\]
Let $\Gamma^* = \bigcup_{n \geq 0} \Gamma_n$.

Each $\Gamma_n$ is consistent: $\Gamma_0$ is consistent by definition.
If $\Gamma_{n+1} = \Gamma_n \cup \{!A_n\}$, this is because the latter
is consistent.  If it is not, $\Gamma_{n+1} = \Gamma_n \cup \{\lnot
!A_n\}$. We have to verify that $\Gamma_n \cup \{\lnot !A_n\}$ is
consistent. Suppose it is not. Then \emph{both} $\Gamma_n \cup
\{!A_n\}$ and $\Gamma_n \cup \{\lnot !A_n\}$ are inconsistent.  This
means that $\Gamma_n$ would be inconsistent by
the exhaustive cases property of derivability (see \S\ref{DED.1}),
contrary to the induction hypothesis.

For every~$n$ and every $i < n$, $\Gamma_i \subseteq \Gamma_n$. This
follows by a simple induction on~$n$. For $n=0$, there are no $i < 0$,
so the claim holds automatically.  For the inductive step, suppose it
is true for~$n$. We show that if $i < n+1$ then $\Gamma_i \subseteq
\Gamma_{n+1}$. We have $\Gamma_{n+1} = \Gamma_n \cup \{!A_n\}$ or $=
\Gamma_n \cup \{\lnot !A_n\}$ by construction. So $\Gamma_n \subseteq
\Gamma_{n+1}$. If $i < n+1$, then $\Gamma_i \subseteq \Gamma_n$ by
inductive hypothesis (if $i < n$) or the trivial fact that $\Gamma_n
\subseteq \Gamma_n$ (if $i = n$). We get that $\Gamma_i \subseteq
\Gamma_{n+1}$ by transitivity of~$\subseteq$.

From this it follows that $\Gamma^*$ is consistent. Here is why: Let
$\Gamma' \subseteq \Gamma^*$ be finite. Each $!B \in \Gamma'$ is also
in~$\Gamma_i$ for some~$i$. Let $n$ be the largest of these. Since
$\Gamma_i \subseteq \Gamma_n$ if $i \le n$, every $!B \in \Gamma'$ is
also $\in \Gamma_n$, i.e., $\Gamma' \subseteq \Gamma_n$, and
$\Gamma_n$~is consistent. So, every finite subset $\Gamma' \subseteq
\Gamma^*$ is consistent. By the compactness of derivability
(see \S\ref{DED.1}), $\Gamma^*$ is
  consistent.

Every sentence of $\Frm[L]$ appears on the list used to
define~$\Gamma^*$. If $!A_n \notin \Gamma^*$, then that is because
$\Gamma_n \cup \{!A_n\}$ was inconsistent.  But then $\lnot !A_n
\in \Gamma^*$, so $\Gamma^*$ is complete.
\end{proof}

%%% -----------------------------------------------------------------
%%% META.2.5  Construction of a Model
%%% -----------------------------------------------------------------

\subsection{Construction of a Model}

We first extend~$\Gamma$ to a consistent, complete, and saturated
set~$\Gamma^*$.  The term model~$\Struct{M(\Gamma^*)}$ takes the set
of closed terms of~$\Lang{L'}$ as the domain, assigns every constant
to itself, and defines predicate extensions so that an atomic sentence
is true in $\Struct{M(\Gamma^*)}$ iff it is in~$\Gamma^*$.

\begin{defn}[Term model] % DEF-META-TM
\label{defn:termmodel}
Let $\Gamma^*$ be a complete and consistent,
saturated set of sentences in a language~$\Lang L$. The \emph{term
  model}~$\Struct M(\Gamma^*)$ of $\Gamma^*$ is the structure
defined as follows:
\begin{enumerate}
\item The domain~$\Domain{M(\Gamma^*)}$ is the set of all closed
  terms of~$\Lang L$.
\item The interpretation of a constant $c$ is $c$ itself:
  $\Assign{c}{M(\Gamma^*)} = c$.
\item The function~$f$ is assigned the function which, given as
  arguments the closed terms $t_1$, \dots, $t_n$, has as value the
  closed term $f(t_1, \dots, t_n)$:
\[
\Assign{f}{M(\Gamma^*)}(t_1, \dots, t_n) = f(t_1,\dots, t_n)
\]
\item If $R$ is an $n$-place predicate, then
  \[
  \tuple{t_1, \dots,
  t_n} \in \Assign{R}{M(\Gamma^*)} \text{ iff } \Atom{R}{t_1, \dots,
    t_n} \in \Gamma^*.
  \]
\end{enumerate}
\end{defn}

\begin{lem}
\label{lem:val-in-termmodel} Let $\Struct M(\Gamma^*)$ be the term model
of Definition~\ref{defn:termmodel}; then $\Value{t}{M(\Gamma^*)} = t$.
\end{lem}

\begin{proof}
 The proof is by induction on $t$, where the base case, when $t$
 is a constant, follows directly from the definition of the term
 model. For the induction step assume $t_1, \ldots, t_n$ are closed terms
 such that $\Value{t_i}{M(\Gamma^*)} = t_i$ and that $f$ is an $n$-ary
 function. Then
\begin{align*}
\Value{f(t_1,\ldots,t_n)}{M(\Gamma^*)} &= \Assign{f}{M(\Gamma^*)}(\Value{t_1}
 {M(\Gamma^*)},
\ldots, \Value{t_n}{M(\Gamma^*)}) \\
&= \Assign{f}{M(\Gamma^*)}(t_1, \dots, t_n) \\
&= f(t_1,\dots, t_n),
\end{align*}
and so by induction this holds for every closed term~$t$.
\end{proof}

\begin{prop}
\label{prop:quant-termmodel}
Let $\Struct M(\Gamma^*)$ be the term model of Definition~\ref{defn:termmodel}.
\begin{enumerate}
\item $\Sat{M(\Gamma^*)}{\lexists[x][!A(x)]}$ iff
  $\Sat{M(\Gamma^*)}{!A(t)}$ for at least one closed term~$t$.
\item $\Sat{M(\Gamma^*)}{\lforall[x][!A(x)]}$ iff
  $\Sat{M(\Gamma^*)}{!A(t)}$ for all closed terms~$t$.
\end{enumerate}
\end{prop}

\begin{proof}
\begin{enumerate}
\item By the definition of satisfaction (see DEF-SEM002, \S\ref{SEM.4}),
    $\Sat{M(\Gamma^*)}{\lexists[x][!A(x)]}$ iff for at least one
    variable assignment~$s$, $\Sat{M(\Gamma^*)}{!A(x)}[s]$. As
    $\Domain{M(\Gamma^*)}$ consists of the closed terms of~$\Lang{L}$,
    this is the case iff there is at least one closed term~$t$ such
    that $s(x) = t$ and $\Sat{M(\Gamma^*)}{!A(x)}[s]$.  By
    the Extension Lemma (see \S\ref{SEM.4}),
    $\Sat{M(\Gamma^*)}{!A(x)}[s]$ iff $\Sat{M(\Gamma^*)}{!A(t)}[s]$,
    where $s(x) = t$.  By the Sentence Satisfaction Lemma (see \S\ref{SEM.4}),
    $\Sat{M(\Gamma^*)}{!A(t)}[s]$ iff $\Sat{M(\Gamma^*)}{!A(t)}$,
    since $!A(t)$ is a sentence.
\item By the definition of satisfaction,
    $\Sat{M(\Gamma^*)}{\lforall[x][!A(x)]}$ iff for every variable
    assignment $s$, $\Sat{M(\Gamma^*)}{!A(x)}[s]$. Recall that
    $\Domain{M(\Gamma^*)}$ consists of the closed terms of~$\Lang{L}$,
    so for every closed term~$t$, $s(x) = t$ is such a variable
    assignment, and for any variable assignment, $s(x)$ is some closed
    term~$t$.  By the Extension Lemma,
    $\Sat{M(\Gamma^*)}{!A(x)}[s]$ iff $\Sat{M(\Gamma^*)}{!A(t)}[s]$,
    where $s(x) = t$.  By the Sentence Satisfaction Lemma,
    $\Sat{M(\Gamma^*)}{!A(t)}[s]$ iff $\Sat{M(\Gamma^*)}{!A(t)}$,
    since $!A(t)$ is a sentence.
\end{enumerate}
\end{proof}

\begin{lem}[Truth Lemma] % THM-META-TL
\label{lem:truth}
Suppose $!A$ does not contain~$\eq$. Then
$\Sat{M(\Gamma^*)}{!A}$ iff $!A \in \Gamma^*$.
\end{lem}

\begin{proof}
We prove both directions simultaneously, and by induction on $!A$.
\begin{enumerate}
\item $!A \ident \lfalse$:
  $\Sat/{M(\Gamma^*)}{\lfalse}$
    by definition of satisfaction. On the other hand, $\lfalse \notin
    \Gamma^*$ since $\Gamma^*$ is consistent.

\item $!A \ident \ltrue$:
  $\Sat{M(\Gamma^*)}{\ltrue}$
    by definition of satisfaction. On the other hand, $\ltrue \in
    \Gamma^*$ since $\Gamma^*$ is consistent and complete, and
    $\Gamma^* \Proves \ltrue$.

\item $!A \ident R(t_1, \dots, t_n)$:
  $\Sat{M(\Gamma^*)}{\Atom{R}{t_1, \dots, t_n}}$ iff $\tuple{t_1,
      \dots, t_n} \in \Assign{R}{M(\Gamma^*)}$ (by the definition of
    satisfaction) iff $R(t_1, \dots, t_n) \in \Gamma^*$ (by the
    construction of $\Struct
    M(\Gamma^*)$).

\item $!A \ident \lnot !B$:
    $\Sat{M(\Gamma^*)}{\lnot !B}$ iff
    $\Sat/{M(\Gamma^*)}{!B}$ (by
    definition of satisfaction). By induction hypothesis,
    $\Sat/{M(\Gamma^*)}{!B}$ iff
    $!B \notin \Gamma^*$. Since $\Gamma^*$ is consistent and
    complete, $!B
    \notin \Gamma^*$ iff $\lnot !B \in \Gamma^*$.

\item $!A \ident !B \land !C$:
    $\Sat{M(\Gamma^*)}{!B \land !C}$
    iff we have both
    $\Sat{M(\Gamma^*)}{!B}$ and
    $\Sat{M(\Gamma^*)}{!C}$ (by
    definition of satisfaction) iff both $!B \in \Gamma^*$ and $!C \in
    \Gamma^*$ (by the induction hypothesis). By
    Proposition~\ref{prop:ccs}\ref{prop:ccs-and}, this is the case
    iff $(!B \land !C) \in \Gamma^*$.

\item $!A \ident !B \lor !C$:
    $\Sat{M(\Gamma^*)}{!B \lor !C}$
    iff $\Sat{M(\Gamma^*)}{!B}$ or
    $\Sat{M(\Gamma^*)}{!C}$ (by
    definition of satisfaction) iff $!B \in \Gamma^*$ or $!C \in
    \Gamma^*$ (by induction hypothesis). This is the case iff $(!B
    \lor !C) \in \Gamma^*$ (by
    Proposition~\ref{prop:ccs}\ref{prop:ccs-or}).

\item $!A \ident !B \lif !C$:
    $\Sat{M(\Gamma^*)}{!B \lif !C}$
    iff $\Sat/{M(\Gamma^*)}{!B}$ or $\Sat{M(\Gamma^*)}{!C}$ (by
    definition of satisfaction) iff $!B \notin \Gamma^*$ or $!C \in
    \Gamma^*$ (by induction hypothesis). This is the case iff $(!B
    \lif !C) \in \Gamma^*$ (by
    Proposition~\ref{prop:ccs}\ref{prop:ccs-if}).

\item $!A \ident \lforall[x][!B(x)]$:
    $\Sat{M(\Gamma^*)}{\lforall[x][!B(x)]}$ iff
      $\Sat{M(\Gamma^*)}{!B(t)}$ for all terms~$t$
      (Proposition~\ref{prop:quant-termmodel}).  By induction hypothesis, this
      is the case iff $!B(t) \in \Gamma^*$ for all terms~$t$; by
      Proposition~\ref{prop:saturated-instances}, this in turn is the case
      iff $\lforall[x][!A(x)] \in \Gamma^*$.

\item $!A \ident \lexists[x][!B(x)]$:
    $\Sat{M(\Gamma^*)}{\lexists[x][!B(x)]}$ iff
      $\Sat{M(\Gamma^*)}{!B(t)}$ for at least one term~$t$
      (Proposition~\ref{prop:quant-termmodel}).  By induction hypothesis, this
      is the case iff $!B(t) \in \Gamma^*$ for at least one term~$t$.
      By Proposition~\ref{prop:saturated-instances}, this in turn is the
      case iff $\lexists[x][!B(x)] \in \Gamma^*$.
\end{enumerate}
\end{proof}

%%% -----------------------------------------------------------------
%%% META.2.6  Identity
%%% -----------------------------------------------------------------

\subsection{Identity}

The term model constructed above suffices for sets~$\Gamma$ that do not
contain the identity predicate~$\eq$.  When $\Gamma^*$ contains a
sentence~$\eq[t][t']$ with $t$ and $t'$ distinct terms, the term model
falsifies it (since $\Value{t}{M(\Gamma^*)} = t \neq t'$).  We fix
this by quotienting the term model by provable equality.

\begin{defn} \label{defn:approx}
  Let $\Gamma^*$ be a consistent and complete set of sentences
  in~$\Lang L$.  We define the relation $\approx$ on the set of closed
  terms of~$\Lang L$ by
  \[
  t \approx t' \text{\quad iff \quad} \eq[t][t'] \in \Gamma^*
  \]
\end{defn}

\begin{prop}
\label{prop:approx-equiv}
The relation $\approx$ has the following properties:
\begin{enumerate}
\item $\approx$ is reflexive.
\item $\approx$ is symmetric.
\item  $\approx$ is transitive.
\item If $t \approx t'$, $f$ is a function, and $t_1$, \dots,
  $t_{i-1}$, $t_{i+1}$, \dots, $t_n$ are closed terms, then
\[
\Atom{f}{t_1,\dots, t_{i-1}, t, t_{i+1}, \dots, t_n} \approx
\Atom{f}{t_1,\dots, t_{i-1}, t', t_{i+1}, \dots, t_n}.
\]
\item If $t \approx t'$, $R$ is a predicate, and $t_1$, \dots,
  $t_{i-1}$, $t_{i+1}$, \dots, $t_n$ are closed terms, then
\begin{multline*}
\Atom{R}{t_1,\dots, t_{i-1}, t, t_{i+1}, \dots, t_n} \in \Gamma^* \text{ iff } \\
\Atom{R}{t_1,\dots, t_{i-1}, t', t_{i+1}, \dots, t_n} \in \Gamma^*.
\end{multline*}
\end{enumerate}
\end{prop}

\begin{proof}
Since $\Gamma^*$ is consistent and complete, $\eq[t][t'] \in
\Gamma^*$ iff $\Gamma^* \Proves \eq[t][t']$.  Thus it is enough to
show the following:
\begin{enumerate}
\item $\Gamma^* \Proves \eq[t][t]$ for all closed terms~$t$.
\item If $\Gamma^* \Proves \eq[t][t']$ then $\Gamma^* \Proves \eq[t'][t]$.
\item If $\Gamma^* \Proves \eq[t][t']$ and $\Gamma^* \Proves
  \eq[t'][t'']$, then $\Gamma^* \Proves \eq[t][t'']$.
\item If $\Gamma^* \Proves \eq[t][t']$, then
\[
\Gamma^* \Proves
\eq[\Atom{f}{t_1,\dots,t_{i-1},t,t_{i+1},\dots,t_n}][\Atom{f}{t_1,\dots,t_{i-1},t',t_{i+1},\dots,t_n}]
\]
for every $n$-place function~$f$ and closed terms $t_1$, \dots,
$t_{i-1}$, $t_{i+1}$, \dots,~$t_n$.
\item If $\Gamma^* \Proves \eq[t][t']$ and
$\Gamma^* \Proves
\Atom{R}{t_1,\dots,t_{i-1},t,t_{i+1},\dots,t_n}$, then
$\Gamma^* \Proves \Atom{R}{t_1,\dots,t_{i-1},t',t_{i+1},\dots,t_n}$
for every $n$-place predicate~$R$ and closed terms $t_1$, \dots,
$t_{i-1}$, $t_{i+1}$, \dots,~$t_n$.
\end{enumerate}
\end{proof}

\begin{defn} \label{defn:equiv-class}
Suppose $\Gamma^*$ is a consistent and complete set in a
language~$\Lang L$, $t$ is a closed term, and $\approx$ as in the
previous definition. Then:
\[
\equivrep{t}{\approx} = \Setabs{t'}{t'\in \Trm[L], t \approx t'}
\]
and $\equivclass{\Trm[L]}{\approx} = \Setabs{\equivrep{t}{\approx}}{t \in \Trm[L]}$.
\end{defn}

\begin{defn}[Factored term model] % DEF-META-FTM
\label{defn:term-model-factor}
Let $\Struct M = \Struct M(\Gamma^*)$ be the term model
for~$\Gamma^*$ from Definition~\ref{defn:termmodel}.  Then $\Struct{\equivclass{M}{\approx}}$ is the following
structure:
\begin{enumerate}
\item $\Domain{\equivclass{M}{\approx}} = \equivclass{\Trm[L]}{\approx}$.
\item $\Assign{c}{\equivclass{M}{\approx}} = \equivrep{c}{\approx}$
\item $\Assign{f}{\equivclass{M}{\approx}}(\equivrep{t_1}{\approx}, \dots,
  \equivrep{t_n}{\approx}) = \equivrep{\Atom{f}{t_1,\dots, t_n}}{\approx}$
\item $\tuple{\equivrep{t_1}{\approx}, \dots, \equivrep{t_n}{\approx}} \in
  \Assign{R}{\equivclass{M}{\approx}}$ iff
  $\Sat{M}{\Atom{R}{t_1,\dots, t_n}}$, i.e., iff $\Atom{R}{t_1,\dots,
  t_n} \in \Gamma^*$.
\end{enumerate}
\end{defn}

The definitions of $\Assign{f}{\equivclass{M}{\approx}}$ and
$\Assign{R}{\equivclass{M}{\approx}}$ refer to elements of
$\equivclass{\Trm[L]}{\approx}$ via representatives~$t \in
\equivrep{t}{\approx}$.  Proposition~\ref{prop:approx-equiv}
guarantees that these definitions do not depend on the choice of
representatives.

\begin{prop}
$\Struct{\equivclass{M}{\approx}}$ is well defined, i.e., if $t_1$,
  \dots, $t_n$, $t_1'$, \dots, $t_n'$ are closed terms,
  and $t_i \approx t_i'$ then
\begin{enumerate}
\item $\equivrep{\Atom{f}{t_1,\dots, t_n}}{\approx} =
    \equivrep{\Atom{f}{t_1',\dots, t_n'}}{\approx}$, i.e.,
  \[
  \Atom{f}{t_1,\dots, t_n} \approx \Atom{f}{t_1',\dots, t_n'}
  \]
  and
\item $\Sat{M}{\Atom{R}{t_1,\dots, t_n}}$ iff
  $\Sat{M}{\Atom{R}{t_1',\dots, t_n'}}$, i.e.,
  \[
    \Atom{R}{t_1,\dots, t_n} \in \Gamma^* \text{ iff }
    \Atom{R}{t_1',\dots, t_n'} \in \Gamma^*.
  \]
\end{enumerate}
\end{prop}

\begin{proof}
Follows from Proposition~\ref{prop:approx-equiv} by induction on~$n$.
\end{proof}

\begin{lem}
\label{lem:val-in-termmodel-factored} Let $\Struct M = \Struct M
 (\Gamma^*)$; then $\Value{t}{\equivclass{M}{\approx}} = \equivrep{t}
 {\approx}$.
\end{lem}

\begin{proof}
The proof is similar to that of Lemma~\ref{lem:val-in-termmodel}.
\end{proof}

\begin{lem}[Truth Lemma, with identity] % THM-META-TLI
\label{lem:truth-factored}
$\Sat{\equivclass{M}{\approx}}{!A}$ iff $!A \in \Gamma^*$ for all
  sentences~$!A$.
\end{lem}

\begin{proof}
By induction on~$!A$, just as in the proof of Lemma~\ref{lem:truth}.
The only case that needs additional attention is when $!A \ident
\eq[t][t']$.
\begin{align*}
\Sat{\equivclass{M}{\approx}}{\eq[t][t']} & \text{ iff } \equivrep{t}{\approx} = \equivrep{t'}{\approx}
\text{ (by definition of $\Struct{\equivclass{M}{\approx}}$)}\\
& \text{ iff } t \approx t' \text{ (by definition of $\equivrep{t}{\approx}$)}\\
& \text{ iff } \eq[t][t'] \in \Gamma^* \text{ (by definition of $\approx$).}
\end{align*}
\end{proof}

%%% -----------------------------------------------------------------
%%% META.2.7  The Completeness Theorem
%%% -----------------------------------------------------------------

\subsection{The Completeness Theorem}

Let us combine our results: we arrive at the completeness theorem.

\begin{thm}[Completeness Theorem] % CP-002
\label{CP-002}
Let $\Gamma$ be a set of sentences.  If $\Gamma$ is consistent, it
is satisfiable.
\end{thm}

\begin{proof}
Suppose $\Gamma$ is consistent. By
  Lemma~\ref{lem:henkin}, there is a saturated consistent set
  $\Gamma' \supseteq \Gamma$. By Lemma~\ref{THM-DED005} (Lindenbaum's Lemma), there
is a $\Gamma^* \supseteq \Gamma'$ which is
consistent and complete.
  Since $\Gamma' \subseteq \Gamma^*$, for each
  formula~$!A(x)$, $\Gamma^*$ contains a sentence of the form
        $\lexists[x][!A(x)] \lif !A(c)$
  and so $\Gamma^*$ is saturated.  If $\Gamma$
  does not contain~$\eq$, then by
Lemma~\ref{lem:truth} (Truth Lemma),
$\Sat{M(\Gamma^*)}{!A}$ iff $!A \in
\Gamma^*$.  From this it follows in particular that for all $!A \in
\Gamma$, $\Sat{M(\Gamma^*)}{!A}$, so
$\Gamma$ is satisfiable.
  If $\Gamma$ does contain~$\eq$,
  then by Lemma~\ref{lem:truth-factored}, for all sentences~$!A$,
  $\Sat{\equivclass{M}{\approx}}{!A}$ iff $!A \in \Gamma^*$.  In
  particular, $\Sat{\equivclass{M}{\approx}}{!A}$ for all $!A \in
  \Gamma$, so $\Gamma$ is satisfiable.
\end{proof}

\begin{cor}[Completeness Theorem, Second Version] % CP-002
\label{cor:completeness}
For all $\Gamma$ and sentences~$!A$: if $\Gamma \Entails !A$ then
$\Gamma \Proves !A$.
\end{cor}

\begin{proof}
Note that the $\Gamma$'s in Corollary~\ref{cor:completeness} and
Theorem~\ref{CP-002} are universally quantified.  To make sure we
do not confuse ourselves, let us restate Theorem~\ref{CP-002}
using a different variable: for any set of sentences~$\Delta$, if
$\Delta$ is consistent, it is satisfiable.  By contraposition, if
$\Delta$ is not satisfiable, then $\Delta$ is inconsistent.  We will
use this to prove the corollary.

Suppose that $\Gamma \Entails !A$.  Then $\Gamma \cup \{\lnot !A\}$ is
unsatisfiable by the entailment--unsatisfiability equivalence
(see DEF-SEM002, \S\ref{SEM.4}).  Taking $\Gamma
\cup \{\lnot !A\}$ as our $\Delta$, the previous version of
Theorem~\ref{CP-002} gives us that $\Gamma \cup \{\lnot !A\}$ is
inconsistent.  By
the derivability from inconsistency property (see \S\ref{DED.1}),
$\Gamma \Proves !A$.
\end{proof}


%% ===================================================================
%% META.3: Compactness (CP-003)
%% Source: fol/com/com
%% ===================================================================

\section{Compactness} \label{META.3}

\begin{defn}[Finitely satisfiable] % DEF-SEM003
\label{DEF-SEM003}
  A set $\Gamma$ of formulas is \emph{finitely satisfiable} iff every finite $\Gamma_0 \subseteq \Gamma$ is satisfiable.
\end{defn}

\begin{thm}[Compactness Theorem] % CP-003
\label{CP-003}
The following hold for any sentences $\Gamma$ and $!A$:
\begin{enumerate}
  \item $\Gamma \Entails !A$ iff there is a finite $\Gamma_0
    \subseteq \Gamma$ such that $\Gamma_0 \Entails !A$.
  \item $\Gamma$ is satisfiable iff it is finitely
    satisfiable.
\end{enumerate}
\end{thm}

\begin{proof}
We prove (2).  If $\Gamma$ is satisfiable, then there is
a structure~$\Struct{M}$
such that $\Sat{M}{!A}$ for all $!A \in
\Gamma$.  Of course, this $\Struct{M}$ also
satisfies every finite subset of~$\Gamma$, so $\Gamma$ is finitely
satisfiable.

Now suppose that $\Gamma$ is finitely satisfiable.  Then every finite
subset~$\Gamma_0 \subseteq \Gamma$ is satisfiable.  By soundness
(Corollary~\ref{cor:satisfiable-consistent}, from \S\ref{META.1}),
every finite subset is consistent.  Then $\Gamma$ itself must be
consistent by
the compactness of derivability (see \S\ref{DED.1}).
By the Completeness Theorem (\ref{CP-002}), since $\Gamma$~is
consistent, it is satisfiable.
\end{proof}


%% ===================================================================
%% META.4: Lowenheim-Skolem (CP-004)
%% Source: fol/com/dls
%% ===================================================================

\section{The L\"owenheim--Skolem Theorem} \label{META.4}

First-order logic cannot express that the size of a structure is
non-enumerable: any sentence or set of sentences satisfied in all
non-enumerable structures is also satisfied in some enumerable
structure.

\begin{thm}[Downward L\"owenheim--Skolem] % CP-004
\label{CP-004}
If $\Gamma$ is consistent then it has
an enumerable model, i.e., it is satisfiable in a structure
whose domain is either finite or denumerable.
\end{thm}

\begin{proof}
If $\Gamma$ is consistent, the structure~$\Struct M$ delivered by
the proof of the Completeness Theorem (\ref{CP-002}) has a domain $\Domain{M}$ that
is no larger than the set of the terms of the language~$\Lang L$. So
$\Struct M$ is at most denumerable.
\end{proof}

\begin{thm}[L\"owenheim--Skolem without identity] % CP-004-variant
\label{thm:noidentity-ls}
If $\Gamma$ is a consistent set of sentences
in the language of first-order logic without identity, then it has
a denumerable model, i.e., it is satisfiable in a structure
whose domain is infinite and enumerable.
\end{thm}

\begin{proof}
If $\Gamma$ is consistent and contains no sentences in which identity
appears, then the structure~$\Struct M$ delivered by the proof of
the Completeness Theorem has a domain $\Domain{M}$ identical to the set
of terms of the language~$\Lang L'$. So $\Struct{M}$ is
denumerable, since $\Trm[L']$ is.
\end{proof}


%% ===================================================================
%% META.5: First Incompleteness (CP-005)
%% Sources: inc/inp/fix (KEEP), inc/inp/s1c (KEEP),
%%          inc/inp/1in (ABSORB:inc/tcp/inc), inc/inp/ros (KEEP)
%% ===================================================================

\section{The First Incompleteness Theorem} \label{META.5}

The First Incompleteness Theorem establishes an inherent limitation of
sufficiently strong, axiomatizable theories: no such theory can be both
consistent and complete.  The proof rests on the Fixed-Point Lemma,
$\Sigma_1$-completeness, and a careful analysis of the provability
predicate.

%%% -----------------------------------------------------------------
%%% META.5.1  The Fixed-Point Lemma
%%% -----------------------------------------------------------------

\subsection{The Fixed-Point Lemma}

The fixed-point lemma says that for any formula~$!B(x)$, there is
a sentence~$!A$ such that $\Th{T} \Proves !A \liff !B(\gn{!A})$,
provided $\Th{T}$ extends~$\Th{Q}$.  In the case of the liar sentence,
we would want $!A$ to be equivalent (provably in~$\Th{T}$) to~``$\gn{!A}$
is false,'' i.e., the statement that $\Gn{!A}$ is the G\"odel number
of a false sentence. To understand the idea of the proof, it will
be useful to compare it with Quine's informal gloss of~$!A$ as,
``{}`yields a falsehood when preceded by its own quotation' yields a
falsehood when preceded by its own quotation.''  The operation of
taking an expression, and then forming a sentence by preceding this
expression by its own quotation may be called \emph{diagonalizing} the
expression, and the result its diagonalization. So, the
diagonalization of `yields a falsehood when preceded by its own
quotation' is ``{}`yields a falsehood when preceded by its own
quotation' yields a falsehood when preceded by its own quotation.''
Now note that Quine's liar sentence is not the diagonalization of
`yields a falsehood' but of `yields a falsehood when preceded by its
own quotation.' So the property being diagonalized to yield the liar
sentence itself involves diagonalization!{}

In the language of arithmetic, we form quotations of a formula with
one free variable by computing its G\"odel numbers and then
substituting the standard numeral for that G\"odel number into the
free variable. The diagonalization of~$!E(x)$ is $!E(\num{n})$, where
$n = \Gn{!E(x)}$. (From now on, let us abbreviate $\num{\Gn{!E(x)}}$ as
$\gn{!E(x)}$.)  So if $!B(x)$ is ``is a falsehood,'' then ``yields a
falsehood if preceded by its own quotation,'' would be ``yields a
falsehood when applied to the G\"odel number of its diagonalization.''
If we had a symbol~$\Obj{diag}$ for the function $\fn{diag}(n)$ which
computes the G\"odel number of the diagonalization of the formula
with G\"odel number~$n$, we could write $!E(x)$ as
$!B(\Obj{diag}(x))$. And Quine's version of the liar sentence would
then be the diagonalization of it, i.e., $!E(\gn{!E(x)})$ or
$!B(\Obj{diag}(\gn{!B(\Obj{diag}(x))}))$.  Of course, $!B(x)$ could
now be any other property, and the same construction would work. For
the incompleteness theorem, we take $!B(x)$ to be ``$x$~is not
derivable in~$\Th{T}$.'' Then $!E(x)$ would be ``yields
a sentence not derivable in~$\Th{T}$ when applied to the
G\"odel number of its diagonalization.''

To formalize this in~$\Th{T}$, we have to find a way to formalize
$\fn{diag}$. The function $\fn{diag}(n)$ is computable, in fact, it is
primitive recursive: if $n$ is the G\"odel number of a
formula~$!E(x)$, $\fn{diag}(n)$ returns the G\"odel number
of~$!E(\gn{!E(x)})$. (Recall, $\gn{!E(x)}$ is the standard numeral of
the G\"odel number of~$!E(x)$, i.e., $\num{\Gn{!E(x)}}$). If
$\Obj{diag}$ were a function symbol in $\Th{T}$ representing the
function $\fn{diag}$, we could take $!A$ to be the formula
$!B(\Obj{diag}(\gn{!B(\Obj{diag}(x))}))$. Notice that
\begin{align*}
\fn{diag}(\Gn{!B(\Obj{diag}(x))}) & =
\Gn{!B(\Obj{diag}(\gn{!B(\Obj{diag}(x))}))} \\
& = \Gn{!A}.
\end{align*}
Assuming $\Th{T}$ can derive
\[
\Obj{diag}(\gn{!B(\Obj{diag}(x))}) = \gn{!A},
\]
it can derive $!B(\Obj{diag}(\gn{!B(\Obj{diag}(x))}))
\liff !B(\gn{!A})$. But the left hand side is, by
definition,~$!A$.

Of course, $\Obj{diag}$ will in general not be a function symbol of
$\Th{T}$, and certainly is not one of~$\Th{Q}$. But, since $\fn{diag}$
is computable, it is \emph{representable} in~$\Th{Q}$ by some formula
$!D_{\fn{diag}}(x,y)$. So instead of writing $!B(\Obj{diag}(x))$ we
can write $\lexists[y][(!D_{\fn{diag}}(x,y) \land !B(y))]$. Otherwise,
the proof sketched above goes through, and in fact, it goes through
already in~$\Th{Q}$.

\begin{lem}[Fixed-Point Lemma] % THM-DED006
\label{THM-DED006}
Let $!B(x)$ be any formula with one free
variable~$x$. Then there is a sentence~$!A$ such that $\Th{Q} \Proves
!A \liff !B(\gn{!A})$.
\end{lem}

\begin{proof}
Given $!B(x)$, let $!E(x)$ be the formula
$\lexists[y][(!D_{\fn{diag}}(x,y) \land !B(y))]$ and let $!A$~be its
diagonalization, i.e., the formula $!E(\gn{!E(x)})$.

Since $!D_{\fn{diag}}$ represents $\fn{diag}$, and
$\fn{diag}(\Gn{!E(x)}) = \Gn{!A}$, $\Th{Q}$ can derive
\begin{align}
  & !D_{\fn{diag}}(\gn{!E(x)}, \gn{!A}) \label{repdiag1} \\
  & \lforall[y][(!D_{\fn{diag}}(\gn{!E(x)},y) \lif
  \eq[y][\gn{!A}])]. \label{repdiag2}
\end{align}
Now we show that $\Th{Q} \Proves !A \liff !B(\gn{!A})$. We argue
informally, using just logic and facts derivable in~$\Th{Q}$.

First, suppose~$!A$, i.e., $!E(\gn{!E(x)})$. Going back to the
definition of $!E(x)$, we see that $!E(\gn{!E(x)})$ just is
\[
\lexists[y][(!D_{\fn{diag}}(\gn{!E(x)},y) \land !B(y))].
\]
Consider such a~$y$. Since $!D_{\fn{diag}}(\gn{!E(x)},y)$, by
\eqref{repdiag2}, $y = \gn{!A}$. So, from $!B(y)$ we
have~$!B(\gn{!A})$.

Now suppose $!B(\gn{!A})$. By \eqref{repdiag1}, we have
\begin{align*}
& !D_{\fn{diag}}(\gn{!E(x)}, \gn{!A}) \land !B(\gn{!A}).
\intertext{It follows
that}
& \lexists[y][(!D_{\fn{diag}}(\gn{!E(x)},y) \land !B(y))].
\end{align*}
But that is just $!E(\gn{!E(x)})$, i.e.,~$!A$.
\end{proof}

%%% -----------------------------------------------------------------
%%% META.5.2  Sigma-1 Completeness
%%% -----------------------------------------------------------------

\subsection{$\Sigma_1$-Completeness}

Despite the incompleteness of $\Th{Q}$ and its consistent, axiomatizable
extensions, $\Th{Q}$ does prove many basic facts about
numerals. In fact, this can be extended quite considerably. To understand
the scope of what can be proved in~$\Th{Q}$, we introduce the notions of
$\Delta_0$, $\Sigma_1$, and $\Pi_1$ formulas. Roughly speaking, a
$\Sigma_1$ formula is one of the form $\lexists[x][!B(x)]$, where $!B$
is constructed using only propositional connectives and bounded
quantifiers. We shall show that if $!A$ is a $\Sigma_1$ sentence
which is true in $\Struct{N}$ (the standard model of arithmetic;
see \S\ref{SEM.5}), then $\Th{Q} \Proves !A$.

\begin{defn}[Bounded quantifiers] % DEF-INC015
\label{DEF-INC015}
A \emph{bounded existential formula} is one of the form
$\lexists[x][(x < t \land !A(x))]$ where $t$ is any term, which we
conventionally write as $\bexists{x < t}{!A(x)}$.
A \emph{bounded universal formula} is one of the form
$\lforall[x][(x < t \lif !A(x))]$ where $t$ is any term, which we
conventionally write as $\bforall{x < t}{!A(x)}$.
\end{defn}

\begin{rem}[$\Delta_0$, $\Sigma_1$, $\Pi_1$ formulas] % cf.\ DEF-SYN009, DEF-SYN010
Recall (DEF-SYN009/010, \S\ref{SYN.5}): a formula is $\Delta_0$ if
it is built from atomic formulas using only propositional connectives
and bounded quantification; $\Sigma_1$ if it has the form
$\lexists[x][!B(x)]$ with $!B$ being $\Delta_0$; and $\Pi_1$ if it
has the form $\lforall[x][!B(x)]$ with $!B$ being $\Delta_0$.
\end{rem}

\begin{lem} \label{lem:q-proves-clterm-id}
Suppose $t$ is a closed term such that
$\Value{t}{N} = n$. Then $\Th{Q} \Proves \eq[t][\num n]$.
\end{lem}

\begin{proof}
We prove this by induction on the complexity of~$t$. For the base case,
$\Value{\Obj 0}{N} = 0$, and $\Th{Q} \Proves \eq[\Obj 0][\num 0]$
since $\num 0 \ident \Obj 0$.
For the inductive case, let $t_1$ and $t_2$ be terms such that
$\Value{t_1}{N} = n_1$, $\Value{t_2}{N} = n_2$,
$\Th{Q} \Proves \eq[t_1][\num n_1]$, and
$\Th{Q} \Proves \eq[t_2][\num n_2]$.

Then $\Value{(t_1')}{N} = n_1 + 1$, and we have that $\Th{Q} \Proves
\eq[t_1'][{\num n_1}']$ by the first-order rules for identity applied
to the induction hypothesis and the formula
$\eq[\num{n_1}'][\num{n_1}']$,
so we have $\Th{Q} \Proves \eq[t_1'][\num{n_1 + 1}]$
by the definition of numerals.

For sums we have
\[
      \Value{(t_1 + t_2)}{N}
    = \Value{t_1}{N} + \Value{t_2}{N}
    = n_1 + n_2.
\]
By the induction hypothesis and the rules for identity,
$\Th{Q} \Proves \eq[t_1 + t_2][\num{n_1} + t_2]$, and then
$\Th{Q} \Proves \eq[t_1 + t_2][\num{n_1} + \num{n_2}]$
by a second application of the rules for identity.
By the fact that $\Th{Q}$ proves the standard addition identities
(see DEF-CMP009, Representability, \S\ref{CMP.5}),
$\Th{Q} \Proves \eq[\num{n_1} + \num{n_2}][\num{n_1 + n_2}]$,
so $\Th{Q} \Proves \eq[t_1 + t_2][\num{n_1 + n_2}]$.

Similar reasoning also works for~$\times$, using the corresponding
multiplication identities.
Since this exhausts the closed terms of arithmetic, we have that
$\Th{Q} \Proves \eq[t][\num n]$ for all closed terms~$t$ such that
$\Value{t}{N} = n$.
\end{proof}

\begin{lem} \label{lem:atomic-completeness}
Suppose $t_1$ and $t_2$ are closed terms. Then
\begin{enumerate}
\item If $\Value{t_1}{N} = \Value{t_2}{N}$,
    then $\Th{Q} \Proves \eq[t_1][t_2]$.
\item If $\Value{t_1}{N} \neq \Value{t_2}{N}$,
    then $\Th{Q} \Proves \eq/[t_1][t_2]$.
\item If $\Value{t_1}{N} < \Value{t_2}{N}$,
    then $\Th{Q} \Proves t_1 < t_2$.
\item If $\Value{t_2}{N} \leq \Value{t_1}{N}$,
    then $\Th{Q} \Proves \lnot(t_1 < t_2)$.
\end{enumerate}
\end{lem}

\begin{proof}
Given terms $t_1$ and $t_2$, we fix $n = \Value{t_1}{N}$ and
$m = \Value{t_2}{N}$.

Suppose $!A \ident t_1 = t_2$. By Lemma~\ref{lem:q-proves-clterm-id},
$\Th{Q} \Proves \eq[t_1][\num n]$ and $\Th{Q} \Proves \eq[t_2][\num n]$.
If $n = m$, then $\Th{Q} \Proves \eq[\num n][\num m]$ and hence
$\Th{Q} \Proves \eq[t_1][t_2]$ by the transitivity of identity.
If $n \neq m$ then $\Th{Q} \Proves \eq/[\num n][\num m]$,
and by the transitivity of identity again,
$\Th{Q} \Proves \eq/[t_1][t_2]$.

Now let $!A \ident t_1 < t_2$. For both cases, we rely on axiom~$!Q_8$,
which states that $x < y \liff \lexists[z][\eq[z' + x][y]]$
for all $x,y$.

Suppose $\Sat{N}{t_1 < t_2}$. Then there exists some $k \in \Nat$
such that $n + k + 1 = m$. By Lemma~\ref{lem:q-proves-clterm-id},
$\Th{Q} \Proves \eq[t_1][\num n]$ and $\Th{Q} \Proves \eq[t_2][\num m]$,
and by the first part of this lemma,
$\Th{Q} \Proves \eq[\num n + {\num k}'][\num m]$.
By the transitivity of identity it follows that
$\Th{Q} \Proves \eq[{\num k}' + t_1][t_2]$,
so $\Th{Q} \Proves \lexists[z][\eq[z' + t_1][t_2]]$.
By the right-to-left direction of~$!Q_8$, $\Th{Q} \Proves t_1 < t_2$.

Suppose instead that $\Sat/{N}{t_1 < t_2}$, i.e., $m \leq n$.
We work in~$\Th{Q}$ and assume that $t_1 < t_2$. By the left-to-right
direction of~$!Q_8$, there is some~$z$ such that $\eq[z' + t_1][t_2]$.
Since $\Th{Q} \Proves \eq[t_1][\num n]$ and
$\Th{Q} \Proves \eq[t_2][\num m]$, $\eq[z' + \num n][\num m]$.
By an external induction on~$m$ using~$!Q_5$,
$\eq[z' + \num{n - m}][\Obj 0]$.
If $m = n$ then $\eq/[z'][\Obj 0]$, giving a contradiction via~$!Q_3$.
If $m < n$ then $\eq[(z' + \num{n - m - 1})'][\Obj 0]$ by~$!Q_5$ again,
giving a contradiction via~$!Q_3$.
So $\Th{Q} \Proves \lnot(t_1 < t_2)$.
\end{proof}

\begin{lem} \label{lem:bounded-quant-equiv}
Suppose $!A$ is a formula, $t$ a closed term, and $k=\Value{t}{N}$. Then
\begin{enumerate}
\item $\Th{Q} \Proves \bforall{x<t}{!A(x)}$ iff $\Th{Q} \Proves
    !A(\num 0) \land \dots \land !A(\num{k-1})$.
\item $\Th{Q} \Proves \bexists{x<t}{!A(x)}$ iff $\Th{Q} \Proves
    !A(\num 0) \lor \dots \lor !A(\num{k-1})$.
\end{enumerate}
\end{lem}

\begin{proof}
    We prove the case for the bounded universal quantifier.
    If $\Value{t}{N} = 0$ then the left-hand side of the
    equivalence is provable in~$\Th{Q}$, because there is no
    $x<\num 0$ by properties of~$<$ in~$\Th{Q}$.
    Similarly, we can take an empty disjunction to be simply
    $\ltrue$, which is also provable in~$\Th{Q}$.
    We therefore suppose that $\Value{t}{N} = k+1$ for some
    natural number~$k$. By Lemma~\ref{lem:q-proves-clterm-id} we
    can assume that we are working with a formula of the
    form $\bforall{x<\num{k+1}}{!A(x)}$.

    Suppose that $\Th{Q} \Proves \bforall{x<\num{k+1}}{!A(x)}$,
    and let $n \leq k$. Since $\Th{Q} \Proves \num n < \num{k+1}$
    by Lemma~\ref{lem:atomic-completeness}, it follows by logic that
    $\Th{Q} \Proves !A(\num n)$. Applying this fact $k+1$ times
    for each $n \leq k$, we get that $\Th{Q} \Proves !A(\num 0)
    \land \dots \land !A(\num k)$ as desired.

    For the other direction, suppose that $\Th{Q} \Proves
    !A(\num 0) \land \dots \land !A(\num k)$. Working in
    $\Th{Q}$, suppose that $x < \num{k+1}$.
    By properties of~$<$ in~$\Th{Q}$ we have that
    $x = \num 0 \lor \dots \lor x = \num k$, so by logic it
    follows that~$!A(x)$, and hence the universal claim
    $\bforall{x<\num{k+1}}{!A(x)}$ follows.

    The proof of the equivalence for bounded existentially
    quantified formulas is similar.
\end{proof}

\begin{lem} \label{lem:delta0-completeness}
If $!A$ is a $\Delta_0$ sentence which is true in
$\Struct{N}$, then $\Th{Q} \Proves !A$.
\end{lem}

\begin{proof}
We prove this by induction on formula complexity.
The base case is given by Lemma~\ref{lem:atomic-completeness},
so we move to the induction step. For simplicity we split
the case of negation into subcases depending on the
structure of the formula to which the negation is
applied.

\begin{enumerate}
\item Suppose $(!A \land !B)$ is true in $\Struct{N}$,
so $!A$ and $!B$ are true in~$\Struct{N}$.
By the induction hypothesis, $\Th{Q} \Proves !A$ and
$\Th{Q} \Proves !B$,
so $\Th{Q} \Proves (!A \land !B)$ by logic.

\item Suppose $\lnot (!A \land !B)$ is true in $\Struct{N}$,
so either $\lnot !A$ or $\lnot !B$ is true in $\Struct{N}$.
Without loss of generality, suppose the former. By the
induction hypothesis $\Th{Q} \Proves \lnot !A$, and hence
$\Th{Q} \Proves \lnot (!A \land !B)$ by logic.

\item Suppose $(!A \lor !B)$ is true in $\Struct{N}$, so
either $!A$ is true in $\Struct{N}$ or $!B$ is true in
$\Struct{N}$. Without loss of generality, suppose the former
holds. By the induction hypothesis $\Th{Q} \Proves !A$, and
hence $\Th{Q} \Proves (!A \lor !B)$ by logic.

\item Suppose $\lnot(!A \lor !B)$ is true in $\Struct{N}$,
so $\lnot !A$ and $\lnot !B$ are true in $\Struct{N}$.
Then $\Th{Q} \Proves \lnot !A$ and $\Th{Q} \Proves \lnot !B$
by the induction hypothesis. Consequently,
$\Th{Q} \Proves \lnot(!A \lor !B)$ by logic.

\item Suppose that $\bforall{x<t}{!A(x)}$ is true
in~$\Struct{N}$, where $t$ is a closed term and $k=\Value{t}{N}$. By the induction
hypothesis and logic, if $!A(\num n)$ is true in~$\Struct{N}$
for all $n < \Value{t}{N}$ then $\Th{Q} \Proves
!A(\num 0) \land \dots \land !A(\num{k-1})$.
By Lemma~\ref{lem:bounded-quant-equiv} it follows that
$\Th{Q} \Proves \bforall{x<t}{!A(x)}$.

\item The case for the bounded existential quantifier, where
we have a sentence of the form $\bexists{x < t}{!A(x)}$,
is similar to that for the bounded universal quantifier.

\item Suppose that $\lnot \bforall{x<t}{!A(x)}$ is true
in~$\Struct{N}$, where $t$ is a closed term. This sentence
is equivalent to the sentence $\bexists{x<t}{\lnot !A(x)}$,
with the equivalence derivable in~$\Th{Q}$, so we may apply
the reasoning for bounded existential quantifiers.

\item Similarly, suppose that $\lnot \bexists{x<t}!A(x)$ is
true in $\Struct{N}$, where $t$ is a closed term. This
sentence is equivalent in $\Th{Q}$ to
$\bforall{x<t}{\lnot!A(x)}$, and so we may apply the reasoning
for bounded universal quantifiers.

\item Finally, suppose $\lnot !A$ is true in $\Struct{N}$.
The only cases remaining are when $!A$ is atomic and when
$\lnot !A \ident \lnot\lnot !B$ for some $\Delta_0$
sentence $!B$. If $!A$ is atomic then by
Lemma~\ref{lem:atomic-completeness}, $\Th{Q} \Proves \lnot !A$.
If $\lnot !A \ident \lnot\lnot !B$, then by logic it is
provably equivalent in~$\Th{Q}$ to~$!B$, which is true
in~$\Struct{N}$ since $\lnot !A$ is true in~$\Struct{N}$.
By the induction hypothesis we therefore have that
$\Th{Q} \Proves \lnot !A$.
\end{enumerate}
\end{proof}

\begin{thm}[$\Sigma_1$-Completeness] \label{thm:sigma1-completeness}
If $!A$ is a $\Sigma_1$ sentence which is true
in~$\Struct{N}$, then $\Th{Q} \Proves !A$.
\end{thm}

\begin{proof}
If $\lexists{x}!A(x)$ is a $\Sigma_1$ sentence which
is true in~$\Struct{N}$, then there exists a natural
number~$n$ and a variable assignment~$s$ such that $s(x) = n$ and
$\Sat{N}{!A(x)}[s]$. By standard facts about
the satisfaction relation it follows that
$\Sat{N}{!A(\num n)}$. But $!A(\num n)$ is a
$\Delta_0$ formula, so by Lemma~\ref{lem:delta0-completeness}
we have that $\Th{Q} \Proves !A(\num n)$, and hence by
logic we also have that $\Th{Q} \Proves \lexists[x][!A(x)]$.
\end{proof}

%%% -----------------------------------------------------------------
%%% META.5.3  The First Incompleteness Theorem
%%% -----------------------------------------------------------------

\subsection{G\"odel's First Incompleteness Theorem}

We can now describe G\"odel's original proof of the first
incompleteness theorem. Let $\Th{T}$ be any computably axiomatized theory
in a language extending the language of arithmetic, such that $\Th{T}$
includes the axioms of $\Th{Q}$. This means that, in particular, $\Th{T}$
represents computable functions and relations.

We have argued that, given a reasonable coding of formulas and proofs
as numbers, the relation $\Prf[\Th{T}](x,y)$ is computable, where
$\Prf[\Th{T}](x,y)$ holds if and only if $x$ is the G\"odel number of
a derivation of the formula with G\"odel number~$y$
in~$\Th{T}$. In fact, for the particular theory that G\"odel had in
mind, G\"odel was able to show that this relation is primitive
recursive, using the list of 45 functions and relations in his
paper. The 45th relation, $x B y$, is just $\Prf[\Th{T}](x,y)$ for his
particular choice of~$\Th{T}$. Remember that where G\"odel uses the
word ``recursive'' in his paper, we would now use the phrase
``primitive recursive.''

Since $\Prf[\Th{T}](x,y)$ is computable, it is representable in $\Th{T}$. We
will use $\OPrf[\Th{T}](x,y)$ to refer to the formula that represents
it. Let $\OProv[\Th{T}](y)$ be the formula
$\lexists[x][\OPrf[\Th{T}](x,y)]$. This describes the 46th relation,
$\fn{Bew}(y)$, on G\"odel's list. As G\"odel notes, this is the only
relation that ``cannot be asserted to be recursive.''  What he
probably meant is this: from the definition, it is not clear that it
is computable; and later developments, in fact, show that it is not.

Let $\Th{T}$ be an axiomatizable theory containing~$\Th{Q}$. Then
$\Prf[\Th{T}](x, y)$ is decidable, hence representable in~$\Th{Q}$ by
a formula~$\OPrf[\Th{T}](x, y)$. Let $\OProv[\Th{T}](y)$ be the formula we
described above. By the fixed-point lemma, there is a formula
$!G_\Th{T}$ such that $\Th{Q}$ (and hence $\Th{T}$) derives
\begin{equation}
\label{eqn:qpf}
!G_\Th{T} \liff \lnot \OProv[\Th{T}](\gn{!G_\Th{T}}).
\end{equation}
Note that $!G_\Th{T}$ says, in essence, ``$!G_\Th{T}$ is not
derivable in~$\Th{T}$.''

\begin{lem}\label{lem:cons-G-unprov}
If $\Th{T}$ is a consistent, axiomatizable theory
extending~$\Th{Q}$, then $\Th{T} \Proves/ !G_\Th{T}$.
\end{lem}

\begin{proof}
Suppose $\Th{T}$ derives $!G_\Th{T}$. Then there \emph{is}
a derivation, and so, for some number $m$, the relation $\Prf[\Th{T}](m,
\Gn{!G_\Th{T}})$ holds. But then $\Th{Q}$ derives the sentence
$\OPrf[\Th{T}](\num m, \gn{!G_\Th{T}})$. So $\Th{Q}$ derives
$\lexists[x][\OPrf[\Th{T}](x,\gn{!G_\Th{T}})]$, which is, by definition,
$\OProv[\Th{T}](\gn{!G_\Th{T}})$. By \eqref{eqn:qpf}, $\Th{Q}$ derives
$\lnot !G_\Th{T}$, and since $\Th{T}$ extends $\Th{Q}$, so
does~$\Th{T}$. We have shown that if $\Th{T}$ derives $!G_\Th{T}$, then
it also derives $\lnot !G_\Th{T}$, and hence it would be inconsistent.
\end{proof}

\begin{defn}[$\omega$-consistency]
\label{defn:omega-consistency}
A theory $\Th{T}$ is \emph{$\omega$-consistent} if the following holds: if
$\lexists[x][!A(x)]$ is any sentence and $\Th{T}$ derives $\lnot
!A(\num 0)$, $\lnot !A(\num 1)$, $\lnot !A(\num 2)$, \dots then $\Th{T}$
does not prove $\lexists[x][!A(x)]$.
\end{defn}

Note that every $\omega$-consistent theory is also consistent. This
follows simply from the fact that if $\Th{T}$ is inconsistent, then
$\Th{T} \Proves !A$ for every~$!A$. In particular, if $\Th{T}$ is
inconsistent, it derives both $\lnot !A(\num n)$ for every~$n$ and
also derives~$\lexists[x][!A(x)]$. So, if $\Th{T}$ is
inconsistent, it is $\omega$-inconsistent. By contraposition, if
$\Th{T}$ is $\omega$-consistent, it must be consistent.

\begin{lem}\label{lem:omega-cons-G-unref}
If $\Th{T}$ is an $\omega$-consistent, axiomatizable theory
extending~$\Th{Q}$, then $\Th{T} \Proves/ \lnot !G_\Th{T}$.
\end{lem}

\begin{proof}
We show that if $\Th{T}$ derives $\lnot !G_\Th{T}$, then it is
$\omega$-inconsistent. Suppose $\Th{T}$ derives $\lnot !G_\Th{T}$. If
$\Th{T}$ is inconsistent, it is $\omega$-inconsistent, and we are
done. Otherwise, $\Th{T}$ is consistent, so it does not derive
$!G_\Th{T}$ by Lemma~\ref{lem:cons-G-unprov}. Since there is no
derivation of $!G_\Th{T}$ in $\Th{T}$, $\Th{Q}$ derives
\[
\lnot \OPrf[\Th{T}](\num 0, \gn{!G_\Th{T}}), \lnot \OPrf[\Th{T}](\num 1,
\gn{!G_\Th{T}}), \lnot \OPrf[\Th{T}](\num 2, \gn{!G_\Th{T}}), \dots
\]
and so does~$\Th{T}$.  On the other hand, by \eqref{eqn:qpf}, $\lnot
!G_\Th{T}$ is equivalent to
$\lexists[x][\OPrf[\Th{T}](x,\gn{!G_\Th{T}})]$. So $\Th{T}$ is
$\omega$-inconsistent.
\end{proof}

\begin{thm}[First Incompleteness Theorem --- G\"odel's version] % CP-005
\label{CP-005}
Let $\Th{T}$ be any
$\omega$-consistent, axiomatizable theory extending~$\Th{Q}$. Then
$\Th{T}$ is not complete.
\end{thm}

\begin{proof}
  If $\Th{T}$ is $\omega$-consistent, it is consistent, so $\Th{T}
  \Proves/ !G_\Th{T}$ by Lemma~\ref{lem:cons-G-unprov}.  By
  Lemma~\ref{lem:omega-cons-G-unref}, $\Th{T} \Proves/ \lnot !G_\Th{T}$.
  This means that $\Th{T}$ is incomplete, since it derives neither
  $!G_\Th{T}$ nor $\lnot !G_\Th{T}$.
\end{proof}

\begin{rem}[Computability-theoretic proof]
\label{rem:comp-incompleteness}
There is an alternative, more direct proof of the First Incompleteness
Theorem via computability theory: if $\Th{T}$ were a complete,
consistent, axiomatizable extension of~$\Th{Q}$, then $\Th{T}$ would
be decidable, contradicting the undecidability of~$\Th{Q}$ (see
Theorem~\ref{CP-008} below). This computability-theoretic argument
avoids the need for the $\omega$-consistency hypothesis entirely,
though it does not construct an explicit independent sentence.
\end{rem}

%%% -----------------------------------------------------------------
%%% META.5.4  Rosser's Theorem
%%% -----------------------------------------------------------------

\subsection{Rosser's Theorem}

Can we modify G\"odel's proof to get a stronger result, replacing
``$\omega$-consistent'' with simply ``consistent''? The answer is
``yes,'' using a trick discovered by Rosser.  Rosser's trick is to use
a ``modified'' derivability predicate $\ORProv_T(y)$ instead of
$\OProv[\Th{T}](y)$.

\begin{thm}[Rosser's Theorem] % CP-005 (strengthened)
\label{thm:rosser}
Let $\Th{T}$ be any consistent, axiomatizable theory
extending $\Th{Q}$. Then $\Th{T}$ is not complete.
\end{thm}

\begin{proof}
Recall that $\OProv[\Th{T}](y)$ is defined as $\lexists[x][\OPrf[\Th{T}](x,
  y)]$, where $\OPrf[\Th{T}](x, y)$ represents the decidable relation which
holds iff $x$ is the G\"odel number of a derivation of the
sentence with G\"odel number~$y$. The relation that holds between
$x$ and~$y$ if $x$~is the G\"odel number of a \emph{refutation} of the
sentence with G\"odel number~$y$ is also decidable. Let $\fn{not}(x)$
be the primitive recursive function which does the following: if $x$
is the code of a formula $!A$, $\fn{not}(x)$ is a code of $\lnot
!A$. Then $\Refut[\Th{T}](x, y)$ holds iff $\Prf[\Th{T}](x, \fn{not}(y))$.  Let
$\ORefut[\Th{T}](x, y)$ represent it.  Then, if $\Th{T} \Proves \lnot !A$
and $\delta$ is a corresponding derivation, $\Th{Q} \Proves
\ORefut[\Th{T}](\gn{\delta}, \gn{!A})$.  We define $\ORProv[\Th{T}](y)$ as
\[
\lexists[x][(\OPrf[\Th{T}](x,y) \land \lforall[z][(z < x \lif \lnot
  \ORefut[\Th{T}](z,y))])].
\]
Roughly, $\ORProv[\Th{T}](y)$ says ``there is a proof of $y$ in $\Th{T}$,
and there is no shorter refutation of~$y$.''  Assuming $\Th{T}$ is
consistent, $\ORProv[\Th{T}](y)$ is true of the same numbers as
$\OProv[\Th{T}](y)$; but from the point of view of \emph{provability}
in~$\Th{T}$ (and we now know that there is a difference between truth
and provability!) the two have different properties. If $\Th{T}$ is
\emph{in}consistent, then the two do \emph{not} hold of the same
numbers!

By the fixed-point lemma, there is a formula $!R_\Th{T}$ such that
\begin{equation}
  \Th{Q} \Proves !R_\Th{T} \liff \lnot \ORProv[\Th{T}](\gn{!R_\Th{T}}).
  \label{RT}
\end{equation}
In contrast to the proof of Theorem~\ref{CP-005},
here we claim that if $\Th{T}$ is consistent, $\Th{T}$ does not derive
$!R_\Th{T}$, and $\Th{T}$ also does not derive $\lnot !R_\Th{T}$. (In
other words, we do not need the assumption of $\omega$-consistency.)

First, let us show that $\Th{T} \Proves/ !R_{\Th{T}}$.  Suppose it did, so
there is a derivation of~$!R_{\Th{T}}$ from~$T$; let $n$ be its G\"odel
number. Then $\Th{Q} \Proves \OPrf[\Th{T}](\num{n}, \gn{!R_{\Th{T}}})$, since
$\OPrf[\Th{T}]$ represents $\Prf[\Th{T}]$ in~$\Th{Q}$. Also, for each $k < n$,
$k$ is not the G\"odel number of a derivation of $\lnot !R_{\Th{T}}$, since $\Th{T}$ is
consistent. So for each $k < n$, $\Th{Q} \Proves \lnot
\ORefut[\Th{T}](\num{k}, \gn{!R_{\Th{T}}})$. By properties of~$<$ in~$\Th{Q}$,
$\Th{Q} \Proves \lforall[z][(z < \num{n} \lif \lnot \ORefut[\Th{T}](z,
  \gn{!R_{\Th{T}}}))]$. Thus,
\[
\Th{Q} \Proves \lexists[x][(\OPrf[\Th{T}](x,\gn{!R_{\Th{T}}}) \land \lforall[z][(z
    < x \lif \lnot \ORefut[\Th{T}](z,\gn{!R_{\Th{T}}}))])],
\]
but that is just $\ORProv[\Th{T}](\gn{!R_{\Th{T}}})$. By \eqref{RT}, $\Th{Q}
\Proves \lnot !R_{\Th{T}}$. Since $\Th{T}$ extends $\Th{Q}$, also $\Th{T}
\Proves \lnot !R_{\Th{T}}$. We have assumed that $\Th{T} \Proves !R_{\Th{T}}$, so
$\Th{T}$ would be inconsistent, contrary to the assumption of the
theorem.

Now, let us show that $\Th{T} \Proves/ \lnot !R_{\Th{T}}$. Again, suppose it
did, and suppose $n$ is the G\"odel number of a derivation
of~$\lnot !R_{\Th{T}}$. Then $\Refut[\Th{T}](n, \Gn{!R_{\Th{T}}})$ holds, and since
$\ORefut[\Th{T}]$ represents $\Refut[\Th{T}]$ in $\Th{Q}$, $\Th{Q} \Proves
\ORefut[\Th{T}](\num{n}, \gn{!R_{\Th{T}}})$. We will again show that $\Th{T}$ would
then be inconsistent because it would also derive~$!R_{\Th{T}}$.  Since
\begin{align*}
\Th{Q} & \Proves !R_{\Th{T}} \liff \lnot \ORProv[\Th{T}](\gn{!R_{\Th{T}}}),
\intertext{and since $\Th{T}$ extends~$\Th{Q}$, it suffices to show that}
\Th{Q} & \Proves \lnot \ORProv[\Th{T}](\gn{!R_{\Th{T}}}).
\end{align*}
The sentence $\lnot
\ORProv[\Th{T}](\gn{!R_{\Th{T}}})$, i.e.,
\begin{align*}
  \lnot & \lexists[x][(\OPrf[\Th{T}](x,\gn{!R_{\Th{T}}}) \land
    \lforall[z][(z < x \lif \lnot \ORefut[\Th{T}](z,\gn{!R_{\Th{T}}}))])],
  \intertext{is logically equivalent to}
  & \lforall[x][(\OPrf[\Th{T}](x,\gn{!R_{\Th{T}}}) \lif
    \lexists[z][(z < x \land \ORefut[\Th{T}](z,\gn{!R_{\Th{T}}}))])].
\end{align*}
We argue informally using logic, making use of facts about what
$\Th{Q}$ derives. Suppose $x$ is arbitrary and $\OPrf[\Th{T}](x,
\gn{!R_{\Th{T}}})$. We already know that $\Th{T} \Proves/ !R_{\Th{T}}$, and so for
every $k$, $\Th{Q} \Proves \lnot \OPrf[\Th{T}](\num{k}, \gn{!R_{\Th{T}}})$. Thus,
for every $k$ it follows that $\eq/[x][\num{k}]$. In particular, we
have (a) that $\eq/[x][\num{n}]$.  We also have $\lnot(\eq[x][\num{0}]
\lor \eq[x][\num{1}] \lor \dots \lor \eq[x][\num{n-1}])$ and so by
properties of~$<$ in~$\Th{Q}$, (b) $\lnot(x < \num{n})$. By
trichotomy in~$\Th{Q}$, $\num{n} < x$. Since $\Th{Q} \Proves
\ORefut[\Th{T}](\num{n}, \gn{!R_{\Th{T}}})$, we have $\num{n} < x \land
\ORefut[\Th{T}](\num{n}, \gn{!R_{\Th{T}}})$, and from that $\lexists[z][(z < x
\land \ORefut[\Th{T}](z,\gn{!R_{\Th{T}}}))]$. Since $x$ was arbitrary we get, as
required, that
\[
\lforall[x][(\OPrf[\Th{T}](x,\gn{!R_{\Th{T}}}) \lif
  \lexists[z][(z < x \land \ORefut[\Th{T}](z,\gn{!R_{\Th{T}}}))])].
\]
\end{proof}


%% ===================================================================
%% META.6: Second Incompleteness (CP-006)
%% Sources: inc/inp/prc (KEEP), inc/inp/2in (KEEP), inc/inp/lob (KEEP)
%% ===================================================================

\section{The Second Incompleteness Theorem} \label{META.6}

G\"odel's Second Incompleteness Theorem shows that no consistent,
sufficiently strong theory can prove its own consistency. The proof
rests on the derivability conditions for the provability predicate.
We also present L\"ob's Theorem, which gives a sharp characterization
of when a reflection principle can be derived.

%%% -----------------------------------------------------------------
%%% META.6.1  Derivability Conditions
%%% -----------------------------------------------------------------

\subsection{The Derivability Conditions for $\Th{PA}$}

Peano arithmetic, or $\Th{PA}$, is the theory extending $\Th{Q}$ with
induction axioms for all formulas. In other words, one adds to $\Th{Q}$
axioms of the form
\[
(!A(0) \land \lforall[x][(!A(x) \lif !A(x'))]) \lif \lforall[x][!A(x)]
\]
for every formula~$!A$. Notice that this is really a
\emph{schema}, which is to say, infinitely many axioms (and it turns
out that $\Th{PA}$ is {\em not} finitely axiomatizable). But since one
can effectively determine whether or not a string of symbols is an
instance of an induction axiom, the set of axioms for $\Th{PA}$ is
computable. $\Th{PA}$ is a much more robust theory than~$\Th{Q}$. For
example, one can easily prove that addition and multiplication are
commutative, using induction in the usual way. In fact, most finitary
number-theoretic and combinatorial arguments can be carried out
in~$\Th{PA}$.

Since $\Th{PA}$ is computably axiomatized, the derivability
predicate $\Prf[\Th{PA}](x,y)$ is computable and hence represented
in~$\Th{Q}$ (and so, in~$\Th{PA}$). As before, we will take
$\OPrf[\Th{PA}](x,y)$ to denote the formula representing the relation.
Let $\OProv[\Th{PA}](y)$ be the formula
$\lexists[x][\OPrf[\Th{PA}](x,y)]$, which intuitively says, ``$y$ is
derivable from the axioms of $\Th{PA}$.''  The reason we need a
little bit more than the axioms of $\Th{Q}$ is we need to know that
the theory we are using is strong enough to derive a few basic
facts about this derivability predicate. In fact, what we need are
the following facts:

\begin{defn}[Derivability Conditions] % DEF-DED014
\label{DEF-DED014}
A formula $\OProv[\Th{T}](y)$ satisfies the
\emph{Hilbert--Bernays--L\"ob derivability conditions} for a
theory~$\Th{T}$ if the following hold:
\begin{enumerate}
\item[P1.] If $\Th{T} \Proves !A$, then $\Th{T} \Proves
  \OProv[\Th{T}](\gn{!A})$.
\item[P2.] For all formulas $!A$ and $!B$,
  \[
  \Th{T} \Proves \OProv[\Th{T}](\gn{!A \lif !B}) \lif
  (\OProv[\Th{T}](\gn{!A}) \lif \OProv[\Th{T}](\gn{!B})).
  \]
\item[P3.] For every formula~$!A$,
  \[
  \Th{T} \Proves \OProv[\Th{T}](\gn{!A})
  \lif \OProv[\Th{T}](\gn{\OProv[\Th{T}](\gn{!A})}).
  \]
\end{enumerate}
\end{defn}

The only way to verify that these three properties hold is to describe
the formula $\OProv[\Th{PA}](y)$ carefully and use the axioms of
$\Th{PA}$ to describe the relevant formal derivations. Conditions (1)
and~(2) are easy; it is really condition~(3) that requires
work. Carrying out the
details would be tedious and uninteresting, so here we will ask you to
take it on faith that $\Th{PA}$ has the three properties listed
above. A reasonable choice of $\OProv[\Th{PA}](y)$ will also satisfy
\begin{enumerate}
\item[P4.] If $\Th{PA} \Proves \OProv[\Th{PA}](\gn{!A})$, then
  $\Th{PA} \Proves !A$.
\end{enumerate}
But we will not need this fact.

%%% -----------------------------------------------------------------
%%% META.6.2  The Second Incompleteness Theorem
%%% -----------------------------------------------------------------

\subsection{The Second Incompleteness Theorem}

How can we express the assertion that $\Th{PA}$ does not prove its own
consistency? Saying $\Th{PA}$ is inconsistent amounts to saying that
$\Th{PA} \Proves \eq[0][1]$. So we can take the consistency statement
$\OCon[\Th{PA}]$ to be the sentence $\lnot
\OProv[\Th{PA}](\gn{\eq[0][1]})$, and then the following theorem does
the job:

\begin{thm}[Second Incompleteness Theorem] % CP-006
\label{CP-006}
Assuming $\Th{PA}$ is consistent, then $\Th{PA}$ does not derive
$\OCon[\Th{PA}]$.
\end{thm}

It is important to note that the theorem depends on the particular
representation of $\OCon[\Th{PA}]$ (i.e., the particular
representation of $\OProv[\Th{PA}](y)$). All we will use is that the
representation of $\OProv[\Th{PA}](y)$ satisfies the three
derivability conditions, so the theorem generalizes to any theory
with a derivability predicate having these properties.

It is informative to read G\"odel's sketch of an argument, since the
theorem follows like a good punch line. It goes like this. Let
$!G_\Th{PA}$ be the G\"odel sentence that we constructed in the proof
of Theorem~\ref{CP-005}. We have shown ``If $\Th{PA}$
is consistent, then $\Th{PA}$ does not derive $!G_\Th{PA}$.'' If we
formalize this \emph{in} $\Th{PA}$, we have a proof of
\[
\OCon[\Th{PA}] \lif \lnot \OProv[\Th{PA}](\gn{!G_\Th{PA}}).
\]
Now suppose $\Th{PA}$ derives $\OCon[\Th{PA}]$. Then it derives $\lnot
\OProv[\Th{PA}](\gn{!G_\Th{PA}})$. But since $!G_\Th{PA}$ is a G\"odel
sentence, this is equivalent to $!G_\Th{PA}$. So $\Th{PA}$ derives
$!G_\Th{PA}$.

But: we know that if $\Th{PA}$ is consistent, it does not derive
$!G_\Th{PA}$!{}  So if $\Th{PA}$ is consistent, it cannot derive
$\OCon[\Th{PA}]$.

To make the argument more precise, we will let $!G_\Th{PA}$ be the
G\"odel sentence for~$\Th{PA}$ and use the derivability conditions
(P1)--(P3) to show that $\Th{PA}$ derives $\OCon[\Th{PA}] \lif
!G_\Th{PA}$. This will show that $\Th{PA}$ does not derive
$\OCon[\Th{PA}]$. Here is a sketch of the proof, in~$\Th{PA}$. (For
simplicity, we drop the $\Th{PA}$ subscripts.)
\begin{align}
& !G \liff \lnot \OProv(\gn{!G}) \label{G2-1}\\
& \qquad\text{$!G$ is a G\"odel sentence}\notag \\
& !G \lif \lnot \OProv(\gn{!G}) \label{G2-2}\\
  & \qquad\text{from \eqref{G2-1}} \notag\\
& !G \lif
  (\OProv(\gn{!G}) \lif \lfalse) \label{G2-3}\\
  & \qquad\text{from \eqref{G2-2} by logic}\notag\\
& \OProv(\gn{
    !G \lif
    (\OProv(\gn{!G}) \lif \lfalse)
  }) \label{G2-4}\\
  & \qquad\text{from \eqref{G2-3} by condition P1} \notag\\
& \OProv(\gn{!G}) \lif
  \OProv(\gn{
    (\OProv(\gn{!G}) \lif \lfalse)
    }) \label{G2-5}\\
  & \qquad\text{from \eqref{G2-4} by condition P2} \notag\\
& \OProv(\gn{!G}) \lif (\OProv(\gn{\OProv(\gn{!G})}) \lif \OProv(\gn{\lfalse})) \label{G2-6}\\
  & \qquad\text{from \eqref{G2-5} by condition P2 and logic} \notag\\
& \OProv(\gn{!G}) \lif
  \OProv(\gn{\OProv(\gn{!G})}) \label{G2-7}\\
   & \qquad\text{by P3} \notag\\
& \OProv(\gn{!G}) \lif \OProv(\gn{\lfalse}) \label{G2-8}\\
  & \qquad \text{from \eqref{G2-6} and \eqref{G2-7} by logic}\notag\\
& \OCon \lif \lnot \OProv(\gn{!G}) \label{G2-9}\\
  & \qquad\text{contraposition of \eqref{G2-8} and $\OCon \ident \lnot \OProv(\gn{\lfalse})$}\notag \\
& \OCon \lif !G \notag\\
  & \qquad\text{from \eqref{G2-1} and \eqref{G2-9} by logic}\notag
\end{align}
The use of logic in the above involves just elementary facts from propositional
logic, e.g., \eqref{G2-3} uses $\Proves \lnot!A \liff (!A\lif
\lfalse)$ and \eqref{G2-8} uses $!A \lif (!B \lif !C), !A \lif !B
\Proves !A \lif !C$. The use of condition~P2 in \eqref{G2-5} and
\eqref{G2-6} relies on instances of~P2, $\OProv(\gn{!A \lif !B}) \lif
(\OProv(\gn{!A}) \lif \OProv(\gn{!B}))$. In the first one, $!A \ident
!G$ and $!B \ident \OProv(\gn{!G}) \lif \lfalse$; in the second, $!A
\ident \OProv(\gn{G})$ and $!B \ident \lfalse$.

The more abstract version of the second incompleteness theorem is as follows:

\begin{thm}[Second Incompleteness Theorem --- general version] % CP-006
\label{thm:second-incompleteness-gen}
Let $\Th{T}$ be any
consistent, axiomatized theory extending $\Th{Q}$ and let
$\OProv[\Th{T}](y)$ be any formula satisfying derivability conditions
P1--P3 for~$\Th{T}$. Then $\Th{T}$ does not derive~$\OCon[T]$.
\end{thm}

The moral of the story is that no ``reasonable'' consistent theory for
mathematics can derive its own consistency statement. Suppose
$\Th{T}$ is a theory of mathematics that includes $\Th{Q}$ and
Hilbert's ``finitary'' reasoning (whatever that may be). Then, the
whole of $\Th{T}$ cannot derive the consistency statement of
$\Th{T}$, and so, a fortiori, the finitary fragment cannot derive
the consistency statement of~$\Th{T}$ either. In that sense, there
cannot be a finitary consistency proof for ``all of mathematics.''

There is some leeway in interpreting the term ``finitary,'' and
G\"odel, in the 1931 paper, grants the possibility that something we
may consider ``finitary'' may lie outside the kinds of mathematics
Hilbert wanted to formalize. But G\"odel was being charitable; today,
it is hard to see how we might find something that can reasonably be
called finitary but is not formalizable in, say, $\Th{ZFC}$,
Zermelo--Fraenkel set theory with the axiom of choice.

%%% -----------------------------------------------------------------
%%% META.6.3  Lob's Theorem
%%% -----------------------------------------------------------------

\subsection{L\"ob's Theorem}

The G\"odel sentence for a theory~$\Th{T}$ is a fixed point of $\lnot
\OProv[\Th{T}](y)$, i.e., a sentence~$!G$ such that
\[
\Th{T} \Proves \lnot \OProv[\Th{T}](\gn{!G}) \liff !G.
\]
It is not derivable, because if $\Th{T} \Proves !G$, (a) by derivability
condition~(1), $\Th{T} \Proves \OProv[\Th{T}](\gn{!G})$, and (b) $\Th{T}
\Proves !G$ together with $\Th{T} \Proves \lnot \OProv[\Th{T}](\gn{!G})
\liff !G$ gives $\Th{T} \Proves \lnot \OProv[\Th{T}](\gn{!G})$, and so
$\Th{T}$ would be inconsistent.  Now it is natural to ask about the
status of a fixed point of $\OProv[\Th{T}](y)$, i.e., a sentence~$!H$
such that
\[
\Th{T} \Proves \OProv[\Th{T}](\gn{!H}) \liff !H.
\]
If it were derivable, $\Th{T} \Proves \OProv[\Th{T}](\gn{!H})$ by
condition~(1), but the same conclusion follows if we apply modus
ponens to the equivalence above. Hence, we do not get that $\Th{T}$ is
inconsistent, at least not by the same argument as in the case of the
G\"odel sentence. This of course does not show that $\Th{T}$
\emph{does} derive~$!H$.

We can make headway on this question if we generalize it a bit. The
left-to-right direction of the fixed point equivalence,
$\OProv[\Th{T}](\gn{!H}) \lif !H$, is an instance of a general schema
called a \emph{reflection principle}: $\OProv[\Th{T}](\gn{!A}) \lif !A$.
It is called that because it expresses, in a sense, that $\Th{T}$ can
``reflect'' about what it can derive; basically it says, ``If $\Th{T}$
can derive~$!A$, then~$!A$ is true,'' for any~$!A$.  This is true for
sound theories only, of course, and this suggests that theories will
in general not derive every instance of it.  So which instances can a
theory (strong enough, and satisfying the derivability conditions)
derive?  Certainly all those where $!A$ itself is derivable. And that is
it, as the next result shows.

The heuristic for the proof of L\"ob's theorem is a clever proof that
Santa Claus exists. (If you do not like that conclusion, you are free
to substitute any other conclusion you would like.) Here it is:
\begin{enumerate}
\item Let $X$ be the sentence, ``If $X$ is true, then Santa Claus
  exists.''
\item Suppose $X$ is true.
\item Then what it says holds; i.e., we have: if $X$ is true, then
  Santa Claus exists.
\item Since we are assuming $X$ is true, we can conclude that
  Santa Claus exists, by modus ponens from (2) and~(3).
\item We have succeeded in deriving (4), ``Santa Claus exists,'' from
  the assumption~(2), ``$X$ is true.'' By conditional proof, we have
  shown: ``If $X$ is true, then Santa Claus exists.''
\item But this is just the sentence~$X$. So we have shown that $X$ is
  true.
\item But then, by the argument (2)--(4) above, Santa Claus exists.
\end{enumerate}
A formalization of this idea, replacing ``is true'' with ``is
derivable,'' and ``Santa Claus exists'' with~$!A$, yields the proof of
L\"ob's theorem. The trick is to apply the fixed-point lemma to the
formula~$\OProv[\Th{T}](y) \lif !A$. The fixed point of that
corresponds to the sentence~$X$ in the preceding sketch.

\begin{thm}[L\"ob's Theorem] % THM-DED007
\label{THM-DED007}
Let $\Th{T}$ be an axiomatizable theory extending $\Th{Q}$, and
suppose $\OProv[\Th{T}](y)$ is a formula satisfying conditions P1--P3
(Definition~\ref{DEF-DED014}). If $\Th{T}$ derives $\OProv[\Th{T}](\gn{!A}) \lif !A$,
then in fact $\Th{T}$ derives $!A$.
\end{thm}

Put differently, if $\Th{T} \Proves/ !A$, then $\Th{T} \Proves/
\OProv[\Th{T}](\gn{!A}) \lif !A$.

\begin{proof}
Suppose $!A$ is a sentence such that $\Th{T}$ derives
$\OProv[\Th{T}](\gn{!A}) \lif !A$. Let $!B(y)$ be the formula~$\OProv[\Th{T}](y)
\lif !A$, and use the fixed-point lemma to find a sentence~$!D$
such that $\Th{T}$ derives $!D \liff !B(\gn{!D})$. Then each of the
following is derivable in $\Th{T}$:
\begin{align}
  & !D \liff (\OProv[\Th{T}](\gn{!D}) \lif !A) \label{L-1}\\
  & \qquad \text{$!D$ is a fixed point of~$!B(y)$}\notag \\
  & !D \lif (\OProv[\Th{T}](\gn{!D}) \lif !A) \label{L-2}\\
  & \qquad\text{from \eqref{L-1}}\notag\\
  & \OProv[\Th{T}](\gn{!D \lif (\OProv[\Th{T}](\gn{!D}) \lif !A)}) \label{L-3}\\
  & \qquad \text{from \eqref{L-2} by condition P1}\notag \\
  & \OProv[\Th{T}](\gn{!D}) \lif \OProv[\Th{T}](\gn{\OProv[\Th{T}](\gn{!D}) \lif !A})
  \label{L-4}\\
  &\qquad \text{from \eqref{L-3} using condition P2}\notag \\
  & \OProv[\Th{T}](\gn{!D}) \lif (\OProv[\Th{T}](\gn{\OProv[\Th{T}](\gn{!D})}) \lif \OProv[\Th{T}](\gn{!A})) \label{L-5}\\
  &\qquad \text{from \eqref{L-4} using P2 again} \notag\\
& \OProv[\Th{T}](\gn{!D}) \lif \OProv[\Th{T}](\gn{\OProv[\Th{T}](\gn{!D})}) \label{L-6}\\
  & \qquad\text{by derivability condition P3} \notag\\
  & \OProv[\Th{T}](\gn{!D}) \lif \OProv[\Th{T}](\gn{!A}) \label{L-7} \\
  &\qquad\text{from \eqref{L-5} and \eqref{L-6}}\notag\\
  & \OProv[\Th{T}](\gn{!A}) \lif !A \label{L-8}\\
  &\qquad\text{by assumption of the theorem} \notag\\
  & \OProv[\Th{T}](\gn{!D}) \lif !A \label{L-9}\\
  &\qquad\text{from \eqref{L-7} and \eqref{L-8}}\notag\\
  & (\OProv[\Th{T}](\gn{!D}) \lif !A) \lif !D \label{L-10}\\
  & \qquad \text{from \eqref{L-1}}\notag \\
  & !D \label{L-11}\\
  & \qquad\text{from \eqref{L-9} and \eqref{L-10}}\notag \\
  & \OProv[\Th{T}](\gn{!D}) \label{L-12}\\
  & \qquad\text{from \eqref{L-11} by condition~P1}\notag \\
  & !A \qquad\qquad\text{from \eqref{L-8} and \eqref{L-12}}\notag
\end{align}
\end{proof}

With L\"ob's theorem in hand, there is a short proof of the second
incompleteness theorem (for theories having a derivability predicate
satisfying conditions P1--P3): if $\Th{T} \Proves
\OProv[\Th{T}](\gn{\lfalse}) \lif \lfalse$, then $\Th{T} \Proves \lfalse$.
If $\Th{T}$ is consistent, $\Th{T} \Proves/ \lfalse$. So, $\Th{T}
\Proves/ \OProv[\Th{T}](\gn{\lfalse}) \lif \lfalse$, i.e., $\Th{T} \Proves/
\OCon[\Th{T}]$.  We can also apply it to show that~$!H$, the fixed
point of $\OProv[\Th{T}](x)$, is derivable. For since
\begin{align*}
  \Th{T} & \Proves \OProv[\Th{T}](\gn{!H}) \liff !H\\
  \intertext{in particular}
    \Th{T} & \Proves \OProv[\Th{T}](\gn{!H}) \lif !H
\end{align*}
and so by L\"ob's theorem, $\Th{T} \Proves !H$.


%% ===================================================================
%% META.7: Undefinability (CP-007)
%% Sources: inc/inp/tar (KEEP)
%% ===================================================================

\section{The Undefinability of Truth} \label{META.7}

The notion of \emph{definability} depends on having a formal semantics
for the language of arithmetic.  We have described a set of formulas
and sentences in the language of arithmetic. The ``intended
interpretation'' is to read such sentences as making assertions about
the natural numbers, and such an assertion can be true or false. Let
$\Struct{N}$ be the structure with domain $\Nat$ and the standard
interpretation for the symbols in the language of arithmetic.  Then
$\Sat{N}{!A}$ means ``$!A$ is true in the standard interpretation.''

\begin{defn}[Definability in $\Struct{N}$] \label{defn:definable-N}
A relation $R(x_1,\dots,x_k)$ of natural numbers is \emph{definable}
in $\Struct{N}$ if and only if there is a formula $!A(x_1,\dots,x_k)$
in the language of arithmetic such that for every $n_1,\dots,n_k$,
$R(n_1,\dots,n_k)$ if and only if $\Sat{N}{!A(\num n_1,\dots,\num
  n_k)}$.
\end{defn}

Put differently, a relation is definable in $\Struct{N}$ if and
only if it is representable in the theory $\Th{TA}$, where $\Th{TA} =
\Setabs{!A}{\Sat{N}{!A}}$ is the set of true sentences of
arithmetic.

\begin{lem} \label{lem:comp-definable}
Every computable relation is definable in~$\Struct{N}$.
\end{lem}

\begin{proof}
It is easy to check that the formula representing a relation in
$\Th{Q}$ defines the same relation in $\Struct{N}$.
\end{proof}

Now one can ask, is the converse also true?  That is, is every
relation definable in~$\Struct{N}$ computable? The answer is no. For
example:

\begin{lem} \label{lem:halting-definable}
The halting relation is definable in $\Struct{N}$.
\end{lem}

\begin{proof}
Recall that the Kleene normal form theorem states that every partial
computable function~$f$ has an index~$e$ such that $f(x) =
U(\umin{s}{T(e,x,s)})$ for all $x \in \Nat$, where $U$ and $T$ are
primitive recursive and therefore total. Thus, $f(x)$ is defined
(i.e., the computation halts) iff there is an~$s$ such that $T(e,x,s)$
holds.

Now let $H$ be the halting relation, i.e.,
\[
H = \Setabs{\tuple{e,x}}{\lexists[s][T(e, x, s)]}.
\]
Let $!D_T$ define $T$ in $\Struct{N}$. Then
\[
H = \Setabs{\tuple{e,x}}{\Sat{N}{\lexists[s][!D_T(\num e, \num x, s)]}},
\]
so $\lexists[s][!D_T(z, x, s)]$ defines~$H$ in $\Struct{N}$.
\end{proof}

What about $\Th{TA}$ itself? Is it definable in arithmetic? That
is: is the set $\Setabs{\Gn{!A}}{\Sat{N}{!A}}$ definable in
arithmetic? Tarski's theorem answers this in the negative.

\begin{thm}[Tarski's Undefinability Theorem] % CP-007
\label{CP-007}
The set of true sentences of arithmetic is not definable in arithmetic.
\end{thm}

\begin{proof}
Suppose $!D(x)$ defined it, i.e., $\Sat{N}{!A}$ iff
$\Sat{N}{!D(\gn{!A})}$. By the fixed-point lemma
(Lemma~\ref{THM-DED006}), there is a formula
$!A$ such that $\Th{Q} \Proves !A \liff \lnot !D(\gn{!A})$, and hence
$\Sat{N}{!A \liff \lnot !D(\gn{!A})}$. But then $\Sat{N}{!A}$ if and
only if $\Sat{N}{\lnot !D(\gn{!A})}$, which contradicts the fact that
$!D(y)$ is supposed to define the set of true statements of
arithmetic.
\end{proof}

Tarski applied this analysis to a more general philosophical notion of
truth. Given any language $L$, Tarski argued that an adequate notion
of truth for $L$ would have to satisfy, for each sentence $X$,
\begin{quote}
`$X$' is true if and only if $X$.
\end{quote}
Tarski's oft-quoted example, for English, is the sentence
\begin{quote}
`Snow is white' is true if and only if snow is white.
\end{quote}
However, for any language strong enough to represent the diagonal
function, and any linguistic predicate $T(x)$, we can construct a
sentence $X$ satisfying ``$X$ if and only if not $T(\text{`$X$'})$.''
Given that we do not want a truth predicate to declare some sentences
to be both true and false, Tarski concluded that one cannot specify a
truth predicate for all sentences in a language without, somehow,
stepping outside the bounds of the language. In other words, the
truth predicate for a language cannot be defined in the language
itself.


%% ===================================================================
%% META.8: Undecidability (CP-008)
%% Sources: inc/req/und (KEEP)
%% ===================================================================

\section{Undecidability} \label{META.8}

We call a theory $\Th{T}$ \emph{undecidable} if there is no
computational procedure which, after finitely many steps and
unfailingly, provides a correct answer to the question ``does $\Th{T}$
prove~$!A$?'' for any sentence~$!A$ in the language of~$\Th{T}$.  So
$\Th{Q}$ would be decidable iff there were a computational procedure
which decides, given a sentence~$!A$ in the language of arithmetic,
whether $\Th{Q} \Proves !A$ or not.  We can make this more precise by
asking: Is the relation~$\Prov[\Th{Q}](y)$, which holds of~$y$
iff $y$ is the G\"odel number of a sentence provable in~$\Th{Q}$,
recursive?  The answer is: no.

\begin{thm}[Undecidability of $\Th{Q}$] % CP-008
\label{CP-008}
$\Th{Q}$ is undecidable, i.e., the relation
\[
\Prov[\Th{Q}](y) \defiff \fn{Sent}(y) \land
\lexists[x][\Prf[\Th{Q}](x, y)]
\]
is not recursive.
\end{thm}

\begin{proof}
Suppose it were.  Then we could solve the halting problem as follows:
Given $e$ and $n$, we know that $\cfind{e}(n) \fdefined$ iff there is
an~$s$ such that $T(e, n, s)$, where $T$ is Kleene's predicate from
the Kleene Normal Form Theorem (see DEF-CMP005, \S\ref{CMP.3}).
Since $T$ is primitive recursive
it is representable in~$\Th{Q}$ by a formula $!B_T$, that is, $\Th{Q}
\Proves !B_T(\num{e}, \num{n}, \num{s})$ iff $T(e, n, s)$.  If $\Th{Q}
\Proves !B_T(\num{e}, \num{n}, \num{s})$ then also $ \Th{Q} \Proves
\lexists[y][!B_T(\num{e}, \num{n}, y)]$.  If no such $s$ exists, then
$\Th{Q} \Proves \lnot !B_T(\num{e}, \num{n}, \num{s})$ for
every~$s$.  But $\Th{Q}$ is $\omega$-consistent, i.e., if $\Th{Q}
\Proves \lnot !A(\num{n})$ for every~$n \in \Nat$, then $\Th{Q}
\Proves/ \lexists[y][!A(y)]$.  We know this because the axioms of
$\Th{Q}$ are true in the standard model~$\Struct{N}$.  So, $\Th{Q}
\Proves/ \lexists[y][!B_T(\num{e}, \num{n}, y)]$.  In other words,
$\Th{Q} \Proves \lexists[y][!B_T(\num{e}, \num{n}, y)]$ iff there is
an $s$ such that $T(e, n, s)$, i.e., iff $\cfind{e}(n) \fdefined$.
From $e$ and~$n$ we can compute $\Gn{\lexists[y][!B_T(\num{e},
    \num{n}, y)]}$, let $g(e, n)$ be the primitive recursive function
which does that.  So
\[
h(e, n) =
\begin{cases}
1 & \text{if $\Prov[\Th{Q}](g(e, n))$}\\
0 & \text{otherwise}.
\end{cases}
\]
This would show that $h$ is recursive if $\Prov[\Th{Q}]$ is. But~$h$
is not recursive, by the unsolvability of the Halting Problem
(see THM-CMP002, Unsolvability of the Halting Problem, \S\ref{CMP.4}), so
$\Prov[\Th{Q}]$ cannot be either.
\end{proof}

\begin{cor}[Undecidability of first-order logic] % CP-008
\label{cor:fol-undecidable}
First-order logic is undecidable.
\end{cor}

\begin{proof}
If first-order logic were decidable, provability in~$\Th{Q}$ would be
as well, since $\Th{Q} \Proves !A$ iff $\Proves !T \lif !A$, where
$!T$ is the conjunction of the axioms of~$\Th{Q}$.
\end{proof}


%% ===================================================================
%% META.9: Craig Interpolation (CP-011)
%% Sources: mod/int/sep (CONDENSE), mod/int/prf (KEEP)
%% ===================================================================

\section{Craig's Interpolation Theorem} \label{META.9}

The interpolation theorem states that whenever a valid conditional
$\Entails !A \lif !B$ holds, there exists a ``mediating'' sentence~$!C$
whose non-logical vocabulary is common to both~$!A$ and~$!B$.  Finding
such an interpolant amounts to finding a sentence that \emph{separates}
$!A$ from $\lnot !B$.

\subsection{Separation}

An interpolant for $!A$ and $!B$ is a sentence~$!C$ such that
$!A \Entails !C$ and $!C \Entails !B$.  By contraposition, the latter
holds iff $\lnot !B \Entails \lnot !C$.  A sentence~$!C$ with this
property is said to \emph{separate} $!A$ and $\lnot !B$.  So finding an
interpolant for $!A$ and $!B$ amounts to finding a sentence that
separates $!A$ and $\lnot !B$.  It will be useful to consider the
generalization to sets of sentences.

\begin{defn}
A sentence $!C$ \emph{separates} sets of sentences $\Gamma$ and
$\Delta$ if and only if $\Gamma \Entails !C$ and $\Delta \Entails
\lnot !C$. If no such sentence exists, then $\Gamma$ and $\Delta$
are \emph{inseparable}.
\end{defn}

\begin{lem}\label{lem:sep1}
Suppose $\Lang{L}_0$ is the language containing every constant,
function and predicate (other than $\doteq$) that occurs in
\emph{both} $\Gamma$ and $\Delta$, and let $\Lang{L}'_0$ be obtained
by the addition of infinitely many new constants $\Obj c_n$ for $n
\ge 0$. Then if $\Gamma$ and $\Delta$ are inseparable in $\Lang{L}_0$,
they are also inseparable in $\Lang{L}'_0$.
\end{lem}

\begin{proof}
We proceed indirectly: suppose by way of contradiction that $\Gamma$
and $\Delta$ are separated in $\Lang{L}'_0$. Then $\Gamma \Entails
\Subst{!C}{c}{x}$ and $\Delta \Entails \lnot \Subst{!C}{c}{x}$ for some $!C \in
\Lang{L}_0$ (where $c$ is a new constant---the case where $!C$
contains more than one such new constant is similar). By
compactness (Theorem~\ref{CP-003}), there are \emph{finite} subsets $\Gamma_0$ of $\Gamma$
and $\Delta_0$ of $\Delta$ such that $\Gamma_0 \Entails \Subst{!C}{c}{x}$
and $\Delta_0 \Entails \lnot \Subst{!C}{c}{x}$. Let $!G$ be the
conjunction of all formulas in $\Gamma_0$ and $!H$ the
conjunction of all formulas in $\Delta_0$. Then
\begin{align*}
  !G & \Entails \Subst{!C}{c}{x}, & !H  \Entails \lnot \Subst{!C}{c}{x}.
\end{align*}
From the former, by Generalization, we have $!G \Entails
\lforall[x][!C]$, and from the latter by contraposition,
$\Subst{!C}{c}{x} \Entails \lnot !H$, whence also $\lforall[x][!C]
\Entails \lnot !H$. Contraposition again gives $!H \Entails
\lnot \lforall[x][!C]$. By monotonicity,
\begin{align*}
  \Gamma &\Entails \lforall[x][!C], &
  \Delta & \Entails \lnot \lforall[x][!C],
\end{align*}
so that $\lforall[x][!C]$ separates $\Gamma$ and $\Delta$ in
$\Lang{L}_0$.
\end{proof}

\begin{lem}\label{lem:sep2}
Suppose that $\Gamma \cup \{ \lexists[x][!S] \}$ and $\Delta$ are
inseparable, and $c$ is a new constant not in $\Gamma$, $\Delta$,
or $!S$. Then $\Gamma \cup \{ \lexists[x][!S], \Subst{!S}{c}{x} \}$
and $\Delta$ are also inseparable.
\end{lem}

\begin{proof}
Suppose for contradiction that $!C$ separates $\Gamma \cup \{
\lexists[x][!S], \Subst{!S}{c}{x}\}$ and $\Delta$, while at the same
time $\Gamma \cup \{\lexists[x]{!S} \}$ and $\Delta$ are
inseparable. We distinguish two cases:
\begin{enumerate}
\item $c$ does not occur in $!C$: in this case $\Gamma \cup
  \{\lexists[x][!S], \lnot!C \}$ is satisfiable (otherwise $!C$
  separates $\Gamma \cup \{\lexists[x][!S] \}$ and $\Delta$). It
  remains so if $\Subst{!S}{c}{x}$ is added, so $!C$ does not separate
  $\Gamma \cup \{ \lexists[x][!S], \Subst{!S}{c}{x} \}$ and $\Delta$
  after all.
\item $c$ does occur in $!C$ so that $!C$ has the form
  $\Subst{!C}{c}{x}$. Then we have that
  \[
  \Gamma \cup \{ \lexists[x][!S], \Subst{!S}{c}{x}\} \Entails \Subst{!C}{c}{x},
  \]
  whence $\Gamma, \lexists[x][!S] \Entails \lforall[x][(!S \lif !C)]$
  by the Deduction Theorem and Generalization, and finally $\Gamma
  \cup \{ \lexists[x][!S] \} \Entails \lexists[x][!C]$. On the other
  hand, $\Delta \Entails \lnot \Subst{!C}{c}{x}$ and hence by
  Generalization $\Delta \Entails \lnot \lexists[x][!C]$. So $\Gamma
  \cup \{\lexists[x][!S] \}$ and $\Delta$ are separable, a
  contradiction.
\end{enumerate}
\end{proof}

\subsection{The Interpolation Theorem}

\begin{thm}[Craig's Interpolation Theorem] % CP-011
\label{CP-011}
If $\Entails !A \lif !B$, then there is a sentence $!C$ such that
$\Entails !A \lif !C$ and $\Entails !C \lif !B$, and every
constant, function, and predicate (other than $\eq$) in
$!C$ occurs both in $!A$ and~$!B$. The sentence $!C$ is called an
\emph{interpolant} of $!A$ and~$!B$.
\end{thm}

\begin{proof}
Suppose $\Lang{L}_1$ is the language of $!A$ and $\Lang{L}_2$ is the
language of $!B$. Let $\Lang{L}_0 = \Lang{L}_1 \cap \Lang{L}_2$. For
each $i \in \{0, 1, 2 \}$, let $\Lang{L}'_i$ be obtained from
$\Lang{L}_i$ by adding the infinitely many new constants $\Obj c_0,
\Obj c_1, \Obj c_2, \dots$.

If $!A$ is unsatisfiable, $\lexists[x][\eq/[x][x]]$ is an
interpolant. If $\lnot !B$ is unsatisfiable (and hence $!B$ is valid),
$\lexists[x][\eq[x][x]]$ is an interpolant. So we may assume also that
both $!A$ and $\lnot !B$ are satisfiable.

In order to prove the contrapositive of the Interpolation Theorem,
assume that there is no interpolant for $!A$ and $!B$. In other words,
assume that $\{!A\}$ and $\{\lnot !B\}$ are inseparable in
$\Lang{L}_0$.

Our goal is to extend the pair $(\{ !A \}, \{\lnot!B\})$ to a
maximally inseparable pair $(\Gamma^*, \Delta^*)$.  Let $!A_0$,
$!A_1$, $!A_2$, \dots enumerate the sentences of $\Lang{L}_1$, and
$!B_0$, $!B_1$, $!B_2$, \dots enumerate the sentences
of~$\Lang{L}_2$. We define two increasing sequences of sets of
sentences $(\Gamma_n, \Delta_n)$, for $n \ge 0$, as follows. Put
$\Gamma_0 = \{ !A\}$ and $\Delta_0 = \{\lnot !B \}$. Assuming
$(\Gamma_n, \Delta_n)$ are already defined, define $\Gamma_{n+1}$ and
$\Delta_{n+1}$ by:
\begin{enumerate}
\item If $\Gamma_n \cup \{!A_n \}$ and $\Delta_n$ are inseparable in
  $\Lang{L}'_0$, put $!A_n$ in $\Gamma_{n+1}$. Moreover, if $!A_n$ is
  an existential formula $\lexists[x][!S]$ then pick a new
  constant $c$ not occurring in $\Gamma_n$, $\Delta_n$, $!A_n$ or
  $!B_n$, and put $\Subst{!S}{c}{x}$ in $\Gamma_{n+1}$.
\item If $\Gamma_{n+1}$ and $\Delta_n \cup \{!B_n \}$ are inseparable
  in $\Lang{L}'_0$, put $!B_n$ in $\Delta_{n+1}$. Moreover, if $!B_n$
  is an existential formula $\lexists[x][!S]$, then pick a new
  constant $c$ not occurring in $\Gamma_{n+1}$, $\Delta_n$, $!A_n$
  or $!B_n$, and put $\Subst{!S}{c}{x}$ in $\Delta_{n+1}$.
\end{enumerate}
Finally, define:
\begin{align*}
  \Gamma^* & = \bigcup_{n\ge 0} \Gamma_n, &
  \Delta^* & = \bigcup_{n\ge 0} \Delta_n.
\end{align*}
By simultaneous induction on $n$ we can now prove:
\begin{enumerate}
\item\label{part-a} $\Gamma_n$ and $\Delta_n$ are inseparable in
  $\Lang{L}'_0$;
\item\label{part-b} $\Gamma_{n+1}$ and $\Delta_n$ are inseparable in
    $\Lang{L}'_0$.
\end{enumerate}
The basis for \ref{part-a} is given by Lemma~\ref{lem:sep1}. For
part \ref{part-b}, we need to distinguish three cases:
\begin{enumerate}
\item If $\Gamma_0 \cup \{!A_0 \}$ and $\Delta_0$ are separable, then
  $\Gamma_1 = \Gamma_0$ and \ref{part-b} is just \ref{part-a};
\item If $\Gamma_1 = \Gamma_0 \cup\{ !A_0\}$, then $\Gamma_1$ and
  $\Delta_0$ are inseparable by construction.
\item It remains to consider the case where $!A_0$ is existential, so
  that $\Gamma_1 = \Gamma_0 \cup \{ \lexists[x][!S], \Subst{!S}{c}{x}
  \}$. By construction, $\Gamma_0 \cup \{ \lexists[x][!S]\}$ and
  $\Delta_0$ are inseparable, so that by Lemma~\ref{lem:sep2} also
  $\Gamma_0 \cup \{ \lexists[x][!S], \Subst{!S}{c}{x} \}$ and
  $\Delta_0$ are inseparable.
\end{enumerate}
This completes the basis of the induction for \ref{part-a} and
\ref{part-b} above. Now for the inductive step. For \ref{part-a}, if
$\Delta_{n+1} = \Delta_n \cup \{ !B_n \}$ then $\Gamma_{n+1}$ and
$\Delta_{n+1}$ are inseparable by construction (even when $!B_n$ is
existential, by Lemma~\ref{lem:sep2}); if $\Delta_{n+1} = \Delta_n$
(because $\Gamma_{n+1}$ and $\Delta_n \cup \{!B_n\}$ are separable),
then we use the induction hypothesis on \ref{part-b}. For the
inductive step for \ref{part-b}, if $\Gamma_{n+2} = \Gamma_{n+1} \cup
\{!A_{n+1} \}$ then $\Gamma_{n+2}$ and $\Delta_{n+1}$ are
inseparable by construction (even when $!A_{n+1}$ is existential,
by Lemma~\ref{lem:sep2}); and if  $\Gamma_{n+2} = \Gamma_{n+1}$ then
we use the inductive case for \ref{part-a} just proved. This
concludes the induction on \ref{part-a} and \ref{part-b}.

It follows that $\Gamma^*$ and $\Delta^*$ are inseparable; if not, by
compactness, there is $n \ge 0$ that separates $\Gamma_n$ and
$\Delta_n$, against \ref{part-a}. In particular, $\Gamma^*$ and
$\Delta^*$ are consistent: for if the former or the latter is
inconsistent, then they are separated by $\lexists[x][\eq/[x][x]]$ or
  $\lforall[x][\eq[x][x]]$, respectively.

We now show that $\Gamma^*$ is maximally consistent in
$\Lang{L}'_1$ and likewise $\Delta^*$ in $\Lang{L}'_2$. For the
former, suppose that $!A_n \notin \Gamma^*$ and $\lnot !A_n
\notin \Gamma^*$, for some $n \ge 0$. If $!A_n \notin \Gamma^*$
then $\Gamma_n \cup \{!A_n \}$ is separable from $\Delta_n$, and
so there is $!C \in \Lang{L}'_0$ such that both:
\begin{align*}
  \Gamma^* & \Entails !A_n \lif !C, &
  \Delta^* & \Entails \lnot !C.
\end{align*}
Likewise, if $\lnot !A_n \notin \Gamma^*$, there is $!C' \in
\Lang{L}'_0$ such that both:
\begin{align*}
  \Gamma^* & \Entails \lnot !A_n \lif !C', &
  \Delta^* & \Entails \lnot !C'.
\end{align*}
By propositional logic, $\Gamma^* \Entails !C \lor !C'$ and
$\Delta^* \Entails \lnot (!C \lor !C')$, so $!C \lor
!C'$ separates $\Gamma^*$ and $\Delta^*$. A similar argument
establishes that $\Delta^*$ is maximal.

Finally, we show that $\Gamma^* \cap \Delta^*$ is maximally consistent
in $\Lang{L}'_0$. It is obviously consistent, since it is the
intersection of consistent sets. To show maximality, let $!S \in
\Lang{L}'_0$. Now, $\Gamma^*$ is maximal in $\Lang{L'_1}
\supseteq \Lang{L'_0}$, and similarly $\Delta^*$ is maximal in
$\Lang{L'_2} \supseteq \Lang{L'_0}$. It follows that either
$!S \in \Gamma^*$ or $\lnot !S \in \Gamma^*$, and either
$!S \in \Delta^*$ or $\lnot !S \in \Delta^*$. If $!S \in
\Gamma^*$ and $\lnot !S \in \Delta^*$ then $!S$ would
separate $\Gamma^*$ and $\Delta^*$; and if $\lnot !S \in
\Gamma^*$ and $!S \in \Delta^*$ then $\Gamma^*$ and $\Delta^*$
would be separated by $\lnot !S$. Hence, either $!S \in
\Gamma^* \cap \Delta^*$ or $\lnot !S \in \Gamma^* \cap \Delta^*$,
and $\Gamma^* \cap \Delta^*$ is maximal.

Since $\Gamma^*$ is maximally consistent, it has a model
$\Struct{M}'_1$ whose domain $\Domain{M'_1}$ comprises all and
only the elements $\Assign{c}{M'_1}$ interpreting the
constants---just like in the proof of the Completeness Theorem
(Theorem~\ref{CP-002}). Similarly, $\Delta^*$ has a
model $\Struct{M}'_2$ whose domain $\Domain{M'_2}$ is given by the
interpretations $\Assign{c}{M'_2}$ of the constants.

Let $\Struct{M_1}$ be obtained from $\Struct{M'_1}$ by dropping
interpretations for constants, functions, and predicates in
$\Lang{L'_1} \setminus \Lang{L'_0}$, and similarly for
$\Struct{M_2}$. Then the map $h \colon M_1 \to M_2$ defined by
$h(\Assign{c}{M'_1}) = \Assign{c}{M'_2}$ is an
isomorphism in $\Lang{L}'_0$, because $\Gamma^* \cap \Delta^*$ is
maximally consistent in $\Lang{L}'_0$, as shown. This follows
because any $\Lang{L}'_0$-sentence either belongs to both
$\Gamma^*$ and $\Delta^*$, or to neither: so $\Assign{c}{M'_1} \in
\Assign{P}{M'_1}$ if and only if $\Atom{P}{c} \in \Gamma^*$ if and only if
$\Atom{P}{c} \in \Delta^*$ if and only if $\Assign{c}{M'_2} \in
\Assign{P}{M'_2}$. The other conditions satisfied by isomorphisms
can be established similarly.

Let us now define a model $\Struct{M}$ for the language
$\Lang{L_1} \cup \Lang{L_2}$ as follows:
\begin{enumerate}
\item The domain $\Domain{M}$ is just $\Domain{M_2}$, i.e., the
  set of all elements $\Assign{c}{M'_2}$;
\item If a predicate~$P$ is in $\Lang{L_2} \setminus
  \Lang{L_1}$ then $\Assign{P}{M} = \Assign{P}{M'_2}$;
\item If a predicate $P$ is in $\Lang{L}_1\setminus \Lang{L}_2$ then
  $\Assign{P}{M} = h(\Assign{P}{M'_2})$, i.e.,
  $\tuple{\Assign{c_1}{M'_2}, \dots, \Assign{c_n}{M'_2}} \in
  \Assign{P}{M}$ if and only if $\tuple{\Assign{c_1}{M'_1}, \dots,
  \Assign{c_n}{M'_1}} \in \Assign{P}{M'_1}$.
\item If a predicate $P$ is in $\Lang{L}_0$ then $\Assign{P}{M} =
  \Assign{P}{M'_2} = h(\Assign{P}{M'_1})$.
\item Functions of $\Lang{L}_1 \cup \Lang{L}_2$, including
  constants, are handled similarly.
\end{enumerate}

Finally, one shows by induction on formulas that $\Struct{M}$ agrees
with $\Struct{M'_1}$ on all formulas of $\Lang{L'_1}$ and with
$\Struct{M'_2}$ on all formulas of $\Lang{L'_2}$. In particular,
$\Struct{M} \Entails \Gamma^* \cup \Delta^*$, whence $\Struct{M}
\Entails !A$ and $\Struct{M} \Entails \lnot!B$, and
$\not\Entails !A \lif !B$. This concludes the proof of
Craig's Interpolation Theorem.
\end{proof}


%% ===================================================================
%% META.10: Beth Definability (CP-012)
%% Sources: mod/int/def (KEEP)
%% ===================================================================

\section{Beth's Definability Theorem} \label{META.10}

One important application of the interpolation theorem is Beth's
definability theorem.  To define an $n$-place relation~$R$ we can give
a formula~$!C$ with $n$ free variables which does not
involve~$R$. This would be an \emph{explicit} definition of~$R$ in
terms of~$!C$.  We can then say also that a theory~$\Sigma(P)$ in a
language containing the $n$-place predicate~$P$ explicitly
defines~$P$ if it contains (or at least entails) a formalized explicit
definition, i.e.,
\[
\Sigma(P) \Entails \lforall[x_1][\dots
  \lforall[x_n][(\Atom{P}{x_1,\dots, x_n} \liff !C(x_1, \dots,
    x_n))]].
\]
But an explicit definition is only one way of defining---in the sense
of determining completely---a relation.  A theory may also be such
that the interpretation of~$P$ is fixed by the interpretation of the
rest of the language in any model.  The definability theorem
states that whenever a theory fixes the interpretation of~$P$ in this
way---whenever it \emph{implicitly defines}~$P$---then it also
explicitly defines it.

\begin{defn}
Suppose $\Lang{L}$ is a language not containing the
predicate~$P$.  A set $\Sigma(P)$ of sentences of $\Lang{L}
\cup \{P\}$ \emph{explicitly defines}~$P$ if and only if there is
a formula~$!C(x_1, \dots, x_n)$ of $\Lang{L}$ such that
\[
\Sigma(P) \Entails \lforall[x_1][\dots
  \lforall[x_n][(\Atom{P}{x_1,\dots, x_n} \liff !C(x_1, \dots,
    x_n))]].
\]
\end{defn}

\begin{defn}
Suppose $\Lang{L}$ is a language not containing the
predicates~$P$ and~$P'$.  A set $\Sigma(P)$ of sentences of
$\Lang{L} \cup \{P\}$ \emph{implicitly defines} $P$ if and only if
\[
\Sigma(P) \cup \Sigma(P') \Entails \lforall[x_1][\dots
  \lforall[x_n][(\Atom{P}{x_1,\dots, x_n} \liff \Atom{P'}{x_1,\dots,
      x_n})]],
\]
where $\Sigma(P')$ is the result of uniformly replacing $P$ with $P'$
in $\Sigma(P)$.
\end{defn}

In other words, for any model $\Struct{M}$ and $R, R' \subseteq
\Domain{M}^n$, if both $\Expan{M}{R} \Entails \Sigma(P)$ and
$\Expan{M}{R'} \Entails \Sigma(P')$, then $R=R'$; where
$\Expan{M}{R}$ is the structure~$\Struct{M'}$ for the
expansion of $\Lang{L}$ to $\Lang{L} \cup \{P\}$ such that
$\Assign{P}{M'} = R$, and similarly for $\Expan{M}{R'}$.

\begin{thm}[Beth Definability Theorem] % CP-012
\label{CP-012}
A set $\Sigma(P)$ of $\Lang{L}
  \cup\{P\}$-formulas implicitly defines $P$ if and only $\Sigma(P)$
  explicitly defines $P$.
\end{thm}

\begin{proof}
If $\Sigma(P)$ explicitly defines $P$ then both
\begin{align*}
  \Sigma(P) & \Entails & \lforall[x_1][\dots \lforall[x_n]
    [(\Atom{P}{x_1,\dots, x_n} \liff !C(x_1,\dots,x_n))]]\\
  \Sigma(P') & \Entails & \lforall[x_1][\dots \lforall[x_n]
    [(\Atom{P'}{x_1,\dots, x_n} \liff !C(x_1,\dots,x_n))]]
\end{align*}
and the conclusion follows. For the converse: assume that $\Sigma(P)$
implicitly defines $P$. First, we add constants $c_1$, \dots,~$c_n$ to
$\Lang{L}$. Then
\[
\Sigma(P) \cup \Sigma(P') \Entails
\Atom{P}{c_1, \dots, c_n} \to  \Atom{P'}{c_1, \dots, c_n}.
\]
By compactness (Theorem~\ref{CP-003}), there are finite sets $\Delta_0 \subseteq \Sigma(P)$
and $\Delta_1 \subseteq \Sigma(P')$ such that
\[
\Delta_0 \cup \Delta_1 \Entails
\Atom{P}{c_1, \dots, c_n} \to \Atom{P'}{c_1, \dots, c_n}.
\]
Let $!D(P)$ be the conjunction of all sentences $!A(P)$ such that
either $!A(P) \in \Delta_0$ or $!A(P') \in \Delta_1$ and let $!D(P')$
be the conjunction of all sentences $!A(P')$ such that either
$!A(P) \in \Delta_0$ or $!A(P') \in \Delta_1$. Then $!D(P) \land
!D(P') \Entails \Atom{P}{c_1, \dots, c_n} \to P'c_1\dots c_n$. We can
re-arrange this so that each predicate occurs on one side of
$\Entails$:
\[
!D(P) \land \Atom{P}{c_1, \dots, c_n} \Entails
!D(P') \to \Atom{P'}{c_1, \dots, c_n}.
\]
By Craig's Interpolation Theorem (Theorem~\ref{CP-011}) there is a sentence $!C(c_1,\dots, c_n)$
not containing $P$ or $P'$ such that:
\begin{align*}
  !D(P) \land \Atom{P}{c_1, \dots, c_n} & \Entails !C(c_1,\dots, c_n); \\
  !C(c_1,\dots, c_n) & \Entails !D(P') \to \Atom{P'}{c_1, \dots, c_n}.
\end{align*}
From the former of these two entailments we have: $!D(P) \Entails
\Atom{P}{c_1,\dots, c_n} \lif !C(c_1,\dots, c_n)$. And from the
latter, since an $\Lang{L} \cup \{P\}$-model $\Expan{M}{R}
\Entails !A(P)$ if and only if the corresponding $\Lang{L} \cup
\{P'\}$-model $\Expan{M}{R} \models !A(P')$, we have
$!C(c_1,\dots, c_n) \Entails !D(P) \lif \Atom{P}{c_1,\dots, c_n}$,
from which:
\[
!D(P) \Entails !C(c_1,\dots,c_n) \to \Atom{P}{c_1,\dots, c_n}.
\]
Putting the two together, $!D(P) \Entails \Atom{P}{c_1,\dots, c_n}
\liff !C(c_1, \dots, c_n)$, and by monotonicity and generalization also
\[
\Sigma(P) \Entails
\lforall[x_1][\dots\lforall[x_n][(\Atom{P}{x_1,\dots, x_n} \liff
    !C(x_1,\dots, x_n))]].
\]
\end{proof}


%% ===================================================================
%% META.11: Lindstrom's Theorem (CP-013)
%% Sources: mod/lin/alg (ABSORB), mod/lin/lsp (ABSORB),
%%          mod/bas/pis (CONDENSE), mod/lin/prf (KEEP)
%% ===================================================================

\section{Lindstr\"om's Theorem} \label{META.11}

Lindstr\"om's theorem characterizes first-order logic as the maximal
logic---in a precisely defined sense---for which both the Compactness
Theorem and the Downward L\"owenheim--Skolem Theorem hold.  To state
the theorem we need the notions of abstract logic, partial
isomorphism, and the back-and-forth characterization of
$n$-equivalence.  Throughout this section we restrict to purely
relational languages (containing only predicates and individual
constants, no functions).

\subsection{Abstract Logics}

\begin{defn}
An \emph{abstract logic} is a pair $\tuple{L, \models_L}$, where $L$
is a function that assigns to each language~$\Lang{L}$ a set
$L(\Lang{L})$ of sentences, and $\models_L$ is a relation between
structures for the language~$\Lang{L}$ and elements of
$L(\Lang{L})$. In particular, $\tuple{F, \models}$ is ordinary
first-order logic, i.e., $F$ is the function assigning to the
language~$\Lang{L}$ the set of first-order sentences built from
the constants in $\Lang{L}$, and $\models$ is the satisfaction relation
of first-order logic.
\end{defn}

\begin{defn}
Let $\Mod(L){!E}$ denote the class $\Setabs{\Struct{M}}{\Struct{M}
  \models_L !E}$. If the language needs to be made explicit, we
write $\Mod[L](L){!E}$. Two structures $\Struct{M}$ and
$\Struct{N}$ for $\Lang{L}$ are \emph{elementarily equivalent in}
$\tuple{L, \models_L}$, written $\Struct{M} \elemequiv[L] \Struct{N}$, if
the same sentences from $L(\Lang{L})$ are true in each.
\end{defn}

\begin{defn}
An abstract logic $\tuple{L,\models_L}$ for the language $\Lang{L}$
is \emph{normal} if it satisfies the following properties:
\begin{enumerate}
\item (\emph{$L$-Monotonicity}) For languages $\Lang{L}$ and
  $\Lang{L'}$, if $\Lang{L} \subseteq \Lang{L'}$, then
  $L(\Lang{L}) \subseteq L(\Lang{L'})$.
\item (\emph{Expansion Property}) For each $!E \in L(\Lang{L})$
  there is a \emph{finite} subset $\Lang{L'}$ of $\Lang{L}$ such that
  the relation $\Struct{M} \models_L !E$ depends only on the
  reduct of $\Struct{M}$ to $\Lang{L'}$.
\item (\emph{Isomorphism Property}) If $\Struct{M} \models_L !E$
  and $\Struct{M} \simeq \Struct{N}$ then also $\Struct{N} \models_L
  !E$.
\item (\emph{Renaming Property}) The relation $\models_L$ is preserved
  under renaming of non-logical symbols.
\item (\emph{Boolean Property}) $\tuple{L, \models_L}$ is closed
  under the Boolean connectives: for each $!E \in L(\Lang{L})$ there
  is $!F$ with $\Mod(L){!F} = \Mod(L){!E}^c$, and for each pair
  $!E$, $!F$ there is $!G$ with $\Mod(L){!G} = \Mod(L){!E} \cap
  \Mod(L){!F}$.
\item (\emph{Quantifier Property}) For each constant $c$ in $\Lang{L}$
  and $!E \in L(\Lang{L})$ there is $!F \in L(\Lang{L} \setminus \{c\})$
  such that $\Struct{M} \models_L !F$ iff $\Expan{M}{a} \models_L !E$
  for some $a \in \Domain{M}$.
\item (\emph{Relativization Property}) Given a sentence $!E \in
  L(\Lang{L})$ and symbols $R$, $c_1$, \dots, $c_n$ not in $\Lang{L}$,
  there is a sentence $!F$ (the \emph{relativization} of $!E$) such
  that satisfaction of~$!F$ in an expansion of~$\Struct{M}$
  corresponds to satisfaction of~$!E$ in the substructure picked out by~$R$.
\end{enumerate}
First-order logic $\tuple{F, \models}$ is normal.  Moreover, if
$\tuple{L, \models_L}$ is normal, then $\tuple{F, \models} \leq
\tuple{L, \models_L}$.
\end{defn}

\begin{defn}
Given two abstract logics $\tuple{L_1, \models_{L_1}}$ and
$\tuple{L_2, \models_{L_2}}$ we say that the latter is \emph{at least
  as expressive} as the former, written $\tuple{L_1, \models_{L_1}}
\leq \tuple{L_2, \models_{L_2}}$, if for each language $\Lang{L}$
and sentence $!E \in L_1(\Lang{L})$ there is a sentence $!F
\in L_2(\Lang{L})$ such that $\Mod[L](L_1){!E} =
\Mod[L](L_2){!F}$. The logics are \emph{equivalent} if the inequality
holds in both directions.
\end{defn}

\subsection{Compactness and L\"owenheim--Skolem Properties}

\begin{defn}
An abstract logic $\tuple{L, \models_L}$ has the \emph{Compactness
  Property} if each set $\Gamma$ of $L(\Lang{L})$-sentences is
satisfiable whenever each finite $\Gamma_0 \subseteq \Gamma$ is
satisfiable.
\end{defn}

\begin{defn}
$\tuple{L, \models_L}$ has the \emph{Downward L\"owenheim--Skolem
  Property} if any satisfiable $\Gamma$ has an enumerable model.
\end{defn}

\subsection{Partial Isomorphisms}

\begin{defn}
  Given two structures $\Struct{M}$ and $\Struct{N}$, a
  \emph{partial isomorphism} from $\Struct{M}$ to $\Struct{N}$ is a
  finite partial function $p$ taking arguments in $\Domain M$ and returning
  values in $\Domain N$, which is injective and preserves the
  interpretations of all constants, predicates, and functions on
  its domain.
\end{defn}

\begin{defn}\label{defn:partialisom}
  Two structures $\Struct{M}$ and $\Struct{N}$ are
  \emph{partially isomorphic}, written $\Struct{M} \iso[p]
  \Struct{N}$, if and only if there is a non-empty set $I$
  of partial isomorphisms between $\Struct{M}$ and $\Struct{N}$
  satisfying the \emph{back-and-forth} property:
  \begin{enumerate}
  \item (\emph{Forth}) For every $p \in I$ and $a \in \Domain M$
    there is $q \in I$ such that $p \subseteq q$ and $a$ is
    in the domain of $q$;
  \item (\emph{Back}) For every $p \in I$ and $b \in \Domain N$
    there is $q \in I$ such that $p \subseteq q$ and $b$ is
    in the range of $q$.
  \end{enumerate}
\end{defn}

\begin{thm}\label{thm:p-isom1}
  If $\Struct{M} \iso[p] \Struct{N}$ and $\Struct{M}$ and
  $\Struct{N}$ are enumerable, then $\Struct{M} \iso
  \Struct{N}$.
\end{thm}

\begin{proof}[Proof sketch]
Enumerate $\Domain{M} = \{a_0, a_1, \ldots\}$ and $\Domain{N} =
\{b_0, b_1, \ldots\}$.  Starting from an arbitrary $p_0 \in I$,
alternately apply the Forth property (to include $a_r$ in the domain)
and the Back property (to include $b_r$ in the range), building an
increasing chain $p_0 \subseteq p_1 \subseteq \cdots$.  The union
$p = \bigcup_n p_n$ is an isomorphism.
\end{proof}

\begin{thm}\label{thm:p-isom2}
  Suppose $\Struct{M}$ and $\Struct{N}$ are structures for a
  purely relational language. Then if
  $\Struct{M} \iso[p] \Struct{N}$, also $\Struct{M} \elemequiv
  \Struct{N}$.
\end{thm}

\begin{proof}[Proof sketch]
By induction on formulas, one shows that if $p$ maps each $a_i$ to
$b_i$, then $\Sat{M}{!A}[s_1]$ iff $\Sat{N}{!A}[s_2]$ whenever
$s_1(x_i)=a_i$ and $s_2(x_i)=b_i$.  The base case uses the
isomorphism conditions on $p$; the quantifier step uses the
back-and-forth property.  The case $n=0$ gives
$\Struct{M} \elemequiv \Struct{N}$.
\end{proof}

\subsection{Quantifier Rank and $n$-Equivalence}

\begin{defn}
  For any formula~$!A$, the \emph{quantifier rank} of $!A$, denoted
  by $\QuantRank{!A} \in \Nat$, is recursively defined as
  the highest number of nested quantifiers in $!A$.  Two
  structures $\Struct{M}$ and $\Struct{N}$ are \emph{$n$-equivalent},
  written $\Struct{M} \elemequiv[n] \Struct{N}$, if they agree on all
  sentences of quantifier rank less than or equal to~$n$.
\end{defn}

\begin{prop}\label{prop:qr-finite}
  Let $\Lang{L}$ be a finite purely relational language. Then for each $n \in \Nat$ there are
  only finitely many first-order sentences in $\Lang{L}$
  that have quantifier rank no greater than $n$, up to
  logical equivalence.
\end{prop}

\begin{proof}
  By induction on $n$.
\end{proof}

\begin{defn}
  Given structures $\Struct{M}$ and $\Struct{N}$, we define
  relations $I_n \subseteq \Domain M^{<\omega} \times \Domain N^{<\omega}$ between
  sequences of equal length, by recursion on $n$ as follows:
   \begin{enumerate}
   \item $I_0(\mathbf{a},\mathbf{b})$ iff $\mathbf{a}$ and
     $\mathbf{b}$ satisfy the same atomic formulas in $\Struct{M}$
     and $\Struct{N}$.
   \item $I_{n+1} (\mathbf{a},\mathbf{b})$ iff for every
     $a\in \Domain M$ there is a $b\in \Domain N$ such that $I_n
     (\mathbf{a}a,\mathbf{b}b)$, and vice-versa.
   \end{enumerate}
\end{defn}

\begin{defn}
  Write $\Struct{M} \approx_n \Struct{N}$ if
  $I_n(\emptyseq,\emptyseq)$ holds of $\Struct{M}$ and
  $\Struct{N}$ (where $\emptyseq$ is the empty sequence).
\end{defn}

\begin{thm}\label{thm:b-n-f}
  Let $\Lang{L}$ be a purely relational language. Then $I_n
  (\mathbf{a},\mathbf{b})$ implies that for every $!A$ such that
  $\QuantRank{!A} \le n$, we have $\Sat{M}{!A}[\mathbf{a}]$ if and
  only if $\Sat{N}{!A}[\mathbf{b}]$. Moreover, if $\Lang{L}$ is finite, the converse also holds.
\end{thm}

\begin{proof}[Proof sketch]
The forward direction proceeds by induction on~$!A$.  For the
converse, one proceeds by induction on~$n$.  The key step uses
Proposition~\ref{prop:qr-finite}: given $a \in \Domain{M}$, let
$!T^a_n$ be the finite set of formulas of rank $\le n$ satisfied by
$\mathbf{a}a$ in $\Struct{M}$.  Then $\mathbf{a}$ satisfies
$\lexists[x][!T^a_n]$ (rank $\le n+1$), so by hypothesis $\mathbf{b}$
does too in $\Struct{N}$, yielding the required~$b$.
\end{proof}

\begin{cor}\label{cor:b-n-f}
  If $\Struct{M}$ and $\Struct{N}$ are purely relational structures
  in a finite language, then $\Struct{M} \approx_n\Struct{N}$ if and
  only if $\Struct{M} \elemequiv[n] \Struct{N}$. In particular
  $\Struct{M} \elemequiv \Struct{N}$ if and only if for each $n$,
  $\Struct{M} \approx_n \Struct{N}$.
\end{cor}

\subsection{Partially Isomorphic Structures in Abstract Logics}

The notion of partial isomorphism is purely algebraic and hence applies
to abstract logics.  The following key theorem shows that if
$\tuple{L,\models_L}$ has the L\"owenheim--Skolem property, partially
isomorphic structures are $L$-equivalent.

\begin{thm}\label{thm:abstract-p-isom}
Suppose $\tuple{L, \models_L}$ is a normal logic with the
L\"owenheim--Skolem property. Then any two structures that are
partially isomorphic are elementarily equivalent in $\tuple{L,
  \models_L}$.
\end{thm}

\begin{proof}[Proof sketch]
Suppose $\Struct{M} \simeq_p \Struct{N}$ but $\Struct{M} \models_L
!E$ while $\Struct{N} \not\models_L !E$.  Using the Isomorphism and
Expansion Properties, assume $\Domain{M}$ and $\Domain{N}$ are
disjoint and $!E \in L(\Lang{L})$ for a finite language.  Encode the
partial isomorphism $I$ and the extended structures $\Struct{M}^*$,
$\Struct{N}^*$ (with their finite-sequence domains and concatenation
predicates) into a single structure~$\Struct{M}$.  The Relativization
Property yields a first-order sentence~$!D_1$ true in~$\Struct{M}$
expressing that $\Struct{M} \models_L !E$ and $\Struct{N} \not\models_L
!E$, and a sentence~$!D_2$ expressing that $\Struct{M} \simeq_p
\Struct{N}$ via~$I$.  By the L\"owenheim--Skolem Property,
$!D_1 \land !D_2$ has an enumerable model containing enumerable
partially isomorphic substructures.  But enumerable partially
isomorphic structures are isomorphic (Theorem~\ref{thm:p-isom1}),
contradicting the Isomorphism Property.
\end{proof}

\subsection{Lindstr\"om's Theorem}

\begin{lem}
\label{lem:lindstrom}
Suppose $!E \in L(\Lang{L})$, with $\Lang{L}$ finite, and assume
also that there is an $n \in \Nat$ such that for any two
structures $\Struct{M}$ and~$\Struct{N}$, if $\Struct{M} \equiv_n
\Struct{N}$ and $\Struct{M} \models_L !E$ then also $\Struct{N}
\models_L !E$. Then $!E$ is equivalent to a first-order
sentence, i.e., there is a first-order $!D$ such that
$\Mod(L){!E} = \Mod(L){!D}$.
\end{lem}

\begin{proof}
Let $n$ be such that any two $n$-equivalent structures
$\Struct{M}$ and $\Struct{N}$ agree on the value assigned to~$!E$.
Recall Proposition~\ref{prop:qr-finite}: there are only finitely many
first-order sentences in a finite language that have
quantifier rank no greater than~$n$, up to logical equivalence. Now,
for each fixed structure~$\Struct{M}$ let $!D_{\Struct{M}}$ be the
conjunction of all first-order sentences~$!E$ true in~$\Struct{M}$
with $\QuantRank{!E} \le n$ (this conjunction is finite), so that
$\Struct{N} \models !D_{\Struct{M}}$ if and only if $\Struct{N}
\equiv_n \Struct{M}$. Then put $!D = \textstyle\bigvee
\Setabs{!D_{\Struct{M}}}{\Struct{M} \models_L !E}$; this disjunction
is also finite (up to logical equivalence).

The conclusion $\Mod(L){!E} = \Mod(L){!D}$ follows. In fact, if
$\Struct{N} \models_L !D$ then for some $\Struct{M} \models_L
!E$ we have $\Struct{N} \models !D_{\Struct{M}}$, whence also
$\Struct{N} \models_L !E$ (by the hypothesis of the
lemma). Conversely, if $\Struct{N} \models_L !E$ then
$!D_\Struct{N}$ is a disjunct in $!D$, and since $\Struct{N}
\models !D_\Struct{N}$, also $\Struct{N} \models_L !D$.
\end{proof}

\begin{thm}[Lindstr\"om's Theorem] % CP-013
  \label{CP-013} Suppose $\tuple{L, \models_L}$ has the
  Compactness and the L\"owenheim--Skolem Properties. Then
  $\tuple{L, \models_L} \le \tuple{F, \models}$ (so
  $\tuple{L, \models_L}$ is equivalent to first-order logic).
\end{thm}

\begin{proof}
By Lemma~\ref{lem:lindstrom}, it suffices to show that for any $!E
\in L(\Lang{L})$, with $\Lang{L}$ finite, there is $n \in \Nat$
such that for any two structures $\Struct{M}$ and~$\Struct{N}$: if
$\Struct{M} \equiv_n \Struct{N}$ then $\Struct{M}$ and $\Struct{N}$
agree on~$!E$. For then $!E$ is equivalent to a first-order
sentence, from which $\tuple{L, \models_L} \le \tuple{F, \models}$
follows. Since we are working in a finite, purely relational
language, by Theorem~\ref{thm:b-n-f} we can replace the statement
that $\Struct{M} \equiv_n \Struct{N}$ by the corresponding algebraic
statement that $I_n(\emptyset,\emptyset)$.

Given $!E$, suppose towards a contradiction that for each $n$ there
are structures $\Struct{M}_n$ and $\Struct{N}_n$ such that
$I_n(\emptyset, \emptyset)$, but (say) $\Struct{M}_n \models_L !E$
whereas $\Struct{N}_n \not\models_L !E$. By the Isomorphism Property
we can assume that all the $\Struct{M}_n$'s interpret the constants of
the language by the same objects; furthermore, since there are only
finitely many atomic sentences in the language, we may also assume
that they satisfy the same atomic sentences (we can take a
subsequence of the $\Struct{M}$'s otherwise). Let $\Struct{M}$ be the
union of all the $\Struct{M}_n$'s, i.e., the unique minimal
structure having each $\Struct{M}_n$ as a substructure.  As in the
proof of Theorem~\ref{thm:abstract-p-isom}, let $\Struct{M}^*$ be the
extension of $\Struct{M}$ with domain $\Domain{M} \cup
\Domain{M}^{<\omega}$, in the expanded language comprising the
concatenation predicates $P$ and~$Q$.

Similarly, define $\Struct{N}_n$, $\Struct{N}$ and $\Struct{N}^*$. Now
let $\Struct{M}$ be the structure whose domain comprises the
domains of $\Struct{M}^*$ and $\Struct{N}^*$ as well as the natural
numbers~$\Nat$ along with their natural ordering~$\le$, in the
language with extra predicates representing the domains
$\Domain{M}$, $\Domain{N}$, $\Domain{M}^{<\omega}$ and
$\Domain{N}^{<\omega}$ as well as predicates coding the domains of
$\Struct{M}_n$ and $\Struct{N}_n$ in the sense that:
\begin{align*}
  \Domain{M_n} & = \Setabs{a \in \Domain{M}}{R(a, n)}; &
  \Domain{N_n} & = \Setabs{a \in \Domain{N}}{S(a,n)}; \\
  \Domain{M}^{<\omega}_n & = \Setabs{a \in \Domain{M}^{<\omega}}{R(a,n)}; &
  \Domain{N}^{<\omega}_n & = \Setabs{a \in \Domain{N}^{<\omega}}{S(a,n)}.
\end{align*}
The structure~$\Struct{M}$ also has a ternary relation $J$ such
that $J(n, \mathbf{a}, \mathbf{b})$ holds if and only if
$I_n(\mathbf{a}, \mathbf{b})$.

Now there is a sentence~$!D$ in the language~$\Lang{L}$ augmented
by $R$, $S$, $J$, etc., saying that $\le$ is a discrete linear ordering
with first but no last element and such that $\Struct{M}_n \models
!E$, $\Struct{N}_n \not\models !E$, and for each $n$ in the
ordering, $J(n, \mathbf{a}, \mathbf{b})$ holds if and only if
$I_n(\mathbf{a}, \mathbf{b})$.

Using the Compactness Property, we can find a model $\Struct{M}^*$ of
$!D$ in which the ordering contains a non-standard element~$n^*$. In
particular then $\Struct{M^*}$ will contain substructures
$\Struct{M_{n^*}}$ and $\Struct{N_{n^*}}$ such that $\Struct{M_{n^*}}
\models_L !E$ and $\Struct{N_{n^*}} \not\models_L !E$. But now we can
define a set $\mathcal{I}$ of pairs of $k$-tuples from
$\Domain{M_{n^*}}$ and $\Domain{N_{n^*}}$ by putting
$\tuple{\mathbf{a}, \mathbf{b}} \in \mathcal{I}$ if and only if
$J(n^*-k, \mathbf{a}, \mathbf{b})$, where $k$ is the length of
$\mathbf{a}$ and $\mathbf{b}$. Since $n^*$ is non-standard, for each
standard $k$ we have that $n^* - k >0$, and the set $\mathcal{I}$
witnesses the fact that $\Struct{M_{n^*}} \simeq_p
\Struct{N_{n^*}}$. But by Theorem~\ref{thm:abstract-p-isom},
$\Struct{M_{n^*}}$ is $L$-equivalent to $\Struct{N_{n^*}}$, a
contradiction.
\end{proof}


%% ===================================================================
%% META.12: Equivalence of Proof Systems (THM-DED002)
%% Sources: NEW-CONTENT
%% ===================================================================

\section{Equivalence of Proof Systems} \label{META.12}

We have introduced four architectures for deriving theorems of
classical first-order logic: axiomatic (Hilbert-style) deduction
(see DEF-DED005, \S\ref{DED.2}), natural deduction
(see DEF-DED006, \S\ref{DED.3}), the sequent calculus
(see DEF-DED007, \S\ref{DED.4}), and analytic tableaux
(see DEF-DED008, \S\ref{DED.5}).  Despite their very different
structures, all four systems derive exactly the same formulas.

\begin{thm}[Equivalence of Proof Systems] % THM-DED002
\label{THM-DED002}
For classical first-order logic, the following are equivalent for any
set of sentences~$\Gamma$ and sentence~$!A$:
\begin{enumerate}
\item $\Gamma \Proves_H !A$ \quad (derivable in the Hilbert calculus);
\item $\Gamma \Proves_{ND} !A$ \quad (derivable in natural deduction);
\item The sequent $\Gamma \Sequent !A$ is derivable in the sequent
  calculus;
\item The tableau for $\Gamma \cup \{\lnot !A\}$ closes.
\end{enumerate}
All four proof system architectures derive exactly the same formulas.
\end{thm}

\begin{proof}[Proof sketch]
One establishes a cycle of mutual simulations.

\textbf{Hilbert $\Rightarrow$ Natural Deduction.}  Every axiom schema
of the Hilbert calculus is derivable in natural deduction (using
introduction rules to construct proofs of the corresponding tautological
schemas).  Modus ponens corresponds to an application of
$\lif$-elimination.  Hence any Hilbert derivation can be transformed
step-by-step into a natural deduction derivation.

\textbf{Natural Deduction $\Rightarrow$ Sequent Calculus.}  Each
introduction rule of natural deduction corresponds to a right rule of
the sequent calculus, and each elimination rule corresponds to a left
rule followed by a cut.  Discharged assumptions in natural deduction
become formulas on the left side of the sequent.  The translation
proceeds by induction on the structure of the natural deduction
derivation tree.

\textbf{Sequent Calculus $\Rightarrow$ Tableaux.}  A sequent
derivation can be read ``upside down'' as a closed tableau.  An initial
sequent $!A \Sequent !A$ corresponds to a branch containing both $!A$
and $\lnot !A$, and hence closed.  Left and right rules of the sequent
calculus correspond to the tableau expansion rules for signed formulas.
Branching in the sequent calculus (e.g., $\lor$-right, $\land$-left)
corresponds to branching in the tableau.

\textbf{Tableaux $\Rightarrow$ Hilbert.}  A closed tableau for $\Gamma
\cup \{\lnot !A\}$ witnesses the unsatisfiability of $\Gamma \cup
\{\lnot !A\}$.  By soundness of tableaux (see \S\ref{DED.5}) we have
$\Gamma \Entails !A$, and by the Completeness Theorem
(Theorem~\ref{CP-002}), $\Gamma \Proves_H !A$.

Alternatively, each direction can be verified by a direct syntactic
translation.  The indirect route via soundness and completeness gives
the result with less effort: since each system is sound and complete
with respect to the same semantics, they derive the same formulas.
\end{proof}

The indirect argument deserves emphasis.  By the Soundness Theorem
(Theorem~\ref{CP-001}), for each system $S$ we have: if $\Gamma
\Proves_S !A$ then $\Gamma \Entails !A$.  By the Completeness Theorem
(Theorem~\ref{CP-002}), for each system $S$: if $\Gamma \Entails !A$
then $\Gamma \Proves_S !A$.  Chaining these two facts for any pair of
systems $S_1$ and $S_2$ yields: $\Gamma \Proves_{S_1} !A$ iff $\Gamma
\Proves_{S_2} !A$.  Thus the equivalence is an immediate corollary of
soundness and completeness, and does not require an explicit syntactic
simulation---though such simulations are of independent interest for
understanding the computational relationships between proof systems.
  % CH-META: Metatheory (Composition Patterns)
\chapter{Formal Set Theory} \label{ch:set}

%% ===================================================================
%% SET.1: The Language of Set Theory
%% Sources: sth/z/sep (partial, language intro), FORMALIZE from domain spec
%% ===================================================================

\section{The Language of Set Theory} \label{SET.1}

In Chapter~\ref{ch:bst} we developed set-theoretic foundations
informally, treating sets as intuitively given collections of objects.
We now turn to \emph{formal} set theory: a first-order theory in which
every concept is defined from a single binary relation symbol, and
every existence claim is justified by explicit axioms.

\begin{defn}[Set (Formal)] % PRIM-SET001
\label{PRIM-SET001}
In ZFC, the variables of $\Lang{L}_\in$ range over \emph{sets}.
Unlike the naive sets of Chapter~\ref{ch:bst}, formal sets are objects
within a first-order theory: they exist only to the extent that the
ZFC axioms guarantee their existence.
\end{defn}

The formal language of set theory is remarkably austere:

\begin{defn}[Membership] % PRIM-SET002
\label{PRIM-SET002}
The language $\Lang{L}_\in = \{\in\}$ has a single binary relation
symbol~$\in$. All set-theoretic concepts---subset, union, power set,
ordinal, cardinal, function---are defined in terms of $\in$ and the
logical connectives of first-order logic (see SYN.1--SYN.2,
Chapter~\ref{ch:syn}).
\end{defn}

The notation $(\forall x \in A)\phi$ abbreviates $\forall x(x \in A
\lif \phi)$, and $(\exists x \in A)\phi$ abbreviates $\exists x(x \in
A \land \phi)$.

\begin{rem} % PRIM-SET003
\label{PRIM-SET003}
It is sometimes useful to speak of \emph{classes}---collections that
may be ``too large'' to be sets (for instance, the collection of all
ordinals). Proper classes will be introduced in SET.3, where the
Burali-Forti paradox motivates the distinction between sets and
proper classes.
\end{rem}


%% ===================================================================
%% SET.2: ZFC Axioms
%% Sources: sth/story/extensionality (KEEP), sth/z/sep (KEEP),
%%          sth/z/pairs (KEEP), sth/z/union (KEEP),
%%          sth/z/powerset (KEEP), sth/z/infinity-again (KEEP),
%%          sth/ordinals/replacement (KEEP),
%%          sth/spine/foundation (CONDENSE),
%%          sth/z/story (CONDENSE), sth/z/milestone (CONDENSE),
%%          sth/ordinals/zfm (CONDENSE), sth/spine/zf (CONDENSE),
%%          sth/z/arbintersections (CONDENSE),
%%          sth/cardinals/zfc (CONDENSE)
%% ===================================================================

\section{ZFC Axioms} \label{SET.2}

The axioms of ZFC are motivated by the \emph{cumulative-iterative
conception} of set, which rests on three informal principles:
\begin{enumerate}
\item[] \stageshier. Every set is formed at some stage.
\item[] \stagesord. Stages are ordered: some come \emph{before} others.
\item[] \stagesacc. For any stage $S$, and for any sets which were
formed \emph{before} stage~$S$: a set is formed at stage~$S$ whose
members are exactly those sets. Nothing else is formed at stage~$S$.
\end{enumerate}
These informal principles do not constitute a formal theory, but they
serve to justify each axiom we adopt. We will introduce additional
stage-theoretic principles as needed.


%%% -----------------------------------------------------------------
%%% SET.2.1  Extensionality
%%% -----------------------------------------------------------------

\subsection{Extensionality}

The very first thing to say is that sets are individuated by their
elements. More precisely:

\begin{axiom}[Extensionality] % AX-SET001
\label{AX-SET001}
If sets $A$ and $B$ have the same elements, then $A$ and $B$ are
the same set.
\[
  \lforall[A][\lforall[B][(\lforall[x][(x \in A \liff x \in B)] \lif
  \eq[A][B])]]
\]
\end{axiom}

The Axiom of Extensionality expresses the basic idea that a set is
determined by its elements. (So sets might be contrasted with
\emph{concepts}, where precisely the same objects might fall under many
different concepts.)

Why embrace this principle? Well, it is plausible to say that any
denial of Extensionality is a decision to abandon anything which might
even be called \emph{set theory}. Set theory is no more nor less than
the theory of extensional collections.

The real challenge, though, is to lay down principles which tell us
\emph{which sets exist}. And it turns out that the only truly
``obvious'' answer to this question is provably wrong.


%%% -----------------------------------------------------------------
%%% SET.2.2  Separation
%%% -----------------------------------------------------------------

\subsection{Separation}

We start with a principle to replace Naive Comprehension:

\begin{axiom}[Scheme of Separation] % AX-SET006
\label{AX-SET006}
For every formula $\phi(x)$, this is an axiom: for any $A$, the set
$\Setabs{x \in A}{\phi(x)}$ exists.
\end{axiom}

Note that this is not a single axiom. It is a \emph{scheme} of axioms.
There are \emph{infinitely many} Separation axioms; one for every
formula $\phi(x)$. The scheme can equally well be (and normally is)
written down as follows:

\begin{defish}
For any formula $\phi(x)$ which does not contain ``$S$'', this is an
axiom:
\[
	\forall A \exists S \forall x(x \in S \liff (\phi(x) \land x \in A)).
\]
\end{defish}

The formulas~$\phi$ in the Separation axioms may have parameters.

Separation is immediately justified by the cumulative-iterative
conception. To see why, let $A$ be a set. So $A$ is formed by some
stage~$S$ (by \stageshier). Since $A$ was formed at stage~$S$, all of
$A$'s members were formed before stage $S$ (by \stagesacc). Now in
particular, consider all the sets which are members of $A$ and which
also satisfy $\phi$; clearly all of these sets, too, were formed
before stage~$S$. So they are formed into a set $\Setabs{x \in
A}{\phi(x)}$ at stage~$S$ too (by \stagesacc).

Unlike Naive Comprehension, this avoids Russell's Paradox. For we
cannot simply assert the existence of the set $\Setabs{x}{x \notin
x}$. Rather, \emph{given} some set~$A$, we can assert the existence of
the set $R_A = \Setabs{x \in A}{x \notin x}$. But all this proves is
that $R_A \notin R_A$ and $R_A \notin A$, none of which is very
worrying.

However, Separation has an immediate and striking consequence: there is
no \emph{universal} set, i.e., $\Setabs{x}{x = x}$ does not exist.
For if $V$ were a universal set, then by Separation, $R = \Setabs{x
\in V}{x \notin x} = \Setabs{x}{x \notin x}$ would exist,
contradicting Russell's Paradox.

The absence of a universal set---indeed, the open-endedness of the
hierarchy of sets---is one of the most fundamental ideas behind the
cumulative-iterative conception.

Here are a few more consequences of Separation and Extensionality.

\begin{prop} % AX-SET002
\label{AX-SET002}
If any set exists, then $\emptyset$ exists.
\end{prop}

\begin{proof}
If $A$ is a set, $\emptyset = \Setabs{x \in A}{x \neq x}$ exists by Separation.
\end{proof}

\begin{rem}[Empty Set] \label{AX-SET002:rem}
The existence of the empty set is thus \emph{derived} from Separation
(given the existence of at least one set). In many formulations of ZFC,
the Empty Set axiom is listed separately; in our development it follows
from AX-SET006 (Separation, \S\ref{AX-SET006}).
\end{rem}

\begin{rem}[Intersection existence] \label{SET.2:arbint}
Given Separation, if $A \neq \emptyset$, then $\bigcap A = \Setabs{x}{
(\forall y \in A)\, x \in y}$ exists. For let $c \in A$; then $\bigcap
A = \Setabs{x \in c}{(\forall y \in A)\, x \in y}$, which exists by
Separation. Note that $\bigcap \emptyset$ would be the universal set
(vacuously), so the condition $A \neq \emptyset$ is essential. More
generally, definitions of the form $C = \bigcap \Setabs{X}{\phi(X)}$
are justified whenever some witness $S$ with $\phi(S)$ can be exhibited:
one simply defines $C = \Setabs{x \in S}{\forall X(\phi(X) \lif x \in
X)}$ using Separation.
\end{rem}


%%% -----------------------------------------------------------------
%%% SET.2.3  Pairing
%%% -----------------------------------------------------------------

\subsection{Pairing}

\begin{axiom}[Pairs] % AX-SET003
\label{AX-SET003}
For any sets $a, b$, the set $\{a, b\}$ exists.
\[
	\forall a \forall b \exists P \forall x (x \in P \liff (x = a \lor x = b))
\]
\end{axiom}

Here is how to justify this axiom, using the iterative conception.
Suppose $a$ is available at stage $S$, and $b$ is available at stage
$T$. Let $M$ be whichever of stages $S$ and $T$ comes later. Then
since $a$ and $b$ are both available at stage $M$, the set $\{a,b\}$
is a possible collection available at any stage after $M$.

But why assume that there \emph{are} any stages after $M$? If there are
none, then our justification will fail. So, to justify Pairs, we add
another principle to the story:
\begin{enumerate}
	\item[] \stagessucc. There is no last stage.
\end{enumerate}
Even if this principle was not stated explicitly in the story of stages,
it fits well with the basic idea that sets are formed in stages. We
accept it in what follows, and with it, the Axiom of Pairs.

\begin{rem}[Consequences of Pairing] \label{AX-SET003:consequences}
For any sets $a$ and $b$, the following sets exist:
\begin{enumerate}
\item $\{a\}$ (the singleton): by Pairs, $\{a, a\}$ exists, which is
$\{a\}$ by Extensionality.
\item $a \cup b$ (binary union): by Pairs, $\{a, b\}$ exists; now $a
\cup b = \bigcup \{a, b\}$ exists by Union (see below).
\item $\tuple{a, b}$ (the ordered pair): $\{a\}$ exists by~(1); $\{a,
b\}$ by Pairs; so $\{\{a\}, \{a, b\}\} = \tuple{a, b}$ exists by Pairs
again.
\end{enumerate}
\end{rem}


%%% -----------------------------------------------------------------
%%% SET.2.4  Union
%%% -----------------------------------------------------------------

\subsection{Union}

\begin{axiom}[Union] % AX-SET004
\label{AX-SET004}
For any set $A$, the set $\bigcup A = \Setabs{x}{(\exists b \in A)\, x
\in b}$ exists.
\[
	\forall A \exists U \forall x(x \in U \liff (\exists b \in A)x \in b)
\]
\end{axiom}

This axiom is also justified by the cumulative-iterative conception.
Let $A$ be a set, so $A$ is formed at some stage $S$ (by \stageshier).
Every member of $A$ was formed \emph{before} $S$ (by \stagesacc); so,
reasoning similarly, every member of every member of $A$ was formed
before $S$. Thus all of \emph{those} sets are available before $S$, to
be formed into a set at $S$. And that set is just $\bigcup A$.


%%% -----------------------------------------------------------------
%%% SET.2.5  Power Set
%%% -----------------------------------------------------------------

\subsection{Power Set}

\begin{axiom}[Power Set] % AX-SET005
\label{AX-SET005}
For any set $A$, the set $\Pow{A} = \Setabs{x}{x \subseteq A}$ exists.
\[
	\forall A \exists P \forall x(x \in P \liff (\forall z \in x)z \in A)
\]
\end{axiom}

Our justification is straightforward. Suppose $A$ is formed at stage
$S$. Then all of $A$'s members were available before $S$ (by
\stagesacc). So, reasoning as in our justification for Separation,
every subset of $A$ is formed by stage $S$. So they are all available,
to be formed into a single set, at any stage after $S$. And we know
that there is some such stage, since $S$ is not the last stage (by
\stagessucc). So $\Pow{A}$ exists.

\begin{rem}[Cartesian products] \label{AX-SET005:products}
Given any sets $A, B$, their Cartesian product $A \times B$ exists.
This follows because $\Pow{\Pow{A \cup B}}$ exists by Power Set, and
$A \times B$ can be carved out by Separation: $A \times B = \Setabs{z
\in \Pow{\Pow{A \cup B}}}{(\exists x \in A)(\exists y \in B)\, z =
\tuple{x, y}}$.
\end{rem}


%%% -----------------------------------------------------------------
%%% SET.2.6  Infinity
%%% -----------------------------------------------------------------

\subsection{Infinity}

We already have enough axioms to ensure that there are infinitely many
sets (if there are any). For suppose some set exists, and so
$\emptyset$ exists (by Proposition~\ref{AX-SET002}). Now for any
set~$x$, the set $x \cup \{x\}$ exists by
Remark~\ref{AX-SET003:consequences}. So, applying this a few times, we
obtain sets as follows:
\begin{enumerate}
	\item[0.] $\emptyset$
	\item[1.] $\{\emptyset\}$
	\item[2.] $\{\emptyset, \{\emptyset\}\}$
	\item[3.] $\{\emptyset, \{\emptyset\}, \{\emptyset, \{\emptyset\}\}\}$
	\item[4.] $\{\emptyset, \{\emptyset\}, \{\emptyset, \{\emptyset\}\},
	\{\emptyset, \{\emptyset\}, \{\emptyset, \{\emptyset\}\}\}\}$
\end{enumerate}
and we can check that each of these sets is distinct. It is not hard to
verify that the set labelled ``$n$'' has exactly $n$ members, and
(intuitively) is formed at the $n$th stage.

But this gives us \emph{infinitely many} sets without guaranteeing that
there is an \emph{infinite set}, i.e., a set with infinitely many
members. And this really matters: unless we can find a (Dedekind)
infinite set, we cannot construct a Dedekind algebra to serve as the
natural numbers (compare PRIM-BST012, Dedekind algebra,
Chapter~\ref{ch:bst}).

The axioms we have laid down so far do \emph{not} guarantee the
existence of any infinite set. So we lay down a new axiom:

\begin{axiom}[Infinity] % Infinity axiom (no separate AX-SET ID in lean-outline; cf.\ AX-SET009 for Choice)
\label{SET.2:infinity}
There is a set, $I$, such that $\emptyset \in I$ and $x \cup \{x\} \in
I$ whenever $x \in I$.
\begin{align*}
	\exists I( & (\exists o \in I)\forall x\ x \notin o \land {}\\
	& (\forall x \in I)(\exists s \in I)\forall z(z \in s \liff (z \in x \lor z = x)))
\end{align*}
\end{axiom}

It is easy to see that the set $I$ given to us by the Axiom of Infinity
is Dedekind infinite. Its distinguished element is $\emptyset$, and the
injection on $I$ is given by $s(x) = x \cup \{x\}$.

\begin{defn}[$\omega$] % defines omega
\label{SET.2:defnomega}
Let $I$ be any set given to us by the Axiom of Infinity. Let $s$ be the
function $s(x) = x \cup \{x\}$. Let $\omega =
\closureofunder{s}{\emptyset}$. We call the members of $\omega$ the
\emph{natural numbers}, and say that $n$ is the result of $n$-many
applications of $s$ to $\emptyset$.
\end{defn}

To justify the Axiom of Infinity, we add another principle:
\begin{enumerate}
	\item[] \stagesinf. There is an infinite stage. That is, there is a
	stage which (a) is not the first stage, (b) has some stages before
	it, but (c) has no immediate predecessor.
\end{enumerate}
The Axiom of Infinity follows: natural number $n$ is formed at stage
$n$, so $\omega$ is formed at the first infinite stage. Unlike
\stagessucc, the principle \stagesinf{} is not ``forced upon us'' by
the iterative conception---it seems perfectly coherent to think that the
stages are ordered like the natural numbers. We simply accept
\stagesinf{} in what follows.


%%% -----------------------------------------------------------------
%%% Milestone: Z^-
%%% -----------------------------------------------------------------

\begin{rem}[$\Zminus$: A Milestone] \label{SET.2:zminus}
The theory $\Zminus$ has these axioms: Extensionality, Union, Pairs,
Power Set, Infinity, and all instances of the Separation scheme.
The name stands for \emph{Zermelo} set theory (minus Foundation, which
we will come to below). Zermelo essentially formulated this theory in
1908. $\Zminus$ is powerful enough to carry out an enormous amount of
mathematics; in particular, the naive set-theoretic constructions of
Chapter~\ref{ch:bst} can be made rigorous within~$\Zminus$.
\end{rem}


%%% -----------------------------------------------------------------
%%% SET.2.7  Replacement
%%% -----------------------------------------------------------------

\subsection{Replacement}

In order to prove that every well-ordering is isomorphic to some ordinal
(a key result for ordinal theory in SET.3), we need a new axiom that
goes beyond the power of~$\Zminus$.

\begin{axiom}[Scheme of Replacement] % AX-SET007
\label{AX-SET007}
For any formula $\phi(x, y)$, the following is an axiom:
\begin{quote}
	for any $A$, if $(\forall x \in A)\lexists![y][\phi(x,y)]$, then
	$\Setabs{y}{(\exists x \in A)\phi(x,y)}$ exists.
\end{quote}
\end{axiom}
\noindent
As with Separation, this is a scheme: it yields infinitely many axioms,
for each of the infinitely many different $\phi$'s. It can equally well
be written thus:

\begin{defish}
For any formula $\phi(x,y)$ which does not contain ``$B$'', the
following is an axiom:
\[
\forall A[(\forall x \in A)\lexists![y][\phi(x,y)] \lif \exists B\forall y (y \in B \liff (\exists x \in A)\phi(x,y))]
\]
\end{defish}

On first encounter this is quite a tangled formula. The following quick
consequence gives a \emph{clearer} expression to the intuitive idea:
for any term $\tau(x)$ and any set $A$, the set
$\Setabs{\tau(x)}{x \in A} = \Setabs{y}{(\exists x \in A)\, y =
\tau(x)}$ exists. This is because $\tau$ is a term, so $\forall x
\lexists![y][\tau(x) = y]$. Thus ``Replacement'' is a good name: given
a set $A$, you can form a new set $\Setabs{\tau(x)}{x \in A}$ by
replacing every member of $A$ with its image under~$\tau$. Following
the notation for the image of a set under a function, we might write
$\funimage{\tau}{A}$ for $\Setabs{\tau(x)}{x \in A}$.

Crucially, $\tau$ is a \emph{term}. It need not be a \emph{function}
in the sense of a set of ordered pairs. If $f$ is a function (in that
sense), then $\funimage{f}{A}$ is just a subset of $\ran{f}$, already
guaranteed to exist by the axioms of~$\Zminus$. Replacement, by
contrast, is a \emph{powerful} addition to our axioms.


%%% -----------------------------------------------------------------
%%% Milestone: ZF^-
%%% -----------------------------------------------------------------

\begin{rem}[$\ZFminus$: A Milestone] \label{SET.2:zfminus}
The theory $\ZFminus$ adds all instances of the Replacement scheme
to~$\Zminus$. The name stands for \emph{Zermelo--Fraenkel} set theory
(minus Foundation). Fraenkel is credited with the formulation of
Replacement in 1922, although the first precise formulation was due to
Skolem in the same year.
\end{rem}


%%% -----------------------------------------------------------------
%%% SET.2.8  Foundation
%%% -----------------------------------------------------------------

\subsection{Foundation}

We are \emph{almost} done---but not \emph{quite}---because nothing in
$\ZFminus$ guarantees that \emph{every} set is in some $V_\alpha$,
i.e., that every set is formed at some stage.

There is a fairly straightforward sense in which we don't \emph{care}
whether there are sets outside the hierarchy (if there are any, we can
simply ignore them). But we have motivated our \emph{concept} of set
with the thought that every set is formed at some stage (see
\stageshier). So we preclude the possibility of sets falling outside the
hierarchy by adding a new axiom.

Since the $V_\alpha$s are our stages, we might simply consider adding
Regularity as an axiom:

\begin{defish}
\emph{Regularity.} $\forall A \exists \alpha\, A \subseteq V_\alpha$
\end{defish}

This would be perfectly reasonable. However, for technical reasons we
instead adopt an alternative formulation:

\begin{axiom}[Foundation] % AX-SET008
\label{AX-SET008}
$(\forall A \neq \emptyset)(\exists B \in A)\, A \cap B = \emptyset$.
\end{axiom}

The connection between Foundation and Regularity requires some work.
The key notion is the \emph{transitive closure}:

\begin{defn}[Transitive Closure] % transitive closure trcl(A)
\label{SET.2:trcl}
For each set $A$, let:
\begin{align*}
	\text{cl}_0(A) &= A,\\
	\text{cl}_{n+1}(A) &= \bigcup \text{cl}_n(A),\\
	\trcl{A} &= \bigcup_{n < \omega} \text{cl}_{n}(A).
\end{align*}
We call $\trcl{A}$ the \emph{transitive closure} of $A$.
\end{defn}

One can show that $A \subseteq \trcl{A}$ and that $\trcl{A}$ is a
transitive set. Using Foundation and the transitive closure, one proves:

\begin{thm}[Foundation entails Regularity] % THM: zfentailsregularity
\label{SET.2:zfentailsregularity}
Regularity holds in $\ZFminus + \text{Foundation}$.
\end{thm}

\begin{proof}[Proof sketch]
Fix $A$. Since $A \subseteq \trcl{A}$ and $\trcl{A}$ is transitive, it
suffices to show that every transitive set is contained in some
$V_\alpha$. Let $A$ be transitive, and suppose for contradiction that
the set $D = \Setabs{x \in A}{\forall \delta\; x \nsubseteq V_\delta}$
is non-empty. By Foundation, there is some $B \in D$ with $D \cap B =
\emptyset$. Since $A$ is transitive, every element of $B$ is in $A$ but
not in $D$, so every element of $B$ is contained in some $V_\delta$.
Collecting these witnesses yields $B \subseteq V_\beta$ for some
ordinal~$\beta$, contradicting $B \in D$.
\end{proof}

\begin{rem}[Foundation--Regularity equivalence] \label{SET.2:found-reg}
In $\ZFminus$, Foundation and Regularity are equivalent. The converse
direction (Regularity implies Foundation) is established using the
notion of the rank of a set (see \S\ref{SET.5}). Given $\ZFminus$, we can
justify Foundation by noting that it is equivalent to Regularity, and
Regularity follows immediately from \stageshier.

The reason we take Foundation rather than Regularity as our official
axiom is that Foundation can be stated without using the
$V_\alpha$-hierarchy. The definition of the $V_\alpha$s relies on
Transfinite Recursion, whose proof employs Replacement. So while
Foundation and Regularity are equivalent modulo $\ZFminus$, they are
\emph{not} equivalent modulo $\Zminus$; indeed, both $\Zminus$ and $\Z$
are too weak to define the $V_\alpha$s, so Regularity (as formulated
above) does not even make \emph{sense} in~$\Z$.
\end{rem}


%%% -----------------------------------------------------------------
%%% Milestones: Z, ZF
%%% -----------------------------------------------------------------

\begin{rem}[$\Z$ and $\ZF$: A Milestone] \label{SET.2:z-zf}
The theory $\Z$ adds Foundation to $\Zminus$. Its axioms are:
Extensionality, Union, Pairs, Power Set, Infinity, Foundation, and all
instances of the Separation scheme. The theory $\ZF$ adds Foundation to
$\ZFminus$; equivalently, $\ZF$ adds all instances of Replacement
to~$\Z$. From now on we work in $\ZF$ (unless otherwise stated).
\end{rem}


%%% -----------------------------------------------------------------
%%% Milestone: ZFC
%%% -----------------------------------------------------------------

\begin{rem}[$\ZFC$: Final Milestone] \label{SET.2:zfc}
The theory $\ZFC$ adds Well-Ordering to $\ZF$. Its axioms are:
Extensionality, Union, Pairs, Power Set, Infinity, Foundation,
Well-Ordering, and all instances of the Separation and Replacement
schemes. The name stands for \emph{Zermelo--Fraenkel with Choice},
because Well-Ordering turns out to be equivalent (modulo $\ZF$) to the
Axiom of Choice. The Well-Ordering principle (AX-SET009) is stated in
SET.4 after the ordinal theory needed to define cardinals, and proven
equivalent to Choice in SET.6.
\end{rem}


%% ===================================================================
%% SET.3: Ordinals
%% Sources: sth/ordinals/wo (KEEP), sth/ordinals/iso (CONDENSE),
%%          sth/ordinals/vn (KEEP), sth/ordinals/basic (KEEP),
%%          sth/ordinals/ordtype (KEEP), sth/ordinals/opps (KEEP),
%%          sth/spine/recursion (KEEP), sth/choice/hartogs (CONDENSE)
%% ===================================================================

\section{Ordinals} \label{SET.3}

In SET.2 we postulated that there is an infinite stage of the
hierarchy (\stagesinf). Given \stagessucc, the stages do not stop
there: at the next stage after the first infinite stage, we form all
possible collections of sets available at the first infinite stage; and
repeat; and repeat. Implicitly, we have invoked a notion of number
that extends \emph{beyond} the natural numbers---the notion of a
\emph{transfinite ordinal}. The aim of this section is to make that
idea rigorous: we define well-orderings, introduce von Neumann's
ordinals, prove their fundamental properties (including transfinite
induction and the Burali-Forti paradox), and develop the machinery of
transfinite recursion.


%%% -----------------------------------------------------------------
%%% SET.3.1  Well-Orderings
%%% -----------------------------------------------------------------

\subsection{Well-Orderings}

\begin{defn}[Well-Ordering] % DEF-SET009
\label{DEF-SET009}
The relation $<$ \emph{well-orders} $A$ iff it meets these two
conditions:
\begin{enumerate}
	\item $<$ is connected, i.e., for all $a, b \in A$, either $a < b$
	or $a = b$ or $b < a$;
	\item every non-empty subset of $A$ has a $<$-minimal element,
	i.e., if $\emptyset \neq X \subseteq A$ then $(\exists m \in
	X)(\forall z \in X)\, z \nless m$.
\end{enumerate}
\end{defn}

\begin{prop} \label{SET.3:wo:strictorder}
If $<$ well-orders $A$, then every non-empty subset of $A$ has a unique
$<$-least member, and $<$ is irreflexive, asymmetric and transitive.
\end{prop}

\begin{proof}
If $X$ is a non-empty subset of $A$, it has a $<$-minimal element
$m$, i.e., $(\forall z \in X)\, z \nless m$. Since $<$ is connected,
$(\forall z \in X)\, m \leq z$. So $m$ is the $<$-least element of $X$.

For irreflexivity, fix $a \in A$; the $<$-least element of $\{a\}$ is
$a$, so $a \nless a$. For transitivity, if $a < b < c$, then since
$\{a, b, c\}$ has a $<$-least element, $a < c$. Asymmetry follows from
irreflexivity and transitivity.
\end{proof}

\begin{prop}[Well-Ordering Induction] \label{SET.3:propwoinduction}
If $<$ well-orders $A$, then for any formula $\phi(x)$:
\[
	\text{if }(\forall a \in A)((\forall b < a)\phi(b) \lif
		\phi(a))\text{, then }(\forall a \in A)\phi(a).
\]
\end{prop}

\begin{proof}
Suppose $\lnot(\forall a \in A)\phi(a)$, i.e., $X = \Setabs{x \in
A}{\lnot\phi(x)} \neq \emptyset$. Then $X$ has a $<$-minimal element,
$a$. So $(\forall b < a)\phi(b)$ but $\lnot \phi(a)$.
\end{proof}

This last property should remind the reader of the principle of strong
induction on the naturals: if $(\forall n \in \omega)((\forall m <
n)\phi(m) \lif \phi(n))$, then $(\forall n \in \omega)\phi(n)$. It is
this property that makes well-ordering such a \emph{robust} notion.


%%% -----------------------------------------------------------------
%%% SET.3.2  Order-Isomorphisms
%%% -----------------------------------------------------------------

\subsection{Order-Isomorphisms}

\begin{defn}[Order-Isomorphism] \label{SET.3:deforderiso}
A \emph{well-ordering} is a pair $\tuple{A, <}$ such that $<$
well-orders $A$. The well-orderings $\tuple{A, <}$ and $\tuple{B,
\lessdot}$ are \emph{order-isomorphic} iff there is a bijection $f
\colon A \to B$ such that $x < y$ iff $f(x) \lessdot f(y)$. In this
case, we write $\ordeq{\tuple{A, <}}{\tuple{B, \lessdot}}$, and say
that $f$ is an \emph{order-isomorphism} (or simply an
\emph{isomorphism}).
\end{defn}

\begin{defn}[Initial Segment] \label{SET.3:definitseg}
When $\tuple{A, <}$ is a well-ordering with $a \in A$, let $A_a =
\Setabs{x \in A}{x < a}$. We say that $A_a$ is a proper \emph{initial
segment} of $A$. Let $<_a$ be the restriction of $<$ to $A_a^2$.
\end{defn}

\begin{lem} \label{SET.3:wellordnotinitial}
If $\tuple{A, <}$ is a well-ordering with $a \in A$, then
$\ordneq{\tuple{A, <}}{\tuple{A_a, <_a}}$.
\end{lem}

\begin{proof}
For reductio, suppose $f \colon A \to A_a$ is an isomorphism. Since $f$
is a bijection and $A_a \subsetneq A$, let $b \in A$ be the $<$-least
element such that $b \neq f(b)$. One shows that $(\forall x \in A)(x <
b \liff x < f(b))$, from which $b = f(b)$ by the extensionality of
strict linear orders, completing the reductio.
\end{proof}

\begin{lem} \label{SET.3:wellordinitialsegment}
Let $\tuple{A, <}$ and $\tuple{B, \lessdot}$ be well-orderings. If $f
\colon A \to B$ is an isomorphism and $a \in A$, then
$\funrestrictionto{f}{A_{a}} : A_a \to B_{f(a)}$ is an isomorphism.
\end{lem}

\begin{lem} \label{SET.3:lemordsegments}
Let $\tuple{A, <}$ and $\tuple{B, \lessdot}$ be well-orderings. If
$\ordeq{\tuple{A_{a_1}, <_{a_1}}}{\tuple{B_{b_1}, \lessdot_{b_1}}}$
and $\ordeq{\tuple{A_{{a_2}}, <_{a_2}}}{\tuple{B_{{b_2}},
\lessdot_{b_2}}}$, then ${a_1} < {a_2}$ iff ${b_1} \lessdot {b_2}$.
\end{lem}

\begin{proof}[Proof sketch]
If $a_1 < a_2$, then $A_{a_1} \subsetneq A_{a_2}$.
The isomorphism $A_{a_2} \cong B_{b_2}$ restricts
(by Lemma~\ref{SET.3:wellordinitialsegment}) to an isomorphism
$A_{a_1} \cong (B_{b_2})_{b'}$ for some $b' \lessdot b_2$.
Since isomorphisms between well-orderings are unique
(by Lemma~\ref{SET.3:wellordnotinitial}), $b' = b_1$,
giving $b_1 \lessdot b_2$.  The converse is symmetric.
\end{proof}

\begin{thm}[Comparability of Well-Orderings] \label{SET.3:woalwayscomparable}
Given any two well-orderings, one is isomorphic to an initial segment
(not necessarily proper) of the other.
\end{thm}

\begin{proof}[Proof sketch]
Let $\tuple{A, <}$ and $\tuple{B, \lessdot}$ be well-orderings. Using
Separation, let
\[
	f = \Setabs{\tuple{a, b} \in A \times B}{
		\ordeq{\tuple{A_a, <_a}}{\tuple{B_b, \lessdot_b}}}.
\]
By Lemma~\ref{SET.3:lemordsegments}, $f$ preserves order. One shows
that $\dom{f}$ is an initial segment of $A$ and $\ran{f}$ is an initial
segment of $B$. If both were \emph{proper} initial segments, say
$\dom{f} = A_a$ and $\ran{f} = B_b$, then $f \colon A_a \to B_b$ would
be an isomorphism, forcing $\tuple{a, b} \in f$---a contradiction.
\end{proof}


%%% -----------------------------------------------------------------
%%% SET.3.3  Von Neumann's Ordinals
%%% -----------------------------------------------------------------

\subsection{Von Neumann's Ordinals}

Theorem~\ref{SET.3:woalwayscomparable} gives rise to a thought: we could
introduce certain objects, called \emph{order types}, to go proxy for
the well-orderings. We would hope to secure:
\begin{align*}
	\ordtype{A, <} = \ordtype{B, \lessdot} &
	\text{ iff } \ordeq{\tuple{A, <}}{\tuple{B, \lessdot}}\\
	\ordtype{A, <} < \ordtype{B, \lessdot} &
	\text{ iff }\ordeq{\tuple{A, <}}{\tuple{B_b, \lessdot_b}}\text{ for some }b \in B
\end{align*}
The most common way to achieve this---and the approach we follow---is
to define order types via certain \emph{canonical} well-ordered sets,
first introduced by von Neumann:

\begin{defn}[Transitive Set] % DEF-SET002
\label{DEF-SET002}
The set $A$ is \emph{transitive} iff $(\forall x \in A)\, x \subseteq A$.
\end{defn}

\begin{defn}[Ordinal] % DEF-SET001
\label{DEF-SET001}
$A$ is an \emph{ordinal} iff $A$ is transitive and well-ordered by
$\in$.
\end{defn}

In what follows, we use Greek letters for ordinals. It follows
immediately from the definition that if $\alpha$ is an ordinal, then
$\tuple{\alpha, \in_\alpha}$ is a well-ordering, where $\in_\alpha =
\Setabs{\tuple{x, y} \in \alpha^2}{x \in y}$. So, abusing notation, we
can say that $\alpha$ \emph{itself} is a well-ordering.

Here are our first few ordinals:
\[
	\emptyset, \quad \{\emptyset\}, \quad
	\{\emptyset, \{\emptyset\}\}, \quad
	\{\emptyset, \{\emptyset\}, \{\emptyset, \{\emptyset\}\}\}, \quad \ldots
\]
These are exactly the sets that appeared in our Axiom of Infinity, i.e.,
in the definition of $\omega$ (Definition~\ref{SET.2:defnomega}). This
is no coincidence: von Neumann's construction treats natural numbers as
ordinals, but allows for transfinite ordinals too.


%%% -----------------------------------------------------------------
%%% SET.3.4  Basic Properties of the Ordinals
%%% -----------------------------------------------------------------

\subsection{Basic Properties of the Ordinals}

\begin{lem} \label{SET.3:ordmemberord}
Every element of an ordinal is an ordinal.
\end{lem}

\begin{proof}
Let $\alpha$ be an ordinal with $b \in \alpha$. Since $\alpha$ is
transitive, $b \subseteq \alpha$. So $\in$ well-orders $b$ as $\in$
well-orders $\alpha$.

To see that $b$ is transitive, suppose $x \in c \in b$. So $c \in
\alpha$ as $b \subseteq \alpha$. Again, as $\alpha$ is transitive, $c
\subseteq \alpha$, so that $x \in \alpha$. So $x, c, b \in \alpha$.
Since $\in$ well-orders $\alpha$, $\in$ is transitive on $\alpha$ by
Proposition~\ref{SET.3:wo:strictorder}. Hence $x \in c \in b$ gives $x
\in b$. Generalising, $c \subseteq b$.
\end{proof}

\begin{thm}[Transfinite Induction] % DEF-SET005
\label{DEF-SET005}
For any formula $\phi(x)$:
\[
	\text{if }\exists \alpha \phi(\alpha)\text{, then }\exists \alpha(\phi(\alpha)
	\land  (\forall \beta \in \alpha) \lnot \phi(\beta))
\]
where the displayed quantifiers are implicitly restricted to ordinals.
\end{thm}

\begin{proof}
Suppose $\phi(\alpha)$, for some ordinal $\alpha$. If $(\forall \beta
\in \alpha) \lnot \phi(\beta)$, then we are done. Otherwise, as
$\alpha$ is an ordinal, it has some $\in$-least element which is
$\phi$, and this is an ordinal by Lemma~\ref{SET.3:ordmemberord}.
\end{proof}

We can equally express Transfinite Induction as the scheme:
\[
\text{if }\forall \alpha((\forall \beta \in \alpha)\phi(\beta) \lif
\phi(\alpha))\text{, then }\forall \alpha\phi(\alpha).
\]

\begin{thm}[Trichotomy] \label{SET.3:ordtrichotomy}
$\alpha \in \beta \lor \alpha = \beta \lor \beta \in \alpha$, for any
ordinals $\alpha$ and $\beta$.
\end{thm}

\begin{proof}
The proof is by double induction, using Transfinite Induction
(Theorem~\ref{DEF-SET005}) twice. Say that $x$ is \emph{comparable}
with $y$ iff $x \in y \lor x = y \lor y \in x$.

For induction, suppose that every ordinal in $\alpha$ is comparable with
\emph{every} ordinal. For further induction, suppose that $\alpha$ is
comparable with every ordinal in $\beta$. We show that $\alpha$ is
comparable with $\beta$. By induction on $\beta$, it follows that
$\alpha$ is comparable with every ordinal; and by induction on $\alpha$,
\emph{every} ordinal is comparable with \emph{every} ordinal. It
suffices to assume $\alpha \notin \beta$ and $\beta \notin \alpha$, and
show $\alpha = \beta$.

To show $\alpha \subseteq \beta$: fix $\gamma \in \alpha$; this is an
ordinal by Lemma~\ref{SET.3:ordmemberord}. By the first induction
hypothesis, $\gamma$ is comparable with $\beta$. But if $\gamma = \beta$
or $\beta \in \gamma$, then $\beta \in \alpha$ (using transitivity of
$\alpha$ if necessary), contrary to assumption; so $\gamma \in \beta$.
Similar reasoning shows $\beta \subseteq \alpha$. So $\alpha = \beta$.
\end{proof}

We sometimes write $\alpha < \beta$ rather than $\alpha \in \beta$,
since $\in$ behaves as an ordering relation on the ordinals.

\begin{cor} \label{SET.3:corordtransitiveord}
$A$ is an ordinal iff $A$ is a transitive set of ordinals.
\end{cor}

\begin{proof}
\emph{Left-to-right.} By Lemma~\ref{SET.3:ordmemberord}.
\emph{Right-to-left.} If $A$ is a transitive set of ordinals, then
$\in$ well-orders $A$ by Transfinite Induction
(Theorem~\ref{DEF-SET005}) and Trichotomy
(Theorem~\ref{SET.3:ordtrichotomy}).
\end{proof}

Now, we glossed Theorems~\ref{DEF-SET005} and~\ref{SET.3:ordtrichotomy}
as telling us that $\in$ well-orders the ordinals. However, we must be
cautious, thanks to the following result:

\begin{thm}[Burali-Forti Paradox] % PRIM-SET003
\label{SET.3:buraliforti}
There is no set of all the ordinals.
\end{thm}

\begin{proof}
For reductio, suppose $O$ is the set of all ordinals. If $\alpha \in
\beta \in O$, then $\alpha$ is an ordinal by
Lemma~\ref{SET.3:ordmemberord}, so $\alpha \in O$. So $O$ is
transitive, and hence $O$ is an ordinal by
Corollary~\ref{SET.3:corordtransitiveord}. Hence $O \in O$,
contradicting irreflexivity (Proposition~\ref{SET.3:wo:strictorder}).
\end{proof}

The Burali-Forti paradox shows that the ordinals form a \emph{proper
class} (see Remark~\ref{PRIM-SET003}). Ordinals are sets which are
individually well-ordered by membership, and collectively well-ordered
by membership, without collectively constituting a set.


%%% -----------------------------------------------------------------
%%% SET.3.5  Ordinals as Order-Types
%%% -----------------------------------------------------------------

\subsection{Ordinals as Order-Types}

Armed with Replacement (AX-SET007, \S\ref{AX-SET007}), and so now
working in $\ZFminus$, we can prove the key representation theorem:

\begin{thm}[Ordinal Representation] \label{SET.3:thmOrdinalRepresentation}
Every well-ordering is isomorphic to a unique ordinal.
\end{thm}

\begin{proof}
Let $\tuple{A, <}$ be a well-ordering. By
Theorem~\ref{SET.3:ordtrichotomy} and
Lemma~\ref{SET.3:wellordnotinitial}, it is isomorphic to at most one
ordinal. For reductio, suppose it is not isomorphic to \emph{any}
ordinal. ``Make $\tuple{A, <}$ as small as possible'': if some proper
initial segment $\tuple{A_a, <_a}$ is not isomorphic to any ordinal,
let $a$ be least with that property and set $B = A_a$; otherwise let $B
= A$.

By construction, every proper initial segment of $B$ is isomorphic to
some (unique) ordinal. By Replacement, the following set exists and is a
function:
\[
	f = \Setabs{\tuple{\beta, b}}{b \in B\text{ and }
	\ordeq{\beta}{\tuple{B_b, \lessdot_b}}}
\]
Clearly $\ran{f} = B$. By Lemma~\ref{SET.3:lemordsegments}, $f$
preserves ordering. To show $\dom{f}$ is an ordinal, by
Corollary~\ref{SET.3:corordtransitiveord} it suffices to show that
$\dom{f}$ is transitive: if $\beta \in \dom{f}$, i.e.,
$\ordeq{\beta}{\tuple{B_b, \lessdot_b}}$ for some $b$, and $\gamma \in
\beta$, then $\gamma \in \dom{f}$ by
Lemma~\ref{SET.3:wellordinitialsegment}; so $\beta \subseteq \dom{f}$.
This is a contradiction.
\end{proof}

This licenses the following definition:

\begin{defn}[Order Type] \label{SET.3:defordtype}
If $\tuple{A, <}$ is a well-ordering, then its \emph{order type},
$\ordtype{A, <}$, is the unique ordinal $\alpha$ such that
$\ordeq{\tuple{A, <}}{\alpha}$.
\end{defn}

\begin{cor} \label{SET.3:ordtypesworklikeyouwant}
Where $\tuple{A, <}$ and $\tuple{B, \lessdot}$ are well-orderings:
\begin{align*}
	\ordtype{A, <} = \ordtype{B, \lessdot}&\text{ iff }\ordeq{\tuple{A, <}}{\tuple{B, \lessdot}}\\
	\ordtype{A, <} \in \ordtype{B, \lessdot}&\text{ iff }\ordeq{\tuple{A, <}}{\tuple{B_b, \lessdot_b}}\text{ for some }b \in B
\end{align*}
\end{cor}

\begin{proof}
The first claim holds by Trichotomy and
Lemma~\ref{SET.3:wellordnotinitial}. For the second, let $\ordtype{A,
<} = \alpha$ and $\ordtype{B, \lessdot} = \beta$, and let $f \colon
\beta \to \tuple{B, \lessdot}$ be an isomorphism. Then:
\begin{align*}
	\alpha \in \beta &\text{ iff }\funrestrictionto{f}{\alpha} \colon
	\alpha \to B_{f(\alpha)}\text{ is an isomorphism}\\
	&\text{ iff }\ordeq{\tuple{A, <}}{\tuple{B_{f(\alpha)},
	\lessdot_{f(\alpha)}}}\\
	&\text{ iff }\ordeq{\tuple{A, <}}{\tuple{B_b, \lessdot_b}}\text{
	for some $b \in B$}
\end{align*}
by Lemmas~\ref{SET.3:wellordinitialsegment}
and~\ref{SET.3:lemordsegments}.
\end{proof}


%%% -----------------------------------------------------------------
%%% SET.3.6  Successor and Limit Ordinals
%%% -----------------------------------------------------------------

\subsection{Successor and Limit Ordinals}

\begin{defn}[Successor and Limit Ordinal] % DEF-SET003, DEF-SET004
\label{DEF-SET003}
\label{DEF-SET004}
For any ordinal $\alpha$, its \emph{successor} is $\ordsucc{\alpha} =
\alpha \cup \{\alpha\}$. We say that $\alpha$ is a \emph{successor}
ordinal if $\ordsucc{\beta} = \alpha$ for some ordinal $\beta$. We say
that $\alpha$ is a \emph{limit} ordinal iff $\alpha$ is neither empty
nor a successor ordinal.
\end{defn}

\begin{prop} \label{SET.3:succprops}
For any ordinal $\alpha$: (1) $\alpha \in \ordsucc{\alpha}$;
(2) $\ordsucc{\alpha}$ is an ordinal; (3) there is no ordinal $\beta$
such that $\alpha \in \beta \in \ordsucc{\alpha}$.
\end{prop}

\begin{proof}
Trivially, $\alpha \in \alpha \cup \{\alpha\} = \ordsucc{\alpha}$.
Equally, $\ordsucc{\alpha}$ is a transitive set of ordinals, hence an
ordinal by Corollary~\ref{SET.3:corordtransitiveord}. And $\alpha \in
\beta \in \ordsucc{\alpha}$ is impossible, since then either $\beta \in
\alpha$ or $\beta = \alpha$, contradicting irreflexivity.
\end{proof}

\begin{thm}[Simple Transfinite Induction] \label{SET.3:simpletransrecursion}
Let $\phi(x)$ be a formula such that: (1) $\phi(\emptyset)$;
(2) for any ordinal $\alpha$, if $\phi(\alpha)$ then
$\phi(\ordsucc{\alpha})$; and (3) if $\alpha$ is a limit ordinal and
$(\forall \beta \in \alpha)\phi(\beta)$, then $\phi(\alpha)$. Then
$\forall \alpha\, \phi(\alpha)$.
\end{thm}

\begin{proof}
Suppose there is some ordinal which is $\lnot\phi$; let $\gamma$ be the
least such ordinal. Then either $\gamma = \emptyset$, or $\gamma =
\ordsucc{\alpha}$ for some $\alpha$ such that $\phi(\alpha)$, or
$\gamma$ is a limit ordinal and $(\forall \beta \in
\gamma)\phi(\beta)$. In each case, we obtain a contradiction.
\end{proof}

\begin{defn}[Least Strict Upper Bound] \label{SET.3:defsupstrict}
If $X$ is a set of ordinals, then $\supstrict(X) = \bigcup_{\alpha \in
X} \ordsucc{\alpha}$.
\end{defn}

\begin{prop}
If $X$ is a set of ordinals, $\supstrict(X)$ is the least ordinal
greater than every ordinal in $X$.
\end{prop}

\begin{proof}
Let $Y = \Setabs{\ordsucc{\alpha}}{\alpha \in X}$, so $\supstrict(X) =
\bigcup Y$. Since ordinals are transitive and every element of an
ordinal is an ordinal, $\supstrict(X)$ is a transitive set of ordinals,
hence an ordinal by Corollary~\ref{SET.3:corordtransitiveord}.

If $\alpha \in X$, then $\ordsucc{\alpha} \in Y$, so $\ordsucc{\alpha}
\subseteq \bigcup Y = \supstrict(X)$, hence $\alpha \in \supstrict(X)$.
So $\supstrict(X)$ is strictly greater than every ordinal in $X$.
Conversely, if $\alpha \in \supstrict(X)$, then $\alpha \in
\ordsucc{\beta} \in Y$ for some $\beta \in X$, so $\alpha \leq \beta
\in X$. So $\supstrict(X)$ is the \emph{least} strict upper bound on
$X$.
\end{proof}


%%% -----------------------------------------------------------------
%%% SET.3.7  Transfinite Recursion
%%% -----------------------------------------------------------------

\subsection{Transfinite Recursion}

The overarching moral of this subsection is that Transfinite Induction
plus Replacement guarantee the legitimacy of several versions of
transfinite recursion.

\begin{defn}[$\alpha$-Approximation] \label{SET.3:defapprox}
Let $\tau(x)$ be a term; let $f$ be a function; let $\alpha$ be an
ordinal. We say that $f$ is an \emph{$\alpha$-approximation} for $\tau$
iff $\dom{f} = \alpha$ and $(\forall \beta \in \alpha)\, f(\beta) =
\tau(\funrestrictionto{f}{\beta})$.
\end{defn}

\begin{lem}[Bounded Recursion] \label{SET.3:transrecursionfun}
For any term $\tau(x)$ and any ordinal $\alpha$, there is a unique
$\alpha$-approximation for $\tau$.
\end{lem}

\begin{proof}
We first establish uniqueness. Let $g$ and $h$ be $\gamma$- and
$\delta$-approximations, respectively. A transfinite induction on their
arguments shows $g(\beta) = h(\beta)$ for any $\beta \in \dom{g} \cap
\dom{h} = \min(\gamma, \delta)$. So approximations are unique (if they
exist) and agree on all values.

To establish existence, we use Simple Transfinite Induction
(Theorem~\ref{SET.3:simpletransrecursion}) on ordinals $\delta \leq
\alpha$.

The empty function is trivially an $\emptyset$-approximation.

If $g$ is a $\gamma$-approximation, then $g \cup \{\tuple{\gamma,
\tau(g)}\}$ is a $\ordsucc{\gamma}$-approximation.

If $\gamma$ is a limit ordinal and $g_\delta$ is a
$\delta$-approximation for all $\delta < \gamma$, let $g =
\bigcup_{\delta \in \gamma} g_\delta$. This is a function since the
various $g_\delta$s agree on all values. And if $\delta \in \gamma$ then
$g(\delta) = g_{\ordsucc{\delta}}(\delta) =
\tau(\funrestrictionto{g_{\ordsucc{\delta}}}{\delta}) =
\tau(\funrestrictionto{g}{\delta})$.
\end{proof}

\begin{thm}[General Recursion] % DEF-SET006, THM-SET002
\label{DEF-SET006}
\label{THM-SET002}
For any term $\tau(x)$, we can explicitly define a term $\sigma(x)$
such that $\sigma(\alpha) = \tau(\funrestrictionto{\sigma}{\alpha})$
for any ordinal $\alpha$.
\end{thm}

\begin{proof}
For each $\alpha$, by Lemma~\ref{SET.3:transrecursionfun} there is a
unique $\alpha$-approximation, $f_\alpha$, for $\tau$. Define
$\sigma(\alpha)$ as $f_{\ordsucc{\alpha}}(\alpha)$. Then:
\begin{align*}
	\sigma(\alpha) &=
	f_{\ordsucc{\alpha}}(\alpha) \\&=
	\tau(\funrestrictionto{f_{\ordsucc{\alpha}}}{\alpha}) \\&=
	\tau(\Setabs{\tuple{\beta, f_{\ordsucc{\alpha}}(\beta)}}{\beta \in \alpha}) \\&=
	\tau(\Setabs{\tuple{\beta, f_{\ordsucc{\beta}}(\beta)}}{\beta \in \alpha}) \\&=
	\tau(\funrestrictionto{\sigma}{\alpha})
\end{align*}
noting that $f_{\ordsucc{\beta}}(\beta) =
f_{\ordsucc{\alpha}}(\beta)$ for all $\beta < \alpha$, as in
Lemma~\ref{SET.3:transrecursionfun}.
\end{proof}

Note that Theorem~\ref{DEF-SET006} is a \emph{schema}. Crucially, we
cannot expect $\sigma$ to define a function (i.e., a set), since then
$\dom{\sigma}$ would be the set of all ordinals, contradicting
Burali-Forti (Theorem~\ref{SET.3:buraliforti}).

\begin{thm}[Simple Recursion] \label{SET.3:simplerecursionschema}
For any terms $\tau(x)$ and $\theta(x)$ and any set $A$, we can
explicitly define a term $\sigma(x)$ such that:
\begin{align*}
	\sigma(\emptyset) &= A\\
	\sigma(\ordsucc{\alpha}) &= \tau(\sigma(\alpha)) &&
		\text{for any ordinal }\alpha\\
	\sigma(\alpha) &= \theta(\ran{\funrestrictionto{\sigma}{\alpha}})&&
	\text{when }\alpha\text{ is a limit ordinal}
\end{align*}
\end{thm}

\begin{proof}
Define a term $\xi(x)$ by:
\[
	\xi(x) =
	\begin{cases}
		A & \text{if $x$ is not a function whose}\\
		  & \text{\quad domain is an ordinal; otherwise:}\\
		\tau(x(\alpha)) & \text{if $\dom{x} = \ordsucc{\alpha}$}\\
		\theta(\ran{x}) & \text{if $\dom{x}$ is a limit ordinal}
	\end{cases}
\]
By Theorem~\ref{DEF-SET006}, there is a term $\sigma(x)$ such that
$\sigma(\alpha) = \xi(\funrestrictionto{\sigma}{\alpha})$ for every
ordinal $\alpha$; moreover, $\funrestrictionto{\sigma}{\alpha}$ is a
function with domain $\alpha$. A Simple Transfinite Induction
(Theorem~\ref{SET.3:simpletransrecursion}) confirms:
$\sigma(\emptyset) = \xi(\emptyset) = A$;
$\sigma(\ordsucc{\alpha}) =
\xi(\funrestrictionto{\sigma}{\ordsucc{\alpha}}) =
\tau(\sigma(\alpha))$; and when $\alpha$ is a limit,
$\sigma(\alpha) = \xi(\funrestrictionto{\sigma}{\alpha}) =
\theta(\ran{\funrestrictionto{\sigma}{\alpha}})$.
\end{proof}


%%% -----------------------------------------------------------------
%%% SET.3.8  Hartogs' Lemma
%%% -----------------------------------------------------------------

\subsection{Hartogs' Lemma}

\begin{lem}[Hartogs' Lemma, in $\ZF$] \label{SET.3:HartogsLemma}
For any set $A$, there is an ordinal $\alpha$ such that
$\cardnless{\alpha}{A}$.
\end{lem}

\begin{proof}[Proof sketch]
Using Separation, consider:
\[
	C = \Setabs{\tuple{B, R}}{B \subseteq A
	\text{ and $\tuple{B, R}$ is a well-ordering}}.
\]
Using Replacement and
Theorem~\ref{SET.3:thmOrdinalRepresentation}, form
$\alpha = \Setabs{\ordtype{B, R}}{\tuple{B, R} \in C}$.
By Corollary~\ref{SET.3:corordtransitiveord}, $\alpha$ is an ordinal
(it is a transitive set of ordinals). If there were an injection $f
\colon \alpha \to A$, then $\alpha = \ordtype{\ran{f}, R}$ for a
suitable $R$, giving $\alpha \in \alpha$---a contradiction.
\end{proof}

\begin{rem}[Hartogs' Number] \label{SET.3:hartogsrem}
The ordinal $\alpha$ produced in the proof is called the \emph{Hartogs
number} of $A$, sometimes written $\aleph(A)$. It is the least ordinal
that does not inject into $A$.
\end{rem}


%% ===================================================================
%% SET.4: Cardinals
%% Sources: sth/cardinals/cp (CONDENSE), sth/cardinals/cardsasords (KEEP),
%%          sth/cardinals/classing (CONDENSE),
%%          sth/choice/tarskiscott (CONDENSE),
%%          sth/cardinals/alephs (CONDENSE if exists)
%% ===================================================================

\section{Cardinals} \label{SET.4}

In SET.3, we introduced ordinals to calibrate \emph{well-orderings}.
Two well-orderings have the same order type iff they are isomorphic. We
now turn to a simpler notion: the \emph{size} of a set. Two sets have
the same size iff they are equinumerous, i.e., iff there is a bijection
between them (see Chapter~\ref{ch:bst}). Just as we introduced ordinals
to calibrate order types, we now introduce \emph{cardinals} to
calibrate size. Writing $\card{X}$ for the cardinality of $X$, we want
cardinals to satisfy \emph{Cantor's Principle}:
\[
	\card{A} = \card{B} \text{ iff } \cardeq{A}{B}.
\]


%%% -----------------------------------------------------------------
%%% SET.4.1  Cardinals as Ordinals
%%% -----------------------------------------------------------------

\subsection{Cardinals as Ordinals}

Our theory of cardinals makes shameless use of our theory of ordinals:
we define cardinals as certain specific ordinals.

\begin{defn}[Cardinal Number] % DEF-SET007, DEF-SET008
\label{DEF-SET007}
\label{DEF-SET008}
If $A$ can be well-ordered, then $\card{A}$ is the least ordinal
$\gamma$ such that $\cardeq{A}{\gamma}$. For any ordinal $\gamma$, we
say that $\gamma$ is a \emph{cardinal} iff $\gamma = \card{\gamma}$.
\end{defn}

There is a snag with Definition~\ref{DEF-SET007}: we would like
$\card{A}$ to exist for \emph{every} set $A$, but the definition begins
with a conditional---``if $A$ can be well-ordered''. If some set cannot
be well-ordered, the definition fails to define $\card{A}$. So we need a
guarantee that every set can be well-ordered:

\begin{axiom}[Well-Ordering] % AX-SET009
\label{AX-SET009}
Every set can be well-ordered.
\end{axiom}

This guarantee is unavailable in $\ZF$ alone. The Well-Ordering axiom
is the final axiom of $\ZFC$ (see Remark~\ref{SET.2:zfc}). Its
equivalence to the Axiom of Choice will be established in SET.6
(Theorem~\ref{THM-SET001}).

Using Well-Ordering, it is straightforward to show that cardinals exist
and behave well:

\begin{lem} \label{SET.4:CardinalsExist}
For every set $A$: (1) $\card{A}$ exists and is unique;
(2) $\cardeq{\card{A}}{A}$; (3) $\card{A}$ is a cardinal, i.e.,
$\card{A} = \card{\card{A}}$.
\end{lem}

\begin{proof}
Fix $A$. By Well-Ordering (AX-SET009), there is a well-ordering
$\tuple{A, R}$. By Theorem~\ref{SET.3:thmOrdinalRepresentation},
$\tuple{A, R}$ is isomorphic to a unique ordinal $\beta$, so
$\cardeq{A}{\beta}$. By Transfinite Induction, there is a uniquely
least ordinal $\gamma$ such that $\cardeq{A}{\gamma}$. So $\card{A} =
\gamma$, establishing (1) and (2). For (3), if $\delta \in \gamma$ then
$\cardless{\delta}{A}$ by choice of $\gamma$, so also
$\cardless{\delta}{\gamma}$ since equinumerosity is an equivalence
relation. So $\gamma = \card{\gamma}$.
\end{proof}

\begin{lem} \label{SET.4:CardinalsBehaveRight}
For any sets $A$ and $B$:
\begin{align*}
	\cardeq{A}{B} &\text{ iff } \card{A} = \card{B}\\
	\cardle{A}{B} &\text{ iff } \card{A} \leq \card{B}\\
	\cardless{A}{B}&\text{ iff } \card{A} < \card{B}
\end{align*}
\end{lem}

\begin{proof}
We prove left-to-right of the second claim; the other cases are
similar. If $\cardle{A}{B}$, there is an injection $A \to B$. By
Lemma~\ref{SET.4:CardinalsExist}, there are bijections $\card{A} \to A$
and $B \to \card{B}$. Composing, we obtain an injection $\card{A} \to
\card{B}$, giving $\card{A} \leq \card{B}$.
\end{proof}

\begin{rem}[Cantor's Principle] \label{SET.4:cantorprinciple}
Lemma~\ref{SET.4:CardinalsBehaveRight} guarantees Cantor's Principle:
$\card{A} = \card{B}$ iff $\cardeq{A}{B}$. It also yields a quick
re-proof of Schr\"oder--Bernstein: if $\cardle{A}{B}$ and
$\cardle{B}{A}$ then $\card{A} \leq \card{B}$ and $\card{B} \leq
\card{A}$, so $\card{A} = \card{B}$ by Trichotomy. (This implicitly
uses Replacement and Well-Ordering; the proof in Chapter~\ref{ch:bst}
works in $\Zminus$.)
\end{rem}


%%% -----------------------------------------------------------------
%%% SET.4.2  Finite, Infinite, and Uncountable Cardinals
%%% -----------------------------------------------------------------

\subsection{Finite, Infinite, and Uncountable Cardinals}

\begin{defn}[Finite and Infinite Sets] \label{SET.4:defnfinite}
We say that $A$ is \emph{finite} iff $\card{A} \in \omega$, i.e.,
$\card{A}$ is a natural number. Otherwise, $A$ is \emph{infinite}.
\end{defn}

Note that this definition assumes $\ZFC$, since we need Well-Ordering
to guarantee $\card{A}$ exists. Without Well-Ordering, there can be
sets that are neither finite nor Dedekind infinite (see SET.6 for the
role of Choice).

\begin{cor} \label{SET.4:omegaisacardinal}
$\omega$ is the least infinite cardinal.
\end{cor}

\begin{proof}
$\omega$ is a cardinal: it is Dedekind infinite, and if
$\cardeq{\omega}{n}$ for any $n \in \omega$, then $n$ would be Dedekind
infinite, a contradiction. Now $\omega$ is the least infinite cardinal
by definition, since the finite cardinals are exactly the natural
numbers.
\end{proof}

\begin{thm} \label{SET.4:NoLargestCardinal}
There is no largest cardinal.
\end{thm}

\begin{proof}[Proof sketch]
For any cardinal $\cardfont{a}$, Cantor's Theorem (THM-BST001,
Chapter~\ref{ch:bst}) gives $\cardless{\cardfont{a}}{\Pow{\cardfont{a}}}$.
By Lemma~\ref{SET.4:CardinalsExist}, $\cardfont{a} <
\card{\Pow{\cardfont{a}}}$.
\end{proof}

\begin{prop} \label{SET.4:unioncardinalscardinal}
If every member of $X$ is a cardinal, then $\bigcup X$ is a cardinal.
\end{prop}

\begin{proof}
It is easy to check that $\bigcup X$ is an ordinal. If $\alpha \in
\bigcup X$, then $\alpha \in \cardfont{b} \in X$ for some cardinal
$\cardfont{b}$. Since $\cardfont{b}$ is a cardinal,
$\cardless{\alpha}{\cardfont{b}}$. Since $\cardfont{b} \subseteq
\bigcup X$, we have $\cardle{\cardfont{b}}{\bigcup X}$, and so
$\cardneq{\alpha}{\bigcup X}$. Generalising, $\bigcup X$ is a cardinal.
\end{proof}


%%% -----------------------------------------------------------------
%%% SET.4.3  Tarski--Scott Cardinals (Without Well-Ordering)
%%% -----------------------------------------------------------------

\subsection{Cardinals without Well-Ordering}

\begin{rem}[Tarski--Scott Trick] \label{SET.4:tarskiscott}
Cardinals can be developed \emph{without} Well-Ordering using the
\emph{Tarski--Scott trick}: for any formula $\phi(x)$, let $[x :
\phi(x)]$ be the set of all $x$ of least possible rank such that
$\phi(x)$, where the \emph{rank} of a set is its position in the
cumulative hierarchy $V_0 \subset V_1 \subset \cdots$
(see \S\ref{SET.5}).
Working in $\ZF$, one defines the \textsc{ts}-cardinality of
$A$ as $\text{tsc}(A) = [x : \cardeq{A}{x}]$. This yields a
well-behaved notion of cardinality without assuming Well-Ordering, at
the cost of defining cardinals as \emph{sets of sets} rather than
ordinals.
\end{rem}


%% ===================================================================
%% SET.5: Advanced Topics
%% Sources: sth/spine/valpha (KEEP), sth/spine/recursion (KEEP),
%%          sth/spine/Valphabasic, rank, separation, height (CONDENSE),
%%          sth/ord-arithmetic/addition, mult, expo, using-addition (CONDENSE),
%%          sth/card-arithmetic/opps (KEEP), ch (KEEP),
%%          sth/card-arithmetic/simp, expotough, fix (CONDENSE),
%%          sth/replacement/strength, ref, refproofs, finiteaxiom (CONDENSE),
%%          sth/choice/countablechoice, justifications (CONDENSE)
%% ===================================================================

\section{Advanced Topics} \label{SET.5}

We now develop the major advanced topics of formal set theory: the von
Neumann hierarchy $V_\alpha$, ordinal arithmetic, cardinal arithmetic,
the Continuum Hypothesis, and the role of the Axiom of Choice.


%%% -----------------------------------------------------------------
%%% SET.5.1  The Cumulative Hierarchy
%%% -----------------------------------------------------------------

\subsection{The Cumulative Hierarchy}

With the machinery of transfinite recursion in hand, we can give a
formal, \emph{internal} characterisation of the stages of the hierarchy.

\begin{defn}[Von Neumann Hierarchy] % DEF-SET012
\label{DEF-SET012}
\begin{align*}
	V_\emptyset &\defis \emptyset\\
	V_{\ordsucc{\alpha}} &\defis \Pow{V_\alpha} &&
	\text{for any ordinal }\alpha\\
	V_{\alpha} &\defis \bigcup_{\gamma < \alpha} V_\gamma &&
	\text{when }\alpha\text{ is a limit ordinal}
\end{align*}
\end{defn}

This definition is legitimate by Simple Recursion
(Theorem~\ref{SET.3:simplerecursionschema}): take $A = \emptyset$,
$\tau(x) = \Pow{x}$, and $\theta(x) = \bigcup x$.

\begin{lem} \label{SET.5:Valphabasicprops}
For each ordinal $\alpha$: (1) $V_\alpha$ is transitive; (2) $V_\alpha$
is potent (if $x \subseteq y \in V_\alpha$ then $x \in V_\alpha$);
(3) if $\gamma \in \alpha$, then $V_\gamma \in V_\alpha$ (and hence
$V_\gamma \subseteq V_\alpha$).
\end{lem}

\begin{proof}
By simultaneous transfinite induction on $\alpha$. The case $\alpha =
\emptyset$ is trivial. For successor $\alpha = \ordsucc{\beta}$:
(3) if $\gamma \in \alpha$ then $V_\gamma \subseteq V_\beta$ by
hypothesis, so $V_\gamma \in \Pow{V_\beta} = V_\alpha$; (2) if $A
\subseteq B \in V_\alpha$ then $A \subseteq V_\beta$, so $A \in
V_\alpha$; (1) if $x \in A \in V_\alpha$ then $x \in V_\beta$ and $x
\subseteq V_\beta$ by the induction hypothesis, so $x \in V_\alpha$.
For limit $\alpha$: (3) if $\gamma \in \alpha$ then $V_\gamma \in
V_{\ordsucc{\gamma}} \subseteq V_\alpha$; (1) and (2) hold because a
union of transitive (resp.\ potent) sets is transitive (resp.\ potent).
\end{proof}

\begin{defn}[Rank] \label{SET.5:defnsetrank}
For each set $A$, $\setrank{A}$ is the least ordinal $\alpha$ such that
$A \subseteq V_\alpha$.
\end{defn}

Ranks exist by Foundation (AX-SET008, \S\ref{AX-SET008}) and
Regularity (Theorem~\ref{SET.2:zfentailsregularity}).

\begin{prop} \label{SET.5:rankmemberslower}
If $B \in A$, then $\setrank{B} \in \setrank{A}$.
\end{prop}

\begin{thm}[$\in$-Induction Scheme] \label{SET.5:eininduction}
For any formula $\phi$:
\[
	\forall A((\forall x \in A)\phi(x) \lif \phi(A)) \lif \forall A\, \phi(A).
\]
\end{thm}

\begin{proof}
Suppose $\lnot\forall A\, \phi(A)$. By Transfinite Induction, there is
some $A$ of least rank with $\lnot\phi(A)$. If $x \in A$ then
$\setrank{x} \in \setrank{A}$ by Proposition~\ref{SET.5:rankmemberslower},
so $\phi(x)$. Hence $(\forall x \in A)\phi(x)$ and $\lnot\phi(A)$.
\end{proof}

\begin{prop} \label{SET.5:ordsetrankalpha}
$\setrank{\alpha} = \alpha$ for any ordinal $\alpha$.
\end{prop}


%%% -----------------------------------------------------------------
%%% SET.5.2  Ordinal Arithmetic
%%% -----------------------------------------------------------------

\subsection{Ordinal Arithmetic}

We define addition, multiplication, and exponentiation on ordinals. Each
can be defined synthetically (via an explicit well-ordered set and its
order type) or recursively (via transfinite recursion equations). Both
approaches yield the same result by
Theorem~\ref{SET.3:thmOrdinalRepresentation}.

\begin{defn}[Ordinal Addition] \label{SET.5:defordplus}
The \emph{disjoint sum} of $A$ and $B$ is $A \disjointsum B = (A \times
\{0\}) \cup (B \times \{1\})$. Define the \emph{reverse lexicographic
ordering} $\rlexless$ on $\alpha \disjointsum \beta$ by:
$\tuple{\alpha_1, \alpha_2} \rlexless \tuple{\beta_1, \beta_2}$ iff
either $\alpha_2 \in \beta_2$, or both $\alpha_2 = \beta_2$ and
$\alpha_1 \in \beta_1$. Then $\alpha \ordplus \beta =
\ordtype{\alpha \disjointsum \beta, \rlexless}$.
\end{defn}

Ordinal addition satisfies the following recursion equations:
\begin{align*}
	\alpha \ordplus 0 &= \alpha\\
	\alpha \ordplus (\beta \ordplus 1) &= (\alpha \ordplus \beta) \ordplus 1\\
	\alpha \ordplus \beta &= \supstrict_{\delta < \beta}(\alpha \ordplus \delta)
	&& \text{if $\beta$ is a limit ordinal}
\end{align*}

Addition is associative: $\alpha \ordplus (\beta \ordplus \gamma) =
(\alpha \ordplus \beta) \ordplus \gamma$. It is \emph{not} commutative:
$1 \ordplus \omega = \omega < \omega \ordplus 1$. Intuitively, placing
one element \emph{before} an $\omega$-sequence does not change the order
type (by a Hilbert's Hotel argument), but placing one element
\emph{after} it does.

\begin{defn}[Ordinal Multiplication] \label{SET.5:defordtimes}
$\alpha \ordtimes \beta = \ordtype{\alpha \times \beta, \rlexless}$.
Equivalently, by transfinite recursion:
\begin{align*}
	\alpha \ordtimes 0 &= 0\\
	\alpha \ordtimes (\beta \ordplus 1) &=
		(\alpha \ordtimes \beta) \ordplus \alpha\\
	\alpha \ordtimes \beta &=
		\supstrict_{\delta < \beta}(\alpha \ordtimes \delta) &&
		\text{when $\beta$ is a limit ordinal}
\end{align*}
\end{defn}

Multiplication is associative but \emph{not} commutative: $2 \ordtimes
\omega = \omega < \omega \ordtimes 2$. It distributes over addition from
the right: $\alpha \ordtimes (\beta \ordplus \gamma) = (\alpha
\ordtimes \beta) \ordplus (\alpha \ordtimes \gamma)$.

\begin{defn}[Ordinal Exponentiation] \label{SET.5:defordexpo}
By transfinite recursion:
\begin{align*}
	\ordexpo{\alpha}{0} &= 1\\
	\ordexpo{\alpha}{\beta \ordplus 1} &= \ordexpo{\alpha}{\beta}
	\ordtimes \alpha\\
	\ordexpo{\alpha}{\beta} &= \bigcup_{\delta < \beta}
	\ordexpo{\alpha}{\delta} && \text{when $\beta$ is a limit ordinal}
\end{align*}
\end{defn}

Ordinal exponentiation does not commute: $\ordexpo{2}{\omega} =
\bigcup_{n < \omega} \ordexpo{2}{n} = \omega$, whereas
$\ordexpo{\omega}{2} = \omega \ordtimes \omega$.

\begin{lem}[Characterisation of Infinite Ordinals]
\label{SET.5:ordinfinitycharacter}
For any ordinal $\alpha$, the following are equivalent:
(1) $\alpha \notin \omega$; (2) $\omega \leq \alpha$;
(3) $1 \ordplus \alpha = \alpha$; (4) $\alpha$ and $\alpha \ordplus 1$
are equinumerous; (5) $\alpha$ is Dedekind infinite.
\end{lem}

\begin{proof}
$(1) \Rightarrow (2)$: By Trichotomy.
$(2) \Rightarrow (3)$: Write $\alpha = \beta \ordplus \gamma$ where
$\beta$ is a limit ordinal and $\gamma$ is least. Then $1 \ordplus
\alpha = (1 \ordplus \beta) \ordplus \gamma = \beta \ordplus \gamma =
\alpha$, using the fact that $1 \ordplus \beta =
\supstrict_{\delta < \beta}(1 \ordplus \delta) = \beta$.
$(3) \Rightarrow (4)$: There is a bijection $\alpha \disjointsum 1 \to
1 \disjointsum \alpha$; compose with the isomorphism $1 \disjointsum
\alpha \to \alpha$.
$(4) \Rightarrow (5)$: A bijection $\alpha \disjointsum 1 \to \alpha$
restricts to an injection $\alpha \to \alpha$ that is not surjective.
$(5) \Rightarrow (1)$: No natural number is Dedekind infinite.
\end{proof}


%%% -----------------------------------------------------------------
%%% SET.5.3  Cardinal Arithmetic
%%% -----------------------------------------------------------------

\subsection{Cardinal Arithmetic}

Since we do not need to keep track of order, cardinal arithmetic is
rather easier to define than ordinal arithmetic.

\begin{defn}[Cardinal Operations] \label{SET.5:defcardops}
When $\cardfont{a}$ and $\cardfont{b}$ are cardinals:
\begin{align*}
	\cardfont{a} \cardplus \cardfont{b} &\defis
	\card{\cardfont{a} \disjointsum \cardfont{b}}\\
	\cardfont{a} \cardtimes \cardfont{b} &\defis
	\card{\cardfont{a} \times \cardfont{b}}\\
	\cardexpo{\cardfont{a}}{\cardfont{b}} &\defis
	\card{\funfromto{\cardfont{b}}{\cardfont{a}}}
\end{align*}
where $\funfromto{X}{Y} = \Setabs{f}{f\text{ is a function }X \to Y}$.
\end{defn}

Cardinal addition and multiplication are commutative and associative
(unlike their ordinal counterparts).

\begin{lem} \label{SET.5:SizePowerset2Exp}
$\card{\Pow{A}} = \cardexpo{2}{\card{A}}$, for any $A$.
\end{lem}

\begin{proof}
For each $B \subseteq A$, define the characteristic function $\chi_B
\in \funfromto{A}{2}$ by $\chi_B(x) = 1$ if $x \in B$, and
$\chi_B(x) = 0$ otherwise. The map $B \mapsto \chi_B$ is a bijection
$\Pow{A} \to \funfromto{A}{2}$.
\end{proof}

\begin{cor}[Cantor's Theorem in Cardinal Arithmetic] % THM-SET003, DEF-SET011
\label{THM-SET003}\label{DEF-SET011}
$\cardfont{a} < \cardexpo{2}{\cardfont{a}}$ for any
cardinal~$\cardfont{a}$.
\end{cor}

\begin{proof}
From Cantor's Theorem (THM-BST001, Chapter~\ref{ch:bst}) and
Lemma~\ref{SET.5:SizePowerset2Exp}.
\end{proof}

\begin{thm} \label{SET.5:continuumis2aleph0}
$\card{\Real} = \cardexpo{2}{\omega}$.
\end{thm}

\begin{proof}[Proof skeleton]
Show $\cardle{\Pow{\omega}}{\Real}$ and $\cardle{\Real}{\Pow{\omega}}$;
apply Schr\"oder--Bernstein to get $\cardeq{\Real}{\Pow{\omega}}$; then
$\card{\Real} = \card{\Pow{\omega}} = \cardexpo{2}{\omega}$ by
Lemma~\ref{SET.5:SizePowerset2Exp}.
\end{proof}

\begin{thm}[Simplification of Addition and Multiplication]
\label{SET.5:cardplustimesmax}
If $\cardfont{a}, \cardfont{b}$ are infinite cardinals, then
$\cardfont{a} \cardtimes \cardfont{b} = \cardfont{a} \cardplus
\cardfont{b} = \max(\cardfont{a}, \cardfont{b})$.
\end{thm}

\begin{proof}[Proof sketch]
The key step is showing $\cardeq{\alpha}{\alpha \times \alpha}$ for any
infinite ordinal $\alpha$. This uses the \emph{canonical ordering}
$\canonord$ on pairs: $\tuple{\alpha_1, \alpha_2} \canonord
\tuple{\beta_1, \beta_2}$ iff either $\max(\alpha_1, \alpha_2) <
\max(\beta_1, \beta_2)$, or they have the same max and $\alpha_1 <
\beta_1$, or same max and first coordinate and $\alpha_2 < \beta_2$.
This is a well-ordering of $\alpha \times \alpha$. For the least
infinite ordinal $\alpha$ for which $\cardeq{\alpha}{\alpha \times
\alpha}$ fails, one shows $\alpha$ must be a cardinal, and each segment
$\text{Seg}(\gamma_1, \gamma_2)$ in the canonical ordering has
cardinality $< \alpha$, giving $\ordtype{\alpha \times \alpha,
\canonord} \leq \alpha$, hence $\cardle{\alpha \times \alpha}{\alpha}$.
The reverse injection is obvious, so Schr\"oder--Bernstein applies.

Given this, without loss of generality let $\cardfont{a} =
\max(\cardfont{a}, \cardfont{b})$. Then $\cardfont{a} \cardtimes
\cardfont{a} = \cardfont{a} \leq \cardfont{a} \cardplus \cardfont{b}
\leq \cardfont{a} \cardplus \cardfont{a} \leq \cardfont{a} \cardtimes
\cardfont{a}$.
\end{proof}

\begin{prop} \label{SET.5:kappaunionkappasize}
Let $\cardfont{a}$ be an infinite cardinal. For each $\beta \in
\cardfont{a}$, let $X_\beta$ be a set with $\card{X_\beta} \leq
\cardfont{a}$. Then $\card{\bigcup_{\beta \in \cardfont{a}} X_\beta}
\leq \cardfont{a}$.
\end{prop}

\begin{proof}[Proof sketch]
Each $X_\beta$ injects into $\cardfont{a}$, so $\bigcup_\beta X_\beta$
injects into $\cardfont{a} \times \cardfont{a}$; by the preceding
theorem, $\card{\cardfont{a} \times \cardfont{a}} = \cardfont{a}$.
\end{proof}


%%% -----------------------------------------------------------------
%%% SET.5.4  Aleph and Beth Numbers
%%% -----------------------------------------------------------------

\subsection{Aleph and Beth Numbers}

\begin{defn}[Aleph and Beth Numbers] % DEF-SET013
\label{DEF-SET013}
Where $\cardsucc{\cardfont{a}}$ is the least cardinal strictly greater
than $\cardfont{a}$, we define two sequences by transfinite recursion:
\begin{align*}
	\aleph_{0} &\defis \omega &
	\beth_{0} &\defis \omega\\
	\aleph_{\alpha \ordplus 1} &\defis \cardsucc{(\aleph_{\alpha})} &
	\beth_{\alpha+1} &\defis \cardexpo{2}{\beth_{\alpha}}\\
	\aleph_{\alpha} &\defis \bigcup_{\beta < \alpha} \aleph_{\beta} &
	\beth_{\alpha} &\defis \bigcup_{\beta < \alpha}\beth_{\beta}
	& \text{when $\alpha$ is a limit ordinal}
\end{align*}
\end{defn}

The definition of $\cardsucc{\cardfont{a}}$ is in order: for each
cardinal $\cardfont{a}$, there is some cardinal greater than
$\cardfont{a}$ (Theorem~\ref{SET.4:NoLargestCardinal}), and Transfinite
Induction gives the \emph{least} such. The ``$\aleph$'' notation is due
to Cantor; ``$\beth$'' is due to Peirce.

\begin{prop}
$\aleph_\alpha$ and $\beth_\alpha$ are cardinals for every ordinal
$\alpha$. Moreover, every infinite cardinal is an $\aleph$: if
$\cardfont{a}$ is an infinite cardinal, then $\cardfont{a} =
\aleph_\gamma$ for some unique $\gamma$.
\end{prop}

\begin{proof}
By transfinite induction. $\aleph_0 = \beth_0 = \omega$ is a cardinal
by Corollary~\ref{SET.4:omegaisacardinal}; successors are cardinals by
definition; limits are cardinals by
Proposition~\ref{SET.4:unioncardinalscardinal}.

For the second claim, induct on cardinals. If $\cardfont{a}$ is the
successor of some $\cardfont{b} = \aleph_\gamma$, then $\cardfont{a} =
\aleph_{\gamma+1}$. If $\cardfont{a}$ is not a cardinal successor, then
$\cardfont{a} = \bigcup_{\cardfont{b} < \cardfont{a}} \cardfont{b} =
\bigcup_{\cardfont{b} < \cardfont{a}} \aleph_{\gamma_\cardfont{b}} =
\aleph_\gamma$ for a suitable limit $\gamma$.
\end{proof}


%%% -----------------------------------------------------------------
%%% SET.5.5  The Continuum Hypothesis
%%% -----------------------------------------------------------------

\subsection{The Continuum Hypothesis}

Since every infinite cardinal is an $\aleph$, we ask: is every infinite
cardinal a $\beth$? If so, infinite cardinals would ``play
straightforwardly'' with powersets:

\begin{defish}[Generalised Continuum Hypothesis] % DEF-SET015
\label{DEF-SET015}
GCH: $\aleph_\alpha = \beth_\alpha$, for all $\alpha$.
\end{defish}

If GCH held, cardinal exponentiation could be completely determined: for
$\cardfont{b} < \cardfont{a}$, the value $\cardexpo{\cardfont{a}}{\cardfont{b}}$
would be trapped by $\cardfont{a} \leq
\cardexpo{\cardfont{a}}{\cardfont{b}} \leq \cardsucc{\cardfont{a}}$.

But GCH is a \emph{hypothesis}, not a theorem. G\"odel (1938) proved
that if $\ZFC$ is consistent, then so is $\ZFC + \text{GCH}$. The
simplest non-trivial instance is:

\begin{defish}[Continuum Hypothesis] \label{SET.5:CH}
CH: $\aleph_1 = \beth_1$.
\end{defish}

Cohen (1963) proved that if $\ZFC$ is consistent, then so is $\ZFC +
\lnot\text{CH}$. So the Continuum Hypothesis is \emph{independent} from
$\ZFC$.

The Continuum Hypothesis is so-called because ``the continuum'' is
another name for the real line $\Real$.
Theorem~\ref{SET.5:continuumis2aleph0} tells us $\card{\Real} = \beth_1$.
So CH states there is no cardinal between $\aleph_0 = \beth_0$ (the
cardinality of the natural numbers) and $\beth_1$ (the cardinality of
the continuum).

Two observations are worth emphasising. First, it does not immediately
follow from independence that CH is indeterminate in truth value: perhaps
additional natural axioms will settle it. G\"odel himself suggested this
response. Second, the independence of CH is striking but not incredible:
for all $\ZFC$ tells us, moving from a cardinal to its successor may
involve a more refined tool than simply taking powersets.


%%% -----------------------------------------------------------------
%%% SET.5.6  The Strength of Replacement
%%% -----------------------------------------------------------------

\subsection{The Strength of Replacement}

We briefly note the strength of the Replacement scheme.

\begin{thm} \label{SET.5:Znotomegaomega}
$\Z$ is consistent with the non-existence of $\omega + \omega$.
\end{thm}

\begin{proof}[Proof sketch]
Working in $\ZF$, consider $V_{\omega + \omega}$. One can show that for
every axiom $\phi$ of $\Z$, we have $\ZF \vdash \phi^{V_{\omega +
\omega}}$ (where $\phi^M$ restricts all quantifiers to $M$). But $\omega
+ \omega \notin V_{\omega + \omega}$ (since $\setrank{\omega + \omega} =
\omega + \omega$ by Proposition~\ref{SET.5:ordsetrankalpha}). So $\Z$
does not prove the existence of $\omega + \omega$.
\end{proof}

This is why Theorem~\ref{SET.3:thmOrdinalRepresentation} cannot be
proved without Replacement: within $\Z$ one can define a well-ordering
of order type $\omega + \omega$, but if $\omega + \omega$ does not
exist, this well-ordering is not isomorphic to any ordinal.

More broadly, Replacement forces the hierarchy to be very tall.

\begin{thm}[Reflection Schema] \label{SET.5:reflectionschema}
For any formula $\phi$:
\[
\forall \alpha \exists \beta > \alpha\, (\forall x_1, \ldots, x_n \in
V_\beta)(\phi(x_1, \ldots, x_n) \liff \phi^{V_\beta}(x_1, \ldots, x_n))
\]
\end{thm}

Montague (1961) and L\'evy (1960) showed that Replacement and
Reflection are equivalent modulo $\Z$, so adding either gives $\ZF$.

\begin{thm} \label{SET.5:zfnotfinitely}
$\ZF$ is not finitely axiomatizable: if $\mathcal{T}$ is finite and
$\mathcal{T} \vdash \ZF$, then $\mathcal{T}$ is inconsistent.
\end{thm}

\begin{proof}[Proof sketch]
By Reflection, for any finite set of sentences $\mathcal{T} \subseteq
\ZF$ there is a $\beta$ such that $V_\beta \models \mathcal{T}$.
If $\mathcal{T}$ axiomatised all of $\ZF$, then $\ZF$ would prove
``there exists a set model of $\mathcal{T}$,'' hence
$\ZF \vdash \mathrm{Con}(\mathcal{T}) = \mathrm{Con}(\ZF)$.
By the Second Incompleteness Theorem, $\ZF$ is then inconsistent.
\end{proof}


%%% -----------------------------------------------------------------
%%% SET.5.7  Aleph-Fixed Points
%%% -----------------------------------------------------------------

\subsection{Fixed Points}

The following results illustrate just how tall Replacement forces the
hierarchy to be.

\begin{prop}[$\aleph$-Fixed Point] \label{SET.5:alephfixed}
There is a cardinal $\kappa$ such that $\kappa = \aleph_\kappa$.
\end{prop}

\begin{proof}
Define by recursion: $\kappa_0 = 0$; $\kappa_{n+1} =
\aleph_{\kappa_n}$; $\kappa = \bigcup_{n < \omega} \kappa_n$. Then
$\kappa$ is a cardinal by Proposition~\ref{SET.4:unioncardinalscardinal},
and $\kappa = \bigcup_{n < \omega} \kappa_{n+1} = \bigcup_{n < \omega}
\aleph_{\kappa_n} = \bigcup_{\alpha < \kappa} \aleph_\alpha =
\aleph_\kappa$.
\end{proof}

\begin{prop}[$\beth$-Fixed Point] \label{SET.5:bethfixed}
There is a $\kappa$ such that $\kappa = \beth_\kappa$.
\end{prop}

\begin{proof}
As in Proposition~\ref{SET.5:alephfixed}, using ``$\beth$'' in place of
``$\aleph$''.
\end{proof}

\begin{prop} \label{SET.5:stagesize}
$\card{V_{\omega + \alpha}} = \beth_\alpha$. If $\omega \ordtimes \omega
\leq \alpha$, then $\card{V_\alpha} = \beth_\alpha$.
\end{prop}

\begin{proof}[Proof sketch]
By transfinite induction on $\alpha$.  For the base,
$V_\omega$ is countable, so $\card{V_\omega} = \aleph_0 = \beth_0$.
At successor stages,
$\card{V_{\omega+\alpha+1}} = 2^{\card{V_{\omega+\alpha}}}
= 2^{\beth_\alpha} = \beth_{\alpha+1}$.
At limits, take the union.
\end{proof}

\begin{cor}
There is a $\kappa$ such that $\card{V_\kappa} = \kappa$.
\end{cor}

\begin{proof}
Let $\kappa$ be a $\beth$-fixed point (Proposition~\ref{SET.5:bethfixed}).
Then $\card{V_\kappa} = \beth_\kappa = \kappa$ by
Proposition~\ref{SET.5:stagesize}.
\end{proof}

Intuitively, $V_\kappa$ is ``as wide as it is tall'': there are as many
stages beneath it as there are elements of it. This is a
Tristram-Shandy phenomenon: we move from one stage to the next by taking
powersets, making the hierarchy much wider with each step, yet ``in the
end'' the width catches up with the height.


%%% -----------------------------------------------------------------
%%% SET.5.8  Choice and Countable Choice
%%% -----------------------------------------------------------------

\subsection{The Axiom of Choice}

The Axiom of Well-Ordering (AX-SET009, \S\ref{AX-SET009}) has been
essential to our development of cardinal arithmetic. We now discuss its
justification and its relationship to other principles.

\begin{defn}[Choice Function] \label{SET.5:defchoicefun}
A function $f$ is a \emph{choice function} iff $f(x) \in x$ for all $x
\in \dom{f}$. We say $f$ is a choice function \emph{for} $A$ iff
$\dom{f} = A \setminus \{\emptyset\}$.
\end{defn}

\begin{axiom}[Choice] \label{SET.5:axiomchoice}
Every set has a choice function.
\end{axiom}

The equivalence of Choice and Well-Ordering is established in SET.6
(Theorem~\ref{THM-SET001}).

\begin{defn}[Zorn's Lemma] % DEF-SET010
\label{DEF-SET010}
A \emph{chain} in a partially ordered set $P$ is a totally ordered
subset of~$P$.
\emph{Zorn's Lemma} states: if every chain in a partially ordered set
$P$ has an upper bound in $P$, then $P$ has a maximal element.
\end{defn}

Zorn's Lemma is equivalent (in $\ZF$) to both Choice and
Well-Ordering (Theorem~\ref{THM-SET001}).

\begin{rem}[Justification of Choice] \label{SET.5:choicejust}
One intrinsic justification appeals to the cumulative-iterative
conception. Let $A$'s elements be disjoint and non-empty. By
\stageshier, $A$ is formed at some stage $S$. All elements of $\bigcup
A$ are available before $S$. By \stagesacc, every possible collection of
earlier-available sets exists at $S$. It is certainly \emph{possible} to
select one element from each member of $A$---a choice set for $A$. So
such a choice set exists. (For a careful development of this argument,
see Potter (\S14.8).)
\end{rem}

\begin{rem}[Countable Choice] \label{SET.5:countablechoice}
\emph{Countable Choice} states that every countable set has a choice
function. This special case was used frequently by 19th-century
mathematicians (including Dedekind and Cantor) without awareness of its
role. Two notable consequences requiring Countable Choice: (1) every
set is either finite or contains a countably infinite subset (Dedekind);
(2) a countable union of countable sets is countable (Cantor). Both fail
in $\ZF$ alone: Cohen proved it is consistent with $\ZF$ that some sets
are incomparable with $\omega$; Feferman and Levy proved it is
consistent with $\ZF$ that a countable union of countable sets has
cardinality $\beth_1$.
\end{rem}


%% ===================================================================
%% SET.6: Theorems
%% Sources: sth/choice/wellorderingproblem (KEEP),
%%          sth/choice/hartogs (CONDENSE)
%% ===================================================================

\section{Theorems} \label{SET.6}

We close this chapter by establishing the fundamental equivalences that
unify the theory.


%%% -----------------------------------------------------------------
%%% SET.6.1  Well-Ordering iff Choice
%%% -----------------------------------------------------------------

\subsection{Well-Ordering iff Choice}

\begin{thm}[in $\ZF$] % THM-SET001
\label{THM-SET001}
The following are equivalent:
\begin{enumerate}
	\item Well-Ordering (AX-SET009): every set can be well-ordered.
	\item Choice: every set has a choice function.
	\item Zorn's Lemma (DEF-SET010): if every chain in a partially
	ordered set $P$ has an upper bound in $P$, then $P$ has a maximal
	element.
\end{enumerate}
\end{thm}

\begin{proof}
\emph{Well-Ordering $\Rightarrow$ Choice.} Let $A$ be a set of sets.
Then $\bigcup A$ exists by Union, and by Well-Ordering there is some
$<$ which well-orders $\bigcup A$. Define $f(x) = \text{the $<$-least
member of }x$. This is a choice function for $A$.

\emph{Choice $\Rightarrow$ Well-Ordering.} Fix $A$. By Choice, there is
a choice function $f$ for $\Pow{A} \setminus \{\emptyset\}$. Using
Transfinite Recursion (Theorem~\ref{DEF-SET006}), define:
\begin{align*}
	g(0) &= f(A)\\
	g(\alpha) &=
		\begin{cases}
			\text{stop} & \text{if }A = \funimage{g}{\alpha}\\
			f(A \setminus \funimage{g}{\alpha}) & \text{otherwise}
		\end{cases}
\end{align*}
Since $f$ is a choice function, for each $\alpha$ (when defined) we have
$g(\alpha) = f(A \setminus \funimage{g}{\alpha}) \in A \setminus
\funimage{g}{\alpha}$, so $g(\alpha) \notin \funimage{g}{\alpha}$. If
$g(\alpha) = g(\beta)$ then $g(\beta) \notin \funimage{g}{\alpha}$,
i.e., $\beta \notin \alpha$, and similarly $\alpha \notin \beta$. So
$\alpha = \beta$ by Trichotomy, hence $g$ is injective.

We must stop: i.e., there is some least $\alpha$ with $A =
\funimage{g}{\alpha}$. For if not, then as $g$ is injective we would
have $\cardless{\alpha}{\Pow{A} \setminus \{\emptyset\}}$ for every
ordinal $\alpha$, contradicting Hartogs' Lemma
(Lemma~\ref{SET.3:HartogsLemma}).

Assembling these facts, $g$ is a bijection from some ordinal to $A$, and
can be used to well-order $A$.

\emph{Choice $\Leftrightarrow$ Zorn's Lemma.} This equivalence is a
standard result; the proof that Choice implies Zorn's Lemma uses
transfinite recursion to build a maximal chain. The converse constructs
a choice function by applying Zorn's Lemma to a suitable partial order
of partial choice functions.
\end{proof}


%%% -----------------------------------------------------------------
%%% SET.6.2  Comparability of Sets
%%% -----------------------------------------------------------------

\subsection{Comparability of Sets}

\begin{thm}[in $\ZF$] \label{SET.6:WOiffComparability}
The following are equivalent:
\begin{enumerate}
	\item Well-Ordering.
	\item Either $\cardle{A}{B}$ or $\cardle{B}{A}$, for any sets $A$
	and $B$.
\end{enumerate}
\end{thm}

\begin{proof}
\emph{(1) $\Rightarrow$ (2).} Fix $A$ and $B$. By Well-Ordering, there
are well-orderings $\tuple{A, R}$ and $\tuple{B, S}$. Let $f \colon
\alpha \to \tuple{A, R}$ and $g \colon \beta \to \tuple{B, S}$ be
isomorphisms. By Trichotomy, either $\alpha \subseteq \beta$ or $\beta
\subseteq \alpha$. If $\alpha \subseteq \beta$, then $g^{-1} \circ
(\funrestrictionto{f}{\alpha})$ is an injection $A \to B$, so
$\cardle{A}{B}$; similarly in the other case.

\emph{(2) $\Rightarrow$ (1).} Fix $A$; by Hartogs' Lemma
(Lemma~\ref{SET.3:HartogsLemma}) there is an ordinal $\beta$ with
$\cardnless{\beta}{A}$. By (2), $\cardle{A}{\beta}$. An injection $f
\colon A \to \beta$ induces a well-ordering on $A$ via $a_1 < a_2$ iff
$f(a_1) \in f(a_2)$.
\end{proof}

As an immediate consequence: if Well-Ordering fails, then some sets are
\emph{literally incomparable} with regard to their size, and transfinite
cardinal arithmetic becomes significantly more complex.


%%% -----------------------------------------------------------------
%%% Summary
%%% -----------------------------------------------------------------

\begin{rem}[Summary of SET.3--SET.6] \label{SET.6:summary}
Working in $\ZF$, we developed the ordinals (von Neumann's construction)
and proved the key structural theorems: Transfinite Induction
(Theorem~\ref{DEF-SET005}), Trichotomy
(Theorem~\ref{SET.3:ordtrichotomy}), Burali-Forti
(Theorem~\ref{SET.3:buraliforti}), and Ordinal Representation
(Theorem~\ref{SET.3:thmOrdinalRepresentation}). The machinery of
transfinite recursion (Theorems~\ref{DEF-SET006}
and~\ref{SET.3:simplerecursionschema}) enabled the definition of the
cumulative hierarchy $V_\alpha$ (Definition~\ref{DEF-SET012}) and
ordinal arithmetic. Adding the Well-Ordering axiom (AX-SET009,
\S\ref{AX-SET009}) yielded $\ZFC$ and a robust theory of cardinals.
Cardinal arithmetic
(Theorem~\ref{SET.5:cardplustimesmax}) simplifies dramatically for
infinite cardinals; but the Continuum Hypothesis
(Definition~\ref{DEF-SET015}) remains independent of $\ZFC$. The
equivalence of Well-Ordering, Choice, and Zorn's Lemma
(Theorem~\ref{THM-SET001}) unifies the theory.
\end{rem}
   % CH-SET: Formal Set Theory (Level-1)
\chapter{Extensions} \label{ch:ext}

The preceding seven chapters establish a self-contained development of
classical first-order logic: from set-theoretic foundations (CH-BST)
through syntax, semantics, deduction, computation, metatheory, and formal
set theory. This final chapter catalogues the principal directions in
which the core framework can be extended. Each topic area is the subject
of extensive treatment in the OpenLogic Project source material; we
provide brief orientations and pointers.


%% ===================================================================
%% EXT.1: Modal Logic
%% Sources: 69 normal-modal-logic/ + 16 applied-modal-logic/ sections
%% ===================================================================

\section{Modal Logic} \label{EXT.1}

Modal logic extends propositional and first-order logic with operators
expressing necessity~($\Box$) and possibility~($\Diamond$). The
semantics is given by Kripke structures (relational models): a set of
worlds, an accessibility relation, and a valuation at each world. A
formula~$\Box\varphi$ holds at a world~$w$ iff $\varphi$ holds at every
world accessible from~$w$. By varying the properties of the
accessibility relation---reflexive, transitive, symmetric, Euclidean---one
obtains the standard normal modal systems K, T, S4, and S5.
Completeness, correspondence theory (frame conditions vs.\ axiom
schemas), and filtration-based decidability proofs constitute the core
metatheory. Applied extensions include epistemic logic (knowledge and
belief), temporal logic (operators for future and past), and deontic
logic (obligation and permission).

\emph{Source material}: OpenLogic \texttt{content/normal-modal-logic/}
(69~sections) and \texttt{content/applied-modal-logic/} (16~sections).


%% ===================================================================
%% EXT.2: Intuitionistic Logic
%% Sources: 34 intuitionistic-logic/ sections
%% ===================================================================

\section{Intuitionistic Logic} \label{EXT.2}

Intuitionistic logic rejects the law of excluded middle
$\varphi \lor \lnot\varphi$ and the double-negation elimination rule
$\lnot\lnot\varphi \to \varphi$, restricting provability to
constructive reasoning. Its semantics can be given via Kripke models
(with a partial order of information states), Heyting algebras, or the
Brouwer--Heyting--Kolmogorov interpretation (proofs as constructions).
Natural deduction and sequent calculus formulations are obtained by
restricting the classical rules (e.g., allowing at most one formula in
the succedent of sequent calculus sequents, or restricting
\emph{reductio ad absurdum} in ND). The propositions-as-types correspondence (Curry--Howard
isomorphism) connects intuitionistic proofs to typed lambda terms,
establishing deep links between logic and computation.

\emph{Source material}: OpenLogic \texttt{content/intuitionistic-logic/}
(34~sections).


%% ===================================================================
%% EXT.3: Many-Valued Logic
%% Sources: 25 many-valued-logic/ sections
%% ===================================================================

\section{Many-Valued Logic} \label{EXT.3}

Many-valued logics generalize classical logic by allowing truth values
beyond $\{\True, \False\}$. Three-valued logics
({\L}ukasiewicz, Kleene, Priest) introduce a third value for
``unknown,'' ``undefined,'' or ``both true and false.'' Continuous-valued
logics ({\L}ukasiewicz infinite-valued, G\"odel logic, product logic)
use the real interval $[0,1]$ as the truth-value set. The algebraic
semantics is given by MV-algebras, Heyting algebras, or BL-algebras.
Key metatheoretic results include completeness theorems for various
axiomatic systems and the McNaughton theorem characterizing
{\L}ukasiewicz logic functions as piecewise-linear functions on~$[0,1]$.

\emph{Source material}: OpenLogic \texttt{content/many-valued-logic/}
(25~sections).


%% ===================================================================
%% EXT.4: Second-Order Logic
%% Sources: 20 second-order-logic/ sections
%% ===================================================================

\section{Second-Order Logic} \label{EXT.4}

Second-order logic extends first-order logic by allowing quantification
over predicate and function variables in addition to individual
variables. Under the standard (or ``full'') semantics, second-order
quantifiers range over all subsets (or all functions) on the domain.
This dramatically increases expressive power: the natural numbers,
the real numbers, and well-orderings are categorically axiomatizable
in second-order logic. However, this power comes at a metatheoretic
cost: the Completeness Theorem fails for the standard semantics
(second-order validity is not axiomatizable), and the
L\"owenheim--Skolem theorem fails. Under Henkin semantics (where
second-order quantifiers range over a specified subcollection),
completeness and compactness are restored, but categorical
axiomatizability is lost.

\emph{Source material}: OpenLogic \texttt{content/second-order-logic/}
(20~sections).


%% ===================================================================
%% EXT.5: Lambda Calculus
%% Sources: 44 lambda-calculus/ sections
%% ===================================================================

\section{Lambda Calculus} \label{EXT.5}

The lambda calculus, introduced by Church in the 1930s, provides a
formal system for defining and applying functions. The untyped lambda
calculus uses three term constructors---variables, abstraction
($\lambda x.\, M$), and application ($M\, N$)---with
$\beta$-reduction ($(\lambda x.\, M)\, N \to M[N/x]$) as the
fundamental computation rule. The Church--Rosser theorem guarantees
confluence: if a term reduces in two different ways, both reduction
paths can be completed to a common reduct. The simply-typed lambda
calculus adds type annotations ($\lambda x{:}\alpha.\, M$) and
guarantees strong normalization (every reduction sequence terminates),
establishing a correspondence with intuitionistic propositional logic
via the Curry--Howard isomorphism. Extensions to System~F
(polymorphism), dependent types, and the calculus of constructions
connect to modern proof assistants and programming language theory.

\emph{Source material}: OpenLogic \texttt{content/lambda-calculus/}
(44~sections).


%% ===================================================================
%% EXT.6: Other Extensions
%% Sources: counterfactuals/ + methods/ + history/ + reference/
%% ===================================================================

\section{Other Extensions} \label{EXT.6}

Several additional topics in the OpenLogic Project fall outside the
core development but are of independent interest:

\begin{itemize}
\item \textbf{Counterfactual conditionals} (13~sections): Stalnaker
  and Lewis semantics for counterfactuals using sphere models and
  minimal-change semantics, addressing the failure of antecedent
  strengthening, contraposition, and transitivity for the
  counterfactual conditional.

\item \textbf{Methods} (19~sections): Proof techniques and problem
  sets designed for introductory logic courses, including induction
  templates, proof strategies, and worked examples.

\item \textbf{History of logic} (20~sections): Biographies of key
  figures (Cantor, G\"odel, Turing, Tarski, Church, Gentzen, Robinson,
  Noether, Peter, Russell, Zermelo) and historical developments in set
  theory (Cantor on the line and the plane, space-filling curves,
  infinitesimals, pathological sets).

\item \textbf{Reference material} (3~sections): Notation summaries
  and symbol tables.
\end{itemize}

\emph{Source material}: OpenLogic \texttt{content/counterfactuals/},
\texttt{content/methods/}, \texttt{content/history/}, and
\texttt{content/reference/}.
   % CH-EXT: Extensions

%% ===================================================================
%%% Back Matter
%% ===================================================================

\backmatter

%% notation-index.tex
%% Notation Index for "A Lean Systematization of Mathematical Logic"

\chapter*{Notation Index}
\addcontentsline{toc}{chapter}{Notation Index}

\noindent
The following table collects the principal symbols used throughout this
text.  Symbols are grouped by domain; page numbers refer to the first
defining occurrence.

\bigskip

%%% -----------------------------------------------------------------
%%% Standard Sets
%%% -----------------------------------------------------------------

\subsection*{Standard Sets}
\begin{tabular}{@{}lll@{}}
$\Nat$ & Natural numbers & CH-BST \\
$\Int$ & Integers & CH-BST \\
$\PosInt$ & Positive integers & CH-BST \\
$\Rat$ & Rational numbers & CH-BST \\
$\Real$ & Real numbers & CH-BST \\
$\Bin$ & Binary / Boolean values & CH-BST \\
\end{tabular}

\bigskip

%%% -----------------------------------------------------------------
%%% Set-Theoretic Notation (Naive)
%%% -----------------------------------------------------------------

\subsection*{Set-Theoretic Notation}
\begin{tabular}{@{}lll@{}}
$\Setabs{x}{\phi(x)}$ & Set comprehension $\{x : \phi(x)\}$ & CH-BST \\
$\Pow{A}$ & Power set of $A$ & CH-BST \\
$\tuple{a, b}$ & Ordered pair / tuple & CH-BST \\
$A \times B$ & Cartesian product & CH-BST \\
$\dom{f}$ & Domain of $f$ & CH-BST \\
$\ran{f}$ & Range of $f$ & CH-BST \\
$\comp{f}{g}$ & Composition $g \circ f$ & CH-BST \\
$\card{A}$ & Cardinality of $A$ & CH-BST \\
$\cardle{A}{B}$ & $A$ is no larger than $B$ & CH-BST \\
$\cardless{A}{B}$ & $A$ is strictly smaller than $B$ & CH-BST \\
$\cardeq{A}{B}$ & $A$ and $B$ are equinumerous & CH-BST \\
$\Id{A}$ & Identity relation on $A$ & CH-BST \\
\end{tabular}

\bigskip

%%% -----------------------------------------------------------------
%%% Logical Connectives and Quantifiers
%%% -----------------------------------------------------------------

\subsection*{Logical Connectives and Quantifiers}
\begin{tabular}{@{}lll@{}}
$\lnot !A$ & Negation & CH-SYN \\
$!A \land !B$ & Conjunction & CH-SYN \\
$!A \lor !B$ & Disjunction & CH-SYN \\
$!A \lif !B$ & Material conditional & CH-SYN \\
$!A \liff !B$ & Biconditional & CH-SYN \\
$\ltrue$ & Logical truth & CH-SYN \\
$\lfalse$ & Logical falsum & CH-SYN \\
$\lforall[x][!A(x)]$ & Universal quantification & CH-SYN \\
$\lexists[x][!A(x)]$ & Existential quantification & CH-SYN \\
$\eq[t_1][t_2]$ & Identity / equality & CH-SYN \\
\end{tabular}

\bigskip

%%% -----------------------------------------------------------------
%%% Syntax
%%% -----------------------------------------------------------------

\subsection*{Syntax}
\begin{tabular}{@{}lll@{}}
$\Lang{L}$ & Formal language & CH-SYN \\
$\Var$ & Set of variables & CH-SYN \\
$\PVar$ & Set of propositional variables & CH-SYN \\
$\Trm[L]$ & Set of terms of $\Lang{L}$ & CH-SYN \\
$\Frm[L]$ & Set of formulas of $\Lang{L}$ & CH-SYN \\
$\Sent[L]$ & Set of sentences of $\Lang{L}$ & CH-SYN \\
$\Atom{P}{t_1, \ldots, t_n}$ & Atomic formula & CH-SYN \\
$\SubFrm{!A}$ & Set of subformulas of $!A$ & CH-SYN \\
$\FV{!A}$ & Free variables of $!A$ & CH-SYN \\
$\Subst{!A}{t}{x}$ & Substitution of $t$ for $x$ in $!A$ & CH-SYN \\
$!A \ident !B$ & Syntactic identity & CH-SYN \\
$\num{n}$ & Standard numeral for $n$ & CH-SYN \\
\end{tabular}

\bigskip

%%% -----------------------------------------------------------------
%%% Semantics
%%% -----------------------------------------------------------------

\subsection*{Semantics}
\begin{tabular}{@{}lll@{}}
$\Struct{M}$ & First-order structure & CH-SEM \\
$\Domain{M}$ & Domain (universe) of $\Struct{M}$ & CH-SEM \\
$\Assign{f}{M}$ & Interpretation of $f$ in $\Struct{M}$ & CH-SEM \\
$\Value{t}{M}[s]$ & Value of term $t$ under assignment $s$ & CH-SEM \\
$\Sat{M}{!A}[s]$ & $\Struct{M}$ satisfies $!A$ under $s$ & CH-SEM \\
$\Sat[/]{M}{!A}$ & $\Struct{M}$ does not satisfy $!A$ & CH-SEM \\
$\Entails$ & Semantic consequence (entailment) & CH-SEM \\
$\Entails[/]$ & Does not entail & CH-SEM \\
$\Theory{M}$ & Theory of $\Struct{M}$ & CH-SEM \\
$\Mod(\Gamma)$ & Class of models of $\Gamma$ & CH-SEM \\
$\Struct{M} \elemequiv \Struct{N}$ & Elementary equivalence & CH-SEM \\
$\Struct{M} \iso \Struct{N}$ & Isomorphism & CH-SEM \\
$\pAssign{v}$ & Propositional valuation & CH-SEM \\
$\pSat{v}{!A}$ & Propositional satisfaction & CH-SEM \\
$\QuantRank{!A}$ & Quantifier rank of $!A$ & CH-SEM \\
\end{tabular}

\bigskip

%%% -----------------------------------------------------------------
%%% Deduction
%%% -----------------------------------------------------------------

\subsection*{Deduction}
\begin{tabular}{@{}lll@{}}
$\Gamma \Proves !A$ & $!A$ is derivable from $\Gamma$ & CH-DED \\
$\Gamma \Proves[/] !A$ & $!A$ is not derivable from $\Gamma$ & CH-DED \\
$\Proves !A$ & $!A$ is a theorem & CH-DED \\
$\MP$ & Modus ponens & CH-DED \\
$\Intro{\lif}$, $\Elim{\lif}$ & ND introduction / elimination rule & CH-DED \\
$\Discharge{!A}{n}$ & Discharged assumption (label $n$) & CH-DED \\
$\Gamma \Sequent \Delta$ & Sequent & CH-DED \\
$\LeftR{\lif}$, $\RightR{\lif}$ & SC left / right rule & CH-DED \\
$\Weakening$, $\Contraction$, $\Exchange$, $\Cut$ & Structural rules & CH-DED \\
$\sFmla{\True}{!A}$, $\sFmla{\False}{!A}$ & Signed formula (tableau) & CH-DED \\
\end{tabular}

\bigskip

%%% -----------------------------------------------------------------
%%% Theories and Formal Arithmetic
%%% -----------------------------------------------------------------

\subsection*{Theories and Formal Arithmetic}
\begin{tabular}{@{}lll@{}}
$\Th{T}$ & Named theory & CH-DED \\
$\Th{Q}$ & Robinson arithmetic & CH-DED \\
$\Th{PA}$ & Peano arithmetic & CH-DED \\
$\defis$ & Definitional equality & CH-DED \\
$\bforall{x < t}{!A}$ & Bounded universal quantifier & CH-META \\
$\bexists{x < t}{!A}$ & Bounded existential quantifier & CH-META \\
\end{tabular}

\bigskip

%%% -----------------------------------------------------------------
%%% Computability
%%% -----------------------------------------------------------------

\subsection*{Computability}
\begin{tabular}{@{}lll@{}}
$\Zero$, $\Succ$, $\Add$, $\Mult$ & Basic recursive functions & CH-CMP \\
$\Proj{n}{i}$ & Projection function & CH-CMP \\
$\Char{R}$ & Characteristic function of $R$ & CH-CMP \\
$\umin{x}{R(x)}$ & Unbounded minimization & CH-CMP \\
$\cfind{e}$ & $e$-th partial computable function & CH-CMP \\
$f(x) \fdefined$ & $f(x)$ is defined (converges) & CH-CMP \\
$f(x) \fundefined$ & $f(x)$ is undefined (diverges) & CH-CMP \\
$f \colon A \pto B$ & Partial function from $A$ to $B$ & CH-CMP \\
$\concat$ & String / sequence concatenation & CH-CMP \\
\end{tabular}

\bigskip

%%% -----------------------------------------------------------------
%%% G\"odel Numbering and Provability
%%% -----------------------------------------------------------------

\subsection*{G\"odel Numbering and Provability}
\begin{tabular}{@{}lll@{}}
$\gn{!A}$ & G\"odel number of $!A$ & CH-META \\
$\Gn{!A}$ & G\"odel number (corner-quote style) & CH-META \\
$\OProv[\Th{T}](x)$ & Provability predicate for $\Th{T}$ & CH-META \\
$\ORProv[\Th{T}](x)$ & Rosser provability predicate & CH-META \\
$\OCon[\Th{T}]$ & Consistency statement for $\Th{T}$ & CH-META \\
$\OPrf[\Th{T}](x,y)$ & Proof predicate for $\Th{T}$ & CH-META \\
\end{tabular}

\bigskip

%%% -----------------------------------------------------------------
%%% Formal Set Theory
%%% -----------------------------------------------------------------

\subsection*{Formal Set Theory}
\begin{tabular}{@{}lll@{}}
$\ZF$, $\ZFC$ & Zermelo--Fraenkel (with Choice) & CH-SET \\
$\ZFminus$, $\Zminus$ & ZF / Z without Foundation or Choice & CH-SET \\
$\ordtype{A, R}$ & Order type of $(A, R)$ & CH-SET \\
$\ordsucc{\alpha}$ & Ordinal successor & CH-SET \\
$\cardsucc{\cardfont{a}}$ & Cardinal successor & CH-SET \\
$\cardfont{a}$ & Cardinal (fraktur) & CH-SET \\
$V_\alpha$ & Stage $\alpha$ of the cumulative hierarchy & CH-SET \\
$\setrank{x}$ & Rank of set $x$ & CH-SET \\
$\funimage{f}{A}$ & Image of $A$ under $f$ & CH-SET \\
$\trcl{A}$ & Transitive closure of $A$ & CH-SET \\
\end{tabular}


%% further-reading.tex
%% Further Reading for "A Lean Systematization of Mathematical Logic"

\chapter*{Further Reading}
\addcontentsline{toc}{chapter}{Further Reading}

\noindent
This text is compiled from the Open Logic Project
(\texttt{openlogicproject.org}).  The following sources provide deeper
treatment of each domain.

\subsection*{General Logic Textbooks}

\begin{itemize}
\item H.~B.\ Enderton, \textit{A Mathematical Introduction to Logic},
  2nd ed., Academic Press, 2001.  Covers propositional and
  first-order logic through completeness with careful attention to
  detail.

\item E.\ Mendelson, \textit{Elements of Mathematical Logic},
  6th ed., CRC Press, 2015.  A classic treatment including
  axiomatics, first-order logic, and incompleteness.

\item D.\ Marker, \textit{Model Theory: An Introduction}, Springer,
  2002.  Graduate-level model theory; excellent for the semantic and
  metatheoretic material of CH-SEM and CH-META.
\end{itemize}

\subsection*{Set Theory (CH-BST, CH-SET)}

\begin{itemize}
\item P.~R.\ Halmos, \textit{Naive Set Theory}, Springer, 1960.
  The foundational informal set theory underlying CH-BST.

\item K.\ Kunen, \textit{Set Theory: An Introduction to Independence
  Proofs}, North-Holland, 1980.  Standard graduate reference for the
  formal set theory of CH-SET.

\item T.\ Jech, \textit{Set Theory}, 3rd millennium ed., Springer,
  2003.  Comprehensive reference for ordinals, cardinals, and the
  cumulative hierarchy.
\end{itemize}

\subsection*{Syntax and Semantics (CH-SYN, CH-SEM)}

\begin{itemize}
\item W.\ Hodges, \textit{A Shorter Model Theory}, Cambridge
  University Press, 1997.  Concise introduction to structures,
  satisfaction, and elementary equivalence.

\item C.~C.\ Chang and H.~J.\ Keisler, \textit{Model Theory},
  3rd ed., North-Holland, 1990.  Definitive reference for
  L\"owenheim--Skolem, compactness, and ultraproducts.
\end{itemize}

\subsection*{Deduction (CH-DED)}

\begin{itemize}
\item A.~S.\ Troelstra and H.\ Schwichtenberg, \textit{Basic Proof
  Theory}, 2nd ed., Cambridge University Press, 2000.  The standard
  reference for natural deduction and sequent calculus, including
  cut elimination.

\item R.~M.\ Smullyan, \textit{First-Order Logic}, Dover, 1995
  (reprint).  The original systematic treatment of analytic tableaux.

\item S.~R.\ Buss (ed.), \textit{Handbook of Proof Theory},
  North-Holland, 1998.  Comprehensive survey of all major proof
  systems.
\end{itemize}

\subsection*{Computability (CH-CMP)}

\begin{itemize}
\item N.~J.\ Cutland, \textit{Computability: An Introduction to
  Recursive Function Theory}, Cambridge University Press, 1980.
  Clear development of primitive and general recursive functions,
  Turing machines, and undecidability.

\item R.~I.\ Soare, \textit{Recursively Enumerable Sets and
  Degrees}, Springer, 1987.  Advanced treatment of the computability
  theory underlying representability and Church's thesis.

\item H.\ Rogers, \textit{Theory of Recursive Functions and
  Effective Computability}, MIT Press, 1987.  The classic reference
  for recursion theory.
\end{itemize}

\subsection*{Metatheory and Incompleteness (CH-META)}

\begin{itemize}
\item G.~S.\ Boolos, J.~P.\ Burgess, and R.~C.\ Jeffrey,
  \textit{Computability and Logic}, 5th ed., Cambridge University
  Press, 2007.  Excellent coverage of G\"odel's theorems,
  representability, and Tarski's theorem.

\item C.\ Smorynski, \textit{Self-Reference and Modal Logic},
  Springer, 1985.  Deep treatment of the derivability conditions,
  L\"ob's theorem, and the provability logic GL.

\item P.\ Lindstr\"om, ``On extensions of elementary logic,''
  \textit{Theoria}, 35:1--11, 1969.  The original paper on
  Lindstr\"om's theorem characterizing first-order logic.
\end{itemize}

\subsection*{The Open Logic Project}

\begin{itemize}
\item Open Logic Project, \textit{Open Logic: A Complete Text},
  available at \texttt{openlogicproject.org}.  The upstream source
  of all material in this volume, released under a Creative Commons
  license.

\item The taxonomy and dependency analysis underlying this
  systematization are documented in the \texttt{taxonomy/} directory
  of the project repository.
\end{itemize}


\printindex

\end{document}
